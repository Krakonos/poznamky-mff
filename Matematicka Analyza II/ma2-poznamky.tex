\documentclass[a4paper,10pt]{article}
\usepackage[utf8]{inputenc}
\usepackage{a4wide}
\usepackage[czech]{babel}
%\usepackage{bbm}
\usepackage{amsfonts, amsmath, amsthm, amssymb}
\usepackage{math}
\usepackage{enumerate}

\newcommand{\Nu}{\nu}

\title{Matematická analýza II}
\author{Ladislav Láska}

\begin{document}
\maketitle
Učební text k předmětu \texttt{Matematická analýza II} pro informatiky. Vytvořen podle
požadavků ke zkoušce pro paralelku Y (Šámal). Text je povětšinou výtahem z
přednášek uspořádaným do srozumitelných krátkých celků zaměřený na definice,
věty a probrané důkazy.\\

Poděkování za korekce a připomínky: Martin Pelikán, Marek Feňko.

Na obsahu se podíleli: Martin Pelikán (rozklad na parciální zlomky a spoustu
dalších oprav)

\ \\

Pokud najdete chybu nebo nepřesnost, neváhejte mě kontaktovat (třeba na email
ladislav.laska@gmail.com)

\ \\

\textbf{Upozornění}: Tyto poznámky jsou bez jakékoliv záruky. Nemusí být kompletní a mohou
obsahovat chyby.
\newpage
\newpage

% Titlepage is not numbered, but this is required for page compatibility in pdf
% form
% \setcounter{page}{2}

\tableofcontents
\newpage

\section{Primitivní funkce}
\subsection{Definice primitivní funkce}
\setcounter{equation}{0}
\paragraph{Definice} Nechť $f: I \to \R$, $I \in \R^*$, kde $I$ je otevřený interval. 
Řekneme, že $F$~je primitivní funkce k $f$, pokud platí:
\begin{align*}
	\forall x \in I \quad F'(x) = f(x)
\end{align*}
\paragraph{Pozorování} $F(x)$ je primitivní fce k $f(x) \Rightarrow F(x) + c$ je primitivní fce k $f(x)$

\subsection{Věta o jednoznačnosti primitivní funkce}
\setcounter{equation}{0}
\paragraph{Věta} Nechť $F(x)$, $G(x)$ jsou primitivní funkce $f(x)$ na intervalu $I$, pak 
\begin{align*}
	\exists c \in \R \quad \forall x \in I \quad G(x) = F(x) +c
\end{align*}

\paragraph{Důkaz}
\begin{align}
	F'(x) = f(x) \quad G'(x) = f(x) \qquad \forall x \in I
\end{align}
Mějme funkci $H(x) = G(x) - F(x)$. Po zderivování:
\begin{align}
    H'(x) = G'(x) - F'(x) = f(x) - f(x) = 0
\end{align}
Potom ale funkce $H$ je konstantní, proto také existuje konstanta $c$.

\subsection{Existenční věta}
\paragraph{Věta} Nechť $I$ je otevřený interval a $f$ je spojitá na $I$. Pak má $f$ prim. funkci.
\paragraph{Důkaz} Odložen. (vizte větu o~derivaci integrálu podle horní meze)


\subsection{Věta o linearitě primitivní funkce}
\setcounter{equation}{0}
\paragraph{Věta} Nechť $f$ má primitivní fci $F$ a $g$ má primitivní fci $G$ na intervalu $I$. Pak:
\begin{align*}
	\int \alpha f + \beta g  = \alpha F + \beta G
\end{align*}

\paragraph{Důkaz}
Triviálně z linearity derivace:
\begin{align}
	(\alpha F + \beta G)' = \alpha f + \beta g
\end{align}


\subsection{Věta o substituci}
\setcounter{equation}{0}
\paragraph{Věta}
\begin{enumerate}[i.]
	\item Nechť $F$ je primitivní funkce k $f$ na $(a,b)$. A funkce $\varphi$ je:
        \begin{align*}
            &\varphi: (\alpha, \beta) \to (a,b) \\
            &\forall x \in (\alpha, \beta) \quad \varphi'(x)
        \end{align*}
        Pak na  intervalu $(\alpha, \beta)$ platí:
        \begin{align*}
        	\int f(\varphi(t)) \varphi'(t) \dt = F( \varphi(t)) + c
        \end{align*}
    \item Nechť funkce $\varphi$ je $na$ (tj. každý bod z $(a,b)$ má vzor) a platí:
        \begin{align*}
            &f: (a,b) \to \R \\
            &\varphi: (\alpha, \beta) \to (a, b)
        	\quad : \quad \forall x \in (\alpha, \beta) \quad \varphi'(x) \neq 0 \\
            &\int f(\varphi(t)) \varphi'(t) \dt = G(t)
        \end{align*}
        Pak na intervalu $(a, b)$ platí:
        \begin{align*}
            \int f(x) \dx = G(\varphi^{-1}(x)) + c 
        \end{align*}
\end{enumerate}

\paragraph{Důkaz}
\begin{enumerate}
	\item Ověříme zderivováním (jako složenou funkci):
        \begin{align}
        	[F(\varphi(t))+c]' = F'(\varphi(t)) \varphi'(t) = f(\varphi(t)) \varphi'(t)
        \end{align}
    \item Ověříme zderivováním (jako složenou funkci):
        \begin{align}
        	[G(\varphi^{-1}(x))+c]' = G'(\varphi^{-1}(x)) [ \varphi^{-1}(x)]'
        \end{align}
        Z předpokladu ($\varphi' \neq 0$) je $\varphi$ neklesající/nerostoucí na
		celém intervalu. Podle věty o derivaci inverzní funkce potom platí:
        \begin{align}
        	 \label{vos-der} G'(\varphi^{-1}(x)) [ \varphi^{-1}(x)]' = G'(\varphi^{-1}(x)) \cdot \frac{1}{\varphi'(\varphi^{-1}(x))}
        \end{align}
        Do předpokladu ($\int f(\varphi(t)) \varphi'(t) \dt = G(t)$) dosadíme za
		t = $\varphi^{-1}(x)$ a předpoklad zderivujeme:
		\begin{align}
			f(\varphi(\varphi^{-1}(x))) \varphi'(\varphi^{-1}(x)) =
			G'(\varphi^{-1}(x))
		\end{align}
		Což můžeme dosadit do (\ref{vos-der}):
        \begin{align}
        	f(\varphi(\varphi^{-1}(x))) \varphi'(\varphi^{-1}(x)) \cdot \frac{1}{\varphi'(\varphi^{-1}(x))}
        \end{align}
        Což po zkrácení a upravení je $f(x)$.

\subsection{Per-partes}
\setcounter{equation}{0}
\paragraph{Věta} Nechť $I$ je otevřený interval, $f$ a $g$ spojité funkce na $I$. $F$ je primitivní fce k $f$ na $I$, $G$ je primitivní fce k $g$ na $I$. Pak platí:
    \begin{align*}
    	\int F(x)g(x) \dx = F(x) G(x) - \int f(x) G(x) \dx
    \end{align*}

\paragraph{Důkaz}
Mějme pomocnou funkci $H(x)$, primitivní funkci k $f(x)G(x)$, tedy:
    \begin{align}
    	\int f(x)G(x) \dx = H(x)
    \end{align}
Ověříme zderivováním a upravením:
    \begin{align}
    	F(x)g(x) &= (F(x)G(x) - H(x))' \nonumber \\
                &= F'(x)G(x) + F(x)G'(x) - H(x) \nonumber \\
                &= f(x)G(x) + F(x)g(x) - f(x)G(x) \nonumber \\
                &=F(x)g(x)                
    \end{align}
\end{enumerate}

\newpage
\section{Typové integrace}
\subsection{Rozklad na parciální zlomky}
\setcounter{equation}{0}
\paragraph{Věta}
Nechť jsou $P$ a $Q$ polynomy s reálnými koeficienty takovými, že
\begin{enumerate}
	\item $\deg P < \deg Q$
	\item $Q$ je ve tvaru $ a_n (x-x_1)^{p_1} \dots (x-x_k)^{p_k} 
		(x^2+{\alpha}_1+{\beta}_1)^{q_1} \dots (x^2+{\alpha}_l+{\beta}_l)^{q_l} $
		a platí, že žádné dva nemají společný kořen.
\end{enumerate}
Potom existují jednoznačně určena čísla $ A^i_j, B^i_j, C^i_j $  tak, že
\begin{align*}
	\frac{P(x)}{Q(x)} = \sum_{i=1}^{k} \sum_{j=1}^{p_k} \frac{A^i_j}{(x-x_i)^{j}} 
	+ \sum_{i=1}^l \sum_{j=1}^{q_l} \frac{B^i_jx+C^i_j}{(x^2+{\alpha}_ix+{\beta}_i)^{j}} 
\end{align*}
\paragraph{Důkaz}
(pro reálné kořeny) Matematickou indukcí podle stupně Q:
\begin{enumerate}
	\item stupeň $Q = 1 \Rightarrow$ stupeň $P = 0 \Rightarrow Q(x) = a_1(x-x_1) $ a $ P(x) = C $
		$\Rightarrow  \frac{P(x)}{Q(x)} = \frac{\alpha}{x-x_1} $

	\item
Zkusíme šikovně zvolit $\alpha$, aby na $ \displaystyle \frac{P(x)}{Q(x)} - \frac{\alpha}{(x-x_1)^{p_1}} $ šel užít indukční předpoklad: \\
zavedeme funkci
\begin{align}
	H(x) = a_n(x-x_2)^{p_2} \dots (x-x_k)^{p_k} = \frac{Q(x)}{(x-x_1)^{p_1}}    \parbox{1.5cm}{~} H(x_1) \ne 0
\end{align}
a použijeme ji k vyjádření požadovaného
\begin{align}
	\frac{P(x)}{Q(x)} - \frac{\alpha}{(x-x_1)^{p_1}} 
	 = \frac{P(x)}{Q(x)} - \frac{\displaystyle \frac{\alpha Q(x)}{(x-x_1)^{p_1}} }{Q(x)} 
	 = \frac{P(x) - \displaystyle \frac{\alpha Q(x)}{(x-x_1)^{p_1}} }{Q(x)} 
	 = \frac{P(x) - \alpha H(x)}{Q(x)}
\end{align}
Takže $ \exists \alpha$ : $ P(x_1) - \alpha H(x_1) = 0 $ ($x_1$ je kořen, takže ho můžeme vytknout)
\begin{align}
	P(x) - \alpha H(x) = (x-x_1) P_1(x)
\end{align}
Tedy pro $ \alpha = \displaystyle \frac{P(x_1)}{H(x_1)} $ vypadá upravený výraz takto:
\begin{align}
	\frac{(x-x_1)P_1(x)}{Q(x)} = \frac{P_1(x)}{a_n (x-x_1)^{p_1-1} \dots (x-x_k)^{p_k}}
\end{align}
Snížili jsme o 1 stupeň čitatele i jmenovatele, takže lze výsledek rozložit na parciální zlomky,
\begin{align}
	\displaystyle \frac{P(x)}{Q(x)} - \frac{\alpha}{(x-x_1)^{p_1}}
\end{align}
je lineární kombinace výrazů  
\begin{align}
	\displaystyle \{\frac{1}{(x-x_i)^{s}}; 1 \le s \le p_i \} \setminus 
	\{ \frac{1}{(x-x_1)^{p_1}} \}
\end{align}
$\Rightarrow$ po přidání $\displaystyle \frac{\alpha}{(x-x_1)^{p_1}} $ máme rozklad $ \displaystyle \frac{P(x)}{Q(x)} $
\end{enumerate}


\subsection{Integrace racionální funkce}
\setcounter{equation}{0}
\paragraph{Definice}
Racionální funkce $R(x)$ je funkce tvaru:
\begin{align*}
R(x) = \frac{P(x)}{Q(x)}
\end{align*}
\paragraph{Postup}
\begin{enumerate}
	\item Částečně vydělíme, pokud $\deg P \ge \deg Q$
        \begin{align*}
            R(x) = S(x) + \frac{P_1(x)}{Q(x)}       	
        \end{align*}
 	\item Q(x) rozložíme na součin tvaru $(x-x_1)^{p_1} \cdot ... \cdot (x+\alpha_1x+\beta_1)^{q_1}$.
	\item Rozložíme na parciální zlomky.
	\item Integrál vypočítáme jako:
        \begin{align*}
            \int \frac{P(x)}{Q(x)} = \int S(x) + \int \text{ parciálních zlomků }
        \end{align*}
\end{enumerate}

\subsection{Jednoduché substituce}
\setcounter{equation}{0}
\begin{align}
	\int R(e^{a x}) \dx = \left| 
		\begin{matrix}
			y = e^{ax} \\
			\dy = a e^{ax} \dx
		\end{matrix}
		\right| = \int \frac{R(y)}{ay} \dy \\
	\int R(\log x) \frac{1}{x} \dx = \left| \begin{matrix}y=\log x \\ \dy =
	\frac{1}{x} \end{matrix} \right| = \int R(y) \dy
\end{align}

\subsection{Trigonomické funkce}
\setcounter{equation}{0}
\begin{align*}
	\int R(\sin x, \cos x) \dx
\end{align*}
\begin{enumerate}
	\item  $t = \sin x$
	\item  $t = \cos x$
	\item  $t = \tg x$
	\item  $t = \tg \frac{x}{2}$ - funguje vždy
\end{enumerate} 

\subsection{Integrály s odmocninama}
\setcounter{equation}{0}
\begin{align*}
	\int R\left( x, \left (\frac{ax + b}{cx+d}\right)^\frac{1}{q} \right) \dx
\end{align*}

\subsection{Eulerovy substituce}
\setcounter{equation}{0}
\begin{align*}
	\int R(x, \sqrt{ax^2 + bx + c}) \dx
\end{align*}
Pokud $a \neq 0$ ($a = 0 \Rightarrow$ předchozí případ):
\begin{enumerate}
	\item  kv. polynom má jeden kořen v $\R$:
	\begin{align*}
		ax^2 + bx + c &= a(x-\alpha)^2 \\
		\sqrt{ax^2 + bx + c} &= \sqrt{a} |x - \alpha|
	\end{align*}
	Spočítáme $\int R(x, \sqrt{a}|x-\alpha|)$
	\item  kv. polynom má 2 kořeny v $\R$:
	\begin{align*}
		ax^2 + bx + c &= a(x-x_1)(x-x_2) \\
		&= \frac{a(x-x_1)(x-x_2)^2}{x-x_2}
	\end{align*}
	\item  kv. polynom má 0 kořenů v $\R$:
	\begin{align*}
		\sqrt{ax^2 + bx + c} = \sqrt{a} x + t \\
		ax^2 + bx + c = ax^2 + 2t\sqrt{a}x + t^2 \\
		x = \frac{t^2-c}{b-2t\sqrt{a}} 
	\end{align*}
	Po zbavení se kvadratického členu snadno dopočítáme.
\end{enumerate}


\newpage
\section{Určitý (Riemannův) integrál}
\subsection{Dělení intervalu}
\setcounter{equation}{0}
\paragraph{Definice}
\textbf{Dělení} intervalu $[a,b]$ je posloupnost $D = (x_j)_{j=0}^n$, kde $a = x_0 < x_1 <
... < x_n = b$.
\paragraph{Definice}
Nechť $D$, $D'$ jsou dělení intervalu $[a,b]$. O $D'$ řekneme, že
\textbf{zjemňuje} $D$ pokud $D\subsetq D'$ (tj. $\forall x \in D \quad  x \in D'$)


\subsection{Horní a dolní součet}
\setcounter{equation}{0}
\paragraph{Definice}
Nechť f je omezená funkce na $[a,b]$, $D=(x_j)$ dělení $[a,b]$. Potom: \\
\textbf{Horní součet} je:
\begin{align*}
	S(f, D) = \sum_{j=1}^n |x_j - x_{j-1}| \cdot \sup \{f(x) \quad 
	x \in [x_{j-1},	x_j]\}
\end{align*}
\textbf{Dolní součet} je:
\begin{align*}
	s(f, D) = \sum_{j=1}^n |x_j - x_{j-1}| \cdot \inf \{f(x) \quad 
	x \in [x_{j-1},	x_j]\}
\end{align*}


\paragraph{Poznámka}Pokud plocha pod křivkou je P: $s(f,D) \le P \le S(f,D)$

\subsection{Horní a dolní Riemannův integrál}
\setcounter{equation}{0}
\paragraph{Definice}
\begin{align*}
	(R) \overline{\int_a^b} f(x) \dx = \inf \{ S(f, D) \quad D \text{ dělení }
	[a,b]\} \\
	(R) \underline{\int_a^b} f(x) \dx = \sup \{ s(f, D) \quad D \text{ dělení }
	[a,b]\}
\end{align*}

\paragraph{Definice}
Pokud 
\begin{align*}
(R) \overline{\int_a^b} f(x) \dx = (R) \underline{\int_a^b} f(x) \dx = A
\end{align*}
Potom
\begin{align*}
(R) \int_a^b f(x) \dx := A
\end{align*}
a říkáme, že je $f$ Riemannovsky integrovatelná.

\paragraph{Definice}
\begin{align*}
	R([a,b]) = \{ \text{ Riemannovsky integrovatelných funkcí na } [a,b]\}
\end{align*}

\paragraph{Poznámka}
\setcounter{equation}{0}
\begin{align}
	f \text{ spojitá na } [a,b] \Rightarrow f \in R([a,b]) \\
	f \in R([a,b]) \Rightarrow f \text{ je omezená }
\end{align}

\subsection{Věta o zjemnění dělení}
\setcounter{equation}{0}
\paragraph{Věta}
Nechť $f$ je omezená funkce na $[a,b]$, $D$ a $D'$ jsou dělení $[a,b]$ a $D'$
zjemňuje $D$, potom:
\begin{align*}
	s(f,D) \le^{(1)} s(f,D') \le^{(2)} S(f,D') \le^{(3)} S(f,D)
\end{align*}
% Tady jsem si pridal vlastni inline cislovani, tak proto...
\setcounter{equation}{3}

\paragraph{Důkaz}
\begin{enumerate}
	\item  $s(f,D) \le s(f,D')$ \\ 
	Matematickou indukcí podle počtu přidaných bodů: 
	\begin{enumerate}[i.]
		\item  $|D'| = |D|+1$
		\item  
			\begin{align*}
				s(f, D) = \sum_{j=1}^n |x_j - x_{j-1}| \cdot \inf \{f(x) \quad 
				x \in [x_{j-1},	x_j]\}
			\end{align*}
			Pokud přibyde jeden bod do zjemnění, i-tý člen se rozdělí na dva, ty
			jsou však alespoň stejně tak velké jako původní (protože pokud by
			existoval menší, muselo by i infinum původního členu být menší).
	\end{enumerate}
	\item  triviální: $\inf M \le \sup M$
	\item  analogicky k 1.
\end{enumerate}

\subsection{Věta o dvou děleních}
\setcounter{equation}{0}
\paragraph{Věta}
Nechť $f$ je omezená funkce na $[a,b]$, $D_1$ a $D_2$ jsou dělení na $[a,b]$. Pak
\begin{align*}
	s(f, D_1) \le S(f, D_2)
\end{align*}

\paragraph{Důkaz}
Vytvoříme společné zjemnění $D := D_1 \cup D_2$. Pak je podle předchozí věty
triviální.

\subsection{Norma dělení}
\setcounter{equation}{0}
\paragraph{Definice}
Nechť $D$ je dělení $[a,b]$, $D = (x_j)_{j=0}^n$. Potom \textbf{normou dělení}
nazveme
\begin{align*}
	\Nu(D) := \max \{ |x_j - x_{j-1} |, j = 1..n \}
\end{align*}


\subsection{Věta o aproximaci Riemannova integrálu pomocí součtů}
\setcounter{equation}{0}
\paragraph{Věta}
Nechť $f$ je omezená funkce $[a,b] \to \R$. $(D_n)_{n=1}^\infty$ je posloupnost
dělení taková, že 
\begin{align*}
	\lim_{n\to\infty} \Nu(D_n) = 0
\end{align*}
Potom:
\begin{align*}
	\overline{\int_a^b} f(x) \dx = \inf S(f, D_n) \\
	\underline{\int_a^b} f(x) \dx = \sup s(f, D_n)
\end{align*}

\paragraph{Poznámka}
\begin{enumerate}
	\item  Rovnoměrné dělení: $\Nu(D_n) = \frac{1}{2}$
	\item  Pokud $\lim S(f, D_n) = \lim s(f, D_n) = a$, potom $\int f(x) \dx = A$
	\item  Pokud víme, že je funkce Riemannovsky integrovatelná, stačí jedna z
	limit.
\end{enumerate}

\paragraph{Bez důkazu.}


\subsection{Kritérium existence Riemannova integrálu}
\setcounter{equation}{0}
\paragraph{Věta}
Nechť $f$ je omezená funkce na $[a,b]$. Potom:
\begin{align*}
	f \in R([a,b]) \Leftrightarrow \forall \epsilon > 0 \quad  \exists D \text{
	dělení } [a,b]: \quad S(f, D) - s(f,D) < \epsilon
\end{align*}
\paragraph{Důkaz}
\begin{enumerate}
	\item  $\Rightarrow$ 
		\begin{align}
			(R) \int f(x) \dx = A
			&= \sup\{s(f,D) \} \label{kriterium-r-1}\\
			&= \inf\{S(f,D)\} \label{kriterium-r-2}
		\end{align}
		Tedy podle (\ref{kriterium-r-1}) víme, že $A - \frac{\epsilon}{2}$ není
		horní závora a podle (\ref{kriterium-r-2}) že $A+\frac{\epsilon}{2}$ není
		dolní závora. Tedy:
		\begin{align}
			\exists D_1 \quad s(f, D_1) > A - \frac{\epsilon}{2}
			\label{kriterium-r-d1} \\
			\exists D_2 \quad S(f, D_2) < A + \frac{\epsilon}{2}
			\label{kriterium-r-d2}
		\end{align}
		Vytvořme tedy společné zjemnění $D$ dělení $D_1$ a $D_2$. Podle věty o zjemnění
		dělení tedy:
		\begin{align}
			A-\frac{\epsilon}{2} < s(f, D_1) \le s(f, D) \le S(f,D) \le S(f,
			D_2) < A+ \frac{\epsilon}{2}
		\end{align}
		Po úpravě (odečtení A, přičtení $\frac{\epsilon}{2}$) získáme:
		\begin{align}
			S(f, D) - s(f,D) < \epsilon
		\end{align} 
	\item  $\Leftarrow$ 
		\begin{align}
			\exists D_\epsilon \text{ dělení } \quad S(f, D_\epsilon) -
			s(f,D_\epsilon) < \epsilon
		\end{align}
		Po rozepsání z definic získáme (ne)rovnosti:
		\begin{align} 
			s(f, D_\epsilon) \le \sup s(f,D_\epsilon) &=\\
			= \underline{\int_a^b} f(x) \dx &\le \overline{\int_a^b} f(x) \dx =\\
			&= \inf S(f,D_\epsilon) \le S(f, D_\epsilon) 
		\end{align}
		Z čehož je vidět, že:
		\begin{align} 
			0 \le \overline{\int_a^b} f(x) \dx -
			\underline{\int_a^b} f(x) \dx < \epsilon
		\end{align}
		Pokud to však platí pro $\epsilon$ libovolně malé, pak:
		\begin{align}
			\overline{\int_a^b} f(x) \dx - \underline{\int_a^b} f(x) \dx = 0\\
			\overline{\int_a^b} f(x) \dx = \underline{\int_a^b} f(x) \dx
		\end{align} 
		Funkce $f$ je tedy Riemannovsky integrovatelná.
\end{enumerate}


\subsection{Věta o monotónnii a Riemannovské integrovatelnosti}
\setcounter{equation}{0}
\paragraph{Věta}
Je-li $f$ omezená a monotónní na $[a,b]$, pak $f \in R([a,b])$.
\paragraph{Důkaz}
BÚNO předpokládejme, že $f$ je rostoucí a $M$ je suprémum (z omezenosti existuje):
\begin{align}
	M: \quad \forall x \in [a,b] \quad |f(x)| \le M
\end{align}
Mějme $D_n$ rovnoměrné dělení s krokem $\frac{a-b}{n}$. Potom:
\begin{align}
	S(f, D_n) - s(f, D_n) &= \sum_{j=1}^n \left[ f(x_j) - f(x_j-1) \right] \cdot
	\frac{b-a}{n} \\
	&= \frac{b-a}{n}\left( f(b) - f(a) \right)
\end{align}
Ověříme podmínky pro kritérium riemannovské integrovatelnosti: \\
Pro každé $\epsilon$ zvolíme $n$:
\begin{align}
	\frac{b-a}{n} (f(b)-f(a)) < \epsilon
\end{align}
Podle Archimedovy vlastnosti takové existuje. Použijeme tedy dělení $D_n$ s právě
takovým $n$. Potom platí:
\begin{align}
	S(f, D_n) - s(f, D_n) < \epsilon
\end{align}
A podle kritéria o riemannovské integrovatelnosti je funkce $f$ integrovatelná na
$[a,b]$


\subsection{Stejnoměrná spojitost}
\setcounter{equation}{0}
\paragraph{Definice}
Řekneme, že funkce $f$ je stejnoměrně spojitá na intervalu $I$ pokud:
\begin{align*}
	\forall \epsilon > 0 \quad \exists \delta > 0 \qquad \forall x,y \in I \quad
	|x-y| < \delta \quad \Rightarrow \quad |f(x) - f(y)| < \epsilon
\end{align*}


\subsection{Věta o spojitosti a stejnoměrné spojistosti (bez důkazu)}
\setcounter{equation}{0}
\paragraph{Věta}
Nechť $f$ je spojitá na $[a,b]$, potom je $f$ stejnoměrně spojitá na $[a,b]$.
(bez důkazu)

\subsection{Věta o spojitosti a riemannovské integrovatelnosti}
\setcounter{equation}{0}
\paragraph{Věta}
Nechť $f$ je spojitá na $[a,b]$, pak $f \in R([a,b])$.
\paragraph{Důkaz}
Pro kritérium riemannovské integrovatelnosti chceme:
\begin{align}
	\forall \epsilon > 0 \quad \exists D \quad S(f,D) - s (f,D) < \epsilon
\end{align}
Mějme tedy $\epsilon > 0$, ze stejnoměrné spojistosti existuje $\delta > 0$.
Vezměme dělení $D$: $\Nu(D) < \delta$ (třeba rovnoměrné):
\begin{align}
	S(f,D) - s(f,D) &= \sum_{j=1}^n |x_j - x_{j-1}| \cdot
		\underbrace{(\sup f(t) - \inf f(t))}_{t \in [x_{j-1},x_j]}
\end{align}
Shora odhadneme:
\begin{align}
	\label{vosss-konst} \sum_{j=1}^n |x_j - x_{j-1}| &\le (b-a) \\
	\label{vosss-maxepsilon} \sup f(t) - \inf f(t) &< \epsilon
\end{align}
Vidíme, že (\ref{vosss-konst}) je maximálně konstanta a dokážeme, že
(\ref{vosss-maxepsilon}) se
dá shora odhadnout jako $\epsilon$:\\
Ze spojitosti a omezenosti plyne, že $f$ nabývá na $[x_{j-1},x_j]$ maxima v $M$ a minima v
$m$. Pak tedy platí:
\begin{align}
	|m-M| \le |x_j - x_{j-1}| \le \Nu (D) < \delta
\end{align}
Což podle stejnoměrné spojitosti dokazuje (\ref{vosss-maxepsilon}) a postačuje pro
kritérium existence riemannovského integrálu.

\subsection{Vlastnosti riemannovského integrálu (bez důkazu)}
\setcounter{equation}{0}
\paragraph{Věta}
\begin{enumerate}
	\item Linearita na $[a,b]$ součtu a násobku $\alpha \in \R$: ($f,g \in R([a,b])$)
		\begin{align*}
			(R) \int_a^b f+g = (R) \int_a^b f \quad + \quad (R) \int_a^b g\\
			(R) \int_a^b \alpha \cdot g = \alpha \cdot (R) \int_a^b g
		\end{align*}
	\item Uspořádání: ($f,g \in R([a,b])$)
		\begin{align*}
			f \le g \qquad \Rightarrow \qquad (R) \int_a^b f \le (R) \int_a^b g
		\end{align*}
	\item Aditivita vzhledem k intervalu: ($a<b<c$)
		\begin{align*}
			&(i) \qquad &f \in R([a,c]) \Leftrightarrow f \in R([a,b]) \land f
			\in R([b,c])\\
			&(ii) \qquad &(R) \int_a^c f = (R) \int_a^b f \quad + \quad (R) \int_b^c f
		\end{align*}
\end{enumerate}
(bez důkazu)

\subsection{Věta o derivace integrálu podle horní meze}
\setcounter{equation}{0}
\paragraph{Věta}
Nechť $J$ je neprázdný interval, $c \in J$ konstantní bod a $f$ je funkce pro niž platí:
\begin{align*}
	\forall \alpha, \beta \in J \quad f \in R([\alpha, \beta])
\end{align*}
F(x) přiřadíme:
\begin{align*}
	F(x) &:=  (R) \int_c^x f(t) \dt \qquad &(x > c) \\
	F(x) &:= -  (R) \int_x^c f(t) \dt \qquad &(x < c)
\end{align*}
Potom:
\begin{enumerate}
	\item $F$ je spojitá na $J$
	\item $f$ je spojitá v $x_0 \in J \Rightarrow F'(x_0) = f(x_0)$
\end{enumerate}
\paragraph{Důkaz}
\begin{enumerate}
	\item Ukážeme, že $\forall x_0 \in J$ je $F$ spojitá: 
		\textbf{chceme} $\lim_{x\to x_{0+}} F(x) - F(x_0)=0$\\
		Mějme tedy $c < x_0 < y \in J$. Potom $f \in R([c,y]) \Rightarrow f $ je omezená na $[c,y]
		\Rightarrow \exists M \quad |f([c,y])| \le M $.\\
		\begin{align}
			\label{vodrhm-prvni} F(x) - F(x_0) = \int_{x_0}^x f(t) \dt & \le \int_{x_0}^x M \dt =
			M(x-x_0)\\
			&\ge \int_{x_0}^x - M \dt = -M (x - x_0)
		\end{align}
		Podle definice limity potřebujeme:
		\begin{align}
			\label{vodrhm-lim}\forall \epsilon > 0 \quad \exists \delta > 0 \quad \forall x,x_0
			\in J \quad |x-x_0| < \delta \quad \Rightarrow  \quad |F(x) -
			F(x_0)| < \epsilon
		\end{align}
		Podle (\ref{vodrhm-prvni}) a definice limity (\ref{vodrhm-lim}) však
		víme:
		\begin{align}
			|F(x) - F(x_0)| \le M|x-x_0| \le M\delta = \epsilon
		\end{align}
		Zvolíme tedy $\delta := \frac{\epsilon}{M}$
	\item Víme, že $f$ je spojitá v $x_0 \in J$. Rozpisem podle derivace
		získáme:
		\begin{align}
			F'(x_0) = \lim_{x \to x_0 + } \frac{F(x) - F(x_0)}{x-x_0}
		\end{align}
		Podle věty o vlastnostech integrálu upravíme čitatele:
		\begin{align}
			= \lim_{x \to x_0} \frac{1}{x - x_0} \int_{x_0}^x f(t) \dt
		\end{align}
		Kde můžeme integrál odhadnout jako:
		\begin{align}
			 \int_{x_0}^x (f(x_0)-\epsilon) \dt
			 \le \int_{x_0}^x f(t) \dt 
			 \le \int_{x_0}^x (f(x_0)+\epsilon) \dt
		\end{align}
		Nyní je již vidět, že podle definice je limita rovna $f(x_0)$
\end{enumerate}
\paragraph{Důsledek}
$f$ je spojitá na $(\alpha, \beta) \quad  \Rightarrow \quad \int_\alpha^\beta f(t) \dt =
[F(t)]_\alpha^\beta$



\subsection{Newtonův integrál}
\setcounter{equation}{0}
\paragraph{Definice}
Newtonův integrál funkce f na intervalu (a,b) je:
\begin{align*}
	(N) \int_a^b f(x) \dx = \lim_{x\to b-} F(x) - \lim_{x \to a+} F(x)
\end{align*}
kde F je primitivní funkce k f na (a,b) a limity jsou vlastní. Píšeme:
\begin{align*}
	(N) \int_a^b f(t) \dt = [ F(t) ]_a^b
\end{align*}

\subsection{Per-partes pro určitý integrál}
\setcounter{equation}{0}
\paragraph{Věta}
Nechť $f$, $f'$, $g$, $g'$ jsou spojité na $[a,b]$, potom:
\begin{align*}
	\int_a^b fg' = [fg]_a^b - \int_a^b f'g
\end{align*}

\paragraph{Důkaz}
Nechť $H$ je primitivní funkce k $f'g$ na $(a,b)$, $K$ je primitivní funkce k
$fg'$ na $(a,b)$. \\
Víme, že: 
\begin{align}
	K(x) = f(x)g(x) - H(x)
\end{align}
Analogicky určitý integrál:
\begin{align}
	[K(x)]_a^b = [fg]_a^b - [H]_a^b
\end{align}


\subsection{Substituce pro určitý integrál (bez důkazu)}
\setcounter{equation}{0}
\paragraph{Věta}
\begin{enumerate}[i.]
	\item  Nechť $f$ je spojitá funkce na $[a,b]$. A funkce $\varphi$ je:
	\begin{align*}
		\varphi&: [\alpha,\beta] \to [a,b] \\
		\varphi& \text{ spojitá na } [\alpha, \beta]
	\end{align*}
	Potom:
	\begin{align*}
		\int_\alpha^\beta f(\varphi(t)) \varphi'(t) \dt =
		\int_{\varphi(\alpha)}^{\varphi(\beta)} f(x) \dx
	\end{align*}
	
	\item  Nechť $f$ je spojitá funkce na $[a,b]$. A funkce $\varphi$ je:
	\begin{align*}
		\varphi&: [\alpha,\beta] \to [a,b] \\
		\varphi& \text{ spojitá na } [\alpha, \beta]\\
		\varphi&' \neq 0 
	\end{align*}
	Potom:
	\begin{align*}
		\int_a^b f(x) \dx = \int_{\varphi^{-1}(a)}^{\varphi^{-1}(b)}
		f(\varphi(t)) \varphi'(t) \dt
	\end{align*}
\end{enumerate}
(bez důkazu)


\section{Aplikace určitého integrálu}
\subsection{Obsah plochy pod křivkou}
\paragraph{Definice}
Nechť $f$ je nezáporná funkce spojitá na $[a,b]$. Pak obsahem plochy pod křivkou $S$ nazveme:
\setcounter{equation}{0}
\begin{align*}
	S = \int_a^b f(x) \dx
\end{align*}

\subsection{Délka křivky}
\paragraph{Definice}
Nechť $f$ je spojitá funkce na $[a,b]$ a $D=(x_j)_{j=0}^n$ dělení intervalu
$[a,b]$. Délkou lomené čáry podle dělení nazveme:
\begin{align*}
	L(f,D) = \sum_{k=1}^n \sqrt{(x_k - x_{k-1})^2 + (f(x_k) - f(x_{k-1}))^2}
\end{align*}
\paragraph{Definice}
Délkou křivky nazveme:
\begin{align*}
	L(f([a,b])) = \sup \{ L(f,D), D \text{ dělení } [a,b]\}
\end{align*}
\paragraph{Věta}
\setcounter{equation}{0}
Nechť má funkce $f$ na intervalu $[a,b]$ spojitou první derivaci. Potom:
\begin{align*}
	L(f) = \int_\alpha^\beta \sqrt{1+(f'(x))^2} \dx
\end{align*}
\paragraph{Důkaz}
Podle definice délky lomené čáry rozepíšeme:
\begin{align}
	L(f,D) &= \sum_{k=1}^n \sqrt{(x_k - x_{k-1})^2 + (f(x_k) - f(x_{k-1}))^2}\\
		&= \sum_{k=1}^n (x_k - x_{k-1}) \sqrt{ 1 + 
				\left( \frac{f(x_k) - f(x_{k-1})}{x_k - x_{k-1}} \right)^2
			}
\end{align}
Na výraz v závorce lze uplatnit Lagrangeho větu o střední
hodnotě a odhadnout pomocí derivace:
\begin{align}
	\frac{f(x_k) - f(x_{k-1})}{x_k - x_{k-1}} = f'(\xi_k) \qquad \xi_k \in
	(x_{k-1}, x_k) \\
	\inf\{f'(x), x \in (x_{k-1}, x_k)\}
	\le f'(\xi)
	\le \sup \{f'(x), x \in (x_{k-1}, x_k)\}
\end{align}
Tedy můžeme $L(f,D)$ považovat za integrální součty. Tedy:
\begin{align}
	s(\sqrt{1+(f'(x))^2}, D) \le L(f,D) \le S(\sqrt{1+(f'(x))^2}, D)
\end{align}
Stačí si tedy uvědomit, že $L(f)$ je definovaná jako supremum $L(f,D)$, tedy nutně musí
platit:
\begin{align}
	\int_a^b( \sqrt{1+(f'(x))^2} ) \dx \le L(f)
\end{align}
(opačná nerovnost bez důkazu)

\subsection{Délka křivky v $\R^n$ (bez důkazu)}
\setcounter{equation}{0}
\paragraph{Věta}
Nechť $\varphi: [a,b] \to \R^n$ je spojitá a má spojitou první derivaci. Potom délku
křivky vypočteme:
\begin{align*}
	L(\varphi([a,b])) = \int_a^b \sqrt{ (\varphi_1'(x))^2 + ... +
	(\varphi_n'(x))^2}\dx
\end{align*}
(bez důkazu)

\subsection{Objem a povrch rotačního tělesa}
\setcounter{equation}{0}
Nechť $f$ je nezáporná funkce spojitá na $[a,b]$. Pak definujeme objem $V$ a
povrch $P$
rotačního tělesa vzniklého rotací křivky $f$ okolo osy $x$ jako:
\begin{align*}
	V &= \int_a^b \pi f^2(x) \dx \\
	P &= \int_a^b 
	\underbrace{2\pi f(x)}_{\text{kruh o } r = f(x)} 
	\underbrace{\sqrt{1+(f'(x))^2}} _{\text{délka křivky}}
	\dx
\end{align*}

\subsection{Odhad konečných součtů řad}
\setcounter{equation}{0}
\paragraph{Věta}
Nechť $f$ je nerostoucí funkce na intervalu $[a-1,b]$ (resp. $[a,b+1]$) a $c_k =
f(k)$. Potom (pro nerostoucí):
\begin{align*}
	\sum_{k=a}^b c_k &\le \int_{a-1}^b f(x) \dx \\
		&\ge \int_a^{b+1} f(x) \dx
\end{align*}
(pro neklesající platí obrácené nerovnsti)

\paragraph{Důkaz}
Nechť D rovnoměrné dělení s krokem 1 ( $D=(a-1,a,..., b)$ ), potom S(f,D) je
jeden z integrálních součtů. Nutně tedy (z definice Riemannova integrálu) platí,
že:
\begin{align}
	\int_a^b f(x) \dx \le \overline{\int_a^b} f(x) \dx = \sup \{ S(f,D), \forall D
	\text{ dělení } [a,b] \}
\end{align}
(pro dolní součty analogicky)

\subsection{Integrační kritérium konvergence řad}
\setcounter{equation}{0}
\paragraph{Věta}
Nechť $f$ je nezáporná, nerostoucí a spojitá funkce na intervalu $[n_0 - 1, \infty]$ 
pro nějaké $n_0 \in \N$. Nechť pro posloupnost $a_n$ platí: 
\begin{align*}
	\forall n \ge n_0 \quad a_n = f(n)
\end{align*}
Potom:
\begin{align*}
	(N) \int_{n_0}^\infty f(x) \dx < \infty \quad \Leftrightarrow \quad
	\sum_{n=1}^\infty a_n \text{ konverguje }
\end{align*}

\paragraph{Důkaz}
Triviální důsledek věty o odhadu součtů řad.


\newpage
\section{Funkce více proměnných}
\setcounter{equation}{0}
\subsection{Funkce více proměnných, okolí}
\setcounter{equation}{0}
\paragraph{Definice}
Funkce více proměnných je zobrazení $f: M \to \R, M \in \R^n$
\paragraph{Definice}
Nechť $a \in \R^n$, $a = (a_1, ..., a_n)$. Okolí a prstencové okolí definujeme
jako:
\begin{align*}
	&U(a, \delta) = (a_1 - \delta, a_1 + \delta) \times ... \times (a_n - \delta,
	a_n + \delta) \\
	&P(a, \delta) = U(a, \delta) \setminus \{ a \}
\end{align*}

\subsection{Otevřená a uzavřená množina}
\setcounter{equation}{0}
\paragraph{Definice}
Množina $G$ je otevřená množina právě když platí:
\begin{align*}
	\forall x \in G \quad \exists \delta > 0 \quad U(x, \delta) \subsetq G
\end{align*}
\paragraph{Definice}
Množina $F$ je uzavřená právě když $\R^n \setminus F$ je uzavařená množina

\subsection{Limita funkce více proměnných}
\setcounter{equation}{0}
\paragraph{Definice}
Nechť $a \in \R^n$, $A \in \R^*$ a $f: M \to \R$. Potom definujeme limitu:
\begin{align*}
	\lim_{x\to a} f(x) = A \qquad
	\Leftrightarrow  \qquad
	\forall \epsilon > 0 \quad \exists \delta > 0 \quad \forall x \in P(a,
	\delta) \quad f(x) \in U(A, \epsilon)
\end{align*}

\subsection{Parciální derivace}
\setcounter{equation}{0}
Nechť funkce $f$ je funkce n proměnných. Definujeme \textbf{parciální derivaci}
funkce $f$ podle $x_i$ jako:
\begin{align*}
	\frac{\partial f}{\partial x_i} = \lim_{h \to 0} \frac{f(x_1, ..., x_i + h,
	..., x_n) - f(x_1, ..., x_n)}{h}
\end{align*}

\subsection{Hessova matice}
\setcounter{equation}{0}
\paragraph{Definice}
Nechť existují všechny druhé parciální derivace funkce $f$. Potom definujeme
\textbf{Hessovu matici} předpisem:
\begin{align*}
	\left( \frac{\partial^2 f(a)}{\partial x_i \, \partial x_j} \right)_{i,j=1}^n
\end{align*}


\subsection{Postačující podmínka pro extrém (bez důkazu)}
\setcounter{equation}{0}
\paragraph{Věta}
Nechť je funkce $f: G \to \R$, $G \subsetq \R^n$ otevřená množina a bod $a \in G$.
Nechť jsou všechny první parciální derivace rovny nule nebo neexistují a všechny
druhé derivace existují. Potom existenci extrému ověříme následovně: \\
Matice $H_f(x)$ je:
\begin{enumerate}[i.]
	\item \textbf{pozitivně definitní} - v $a$ je lokální \textbf{minimum}
	\item \textbf{negativně definitní} - v $a$ je lokální \textbf{maximum}
	\item \textbf{indefinitní} - v $a$ není lokální extrém
	\item \textbf{pozitivně/negativně semidefinitní} - nedává informaci
\end{enumerate}
(bez důkazu)

\subsection{Spojitost funkce více proměnných}
\setcounter{equation}{0}
\paragraph{Definice}
Funkce $f: M \to \R$ je spojitá v bodě $a \in \R^n$ pokud $\lim_{x\to a} f(x) = f(a)$.

\subsection{Nutná podmínka na extrém}
\setcounter{equation}{0}
\paragraph{Věta}
Nechť je funkce $f: G\to \R$, $G \subsetq \R^n$ otevřená množina a bod $a \in G$.
Nechť $f$ nabývá v bodě $a$ lokálního extrému. Potom:
\begin{align*}
	\forall i \quad \frac{\partial f(a)}{\partial x_i} = 0 \text{ nebo
	neexistuje}
\end{align*}
\paragraph{Důkaz}
Definujeme funkci jedné proměnné $g_i(h) = f(a_1, ..., a_i+h, ..., a_n)$. Pokud
má $f$ v $a$ lokální extrém, potom $g_i$ nutně musí mít v $0$ také lokální extrém nebo
neexistuje $\forall i$. Taktéž je vidět, že $g_i'(0) = \frac{\partial f(a)}{\partial x_i}$.

\subsection{Totální diferenciál}
\setcounter{equation}{0}
\paragraph{Definice}
Nechť je funkce $f: G \to \R$, $a \in G$ a funkce L definovaná:
\begin{align*}
	L: \R^n \to \R \text{ lineární funkce } L(h) = ch = c_1 h_1 + c_2 h_2 ...
\end{align*}
L je totální diferenciál pokud platí:
\begin{align*}
	\lim_{h\to(0,..)} \frac{f(a+h) - f(a) - L(h)}{h} = 0
\end{align*}
Značíme:
\begin{align*}
	L = Df(a) \\
	L(h) = Df(a)(h)
\end{align*}

\subsection{Tvar totálního diferenciálu}
\setcounter{equation}{0}
\paragraph{Věta}
\begin{enumerate}[i.]
	\item Pokud existuje totální diferenciál, existují všechny parciální derivace.
	\item 
	\begin{align*}
		Df(a)(h) = \sum_{i=1}^n \frac{\partial f(a)}{\partial x_i} \cdot h_i =
		\nabla f(a) \cdot h
	\end{align*}	
\end{enumerate}
\paragraph{Důkaz}
\begin{enumerate}[i.]
	\item Triviálně podle ii. (protože obsahuje všechny parciální derivace,
	které tímto musí existovat)
	\item Víme, že:
		\begin{align}
			\lim_{h\to(0,..)} \frac{f(a+h) - f(a) - Df(a)(h)}{h} = 0
		\end{align}
		Nechť $h = t \cdot e_i$, tedy:
		\begin{align}
			\lim_{t\to0} \frac{f(a+te_i)-f(a)}{t} - \frac{c_i t}{t} = 0 \\
			\Rightarrow c_i = \frac{\partial f}{\partial x_i}
		\end{align}
		Což platí pro každé i.
\end{enumerate}

\subsection{Věta o aritmetice totálního diferenciálu (bez důkazu)}
\setcounter{equation}{0}
\paragraph{Věta}
Nechť jsou $a \in G \subsetq \R^n$, $f,g: G \to \R$, $G$ otevřená množina
a nechť existují $Df(a)$ a $Dg(a)$. Potom:
\begin{align*}
	&D(f+g)(a) = Df(a) + Dg(a) \\
	&D(c\cdot f)(a) = c \cdot Df(a) \\
	&D(f \cdot g)(a) = f(a) \cdot Dg(a) + g(a) \cdot Df(a) \\
	&D( \frac{f}{g})(a) = \frac{Df(a) \cdot g(a) - f(a) \cdot Dg(a)}{g^2(a)}
\end{align*}
(bez důkazu)

\subsection{Diferenciál složeného zobrazení (bez důkazu)}
\setcounter{equation}{0}
\paragraph{Věta}
Nechť $a \in \R^s$, $b\in \R^n$, $f$ je funkce $n$ proměnných a $g_j$, $j = 1...
n$ jsou funkce $s$ proměnných. Nechť:
\begin{align*}
	g_j (a) = b_j \text{ pro } j \in \{1, ..., n\} \\
	\exists Df(b), Dg_1(a), ..., Dg_n(a).
\end{align*}
Definujeme funkci $H: \R^s \to \R$ předpisem:
\begin{align*}
	H(x) = f(g_1(x), ..., g_n(x))
\end{align*}
Pak $H$ má v bodě $a$ totální diferenciál a platí:
\begin{align*}
	DH(a)(h) = \sum_{i=1}^s \left( \sum_{j=1}^n \frac{\partial f}{\partial
	y_j}(b) \frac{\partial g_j}{\partial x_i}(a) \right) h_i
\end{align*}
Pro všechny $h \in \R^s$, neboli v maticovém zápisu:
\begin{align*}
	DH(a)(h) = Df(g(a))Dg(a)h
\end{align*}
(bez důkazu)


\subsection{Věta o existenci extrémů funkce více proměnných}
\setcounter{equation}{0}
\paragraph{Poznámka}
Pro důkaz této věty je třeba teorie metrických prostorů uvedená ke konci
poznámek. Důkaz je doporučen studovat až po metrických prostorech.
\paragraph{Věta}
Nechť $(P, \rho)$ je metrický prostor, $K \subsetq P$ je kompaktní a funkce
$f:K\to \R$ je spojitá na $K$. Potom:
\begin{enumerate}
	\item $f$ nabývá na $K$ maxima a minima
	\item $f$ je na $K$ omezená
\end{enumerate}
\paragraph{Důkaz} 
\begin{enumerate}
	\item pro maximum: \\
	Mějme $s := \sup\{f(x), x \in K\}$. Z definice suprema víme:
	\begin{align}
		\forall n \quad \exists x_n\quad \lim_{n\to\infty} f(x_n) = s
		\quad \Rightarrow \quad \lim_{k\to\infty} f(y_k \text{vybraná}) = s
	\end{align}
	$K$ je kompaktní, proto zároveň:
	\begin{align}
		\exists y_k \text{ v. p. z } x_n \quad \exists y \in K:
		\lim_{k\to\infty} y_k = y
	\end{align}
	Z definice spojitosti v $y$ potom víme, že:
	\begin{align}
		\forall \epsilon > 0 \quad \exists \delta > 0 \quad \forall x \in B_\rho(y,
		\delta) \quad f(x) \in (f(y) - \epsilon, f(y) + \epsilon)
	\end{align}
	Chceme aby $f(y) = s$. Sporem $f(y) \neq s$, tedy $f(y) < s$. \\
	Zvolme $\epsilon := \frac{1}{2} |s - f(y)|$. 
	Z konvergence ale víme:
	\begin{align}
		\exists k_0 \quad \forall k > k_0 \quad y_k \in B_\rho(y, \delta)
		\quad \Rightarrow \quad |f(y_k) - f(y)| < \epsilon
	\end{align}
	Což je spor, protože $\lim f(y_k) = s$.
\end{enumerate}


\subsection{Množina funkcí se spojitými prvními derivacemi na množině}
\setcounter{equation}{0}
\paragraph{Definice}
O funkci $f: G \to \R$ řekneme, že patří do $C^1(G)$ pokud $\forall i \quad
\frac{\partial f}{\partial x_i}$ je spojitá funkce na $G$.



\subsection{Lagrangeho věta o vázaných extrémech (bez důkazu)}
\setcounter{equation}{0}
\paragraph{Věta}
Nechť $G \subsetq \R^n$ je otevřená množina, $s < n$, $f, g_1, ..., g_s \in
C^1(G)$ a množina $M = \{x \in \R^n: \quad g_1(x) = .... = g_s(x) = 0 \}$. \\
\textbf{Pokud}:
\begin{enumerate}
	\item $a \in M$ je bodem lokálního extrému $f$ na $M$
	\item vektory $\nabla g_1(a), ..., \nabla g_s(a)$ jsou lineárně nezávislé
\end{enumerate}
\textbf{Potom}:
\begin{align*}
	\exists \lambda_1, ..., \lambda_s \in \R: \nabla f(a) = \lambda_1 \nabla g_1
	(a) + ... + \lambda_s \nabla g_s (a)
\end{align*}


\newpage
\section{Metrické prostory}
\setcounter{equation}{0}

\subsection{Definice metrického prostoru}
\setcounter{equation}{0}
\paragraph{Definice}
Metrický prostor je $(P, \rho)$, kde $P$ je množina bodů a $\rho$ je funkce $P \times P
\to \R_0^+$ splňující axiomy:
\begin{enumerate}
	\item $\forall x \in P \quad \rho(x,x) = 0$
	\item $\forall x,y \in P \quad \rho(x,y) = \rho(y,x)$
	\item $\forall x,y,z \in P \quad \rho(x,y) \le \rho(x,z) + \rho(z,y)$
\end{enumerate}

\subsection{Otevřená a uzavřená koule}
\setcounter{equation}{0}
\paragraph{Definice}
Nechť $(P, \rho)$ je metrický prostor a $x \in P$, $r \in \R$, $r > 0$. Potom
definujeme:
\begin{description}
	\item[otevřenou] kouli: 
		$B(x, r) = \{ y \in P\quad \rho(x,y) < r\}$
	\item[uzavřenou] kouli: 
		$\overline{B(x, r)} = \{ y \in P\quad \rho(x,y) \le r\}$
\end{description}


\subsection{Otevřená a uzavřená množina}
\setcounter{equation}{0}
\paragraph{Definice}
Nechť $(P, \rho)$ je metrický prostor. Potom definujeme:
\begin{description}
	\item[otevřenou množinu] $G$ pokud $\forall x \in G \quad \exists r > 0 \quad
	B(x,r) \subsetq G$.
	\item[uzavřenou množinu] $F$ pokud $P \setminus F$ je otevřená.
\end{description}

\subsection{Vlastnosti otevřených množin}
\setcounter{equation}{0}
\paragraph{Věta}
Nechť $(P, \rho)$ je metrický prostor. Potom platí:
\begin{enumerate}
	\item $\emptyset$ a $P$ jsou otevřené množiny
	\item $G_1, ..., G_n \subsetq P$ jsou otevřené množiny, potom $G_1 \cap ... \cap
	G_n$ 	je otevřená
	\item $G_\alpha (\alpha \in A)$ jsou otevřené množiny, potom $\bigcup_{\alpha \in
	A} G_\alpha$ je otevřená
\end{enumerate}
\paragraph{Důkaz}
\begin{enumerate}
	\item $\emptyset$ $\forall x \in \emptyset$ platí cokoliv \\
		$P$ triviální
	\item Mějme $G = G_1 \cap ... \cap G_n$, potom z předpokladů víme:
		\begin{align}
			\forall x \in G \quad \forall i \quad \exists r_i > 0 \quad B(x,
			r_i) \subsetq G_i
		\end{align}
		Mějme tedy $r = \min \{ r_1, ..., r_n \} > 0$:
		\begin{align}
			B(x, r) \subsetq G_i \forall i \quad \Rightarrow \quad  B(x,r) \subsetq G
		\end{align}
	\item Mějme $\bigcup_{\alpha \in A} G_\alpha$, podle předpokladů víme:
		\begin{align}
			\forall x \in G \quad \exists \alpha \in A : x \in G_\alpha \quad 
			\Rightarrow \quad  \exists r > 0 \quad B(x,r) \in G_\alpha \subsetq G
		\end{align}
\end{enumerate}


\subsection{Vlastnosti uzavřených množin}
\setcounter{equation}{0}
\paragraph{Věta}
Nechť $(P, \rho)$ je metrický prostor. Potom:
\begin{enumerate}
	\item $\emptyset$, $P$ jsou uzavřené množiny
	\item $F_1, ..., F_n$ jsou uzavřené množiny, potom $F_1 \cup ... \cup F_n$
	je uzavřená
	\item $F_\alpha$ ($\alpha \in A$) jsou uzavřené množiny, 
		 potom $\bigcap_{\alpha \in A} F_\alpha$ je uzavřená
\end{enumerate}
\paragraph{Důkaz}
\begin{enumerate}
	\item triviálně: $P \setminus P$, $P \setminus\emptyset$ jsou otevřené
		množiny
	\item $F_1,... ,F_n$ uzavřené množiny, potom $\forall i \quad  G_i = P
		\setminus F_i$ otevřené množiny. Tedy $G_1 \cap ... \cap G_n$ je
		otevřená množina, $P \setminus ( G_1 \cap ... \cap G_n )$ je uzavřená
		množina. Stačí tedy nahlédnout, že $G_1 \cap ... \cap G_n = F_1 \cup ...
		\cup F_n$
\end{enumerate}

\subsection{Ekvivalence metrik}
\setcounter{equation}{0}
\paragraph{Definice}
Metriky $\rho, \sigma$ na $P$ jsou ekvivalentní právě když:
\begin{align*}
	\exists c_1, c_2 > 0 \quad \forall x, y \in P\quad c_1 \sigma(x,y) \le
	\rho(x,y) \le c_2 \sigma(x,y)
\end{align*}
\paragraph{Fakt}
Metriky $\rho_1$, $\rho_2$, $\rho_\infty$ na $\R^n$ jsou ekvivalentní.

\subsection{Konvergentní posloupnost}
\setcounter{equation}{0}
\paragraph{Definice}
Nechť $(P, \rho)$ je metrický prostor, $(x_n)_{n=1}^\infty$ posloupnost prvků
$P$. Řekneme, že $(x_n)_{n=1}^\infty$ konverguje k $x$ ($\lim_{n\to\infty} x_n =
x)$) pokud:
\begin{align*}
	\lim_{n\to\infty} \rho(x_n, x) = 0 \\
	\forall \epsilon > 0 \quad \exists n_0 \in \N \quad \forall n > n_0 \quad
	&\underbrace{|\rho(x_n,x) - 0|}_{} < \epsilon\\
	&\rho(x_n,x) < \epsilon\\
	&x_n \in B(x,\epsilon)
\end{align*}

\subsection{Vlastnosti konvergence}
\setcounter{equation}{0}
\paragraph{Věta}
\begin{enumerate}
	\item Nechť $(x_n)_{n=1}^\infty$ splňuje, že $\exists n_0 \quad \forall n >
	n_0 \quad x_n = x$ \\
	potom $\lim_{n\to\infty} x_n = x$
	\item Nechť $\lim_{n\to\infty} x_n = x$ a $\lim_{n\to\infty} x_n = y$, potom $x = y$
	\item Nechť $\lim_{n\to\infty} x_n = x$ a $(x_{n_s})_{s=1}^\infty$ je vybraná
	posloupnost z $(x_n)_{n=1}^\infty$, potom $\lim_{s\to\infty} x_{n_s} =
	\lim_{n\to\infty} x_n$
\end{enumerate}
\paragraph{Důkaz}
\begin{enumerate}
	\item triviální z definice
	\item Mějme $\epsilon = \frac{1}{2} \rho(x,y)$. Z předpokladů také víme, že:
		\begin{align}
			\exists n_0 \quad \forall n \ge n_0 \quad \rho(x_n, x) < \epsilon\\
			\exists n_1 \quad \forall n \ge n_0 \quad \rho(x_n, y) < \epsilon
		\end{align}
		Vezměme $n = \max\{n_0, n_1 \}$, potom:
		\begin{align}
			\rho(x_n, x) < \epsilon \\
			\rho(x_n, y) < \epsilon 
		\end{align}
		Z trojúhelníkové nerovnosti:
		\begin{align}
			\rho(x,y) &\lt \rho(x,x_n) + \rho(x_n, y) \\
			& < 2 \epsilon = \rho(x,y)
		\end{align}
		Což je spor.
	\item Víme podle věty o vybrané posloupnosti pro $\R$:
		\begin{align}
			\lim_{n\to\infty} \rho(x_n, x) = 0 \quad \Rightarrow \quad
			\lim_{k\to\infty} \rho(x_{n_k}, x) = 0 \quad \Rightarrow \quad
			\lim_{k\to\infty} x_{n_k} = x
		\end{align}
\end{enumerate}

\subsection{Charakterizace uzavřených množin}
\setcounter{equation}{0}
\paragraph{Věta}
Nechť $(P, \rho)$ je metrický prostor, $F \subset P$. Potom:
\begin{align*}
	F \text{ uzavřená }\quad \Leftrightarrow \quad \forall (x_n)_{n=1}^\infty \quad x_n \to x: \quad
	x_n \in F \Rightarrow x \in F
\end{align*}
\paragraph{Důkaz}
\begin{description}
	\item[$\Rightarrow$] $F$ je uzavřená\\% $(x_n)_{n=1}^\infty \quad x_n \in F$\\
	Sporem: $x_n\to x \quad \land \quad  x \nin F$:
	\begin{align}
		x \nin F \Rightarrow
		x \in P \setminus F
	\end{align}
	Tj. $x$ je v otevřené množině. Potom z definice otevřené množiny:
	\begin{align}
		\label{charuzm-2} \exists r> 0 \quad B(x,r) \subsetq P \setminus F
	\end{align}
	Podle definice limity však platí (pro $\epsilon = r$):
	\begin{align}
		\exists n_0 \quad \forall n \ge n_0 \quad \rho(x_n, x) < r
		\quad \Leftrightarrow \quad x_n \in B(x,r)
	\end{align}
	Tedy $x_{n_0} \in B(x,r)$, což je spor s
	(\ref{charuzm-2}) a předpoklady, že $\forall n \quad x_n \in F$.

	\item[$\Leftarrow$]
	Nechť pro spor $F$ není uzavřená množina. Tedy $P \setminus F$ není otevřená
	množina. Proto platí:
	\begin{align}
		\exists x \in P\setminus F \quad \forall r > 0 \quad B(x,r) \cap F \neq
		\emptyset
	\end{align}
	Vezmeme tedy takové $x$ a sestavíme posloupnost $(x_n)$ z tohoto průniku:
	\begin{align}
		x_i &\in B(x,\frac{1}{i}) \cap F
	\end{align}
	Je vidět,  že tato posloupnost je konvergentní k $x$ a taktéž, že $\forall n
	\quad x_n \in F$. Zároveň však $x \in P\setminus F$, což je spor s
	předpokladem.
\end{description}


\subsection{Kompaktní množina}
\setcounter{equation}{0}
\paragraph{Definice}
Nechť $(P, \rho)$ je metrický prostor. Řekneme, že množina $K \subset P$ je
\textbf{kompaktní} pokud:
\begin{align*}
	\forall (x_n)_{n=1}^\infty \quad x_n \in \K \quad \exists \text{ v.p. }
	(x_{n_s})_{s=1}^\infty \land \exists x \in \K \text{ t.ž. } x_{n_k}\to x
\end{align*}
Tj. pro každou posloupnost existuje konvergentní vybraná podposloupnost.

\subsection{Vlastnosti kompaktních množin (důkaz 2. neúplný!)}
\setcounter{equation}{0}
\paragraph{Věta}
Nechť $(P, \rho)$ je metrický prostor, $K \subsetq P$ kompaktní množina. Potom:
\begin{enumerate}
	\item $K$ je uzavřená množina
	\item $K$ je omezená množina
	\item $F \subsetq K$, $F$ uzavřená množina $\Rightarrow$ $F$ je kompaktní množina
\end{enumerate}
\paragraph{Důkaz}
\begin{enumerate}
	\item Aby K byla uzavřená množina musí platit:
		\begin{align}
			\forall (x_n)_{n=1}^\infty \quad x_n \in K \quad  x_n \to x \quad
			\Rightarrow x \in K
		\end{align}
		K je kompaktní, víme tedy:
		\begin{align}
			\exists x' \in K \quad \exists x_{n_k} \quad x_{n_k} \to x'
		\end{align}
		Zvolme tedy $x:=x'$, což splňuje podmínku podle definice.
	\item Sporem - nechť K není omezená: Mějme tedy $a \in P$ libovolné, tedy:
	\begin{align}
		\forall n \quad K \nsubseteq B(a,n) \quad \Leftrightarrow \quad \exists
		x_n \in K:
		x_n \nin B(a,n)
	\end{align}
	(tedy $\rho(x_n, a) > n$) \\
	Protože K je kompaktní, platí:
	\begin{align}
		\exists (x_{n_k})_{k=1}^\infty \quad \exists x\in K \quad x_{n_k} \to x
	\end{align}
	Tedy podle definice limity 
	\begin{align}
		\forall \epsilon > 0 \quad \rho(x_{n_k}, x) < \epsilon
	\end{align}
	Zvolme $\epsilon := n_{k_0}$. Speciálně tedy musí platit:
	\begin{align}
		\rho(x_{n_{k_0}}, x) < n_{k_0}
	\end{align}
	Což je spor pro $n_k = n_{k_0}$ \par
	pozn: v důkazu se na konci využívalo trojúhelníkové nerovnosti mezi body $a$, $x$ a $x_n$
	\item Pro každou posloupnost $x_n \in F \subsetq K$ platí, že konverguje
	v $K$. Potom podle věty o charakterizaci uzavřených množin pokud $(x_{n_k})
	\to x$, tak $x \in F$, tudíž splňuje podmínky z definice kompaktní množiny.
\end{enumerate}

\subsection{Charakterizace kompaktních množin v $\R^n$}
\setcounter{equation}{0}
\paragraph{Věta}
$K \subsetq \R^n$ je kompaktní množina právě když je \textbf{uzavřená a omezená}.
\paragraph{Důkaz}
\begin{description}
	\item[$\Rightarrow$] triviálně podle věty o vlastnostech kompaktních množin
	\item[$\Leftarrow$] pro $\rho_\infty = \max_i |x_i - y_i|$ \\
	Nechť $x(1), x(2), ...$ je posloupnost v $\R^n$, K je omezená a uzavřená.
	Potom:
	\begin{align}
		\exists c \in \R \quad K \subsetq [-c,c]^n
	\end{align}
	Ukážeme, že $[-c,c]^n$ je kompaktní, potom je $K$ kompakt triviálně podle části 3 věty o
	vlastnostech kompaktních množin. Chceme tedy:
	\begin{align}
		\forall (x_n) \quad x_n \in [-c,c]^d \quad \exists\text{ kg.
		podposloupnost}
	\end{align}
	Označme $x_n = (x_n^1, ..., x_n^d) \quad ... \quad \forall n \forall i \quad
	x_n^i\in[-c,c]^n$. Matematickou indukcí podle souřadnice:
	\begin{enumerate}[1.]
		\item $(x_n^1)_{n=1}^\infty$ - posloupnost v $\R$, má vybranou konvergentní
			posloupnost podle \textit{Bolzano-Weistrassovy} věty pro $\R$
			\begin{align}
				\lim_{k\to\infty} (y_k^1) = y^1 
				(y_k^1) =	(x_{n_k}^1)
			\end{align}
			Mějme tedy $(x_{n_k}^1) := (y_k^1)$.
		\item 	analogicky k 1. $\exists (x_{n_k}^2)$
	\end{enumerate}
	\begin{enumerate}[d.]
		\item analogicky k 1. $\exists (x_{n_k}^d)$
	\end{enumerate}
	Víme tedy, že 
	\begin{align}
		\forall j = 1...d \quad \lim_{i\to\infty} y_i^j = y^j
	\end{align}
	Chceme $\lim_{i\to\infty} \rho_\infty (y_i, y) = 0$. Tedy:
	\begin{align}
		\forall \epsilon>0 \quad ?\exists n_0: \\
		\exists n_1 &... \forall i \ge n_1 \quad |y_i^1 - y^1| < \epsilon \\
		&\vdots \\
		\exists n_d &... \forall i \ge n_d \quad |y_i^d - y^d| < \epsilon
	\end{align}
	Mějme tedy $n_0 = \max \{ n_1 ... n_d \}$. Potom:
	\begin{align}
		\forall \epsilon > 0 \quad \forall i \ge n_0 \quad \rho_\infty (y_i, y)
		< \epsilon
	\end{align}
	Což splňuje požadovanou limitu.
\end{description}

\subsection{Spojitost vzhledem k množině}
\setcounter{equation}{0}
\paragraph{Definice}
Nechť $(P, \rho)$, $(Q, \sigma)$ jsou metrické prostory. $M \subsetq P$ a funkce $f: M\to
Q$, $x_0 \in M$. Řeknem, že $f$ je \textbf{spojitá} v $x_0$ \textbf{vzhledem k $M$}
právě když:
\begin{align*}
	\forall \epsilon > 0 \quad \exists \delta > 0 \quad \forall \epsilon
	B_\rho(x_0, \delta) \cap M: \quad f(x) \in B_\sigma(x_0, \epsilon)
\end{align*}
\paragraph{Definice}
Funkce $f$ je spojitá na $M$ právě když je spojitá v každém jejím bodě vzhledem
k $M$.

\subsection{Limita vzhledem k množině}
\setcounter{equation}{0}
\paragraph{Definice}
Nechť $\forall \delta > 0 \quad B_\rho(x_0, \delta \setminus \{x_0\} \cap M \neq
0$, $y \in Q$. Potom definujeme limitu vzhledem k $M$ jako:
\begin{align*}
	\lim_{x \xrightarrow{\text{vz. k M}} x_0} f(x) = y \quad \Leftrightarrow \quad 
	\forall \epsilon> 0 \quad \exists \delta >0 \quad \forall x \in B_\rho(x_0,
	\delta) \cap M \setminus \{x_0\}: \quad f(x) \in B_\sigma(y, \epsilon)
\end{align*}

\subsection{Charakterizace spojitých zobrazení (bez důkazu)}
\setcounter{equation}{0}
\paragraph{Věta}
Nechť $(P, \rho)$, $(Q, \sigma)$ jsou metrické prostory a zobrazení $f:P\to Q$.
Pak následující tvrzení jsou ekvivalentní:
\begin{enumerate}
	\item f je spojité zobrazení na P
	\item $\forall G$ otevřená množina v $(Q, \sigma)$ je $f^{-1}(G)$ otevřená
	množina v $(P, \rho)$
	\item $\forall F$ uzavřená množina v $(Q, \sigma)$ je $f^{-1}(F)$ uzavřená
	množina v $(P, \rho)$
\end{enumerate}
(bez důkazu)
\end{document}
