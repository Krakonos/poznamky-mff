\documentclass{article}

\usepackage[utf8]{inputenc}
\usepackage[czech]{babel}
\usepackage{a4wide}
\usepackage{math}

\begin{document}

\title{Střípky z datových struktur}
\author{Ladislav Láska}

\renewcommand{\H}{\mathcal{H}}
\newcommand{\U}{\mathcal{U}}

\maketitle

\section{Univerzální a perfektní hashování}
Cílem této sekce bude ukázat konstrukci perfektního hashování, která je 
prakticky použitelná. Plán je ukázat konstrukci dobrého univerzálního systému 
(modulo prvočíslo) a z něj dvouúrovňově zkonstruovat perfektní hashovací funkci.  
Zdroje k tomuto textu je zejména přednáška "Universal Hashing, Perfect Hashing" 
od Charlese Leisersona z MIT (dostupné online) a lecture notů (TODO).

\paragraph{Notace} Zaveďme pro tuto kapitolu notaci. $N$ = počet hashovaných 
prvků, $M$ = velikost hashovací tabulky. Hashovací funkce jsou funkce $h: \U \to 
\{0, \dots, M-1\}$.

\subsection{Univerzální systém}
\df Soubor funkcí $\H$ je univerzální, pokud platí:
\begin{align}
	\forall x,y\in \U, h\in \H \text{ náhodné}\quad Pr(h(x) = h(y)) \leq 1/M
\end{align}

\vt Nechť $h\in \H$ je zvolená náhodně. Potom pro každé $x\in \U$ je 
$E[\#\text{kolizí s } x] < N/M$.

\dk Nechť $C_x$ je náhodná proměnná udávající počet kolizí s $x$. Pro 
zjednodušení si zaveďme indikátor $C_{xy}$, který je $1$ pokud pro danou $h$ 
nastane kolize mezi $x$ a $y$, $0$ jinak. Z definice univerzality hned víme, že 
$E[C_{xy}] \leq 1/M$. Stačí již použít linearitu střední hodnoty $E[C_x] = 
\sum_y E[C_{xy}] < N/M$.

\subsection{Konstrukce univerzálního systému}
Zde použijeme elegantní konstrukci pomocí skalárního součinu. Předpokládáme 
přitom, že $M$ je prvočíslo (prvočísla jsou dost hustá na to, abychom si 
nepřihoršili víc než 2-krát!), navíc předpokládáme, že umíme efektivně pracovat 
s klíči vhodné velikosti a počítat součet a násobit (rozumný předpoklad).

Označme jako $x = (x_0, \dots, x_r)$ rozklad klíče $x$ do soustavy o základu $m 
:= M$ (všimněme si, že $x_i$ nabývá hodnot od $0$ do $m-1$). Nyní už stačí 
zvolit $a$ po složkách náhodně a použít jako hashovací funkci skalární součin:

\begin{align}
	h_a(x) = \langle x,a \rangle = \sum_{i=0}^r a_i x_i \mod m
\end{align}

Nyní by se nám hodilo vědět, kolik takových funkcí existuje a kolik takových 
funkcí vytváří kolize. Zřejmě generujeme-li funkce po složkách, vygenerujeme 
$m^{r+1}$ funkcí. Kolik jich ale vytváří kolize?

Předpokládejme nyní $x \neq y$. Búno se v liší v první složce (děláme sumu přes 
pozice, takže to vyjde pro libovolnou pozici stejně). Podívejme se na to, co se 
musí stát, aby nastala kolize:

\begin{align}
	&\sum_{i=0}^r a_i x_i \equiv \sum_{i=0}^r a_i y_i \mod m \\
	&\sum_{i=0}^r a_i (x_i - y_i) =  a_0 (x_0 - y_0) + \sum_{i=1}^r a_i (x_i - 
	y_i) \equiv 0 \mod m\\
\end{align}

Nyní si stačí všimnou, že $x_0 - y_0$ je nenulové a $m$ je prvočíslo, můžeme k 
němu nalézt inverzní prvek a tedy s ním dělit. Získáme tedy:

\begin{align}
	a_0 =  \left(\sum_{i=1}^r a_i (x_i - y_i)\right)(x_0 - y_0)^{-1}
\end{align}

Z toho vidíme, že pokud má nastat kolize, tak pokud náhodně zvolíme $r$ prvků 
vektoru, poslední je již jimi jednoznačně určen (jinak nenastane kolize). Tedy 
Počet kolizních funkcí je $m^r$.

\vt Výše popsaná konstrukce je univerzální systém hashovacích funkcí.

\dk Stačí ukázat vyžadovanou pravděpodobnost. To uděláme středoškolským 
způsobem: podělíme počet špatných funkcí celkovým počtem funkcí:
\begin{align}
	Pr(h(x) = h(y)) = \frac{\#\text{počet špatných }h}{\#\text{počet všech }h} = 
	\frac{m^{r}}{m^{r+1}} = \frac{1}{m}
\end{align}
A věta je dokázána.

\subsection{Konstrukce perfektní hashovací funkce}
Chtěli bychom rádi lineární hashovací tabulku. Nejprve si však ukážeme 
kvadratickou verzi, která je značně jednoduchá a následně ji použijeme při 
chytřejší konstrukci.
\subsubsection{Kvadratická verze}
Prvky zahashujeme do kvadratické tabulky a ukážeme, že pravděpodobnost kolize je 
nízká pokud zvolíme náhodnou univerzální funkci (kterou si poznamenáme k 
tabulce, abychom věděli, jakou použít). Pokud nastane kolize, zvolíme funkci 
jinou.

\vt Nechť $m = N^2$. Potom je pravděpodobnost, že nastane kolize $<1/2$.

\dk Hashujeme $N$ prvků, kolize tedy může nastat mezi libovolným párem, kterých 
je $\binom{N}{2}$. Pravděpodobnost, že konkrétní jeden pár zkoliduje je $1/m$ 
(víme z definice univerzálního systému). Spočítejme střední hodnotu počtu kolizí 
zdravým rozumem (\ref{hash-e-kolize}) a použitím Markovovy nerovnosti 
(\ref{hash-markov}) získáme kýžený výsledek (\ref{hash-vysledek}).

\begin{align}
	\label{hash-e-kolize}
	&E[\text{\it\#počet kolizí}] = \binom{N}{2} \cdot 1/m = \frac{N(N-1)}{2} 
	\cdot \frac{1}{N^2}< 1/2\\
	\label{hash-markov}
	&Pr(A\geq t) \leq E[A] / t \\
	\label{hash-vysledek}
	&Pr(\text{\it\#počet kolizí}\geq1) \leq E[\text{\it\#počet kolizí}]/1
\end{align}
\qed

\subsection{Lineární konstrukce}
Na lineární konstrukci použijeme dvojúrovňové hashování -- nejprve zahashujeme 
do tabulky velikosti $N$, kde nám mohou nastat kolize -- tam kde nastaly 
vytvoříme sekundární tabulku, pro kterou použijeme kvadratickou variantu -- tím 
získáme jistě perfektní hashovací funkci. Protože ale druhá úroveň má
kvadratickou velikost, potřebujeme vědět, jak velké budou tabulky druhé úrovně.
Spočítejme tedy sumu jejich velikostí, pro políčko $i$ označme velikost $n_i$.

\vt Nechť $n_i$ je počet prvků, které se zahashovaly na první úrovni do 
přihrádky $i$. Potom $E[\sum_i n_i^2] < 2N$.

\dk Větu dokážeme trikem. Všimneme si, že chtěná střední hodnota je vlastně 
počet kolizí, které nastanou, počítáme-li jako kolizi každou kolidující 
uspořádanou dvojci. To se nám hodí, stačí si totiž zavést indikátor $C_{xy}$
(jak jinak, než že je $1$ pokud $x$ a $y$ kolidují, $0$ jinak) a zapsat střední 
hodnotu jako:
\begin{align}
	E[\sum_in_i^2] = E[\sum_x\sum_y C_{xy}]
\end{align}

Protože ale známe střední hodnotu kolize dvou různých prvků z univerzality (a 
výpočet jsme provedli v kvadratické verzi), a zároveň víme, že prvek $x$ 
koliduje s $x$, můžeme snadno dopočítat:

\begin{align}
	E[\sum_x\sum_y C_{xy}] = N + \sum_x\sum_{x\neq y} E[C_{xy}] \leq N + 
	N(N-1)/M < 2N
\end{align}

Kde v první rovnosti jsme jednak vytkli $N$ ze sumy (kolize $x$ se sebou samým) 
a zároveň použili linearitu střední hodnoty. Zbytek je již triviální dosazení,
pamatujme však, že $M \sim N$ (tak jsme si ho zvolili).  
Věta je dokázána.
\qed

Druhá úroveň tedy bude lineární ve střední hodnotě. Pomocí Markovovy nerovnosti
ale snadno nahlédneme, že pravděpodobnost, že budeme potřebovat více prostoru,
než je střední hodnota, je malá, a proto můžeme najít hashovací funkci pro první úroveň taktéž
randomizovaně (označme potřebný prostor třeba $X$):
\begin{align}
	&Pr(X \geq 4N) \leq E[X] / 4N \leq 2N / 4N = 1/2
\end{align}


\section{AVL stromy}
\subsection{Hloubka}
\vt Hloubka AVL stromů je $\O(\log n)$.

\dk Nejprve ukážeme, že dolní odhad na počet vrcholů ve stromě dané hloubky jsou 
Fibbonaciho čísla. Poté si stačí uvědomit, že Fibbonaciho čísla rostou 
exponenciálně a máme logaritmický odhad hned. Alternativně lze použít zlatý řez

Indukce: Nechť $w_h$ je nejmenší počet vrcholů v AVL stromu hloubky $h$, přičemž 
triviálně $w_0 = 1$ a $w_1 = 2$. Nyní mějme strom $T$ hloubky $h$ s minimálním 
počtem vrcholů.  Jeho kořen má jistě dva podstromy a alespoň jeden z nich je o 
jedna mělčí než $T$.  Pokud jsou oba stejně hluboké, plyne okamžitě spor. Tedy 
jeden má hloubku $h-1$ a druhý $h-2$ (z AVL podmínky), hloubku $T$ lze tedy 
napsat jako $h(T) = 1 + h(T_L) + h(T_R)$. Protože ale všechny stromy musí být 
minimální (jinak je lze nahradit za menší a je to spor s minimalitou $T$), lze 
psát obecně $w_h = 1 + w_{h-1} + w_{h-2}$.

Pokud nechceme spoléhat na znalost exponenciálního odhadu Fibbonaciho čísel, 
můžeme dokázat přímo: Nechť $\phi=\frac{1+\sqrt{5}}{2}$ (což je zhruba $1.6$).  
Ukážeme, že $w_h \geq \phi^h$, což nám dá kýženou exponencielu (s kladným 
základem).  Dokazujme indukcí (pro $w_0$ a $w_1$ platí triviálně):

\begin{align}
	w_{h+1} &= 1 + w_h + w_{h-1}\\
	&\geq 1 + \phi^h + \phi^{h-1} \\
	&= 1 + \phi^{h-1}(\phi + 1) \\
	\label{avl-mocnina}&= 1 + \phi^{h+1} \\
	&> \phi^{h+1}
\end{align}

Přičemž první nerovnost plyne z indukčních předpokladů a rovnost na řádku 
(\ref{avl-mocnina}) z faktu, že $1+\phi = \phi^2$ (zkuste si umocnit a uvidíte).

Z exponenciálního počtu vrcholů pak snadno plyne logaritmická hloubka a důkaz je 
hotov. \qed



\end{document}
