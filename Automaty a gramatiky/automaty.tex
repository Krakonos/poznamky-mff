\documentclass[a4paper,12pt,titlepage]{article}
\usepackage[utf8]{inputenc}
\usepackage{a4wide}
\usepackage[czech]{babel}
\usepackage{amsfonts, amsmath, amsthm, amssymb}
\usepackage{math}

\title{Automaty a gramatiky}
\author{Ladislav Láska}

\begin{document}

\maketitle
\newpage
\tableofcontents
\newpage

\section{Konečné automaty}
\setcounter{equation}{0}
\subsection{Konečný automat, přechodová funkce}
\setcounter{equation}{0}
\paragraph{Definice}
Konečným automatem nazveme pětici 
\begin{align}
	A = (Q, X, \delta, q_0, F)
\end{align}
kde po řadě $(Q, X, \delta, q_0, F)$ jsou stavový prostor, vstupní abeceda,
přechodová funkce, počáteční stav a množina přijímajících stavů.
\paragraph{Definice}
Rozšířená přechodová funkce $\delta^* Q\times X^* \to Q$ definujeme induktivně:
\begin{align}
	& \delta^*(q, \lambda) = q\\
	&\delta^*(q, wx) = \delta(\delta^*(q, w), x), \quad x \in X, w \in X^*
\end{align}
\subsection{Přijímané slovo, jazyk}
\setcounter{equation}{0}
\paragraph{Definice}
Slovo $w$ je přijímáno automatem $A$ právě když
\begin{align}
	\delta^*(q_0, w) \in F
\end{align}
\paragraph{Definice}
Jazyk přijímaný automatem $A$ definujeme jako množinu přijímaných slov:
\begin{align}
	L(A) = \{w | w \in X^* \land \delta^*(q_0, w) \in F \}
\end{align}
\subsection{Regulární jazyky}
\setcounter{equation}{0}
\paragraph{Definice}
\textbf{Regulární} jazyky jsou právě ty jazyky, které jsou rozpoznatelné
konečným automatem.
\subsection{Kongruence}
\setcounter{equation}{0}
\paragraph{Definice}
Nechť $X$ je konečná abeceda a $\sim$ je relace ekvivalence na $X^*$. Potom
$\sim$:
\begin{enumerate}
	\item je \textbf{pravá kongruence}, jestliže
		\begin{align}
			\forall u,v,w\in X^* \quad u\sim v \Rightarrow uw \sim vw
		\end{align}
	\item je \textbf{konečného indexu}, jestliže rozklad $X^*/\sim$ má konečný
	počet tříd.
\end{enumerate}
\subsection{Nerodova věta}
\setcounter{equation}{0}
\paragraph{Věta}
Nechť $L$ je jazyk nad konečnou abecedou X. Potom následující tvrzení jsou
ekvivalentní:
\begin{enumerate}
	\item $L$ je rozpoznatelný konečným automatem.
	\item Existuje pravá kongruence $\sim$ konečného indexu na $X^*$ tak, že $L$
	je sjednocením jistých tříd rozkladu $X^*/\sim$.
\end{enumerate}
\paragraph{Důkaz}
\begin{enumerate}
	\item $\to$ 2. (máme automat, konstruujeme pravou kongruenci)\\
		Definujeme relaci:
		\begin{align}
			u\sim v = \delta^*(q_0, u) = \delta^*(q_0, v)
		\end{align}
		Což je přímo z definice ekvivalence a je to pravá kongruence (díky
		vlastnostem rozšířené předchodové funkce). Navíc protože má automat
		konečně mnoho stavů, má také konečně mnoho tříd ekvivalence.
	\item $\to$ 1. (máme pravou kongruenci konečného indexu, konstruujeme
	automat)
		Definujeme automat jako:
			\begin{enumerate}
				\item abeceda $X$ je již dána
				\item stavy $Q$ nechť jsou třídy rozkladu
				\item stav $q_0 = [\lambda]$
				\item koncové stavy jsou třídy rozkladu definující jazyk:
					$L= \bigcup_{i=1..n} c_i$, 
					$F = \{c_1, ..., c_n\}$ 
				\item přechodovou funkci $\delta([u],x) = [ux]$
			\end{enumerate}
\end{enumerate}
\subsection{Pumping lemma}
\setcounter{equation}{0}
\paragraph{Věta}
Nechť $L$ je regulární jazyk. Potom existuje přirozené číslo $n$ takové, že
libovolné slovo $z \in L$, kde $|z|\ge n$ lze psát ve tvaru $uvw$ a 
\begin{align*}
	|uv| \le n \quad \land \quad |v| \ge 1 \quad \land \quad \forall i \ge 0
	\quad uv^iw \in L
\end{align*}
\paragraph{Důkaz}
Jako $n$ vezměme počet stavů příslušného automatu. Pokud vezmeme slovo délky
alespoň $n$, podle krabičkového principu se nutně musí nějaký stav opakovat.
Vezměme tedy stav $p$ pravní takový. Nahlédněme:
\begin{align}
	\delta(q_0, u) = p \\
	\delta(q_0, uv) = p
\end{align}
kde $u$ a $uv$ jsou slova, která nás dovedla do stavu $p$. Zřejmě tedy $|uv| \le
n$, protože se opakoval pouze stav $p$ a $|v| \ge 1$, protože se $p$ opakoval.
Smyčku $v$ tedy můžeme projít libovolně krát.
\subsection{Ekvivalence automatů a homomorfismus}
\setcounter{equation}{0}
\paragraph{Definice}
Automaty $A$ a $B$ jsou \textbf{ekvivalentní}, právě když rozpoznávají stejný jazyk: $L(A) =
L(B)$.
\paragraph{Definice}
Nechť $A_1$ a $A_2$ jsou konečné automaty a $h: Q_1 \to Q_2$ zobrazení. Řekneme, že 
$h$ je \textbf{automatový homomorfismus}, právě když:
\begin{enumerate}
	\item $h(q_1) = q_2$
	\item $h(\delta_1(q, x)) = \delta_2(h(q), x)$
	\item $q\in F_1 \eq h(q) \in F_2$
\end{enumerate}
\paragraph{Definice}
Homomorfismus nazveme \textbf{isomorfismus} právě když je prostý.
\subsection{Věta o ekvivalenci automatů}
\setcounter{equation}{0}
\paragraph{Věta}
Nechť $A_1$ a $A_2$ jsou konečné automat a existuje homomorfismus $h: A_1 \to
A_2$. Pak jsou $A_1$ a $A_2$ ekvivalentní.
\paragraph{Důkaz}
Konečnou iterací. Podle definice homomorfismu platí:
\begin{align}
	h(\delta_1(q, w)) = \delta_2(h(q), w) \qquad w \in X^*
\end{align}
Tedy
\begin{align}
	w \in L(A_1) \qquad & \eqs \delta_1(q_1, w) \in F_1 \\
	& \eqs h(\delta_1(q_1, w)) \in F_2 \\
	& \eqs \delta_2(h(q_1, w)) \in F_2 \\
	& \eqs \delta_2(q_2, w) \in F_2 \qquad \qquad \eqs w \in L(A_2)
\end{align}
\subsection{Dosažitelné stavy}
\setcounter{equation}{0}
\paragraph{Definice}
Nechť $A$ je konečný automat. Řekneme, že stav $q \in Q$ je dosažitelný, právě když
$\exists w \in X^* \quad \delta^*(q_0, w) = q$.
\paragraph{Věta}
Nechť $A$ je konečný automat a $P \subsetq Q$ jsou právě všechny dosažitelné stavy
$A$.
\paragraph{Důkaz} (triviální)
\subsection{Hledání dosažitelných stavů}
\setcounter{equation}{0}
\paragraph{Algoritmus}
Hledejme dosažitelné stavy po $i$ krocích:
\begin{enumerate}
	\item $M_0 = \{q_0\}$
	\item Spočítej dosažitelné stavy pro $i+1$:
	\begin{align}
		M_{i+1} = M_i \cup \{ q | q \in Q \land \exists p \in M_i, x \in
		X: \delta(p, x) = q \}
	\end{align}
	\item Pokud $M_{i+1} == M_i$, konči. Jinak opakuj 2. pro další $i$.
\end{enumerate}
\paragraph{Důkaz} (triviální)
\subsection{Ekvivalentní stavy}
\setcounter{equation}{0}
\paragraph{Definice}
Nechť $A$ je konečný automat a $p, q \in Q$ jsou jeho stavy. Řekneme, že stavy
$p$ a $q$ jsou ekvivalentní, právě když:
\begin{align}
	\forall w \in X^* \quad \delta^*(p, w) \in F \eq \delta^*(q,w) \in F
\end{align}
\subsection{Ekvivalence stavů po $i$ krocích}
\setcounter{equation}{0}
\paragraph{Definice}
Nechť $A$ je konečný automat a $p, q \in Q$ jsou jeho stavy. Řekneme, že stavy $p$
a $q$ jsou ekvivalentní po $i$ krocích ($p \sim^i q$) právě když:
\begin{align}
\forall w \in X^*, |w| \le i: \quad \delta^*(p, w) \in F \eqs \delta^*(q,w) \in
F
\end{align}
\paragraph{Tvrzení}
Nechť $A$ je konečný automat a $p, q \in Q$ jsou jeho stavy. Potom:
\begin{align}
	p \sim q \eqs \forall i \quad p \sim^i q
\end{align}
\paragraph{Důkaz} (triviální)
\paragraph{Algoritmus}
Iterativní konstrukce ekvivalence po $i$-krocích:
\begin{align}
	p \sim^0 q \qqquad &p \in F \eq q \in F \\
	p \sim^{i+1} q \qqquad &p \sim^i q \quad\land\quad \forall x \in X: \delta(p, x) \sim^i
	\delta(q,x)
\end{align}
\paragraph{Důkaz} (triviální)
\subsection{Vlastnosti ekvivalence po $i$ krocích}
\setcounter{equation}{0}
\paragraph{Věta}
Nechť A je konečný automat a $\sim^i$ je relace ekvivalence po $i$ krocích.
Potom:
\begin{enumerate}
	\item $\forall i \ge 0$ je $\sim^i$ ekvivalence na $Q$. Označme $\Rc =
	Q/\sim^i$.
	\item $\Rc_{i+1}$ zjemňuje $\Rc_i$
	\item $\Rc_{i+1} = \Rc \impl \forall t > i \Rc_t = \Rc_i$
	\item Nechť $|Q| = n$, potom $\exists k \le n-1: \Rc_{k+1} = \Rc_k$
	\item $\Rc_{k+1} = \Rc_k \impl (p \sim q \eq p \sim^k q)$
\end{enumerate}
\paragraph{Důkaz}
\begin{enumerate}
	\item Triviálně z definice
	\item Triviálně z definice
	\item Triviálně z iteračního algoritmu (podmínky se proti minulé iteraci
	nezměnily)
	\item Triviálně z maximálního počtu rozkladových tříd (konečného)
	\item Triviálně podle definice a 3.
\end{enumerate}
\subsection{Algoritmus hledání ekvivalence stavů}
\setcounter{equation}{0}
\paragraph{Algoritmus}
Postupujeme iteračním algoritmem, dokud nenarazíme na dvě po sobě jdoucí stejné
třídy ekvivalence. Stavy nacházející se ve stejné třídě ekvivalence na konci
algoritu jsou si ekvivalentní.
\paragraph{Důkaz} (triviální)
\subsection{Automatová kongruence}
\setcounter{equation}{0}
\paragraph{Definice}
Nechť $\equiv$ je relace ekvivalence na $Q$. Řekneme, že $\equiv$ je
\textbf{automatovou kongruencí}, právě když:
\begin{align*}
	\forall p \equiv q \in Q \quad \impl \quad (p \in F \eq q \in F) \land \forall x
	\in X: (\delta(p, x) \equiv \delta(q, x))
\end{align*}
\paragraph{Tvrzení}
Ekvivalence stavů je automatovou kongruencí.
\paragraph{Důkaz}
(triviální z definice)
\subsection{Podílový automat}
\setcounter{equation}{0}
\paragraph{Věta}
Nechť $A = (Q, X, \delta, q_0, F)$ je konečný automat a $\equiv$ je automatová
kongruence. Potom $A/\equiv \quad=\quad (Q/\equiv, X, \delta_\equiv, [q_0]_\equiv, \{
[q]_\equiv | q \in F\}$, kde $\delta_\equiv([q], x) = [\delta(q,x)]_\equiv$, je
konečný automat ekvivalentní s $A$. 
\paragraph{Definice}
Automat $A/\equiv$ nazýváme \textbf{podílový automat}.
\paragraph{Důkaz}
\begin{enumerate}
	\item Podílový automat je konečný automat: množiny stavů a abeceda jsou z
	definice konečné, přechodová funkce je definována korektně z vlastností
	automatové kongruence.
	\item Ekvivalence automatů: stačí nalézt homomorfismus, ten jsme však
	dostali v podobě kongruence.
\end{enumerate}
\subsection{Podílový automat a ekvivalence stavů}
\setcounter{equation}{0}
\paragraph{Věta}
Nechť $A$ je konečný automat a $\sim$ je ekvivalence stavů.
Potom :
\begin{enumerate}
	\item $A/\sim$ je konečný automat ekvivalentní s $A$
	\item žádné stavy v $A/\sim$ si nejsou ekvivalentní
\end{enumerate}
\paragraph{Důkaz}
\begin{enumerate}
	\item Relace ekvivalence stavů je automatovou kongruencí, tedy automaty si
	jsou ekvivalentní.
	\item Triviálně z definice ekvivalence. (lze dokázat sporem)
\end{enumerate}
\subsection{Redukce automatu}
\setcounter{equation}{0}
\paragraph{Definice}
Konečný automat je \textbf{redukovaný} právě když nemá nedosažitelné stavy a
žádné dva stavy nejsou ekvivalentní. Automat $B$ je \textbf{reduktem} automatu
$A$ právě když je $B$ redukovaný a $A$ a $B$ jsou ekvivalentní.
\paragraph{Věta}
Ke každému konečnému automatu existuje nějaký jeho redukt.
\paragraph{Důkaz}
Konstruktivně vyřazením stavů a nalezením podílového automatu ekvivalencí stavů.
\subsection{Věta o isomorfismu reduktů}
\setcounter{equation}{0}
\paragraph{Věta}
Pro libovolné dva redukované konečné automaty jsou následující tvrzení
ekvivalentní:
\begin{enumerate}
	\item automaty jsou ekvivalentní
	\item automaty jsou isomorfní
\end{enumerate}
\paragraph{Důkaz}
\begin{enumerate}
	\item $\to$ 2. (hledáme isomorfismu z ekvivalentních reduktů) \\
		Zkonstruujeme funkci $h: Q_1\to Q_2$. Pro stav $q \in Q_1$ (který je
		dosažitelný) existuje slovo $u \in X^*$ vedoucí do $q$ a
		v $h(q) = \delta_2(q_2, u)$. Což je funkce (protože jsou automaty
		redukované) a je prostá a na.
	\item $\to$ 1. (víme z definice)
\end{enumerate}
\paragraph{Poznámka}
Pomocí redukce umíme řešit problémy:
\begin{enumerate}
	\item Zjištění ekvivalence automatů (zredukujeme, znormalizujeme a
	porovnáme)
	\item Zištění $L(A) = \emptyset$
	\item Zištění $L(A) = X^*$
\end{enumerate}
\subsection{Normalizace automatu}
\setcounter{equation}{0}
\paragraph{Definice}
Normalizovaný tvar automatu vytvoříme tak, že:
\begin{enumerate}
	\item Zafixujeme pořadí písmen v abecedě
	\item Počáteční stav označíme 1
	\item Tabulku popisující automat vyplňujeme po řádcích zleva doprava a pokud
	narazíme na nový stav, přiřadíme mu následující volné číslo.
\end{enumerate}
\subsection{Hledání nejkratšího slova}
\setcounter{equation}{0}
\paragraph{Pozorování}
Nejkratší slovo nalezenem při redukci automatu - přes rozkladové třídy. První
stavy, kde se liší, udávají začátek slova, pak přečteme všechny stavy až do
počátku.


\section{Nedeterministické konečné automaty}
\setcounter{equation}{0}
\paragraph{Definice}
Nedeterministickým konečným automatem nazveme pětici
\begin{align}
	A = (Q, X, \delta, S, F)
\end{align}
kde $Q$ je množina stavů, $X$ je vstupní abeceda, $\delta: Q \times X \to P(Q)$
přechodová funkce a $S$, $F$ jsou množiny počátečních a koncových stavů.
\subsection{Převod NKA na KA}
\setcounter{equation}{0}
\paragraph{Věta}
Je-li $A$ nedeterministický konečný automat, potom lze sestrojit konečný automat
$B$ takový, že $L(A) = L(B)$.
\paragraph{Důkaz}
Podmnožinová konstrukce: budeme simulovat všechny možné cesty NKA a zapisovat je
do KA. Protože těch může být pouze konečně mnoho, bude i výsledný automat
konečný. Navíc bude rozpoznávat stejný jazyk.
\subsection{$\lambda$-přechod}
\setcounter{equation}{0}
\paragraph{Definice}
$\lambda$-přechod je hrana automatu, při které nedochází ke čtení vstupního
symbolu.
\section{Dvousměrné konečné automaty}
\setcounter{equation}{0}
Dvousměrným konečným automatem nazveme pětici
\begin{align}
	A = (Q, X, \delta, q_0, F)
\end{align}
kde $Q$ je množina stavů, $X$ je vstupní abeceda, $\delta: Q \times X \to
Q\times\{-1, 0, +1\}$
přechodová funkce, $q_0$ počáteční stav a $F$ je množina koncových stavů.
\subsection{Věta o síle dvousměrných automatů}
\setcounter{equation}{0}
\paragraph{Věta}
Jazyky přijímané dvousměrnými konečnými automaty jsou právě jazyky přijímané
konečnými automaty.
\paragraph{Důkaz}
Je zřejmé, že KA je 2KA posouvající hlavu stále doprava. Důkaz obrácené
implikace můžeme provést například pomocí Nerodovy věty. K tomu si ale musíme
připravit podklad: \\
Zavedeme funkci $f_u$, která bude sledovat výpočet automatu nad slovem $u$.
Mějme tedy slovo $uv$ a definume funkci $f$ následovně (a zavedeme stav $q'_0$
jako vstupní stav a definujeme $\delta(q'_0) = \delta(q_0)$ - to nezmění výpočet
a zajistí nám to potřebné rozlišení prvního vstupu do výpočtu):
\begin{enumerate}
	\item $f(q'_0) = p$ kde stav $p$ je stav, kdy poprvé přejdeme hlavou ze slova
	$u$ do $v$ (nebo 0 pokud se tak nikdy nestane)
	\item $f(p) = q$ kde stav $q$ je stav, kdy přejdeme hlavou ze slova $u$ do
	$v$ pokud jsme vstoupili do slova $u$ z $v$ ve stavu $p$.
\end{enumerate}
Všimneme si, že takto defininovaná funkce pro každé slovo přesně udává postup
výpočtu automatu. Zavedeme tedy ekvivalenci slov takto:
\begin{align}
	u \sim w \eq f_u = f_w
\end{align}
Tedy slova jsou ekvivalentní, pokud se nad nimi automat chová stejně. Nyní se
podíváme na vlastnosti relace $\sim$:
\begin{enumerate}
	\item Podle definice jde o ekvivalenci.
	\item Má konečný index (počet takových funkcí je omezený, protože je omezení
	počet stavů).
	\item Jde o pravou kongruenci: $u\sim w \impl uv \sim uw$ platí, protože
	rozhraní slov $u|v$ a $u|w$ je stejné (definované přímo funkcí $f$) a nad
	$v$ je výpočet automatu stejný.
	\item L(A) je sjednocením jistých tříd rozkladu, protože $w \in L(A) \eq
	f_w(q'_0) \in F$
\end{enumerate}
Tedy podle Nerodovy věty je L(A) regulární jazyk a může být přijímán konečným
automatem.


\section{Množinové operace nad jazyky}
\setcounter{equation}{0}
\subsection{Množinové operace nad regulárními jazyky}
\setcounter{equation}{0}
\paragraph{Definice}
Definujeme následující množinové operace nad jazyky:
\begin{enumerate}
	\item \textbf{Sjednocení} 
		$L_1 \cup L_2 = \{ w | w \in L_1 \lor w \in L_2 \}$
	\item \textbf{Průnik} 
		$L_1 \cap L_2 = \{ w | w \in L_1 \land w \in L_2 \}$
	\item \textbf{Rozdíl} 
		$L_1 - L_2 = \{ w | w \in L_1 \land w \nin L_2 \}$
	\item \textbf{Doplněk} 
		$- L = \{ w | \nin L \} = X^* - L$
\end{enumerate}
\subsection{Uzavřenost množinových operací nad regulárními jazyky}
\setcounter{equation}{0}
\paragraph{Věta}
Sjednocení, průnik, rozdíl i doplněk jsou uzavřeny nad regulárními jazyky.
\paragraph{Důkaz}
Zkonstruujeme automat přijímající daný jazyk (v případě průniku a rozdílu
simulujeme běh dvou automatů, tedy musíme rozšířit množinu stavů na $Q =
Q_1 \times Q_2$ a patřičně rozšířit přechodovou funkci):
\begin{enumerate}
	\item Sjednocení - použijeme nedeterminismus
	\item Průnik - koncové stavy budou stavy $F = F_1 \times F_2$
	\item Rozdíl - koncové stavy budou stavy $F = F_1 \times (Q_2 - F_2)$
	\item Doplněk - prohodíme přijímající a nepřijímající stavy
\end{enumerate}
\subsection{Řetězcové operace nad regulárními jazyky}
\setcounter{equation}{0}
Definujeme následující řetězcové operace nad regulárními jazyky:
\begin{enumerate}
	\item \textbf{Zřetězení} $L_1 \cdot L_2 = \{ uv | u \in L_1 \land v \in L_2
	\}$
	\item \textbf{Mocniny jazyka} $L^{i+1} = L^i \cdot L$
	\item \textbf{Pozitivní iterace} $L^+ = L^1 \cup L^2 \cup ...$
	\item \textbf{Obecná iterace} $L^* = L^0 \cup L^+$
	\item \textbf{Otočení jazyka} $L^R = \{ u^R | u \in L \}$
	\item \textbf{Levý kvocient $L_1$ podle $L_2$} $ L_2 \setminus L_1 = \{ v |
	uv \in L_1 \land u \in L_2 \}$
	\item \textbf{Pravý kvocient $L_1$ podle $L_2$} $ L_1 / L_2 = \{ u | uv \in
	L_1 \land v \in L_2 \}$
\end{enumerate}






















\end{document}

