\documentclass[a4paper,12pt,titlepage]{article}
\usepackage[utf8]{inputenc}
\usepackage{a4wide}
\usepackage[czech]{babel}
\usepackage{amsfonts, amsmath, amsthm, amssymb}
\usepackage[small,compact]{titlesec}
\usepackage{anyfontsize}


\newcommand{\shn}{\Theta}
\newcommand{\lm}{\smallskip\noindent\bf Lemma\rm{} }
\newcommand{\dk}{\smallskip\noindent\bf Důkaz\rm{} }
\newcommand{\df}{\smallskip\noindent\bf Definice\rm{} }
\newcommand{\vt}{\smallskip\noindent\bf Věta\rm{} }
\newcommand{\F}{\mathcal{F}}
\renewcommand{\L}{\mathcal{L}}
\newcommand{\Fr}{\mathbb{F}}
\newcommand{\Z}{\mathbb{Z}}
\newcommand{\C}{\mathbb{C}}
\newcommand{\R}{\mathbb{R}}
\newcommand{\Q}{\mathbb{Q}}
\newcommand{\N}{\mathbb{N}}
\newcommand{\todo}[1]{\bf TODO: \rm#1}
\renewcommand{\L}{\mathcal{L}}
%\newcommand{\qed}{\hfill QED}
\DeclareMathOperator{\rank}{rank}
\DeclareMathOperator{\Sp}{Sp}
\DeclareMathOperator{\Tr}{Tr}
\newcommand\bigzero{\makebox(0,0){\text{\huge0}}}
\newcommand\bigone{\makebox(0,0){\text{\huge1}}}

\title{Linearní Algera v Kombinatorice}
\author{Ladislav Láska}

\begin{document}

\maketitle
\newpage
\tableofcontents
\newpage


\section{Lineární nezávislost}
\df Vektory $v_i$ jsou lineárně nezávislé, pokud existuje netriviální řešení 
rovnice $\sum_i \alpha_iv_i=0$.
\subsection{Sudo-licho města}
\df Nechť $|X|=n$ a $A_1, \dots, A_m \subseteq X$ jsou neprázdné podmnožiny.  
Úloha A-B město se ptá, jak velké může být $m$, pokud $|A_i| \sim B$ a $|A_i\cap 
A_j|\sim A$ (tedy pro sudo-licho město máme omezení na liché velikosti a sudé 
průniky).

\vt Pro úlohu sudo-licho město platí $m \leq n$.

\dk Počítejme nad $GF(2)$. Matice $A$ nechť je charakteristická matice dimenze 
$n \times m$. Podívejme se na součin $AA^T$, tedy na matici skalárních součinů:
\begin{align}
	AA^T = \left(\begin{matrix}A_1\\ A_2 \\ \vdots \\ A_m \end{matrix}\right) 
	\cdot \left(\begin{matrix}A_1, A_2, \dots, A_m\end{matrix}\right) =
	\left(\begin{matrix}
	1 & &\bigzero & \\
	& \ddots && \\
	&\bigzero& \ddots & \\
	&&  &1
	\end{matrix}\right)
\end{align}
Tedy víme, že $\rank(AA^T) = m$ a $\rank(A) \leq n$. Z vlastností ranku již 
snadno získáme nerovnost $m=\rank(AA^T) \leq \rank(A) \leq n$. \todo{důkaz 
rankové nerovnosti obrázkem pomocí zobrazení} \qed

\vt Nechť $|X|=n$ a $A_1, \dots, A_m \subseteq X$ že platí $|A_i \cap A_j| = 1 $ 
a $A_i \neq A_j$. Potom $m \leq n$.
\dk Podobně jako v předchozím příkladě vezměme matici charakteristických vektorů 
$A$ a podívejme se na součit $AA^T$, tentokrát již nad $Q$:
\begin{align}
	AA^T = \left(\begin{matrix}
	|A_1| & &\bigone & \\
	& \ddots && \\
	&\bigone& \ddots & \\
	&&  &|A_m|
	\end{matrix}\right)
\end{align}
Dále označme $a_i := |A_i|$. Můžeme předpokládat, že $a_1 \leq a_2 \leq \dots 
\leq a_m$. Zřejmě také $a_2 > 1$ (jinak $A_1 = A_2$). Nyní bychom chtěli 
dokázat, že je matice regulární -- proto se podíváme na determinant této matice:
\begin{align}
	|AA^T| = \left|\left(\begin{matrix}
	a_1 & &\bigone & \\
	& \ddots && \\
	&\bigone& \ddots & \\
	&&  &a_m
	\end{matrix}\right)\right|
	%= \left|\begin{matrix}\text{{\fontsize{120}{120}\selectfont 1}}\end{matrix}
	= \left|\begin{matrix}\text{\bf\Huge 1}\end{matrix}
	+\left(\begin{matrix}
	a_1-1 & &\bigzero & \\
	& \ddots && \\
	&\bigzero& \ddots & \\
	&&  &a_m-1
	\end{matrix}\right)\right|
\end{align}
Zatímco matice jedniček je singulární \todo{Pochopit proč se to dá spočítat, ale 
determinant vyjde kladně}.  \qed

\subsection{Dvouvydálenostní množiny}

\subsection{Fisherova nerovnost}

\section{Skalární součin}
\subsection{Ortogonální doplněk}
\subsection{Dimenze}
\subsection{Sudo-sudo města}
\subsection{Dolní odhad na Ramseyovo číslo}

\section{Shannonova kapacita a Lovászova $\vartheta$ funkce}
\subsection{Shannonova kapacita}
$$
	\shn(G) = \sup_i(\alpha(G_i))^{1/i}
$$

\lm $\shn(G + \overline{G}) \geq \sqrt{2|G|}$

\dk 
\begin{align}
	\alpha ((G+\overline G)^2) \geq 2|G| \\
	V_{G+\overline G} = \{ V_1, ..., V_n, V_1', ..., V_n'\}
\end{align}
Lze vzít nezávislou množinu $A$:
\begin{align}
	A = (V_i, V_i'), (V_i', V_i)
\end{align}
\qed


\subsection{Funkční reprezentace grafu}
\df Nechť $G$ je graf, $\F$ je systém funkcí, $X$ množina reprezentantů. Pak pro vrchol $v$ mějme $f_v \in \F$, že $f_v X \to \F$ a :
\begin{enumerate}
	\item $f_v(c_v) \neq 0$
	\item $uv \notin E_G \Rightarrow f_u(c_v) = 0$
\end{enumerate}

\df Dimenzi $\F$ definujeme jako $\dim\L(\{f_v\})$, tedy chápeme funkce jako vektorový prostor.


\lm(L) G má reprezentaci $\F$, pak $\shn(G) \leq \dim \F$.

\lm(A) G má reprezentaci $\F$, pak $\alpha(G) \leq \dim \F$.\par
\dk Nechť $A$ je nezávislá v $G$. Vyhodnotím reprezentující funkci v bodech $A$.
\begin{align}
\left(
	\begin{matrix}
		f_1(c_1) & f_2(c_2) & \dots \\
		f_2(c_1) & f_2(c_2) & \dots \\
		\vdots &&\\
	\end{matrix}\right)
\end{align}
Tedy všechny reprezentující funkce jsou lineárně nezávislé, tedy je lemma dokázáno.\qed

\lm(B) Pokud $G_1$ má reprezentaci $\F_1$, $G_2$ reprezentaci $\F_2$ nad stejným tělesem, pak $G = G_1 \boxtimes G_2$ má reprezentaci $\F$ a $\dim\F \leq \dim\F_1 \cdot \dim\F_2$.

\dk Definuji $X = X_1 \cup^.$, $c_(v_1,v_2) = (c_{v_1}, c_{v_2})$ a $f_(v_1, v_2) = f_{v_1}(x_1) \cdot f_{v_2}(x_2)$.
Dále stačí ověřit, že platí axiomy a víme, že je to korektně definovaná reprezentace. Stačí omezit dimenzi, stačí ale nahlédnout, že nové funkce jsou lineární kombinací vektorů z nanejvýš dvou bazí, výše uvedená nerovnost tedy platí.\qed

\dk(L) $\shn(G) = \sup_i \alpha(G^i)^{1/i} \leq^{LA} \sup_i(\dim f.r.(G^i))^{1/i} \leq^{LB} \sup_i\dim f.r. (G)^{1/i} = \dim f.r.(G)$

\vt Existuje $G, H$, že $\shn(G+H) > \shn(G) + \shn(H)$.
\dk Zvolím $G$ takový, že $V_G=\binom{S}{3}$, $S = \{1, ..., s\}$ a $E_G = \{ (A, B) | |A\cap B| = 1\}$. Reprezentaci vytvoříme nad tělesem $\Fr = \Z_2$, $C_A = char. vektor A$ a $f_A(x) = \sum_{a\in A} X_a$.

Ověříme, že je o reprezentace. Navíc každá funkce, kterou máme, je nějaká kombinace tří funkcí $b_i(X) = x_i$, těch je $s$.

Dále pro $\overline G$: $\Fr = \R$, $X = \R^s$, $c_A = char. vektor A$, $f_A(x) = \left(\sum_{a\in A} x_a\right) -1$. Nahlédneme stejně jako v předchozím případě, že jde o reprezentaci a dim $\leq s+1$.

Nyní zvolíme $s$ takové, že $\sqrt{2\binom{s}{3}} > 2s + 1$ a věta platí \todo{rozmyslet}. \qed

\df Obecná poloha vektorů množiny \v N v  $\R^d$ je taková, že libovolná podmnožina velikosti $\leq d$ je lineárně nezávislá.

\df Lokálně obecná poloha vektorů reprezentace v $\R^d$ na grafu $G$ jsou takové vrcholy, že $\rho(\overline{N(v)})$ jsou lineárně nezávislé.

\vt Pro $G$ s $|G| = n$ jsou ekvivalentní tyto tvrzení:
\begin{enumerate}
	\item $G$ má ortogonální reprezentaci v $\R^d$ v obecné poloze.
	\item $G$ má ortogonální reprezentaci v $\R^d$ v lokálně obecné poloze.
	\item $G$ je $(n-d)$-souvislý.
\end{enumerate}


\section{Vlastní čísla grafu}
\subsection{Moorovy grafy}
Motivací nechť jsou r-regulární grafy bez krátkých cyklů (troj- a 
čtyř-úhelníků). Triviální konstrukce nám dává odhad na počet vrcholů:
\todo{obrázek konstrukce}
\begin{align}
\label{moorova-podminka}
	|V| \geq 1 + r + r(r-1) = r^2 +1
\end{align}

\df Moorův graf je takový $r$-regulární graf bez troj- a čtyř-úhelníků, kde 
platí v (\ref{moorova-podminka}) rovnost.

\vt Moorův graf existuje pro $r=1,2,3,7$, pro $r=57$ se neví a pro žádné další 
$r$ neexistuje.

\dk (Idea) Mějme graf $G$ Moorův a $A$ jeho matici sousednosti. Zapišme druhou 
mocninu $A$ jako stupeň na diagonále a prohozené 0 a 1 jinde a upravme:
\begin{align}
	&A^2 = rE + {\bf0} + {\bf1}(J- A-E) \\
	&A^2 = rE - J - A - E \\
	\label{moorova-matice}&A^2 + A + (1-r)E = J
\end{align}
Dále pro nějaké $\lambda\in \Sp(A)$:
\begin{align}
	\label{moorova-mocnina}A^2 x = AAx = A\lambda x = \lambda A x = \lambda 
	\lambda x = \lambda^2 x
\end{align}
A dosasíme (\ref{moorova-matice}) za $A$:
\begin{align}
	Jx = (A^2 + A + (1-r)E)x = (\lambda^2 + \lambda + (1-r))x
\end{align}
A tedy $(\lambda^2 + \lambda + 1 -r) \in \Sp(J)$. Vlastní čísla matice $J$ 
(matice samých jedniček) ale známe, jsou to $\{0^{(n-1)}, n^{(1)}\}$. Zjevně pro 
$\lambda = r$ vyjde vlastní číslo $n$, je tedy potřeba vyřešit kvadratickou 
rovnici s parametrem $r$:
\begin{align}
	\lambda^2 + \lambda + 1 - r = 0
\end{align}
Jak na to půjdeme? Vyjádříme si $\lambda$ známým vzorečkem pro kořeny:
\begin{align}
	\lambda_{1,2} = \frac{-1\pm \sqrt{1-4(1-r)}}{2} = \frac{-1\pm\sqrt{4r-3}}{2}
\end{align}
Násobnost označíme $m_1, m_2$. Protože stopa matice je suma vlastních čísel 
včetně násobností, platí dále rovnice (protože matice sousednosti $A$ má na 
diagonále vždy nuly):
\begin{align}
	\Tr(A) = r+m_1\lambda_1 + m_2\lambda_2 = 0
\end{align}
Pro další úravy označme odmocninu z diskriminantu jako $\sqrt{\cdot}$. Nejdříve 
upravíme do formy (násobení dvěma a přeskupení):
\begin{align}
	2r - r^2 + \sqrt{.}(m_1 - m_2) = 0
\end{align}
Všimneme si, že $r\in\N$, tedy máme dvě možnosti:
\begin{enumerate}
	\item $\sqrt{.} \in \Q$: potom $m_1 = m_2$ a tedy $r = 2$.
	\item $\sqrt{.} = s^2 \in \Q$ a $s \in \N$. Po menších úpravách lze zjistit, 
	že $s\in \{1, 3, 5, 15\}$, což dává $r\in\{1, 3,7,57\}$.
\end{enumerate}
\qed

\subsection{Raileighův princip a proplétání}
\vt (Raileighův princip) Nechť $A$ je matice s ortogonální bazí z vlastních 
vektorů $x_i$ a vlastními čísly $\lambda_i \geq \lambda_k$. Potom $x \in\langle 
x_1,\dots,x_k\rangle \Rightarrow x^TAx\geq x^T\lambda_kx$.

\vt (Věta o proplétání) Nechť $A$ a $B$ jsou matice takové, že $B$ vznikla z $A$ 
vymazáním nějakého řádku a sloupce. Potom pro vlastní čísla $\lambda_i,\mu_i$ 
matic $A,B$ platí:
\begin{align}
	\lambda_1 \geq \mu_1 \geq \lambda_2 \geq \dots\geq \mu_{n-1} \geq \lambda_n
\end{align}

\dk Dokazujeme indukcí $\lambda_k \geq \mu_k \geq \lambda_{k+1}$. Označme $x_i$ 
a $y_i$ vlastní vektory matic $A$ a $B$.  Zaveďme následující vektorové 
podprostory $\C^n$ (ačkoli druhý z nich nemá dostatek složek, můžeme mu jednu 
nulovou přidat a nic se nestane):
\begin{align}
S_1 := \L\{x_k, \dots, x_n\} \subseteq \C^n \\
S_2 := \L\{y_1, \dots, y_k\} \subseteq \C^n
\end{align}
Zřejmě $\dim(S_1) + \dim(S_2) > n$, tedy $\exists x \in S_1\cap S_2$. Použijeme 
Reileighův princip pro oba prostory a máme:
\begin{align}
	\mu_k \leq \frac{y^*By}{y^*y} = \frac{x^*Ax}{x^*x} \leq \lambda_k
\end{align}
Stačí ukázat, že $\mu_k \geq \lambda_{k+1}$ -- to je ale snadné, stačí vzít $-A$ 
a $-B$, čímž se obrátí znaménka vlastních čísel a nerovnosti. \qed

\vt (Věta o proplétání podobných matic) Nechť $A$ je symetrická čtvercová matice 
s vlastními čísly a vektory $\lambda_i$ a $x_i$, $S$ reálná matice, že $S^TS=I$.  
Definujeme $B := S^TAB$ a označíme vlastní čísla a vektory matice $B$ jako 
$\mu_i$ a $y_i$. Potom $\mu_i$ proplétají $\lambda_i$ a pokud navíc $\mu_i = 
\lambda_i$ pro nějaké $i$, tak $Sy_i$ je vlastní vektor $A$ příslušící vlastnímu 
číslu $\lambda_i$.

\dk Použijeme Raileighův princip podobně, jako v předchozím tvrzení. Všimneme 
si, že:
\begin{align}
	x \in \L\{ S^Tx_k, \dots, S^Tx_{k-1}\}^\perp \Leftrightarrow
	Sx \in \L\{ x_k, \dots, x_{k-1}\}^\perp
\end{align}
Stačí si opět vzít vhodný prvek $x$ z průniku:
\begin{align}
	x \in \L\{ S^Tx_k, \dots, S^Tx_{k-1}\}^\perp \cap \L\{y_1, \dots, y_k\}
\end{align}
A můžeme použít Reileighův princip:
\begin{align}
	\lambda_i \geq \frac{Sx^TASx}{Sx^TSx} = \frac{x^TBx}{x^Tx} \geq \mu_i \\
\end{align}
Na navíc platí pokud $\lambda_i = \mu_i$, potom:
\begin{align}
	\frac{x^TBx}{x^Tx} = \lambda_i \quad\Rightarrow\quad x^TBx=x^Tx\lambda_i 
	\quad\Rightarrow\quad Bx = \lambda_i x
\end{align}
A $x$ je vlastní vektor příslušící $\lambda_i$, jak jsme chtěli dokázat.\qed



\section{Náhodné procházky}
\subsection{Markovovy řetězce}
\subsection{Stabilní distribuce a konvergence}

\section{Expandéry}
\subsection{Mixing lemma}
\subsection{Vzdálenostní mocniny a zig-zag součin}

\section{Perfektní kódy}
\subsection{?}
\subsection{Lloydova věta}


\end{document}

