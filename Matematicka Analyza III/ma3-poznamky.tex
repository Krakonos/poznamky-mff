\documentclass[a4paper,10pt]{article}
\usepackage[utf8]{inputenc}
\usepackage{a4wide}
\usepackage[czech]{babel}
%\usepackage{bbm}
\usepackage{amsfonts, amsmath, amsthm, amssymb}
\usepackage{math}
\usepackage{enumerate}

\newcommand{\Nu}{\nu}

\title{Matematická analýza III}
\author{Ladislav Láska}

\begin{document}
\maketitle
Učební text k předmětu \texttt{Matematická analýza III} pro informatiky. Je
vytvořen na základě látky z přesnášek Martina Klazara a cvičeních a volně navazuje na předmět
\texttt{Matematická analýza II}. Text je povětšinou výtahem z
přednášek uspořádaným do srozumitelných krátkých celků zaměřený na definice,
věty a probrané důkazy. Některé věty jsou pouze minimálně upraveny a přepsány z
oficiálních poznámek z přednášek, jiné jsou úplně přepracovány tak, aby byly co
nejsrozumitelnější. Seznam všech definic, vět a důkazů potrebných ke 
zkoušce je uveden na konci textu
(zkopírovaný z webu přednášky, pro úplnost). \\

Pokud najdete chybu, nepřesnost nebo máte lepší (hezčí, kratší) důkaz než já, neváhejte mě kontaktovat (třeba na email ladislav.laska@gmail.com)

\ \\

Poděkování patří Martinu Pelikánovi za mnoho oprav chyb a vylepšení důkazů.

\ \\

\textbf{Upozornění}: Tyto poznámky jsou bez jakékoliv záruky. Nemusí být kompletní a mohou
obsahovat chyby.
\newpage
\newpage

% Titlepage is not numbered, but this is required for page compatibility in pdf
% form
% \setcounter{page}{2}

\tableofcontents
\newpage

\section{Metrické prostory}
\paragraph{Poznámka}
Obsah této kapitoly byl z části pokryt na přednáškách předmětu
\texttt{Matematická analýza II}, proto zde uvedu jenom základní přehled a
všechny důkazy naleznete v odkazovaných poznámkách. Protože tyto přednášky vedl
jiný přednášející, bude se lišit některé značení - nenechte se tedy zmást.

\subsection{Metrický prostor}
\setcounter{equation}{0}
\paragraph{Definice}
Metrický prostor je dvojice $(M, d)$ (kde funkci $d$ se říká metrika) splňující axiomy:
\begin{enumerate}
	\item $\forall x,y \in M \quad d(x,y)\gt 0 \quad d(x,y) = 0 \Leftrightarrow x = y$
	\item $\forall x,y \in M \quad d(x,y)=d(y,x)$
	\item $\forall x,y,z \in M \quad d(x,y)+d(y,z)\gt d(x,y)$
\end{enumerate}

\subsection{Izometrie}
\setcounter{equation}{0}
\paragraph{Definice}
Metrické prostory $(M_1, d_1), (M_2, d_2)$ jsou izometrické, právě když existuje
bijekce $f:M_1 \to M_2$, že $\forall x,y \in M_1 \quad d_1(x,y)=d_2(f(x),f(y))$

\subsection{Příklady metrických prostorů}
\setcounter{equation}{0}
\begin{enumerate}
	\item Euklidovský ($\R^n$)
	\item Množina omezených funkcí $M = f:x\to\R$, supremová metrika $d(f,g) =
	\sup_{x\in X} |f(x)-g(x)|$.
	Pokud jsou na požadovaném intervalu spojité, můžeme psát i $\max$
	\item Množina spojitých funkcí $M = C[a,b]$, metrika pro $p \in \R$, $p \gt
	1$:
	\begin{align*}
		d_p(f,g) = \left( \int_a^b|f(t)-g(t)|^p \dt \right)^{\frac{1}{p}}
	\end{align*}
	\item $G = (V,E)$ souvislý, $M = V$, $d(u,v) = |P_{min} (u,v)|$ (počet
	vrcholů na cestě)
	\item Hammingova metrika: $A$ je abeceda, slova délky $n \in \N$. Množina $M
	= \{u = u_1, ..., u_n | u_i \in A\}$. Metrikou potom bude $d(u,v) = \{\#i | u_i
	\neq v_i \}$
	\item Sférická metrika $S=\{(x,y,z)\in\R^3| x^2 + y^2 + z^2 = 1 \}$, $d(x,y)
	= $ délka nejkratšího oblouku z $x$ do $y$.
	\item $p$-adická metrika 
	\begin{align*}
		&p = 3, M = \Z, z \in \Z \\
		&v(z) := \max n \in \N \quad p^n \backslash z \qquad v(0) := \infty \\
		&d(z_1, z_2) = 2^{-v(z_1-z_2)}
	\end{align*}
\end{enumerate}

\subsection{Ultrametrika}
\setcounter{equation}{0}
\paragraph{Definice}
Ultrametrika (také nearchimedovská metrika) je metrika splňující:
\begin{align*}
	d(x,y) \le \max \{ d(x,z); d(y,z) \}
\end{align*}

\subsection{Otevřená a uzavřená koule}
\setcounter{equation}{0}
\paragraph{Definice}
Nechť $(M, d)$ je metrický prostor a $x \in M$, $r \in \R$, $r > 0$. Potom
definujeme:
\begin{description}
	\item[otevřenou] kouli: 
		$B(x, r) = \{ y \in M \quad d(x,y) < r\}$
	\item[uzavřenou] kouli: 
		$\overline{B(x, r)} = \{ y \in M \quad d(x,y) \le r\}$
\end{description}


\subsection{Otevřená a uzavřená množina}
\setcounter{equation}{0}
\paragraph{Definice}
Nechť $(M, d)$ je metrický prostor. Potom definujeme:
\begin{description}
	\item[otevřenou množinu] $G$ pokud $\forall x \in G \quad \exists r > 0 \quad
	B(x,r) \subsetq G$.
	\item[uzavřenou množinu] $F$ pokud $P \setminus F$ je otevřená.
\end{description}

\subsection{Vlastnosti otevřených a uzavřených množin}
\setcounter{equation}{0}
\paragraph{Věta}
Nechť $(M, d)$ je metrický prostor. Potom platí:
\begin{enumerate}
	\item $\emptyset, M$ jsou otevřené i uzavřené množiny
	\item $X_1, ..., X_n$ otevřené množiny, potom $X_1 \cap X_2 \cap ... \cap X_k$ je
	otevřená množina
	\item $X_1, ..., X_n$ uzavřené množiny, potom $\bigcup X_i$ je uzavřená
	množina
	\item $X_i, i \in I$ uzavřené množiny, potom $\bigcap_{i\in I} X_i$ je
	uzavřená množina
\end{enumerate}
(důkaz v zimním semestru)

\subsection{Cauchyovská posloupnost}
\setcounter{equation}{0}
\paragraph{Definice}
Nechť $(M, d)$ je metrický prostor a $a_1, a_2, ... \subset M$ posloupnost.
Tuto posloupnost nazveme \textbf{Cauchyovskou} právě když
\begin{align*}
	\forall \epsilon > 0 \quad \exists N \in \N : \quad m,n > N \quad
	\Rightarrow \quad d(a_m, a_n) < \epsilon
\end{align*}

\subsection{Charakterizace uzavřených množin}
\setcounter{equation}{0}
\paragraph{Věta}
Nechť $(M, d)$ je metrický prostor. 
\begin{align*}
	X \subset M \text{je uzavřená} \Leftrightarrow \left(\forall \text{kg. posl.}
	(a_n): \quad \lim a_n = a \Leftrightarrow a \in X\right)
\end{align*}
(důkaz v zimním semestru)

% to je dobrý, to tu nech :-)
\subsection{Izolovaný a limitní body}
\paragraph{Definice} 
Nechť $U(a)$ značí nějaké okolí bodu $a$.
\begin{description}
	\item[limitní bod] $\forall U(a)$ je $U(a) \cap X$ nekonečný
	\item[izolovaný bod] $\exists U(a) \cap X = a$
\end{description}



\subsection{Uzávěr množiny}
\setcounter{equation}{0}
\paragraph{Definice}
Nechť $(M,d)$ je metrický prostor a $X \subset M$. Potom definujeme uzávěr
množiny:
\begin{align*}
	\overline{X} = X \cup \{ \text{limitní body X} \}
\end{align*}


\subsection{Ekvivalence metriky}
\setcounter{equation}{0}
\paragraph{Definice}
Nechť $(M, d_1)$ a $(M, d_2)$ jsou metrické prostory. Řekneme, že metriky $d_1,
d_2$ jsou ekvivalentní, právě když:
\begin{align*}
	&\forall a \in M \quad \forall r > 0 \quad \exists s > 0: \quad B_1(a,r)
	\supset B_2(a,s) \\
	\land \qquad &\forall a \in M \quad \forall r > 0 \quad \exists s > 0: \quad B_1(a,r)
	\subset B_2(a,s)
\end{align*}

\subsection{Podprostor metriky}
\setcounter{equation}{0}
\paragraph{Definice}
Nechť $(M, d)$ je metrický prostor a $X \subset M$. Potom $(X,d)$ nazveme
podprostorem $(M, d)$.


\subsection{Spojitost zobrazení metrického prostoru}
\setcounter{equation}{0}
\paragraph{Definice}
Nechť $(M_1, d_1)$ a $(M_2, d_2)$ jsou metrické prostory. Řekneme, že $f: M_1 \to M_2$ je
spojité v $a \in M_x$ právě když:
\begin{align*}
	\forall \epsilon > 0 \quad \exists \delta > 0: \quad f(B_1(a, \delta))
	\subset B_2(f(a), \epsilon)
\end{align*}


\subsection{Věta o spojitosti zobrazení}
\setcounter{equation}{0}
\paragraph{Věta}
Nechť $(M_1, d_1)$ a $(M_2, d_2)$ jsou metrické prostory. Zobrazení $f: M_1 \to
M_2$ je spojité, právě když:
\begin{align*}
	\forall Y \subset M_2 \quad Y \text{otevřenou je } f^{-1}(Y) \subset M_1
	\text{otevřená}
\end{align*}
tj. do otevřené množiny se mohla zobrazit pouze množina otevřená (poznámka:
otevřená množina se však stále může zobrazit do množiny uzavřené)
\paragraph{Důkaz}
\begin{enumerate}
	\item $\Rightarrow$
		triviální z definice spojitého zobrazení
	\item $\Leftarrow$
		Mějme libovolné $a \in M_1, \quad \epsilon > 0$. Podle předpokladu
		platí:
		\begin{align}
			B_2(f(a), \epsilon) \text{ otevřená } \Rightarrow V := f^{-1}(B_2(f(a),
			\epsilon)) \text{ otevřená }
		\end{align}
		Protože $V \subset M_1$ a $a \in M_1$:
		\begin{align}
			\exists \delta > 0: \quad B_1(a, \delta) \subset f^{-1}(B_2(f(a),
			\epsilon))
		\end{align}
		Po zobrazení obou množin $f$ platí:
		\begin{align}
			f(B_1(a, \delta)) \subset B_2(f(a), \epsilon)
		\end{align}
		Tedy $f$ je spojité zobrazení (podle definice).
\end{enumerate}


\subsection{Homeomorfismus}
\setcounter{equation}{0}
\paragraph{Definice}
Bijektivní zobrazení $f$ je \textbf{homeomorfní} právě když jsou $f$ i $f^{-1}$
spojitá zobrazení.


\subsection{Kompaktní prostor a kompaktní množina}
\setcounter{equation}{0}
\paragraph{Definice} 
Metrický prostor $(M,d)$ je kompaktní, právě když má každá posloupnost $(a_n)
\subset M$ konvergentní podposloupnost. (navíc $\lim a_{n_k} \in M$, protože, jak uvidíme,
je kompaktní množina vždy uzavřená)
\paragraph{Definice}
Nechť $(M,d)$ je metrický prostor. Množina $X \subset M$ je kompaktní, právě když
metrický prostor $(X, d)$ je kompaktní. 
\paragraph{Definice}
Množina $K$ je kompaktní právě tehdy, pokud existuje pro každou posloupnost vybraná
konvergentní podposloupnost a má vlastní limitu v množině $K$.


\subsection{Nabývání maxima a minima na kompaktní množině}
\setcounter{equation}{0}
\paragraph{Věta}
Nechť $(M, d)$ je metrický prostor a $Z \subset M$ je kompaktní množina. Potom
spojitá funkce $f: Z \to \R$ nabývá na $Z$ maxima i minima.\\
(důkaz v dřívějších poznámkách)


\subsection{Věta o zachování kompaktnosti}
\setcounter{equation}{0}
\paragraph{Věta}
Kompaktnost se zachovává:
\begin{enumerate}
	\item Přechodem k uzavřenému podprostoru
	\item Obrazem spojitým zobrazením
	\item Kartézským součinem
\end{enumerate}
\paragraph{Důkaz}
\begin{enumerate}
	\item Nechť $(M, d)$ je metrický prostor a $X \subset M$ je uzavřená
	množina. Mějme libovolnou posloupnost $(a_n) \subset X$. Její konvergentní
	podposloupnost existuje a má limitu $A \in X$ (z uzavřenosti $X$), tedy
	podle definice je i prostor $(X, d)$ uzavřený.
	\item Nechť $f: M_1 \to M_2$ je spojité zobrazení. Z předpokladu je $M_1$ kompaktní. 
	Vezmeme libovolnou posloupnost $(b_n) \subset f(M_1)$ a $\forall n$ zvolíme $a_n \in M_1$ tak, že $f(a_n) =
	b_n$. Protože $(a_n) \subset M_1$ a $M_1$ je kompaktní, existuje
	konvergentní podposloupnost $(a_{k_n})$ s limitou $a \in M_1$. Podle Heineho
	definice spojitosti zobrazení víme, že $b_{k_n} = f(a_{k_n})$, tedy $f(a) =
	b$ a $b$ je limitou $b_{k_n}$ v $f(M_1)$ $\Rightarrow$ $f(M_1)$ je
	kompaktní.
	\item $M = M_1 \times M_2, d( (x_1, y_1), (x_2, y_2) ) = \sqrt{d_1(x_1, x_2)^2 + d_2(y_1, y_2)^2}$
	Potom tedy posloupnost (dvojic) $(x_n, y_n)$ konverguje k $(\alpha, \beta)$, právě, když 
	$x_n \rightarrow \alpha \text{v} (M_1, d_1)$ a zároveň $y_n \rightarrow \beta \text{v} (M_2, d_2)$
	\textit{...to se do toho nebudu zamotávat... (M. Klazar)}
\end{enumerate}


\subsection{Kompaktní množina je omezená a uzavřená}
\setcounter{equation}{0}
\paragraph{Tvrzení}
Každá kompaktní množina je omezená a uzavřená.
\paragraph{Důkaz}
\begin{enumerate}
\item Mějme množinu, která není omezená. V takovéto množině existuje posloupnost
s limitou v~nekonečnu, tedy nemá konvergentní podposloupnost a podle definice
není kompaktní.
\item Mějme množinu $X$, která není uzavřená. Potom existuje konvergentní
posloupnost $(a_n) \subset X$, ale $\lim a_n = a \nin X$. Takováto posloupnost
však nemá konvergentní posloupnost v X, tudíž podle definice není kompaktní.
\end{enumerate}

\subsection{Uzavřená a omezená podmnožina $\R^n$ je kompaktní}
\setcounter{equation}{0}
\paragraph{Věta}
Množina $X\subset \R^n$ (euklidovského prostoru) je kompaktní, právě když je omezená a uzavřená.
\paragraph{Důkaz}
\begin{enumerate}
	\item Každá kompaktní množina je uzavřená a omezená. (podle předchozí věty)
	\item Množina $X$ je omezená a uzavřená. Vezměme $n$-dimenzionální krychli
	$K^n$ tak velkou, aby platilo $X \subset K^n$ (to můžeme, protože $X$ je
	omezená). $K^n$ je určitě kompaktní (podle Bolzano-Weierstrassovy věty pro
	n-dimenzí). Kompaktnost se však přenáší na uzavřenou podmnožinu, tudíž i $X$
	je kompaktní.
\end{enumerate}



\subsection{Věta o spojitém zobrazení na kompaktu}
\setcounter{equation}{0}
\paragraph{Věta}
Nechť $f: M_1 \to M_2$ je spojité zobrazení a $M_1$ je kompaktní množina. Potom:
\begin{enumerate}
	\item pro $M_2 = \R$ nabývá $f$ na $M_1$ maxima i minima
	\item $f$ je stejnoměrně spojité zobrazení
	\item pokud $f$ je navíc bijekce, potom i $f^{-1}$ je spojité
\end{enumerate}
\paragraph{Důkaz}
\begin{enumerate}
	\item Protože spojité zobrazení zachovává kompaktnost, $f(M_1)$ je
	kompaktní, v $\R$ tedy má supremum a infimum a tedy (protože na $\R$) minimum a maximum.
	\item (důkaz jako cvičení)
	\item Mějme zobrazení $f^{-1}: M_2 \to M_1$ inverzní k $f$. Nechť $Y \subset
	M_1$ je uzavřená množina - ověříme, zda i vzor je uzavřený.
	\begin{align}
		(f^{-1})^{-1} (Y) = f(Y)
	\end{align}
	Protože ale $f$ je spojité a $Y$ uzavřená, i $f(Y)$ je uzavřená.
\end{enumerate}

\subsection{Otevřené pokrytí}
\setcounter{equation}{0}
\paragraph{Definice}
Nechť $(M, d)$ je metrický prostor a $P = \{X_i | i \in \I\}$ jsou otevřené
množiny. \\ $P$ je otevřené pokrytí, právě když:
\begin{align}
	\bigcup_{i \in \I} X_i = M
\end{align}

\subsection{Kompaktnost a otevřené pokrytí}
\setcounter{equation}{0}
\paragraph{Tvrzení}
Metrický prostor $(M,d)$ je kompaktní právě tehdy, když otevřené
pokrytí $P = \{X_i | i \in \I\}$ prostoru M má konečné podpokrytí:
\begin{align*}
	\exists J \subset \I: \quad \bigcup_{i \in J} X_i = M
\end{align*}
\paragraph{Důkaz}
(neznám)



\subsection{Cauchyovská posloupnost}
\setcounter{equation}{0}
\paragraph{Definice}
Posloupnost $(a_n)$ je v metrickém prostoru $(M, d)$ Cauchyovská, právě když:
\begin{align*}
	\forall \epsilon > 0 \quad \exists n_0 \in \N: \quad \forall m > n > n_0
	\quad d(a_m, a_n) < \epsilon
\end{align*}


\subsection{Úplný metrický prostor}
\setcounter{equation}{0}
\paragraph{Definice}
Metrický prostor je úplný, právě když je každá Cauchyovská posloupnost také konvergentní.

\subsection{Kompaktní prostor je úplný}
\setcounter{equation}{0}
\paragraph{Věta}
Každý kompaktní prostor $(M,d)$ je také úplný.
\paragraph{Důkaz}
Nechť $(a_n) \subset M$ je Cauchyovská posloupnost. Potom (z kompaktnosti)
existuje konvergentní podposloupnost $(a_{k_n})$ s limitou v $a$. Tedy zároveň
platí:
\begin{align}
	&\forall \epsilon > 0 \quad \exists n_1 \in \N: \quad \forall m > n > n_1
	\quad &d(a_m, a_n) < \epsilon \\
	&\forall \epsilon > 0 \quad \exists n_2 \in \N: \quad \forall k_n > n_2 \quad
	&d(a_{k_n}, a) < \epsilon
\end{align}
Zvolme tedy $n := \max\{ n_1, n_2 \}$, potom i pro $m := {k_n}$ nerovnosti
platí. Tedy:
\begin{align}
	\forall \epsilon > 0 \quad \exists n_0 \in \N: \quad \forall n > n_0 \quad
	d(a_{k_n}, a_n) + d(a_{k_n}, a) < \epsilon 
\end{align}
Pokud uplatníme trojúhelníkovou nerovnost, platí $(a_n)$ je podle definice
konvergentní.


\subsection{Zachování úplnosti}
\setcounter{equation}{0}
\paragraph{Tvrzení}
Nechť $(M, d)$, $(M', d')$ jsou úplné metrické prostory. Potom platí:
\begin{enumerate}
	\item $X \subset M$ uzavřená $\Rightarrow$ $(X, d)$ úplný
	\item Kartézský součin prostorů $M \times M'$ je také úplný
\end{enumerate}
\paragraph{Důkaz}
(bez důkazu)


\subsection{Kontrakce a pevný bod}
\setcounter{equation}{0}
\paragraph{Definice}
Nechť $(M, d)$ je metrický prostor. Zobrazení $f: M \to M$ je \textbf{kontrakce} pokud:
\begin{align*}
	\exists q \in (0,1): \qquad 
	\forall x, y \in M: \quad d(f(x), f(y)) \lt q \cdot d(x, y)
\end{align*}
\paragraph{Definice}
Nechť $f: X \to X$ je kontrakce. Bod $a \in X$ nazveme \textbf{pevným bodem}, pokud $f(a)
= a$.


\subsection{Banachova věta o pevném bodě}
\setcounter{equation}{0}
\paragraph{Věta}
Nechť $(M, d)$ je úplný metrický prostor a $f: M \to M$ je kontrakce. Potom má
$f$ právě jeden pevný bod $a \in M$.
\paragraph{Důkaz}

Mějme posloupnost prvků $(x_n)$ definovanou jako:
\begin{align*}
	x_0 \in M \qquad
	\forall n > 0 \quad x_n = f(x_{n-1})
\end{align*}
Protože $f$ je kontrakce, můžeme odhadnout členy následovně:
\begin{align}
	\label{banach-odhad}	d(x_{n+2}, x_{n+1}) \le q \cdot d(x_{n+1}, x_n) \le ... \le q^n \cdot d(x_2,
	x_1)
\end{align}
\begin{enumerate}
\item
Ověříme, že taková posloupnost je Cauchyovská: mějme tedy z definice $m > n >
n_0$, odhadmene trojúhelníkovou nerovností:
\begin{align}
	d(x_m, x_n) \le d(x_m, x_{m-1}) + d(x_{m-1}, x_{m-2}) + ... + d(x_n+1, x_n)
\end{align}
Takovýto člen odhadneme podle (\ref{banach-odhad}) a sečteme geometrickou řadu v
závorce:
\begin{align}
	d(x_m, x_n) \le d(x_1, x_0)( q^{m-1} + ... + q^n ) = d(x_1, x_0) \cdot
	\frac{q^n}{q-q}
\end{align}
Protože $q < 1$, součet řady konverguje k $0$ a tudíž posloupnost má limitu a je
Cauchyovská. Nechť tedy $a := \lim x_n$ (z úplnost existuje) \textit{je pevný
bod}.
\item Nechť tedy $a \neq b$ jsou pevné body. Potom nutně se zobrazují samy na
sebe a tedy platí:
\begin{align}
	d(a,b) = d(f(a), f(b))
\end{align}
A podle definice kontrahujícího zobrazení:
\begin{align}
	d(f(a), f(b)) \le q \cdot d(a, b)
\end{align}
Což (protože $0 \lt q$) vynucuje $d(a,b) = 0$ a tudíž $a = b$ (spor s
předpokladem $a \neq b$).
\end{enumerate}

\subsection{Picardova věta}
\setcounter{equation}{0}
\paragraph{Věta}
Nechť $f: \R^2 \to \R$ je spojitá funkce a existuje konstanta $M > 0$ taková,
že:
\begin{align*}
	\forall u,v,w \in \R: \quad |f(u,v) - f(u,w)| \le M|v-w|
\end{align*}
Pak každý bod $a \in \R$ má okolí $I = (a - \delta, a + \delta)$ na něm má
rovnice (\ref{dif-rov}) jednoznačné řešení:
\begin{align}
	\label{dif-rov} y(a) &= b \\
	y'(x) &= f(x, y(x)) \notag
\end{align}
(důkaz není požadován ke zkoušce, bude doplněn později)








\section{Posloupnosti a řady funkcí}
\setcounter{equation}{0}
\subsection{Bodová, stejnoměrná a lokálně stejnoměrná konvergence}
\setcounter{equation}{0}
Nechť $M \subset \R$ je množina a $f_n: M \to \R$ je posloupnost funkcí. Potom
definujeme:
\paragraph{Definice}
Posloupnost funkcí $f_n$ je \textbf{bodově konvergentní} ($f_n
\to f$), pokud:
\begin{align*}
	\forall \alpha \in M \quad \lim f_n(\alpha) = f(\alpha)
\end{align*}
\paragraph{Definice}
Posloupnost funkcí $f_n$ je \textbf{stejnoměrně konvergentní} ($f_n \sto f$), pokud:
\begin{align*}
	\forall \epsilon > 0 \quad \exists n_0 \in \N \quad \forall n > n_0 \quad \forall
	\alpha \in M: \quad |f_n(\alpha) - f(\alpha)| < \epsilon
\end{align*}
\paragraph{Definice}
Posloupnost funkcí $f_n$ je \textbf{lokálně stejnoměrně konvergentní} ($f_n
\stol f$), pokud:
\begin{align*}
	\forall \alpha \in M \quad \exists \delta > 0: \quad f_n \sto f \text{ na }
	(\alpha-\delta, \alpha+\delta) \cap M
\end{align*}



\subsection{Bolzano-Cauchyho podmínka}
\label{bc-podminka}
\setcounter{equation}{0}
\paragraph{Věta}
Nechť je množina $M \subset \R$ a $f_n$ posloupnost funkcí $f: M \to \R$, potom:
\begin{align*}
	\exists f: M \to \R \qquad f_n \sto f \text{ na } M
\end{align*}
právě když
\begin{align*}
	\forall \epsilon > 0 \quad \exists n_0: \quad \forall m > n > n_0 \quad
	\forall x \in M \Rightarrow |f_m(x) - f_n(x)| < \epsilon
\end{align*}
\paragraph{Důkaz}
\begin{enumerate}
	\item $\Rightarrow$ $f_n \sto f$ na M, platí tedy:
	\begin{align}
		\forall \epsilon > 0 \quad \exists n_0: \quad \forall n > n_0 \quad
		\forall x \in M: \qquad |f_n(x) - f(x)| < \epsilon
	\end{align}
	Podle trojúhelníkové nerovnosti odhadneme:
	\begin{align}
		|f_m(x) - f_n(x)| \le |f_m(x) - f(x)| + |f_n(x) - f(x)| < 2\epsilon
	\end{align}
	Což splňuje podmínku.
	\item $\Leftarrow$ Z předpokladu víme, že posloupnost je Cauchyovská, tedy
	existuje $\lim f_n(x) = f(x)$ (podle věty pro posloupnosti reálných čísel).
	Z předpokladů vezmeme $n > n_0$ a $N$ libovolné, a odhadneme požadovaný
	rozdíl trojúhelníkovou nerovností:
	\begin{align}
		|f_n(x) - f(x) | \le |f_N(x) - f(x)| + |f_N(x) - f_n(x)|
	\end{align}
	Což platí pro každé $N$. Zvolme tedy $N > n_0$ a podmínka platí.
	\begin{align}
		|f_N(x) - f(x)| < 2\epsilon
	\end{align}
\end{enumerate}



\subsection{Tvrzení o lokální stejnoměrné konvergenci a kompaktní podmnožině}
\setcounter{equation}{0}
\paragraph{Tvrzení}
Nechť $[c,d] \subset (a,b)$ a $f_n \stol f$ na $(a,b)$. Pak $f_n \sto f$ na
$[c,d]$
\paragraph{Důkaz}
(bez důkazu)
\subsection{Diniho věta}
\setcounter{equation}{0}
Nechť $f_n \to f$ na kompaktním intervalu $I$ a funkce $f_n$ i $f$ jsou spojité
a posloupnost je na daném intervalu monotónní. Pak $f_n \sto f$ na $I$.
\paragraph{Důkaz}
(bez důkazu)


\subsection{Mooreova-Osgoodova věta o záměně pořadí limit}
\setcounter{equation}{0}
\paragraph{Věta}
Nechť jsou funkce $f_n$ a $f$ definované na nějakém prstencovém okolí $M = P(x_0,
\delta)$ bodu $x_0 \in \R$, který může být i nevlastní, existují vlastní limity
\begin{align*}
	a_n = \lim_{x \to x_0} f_n(x) \text{ a dále } f_n \sto f \text{ na } P(x_0,
	\delta)
\end{align*}
Potom existují vlastní limity $\lim_{n\to\infty} a_n$ a $\lim_{x\to x_0} f(x)$ a
rovnají se.
\paragraph{Důkaz}
Protože $f_n \sto f$ na $M$, splňuje $f_n$ Bolzano-Cauchyho podmínku:
\begin{align}
	\forall \epsilon > 0 \quad \exists n_0 \quad \forall m > n > n_0 \forall x
	\in M \qquad |f_m(x) - f_n(x)| < \epsilon
\end{align}
Pro pevné indexy $m,n> n_0$ a limitní přechod $x\to x_0$ (limity existují z
předpokladu) máme nerovnost:
\begin{align}
	|a_m - a_n| \le \epsilon
\end{align}
Tedy posloupnost čísel $(a_n)$ je cauchyovská a podle věty o posloupnostech
reálných čísel má vlastní limitu $A \in \R$:
\begin{align}
	\lim_{n\to\infty} a_n = A
\end{align}
Nyní ukážeme, že $\lim_{x\to x_0} f(x) = A$. Odhadneme tedy trojúhelníkovou
nerovností:
\begin{align}
	| f(x) - A | \le \underbrace{|f(x) - f_n(x)|}_{V_1 < \epsilon} + \underbrace{|f_n(x) -
	a_n |}_{V_2 < \epsilon} + \underbrace{|a_n - A|}_{V_3 < \epsilon}
\end{align}
Nyní ukážeme, že pravá strana je $< \epsilon$: $V_1$ a $V_3$ "platí" z předpokladů
pro nějaké $n > n_1$ a $ n > n_3$. Vezměme tedy $n_0 > \max \{ n_1, n_3 \}$. Pro
takové $n_0$ navíc existuje $\delta_0$ takové, aby pro $x \in P(x_0, \delta_0)$
"platí" i $V_2$ a tedy platí i rovnost $\lim_{x\to x_0} f(x) = A$, což jsme
chtěli dokázat.


\subsection{Věta o záměně limity a integrování}
\setcounter{equation}{0}
\paragraph{Věta}
Nechť $f_n, f: [a,b] \to \R$ jsou riemannovsky integrovatelné funkce na
intervalu $[a,b]$ a $f_n \to f$ na $[a,b]$. Potom $f$ je riemannovsky
integrovatelná na $[a,b]$ a platí:
\begin{align*}
	\lim_{n \to \infty} \int_a^b f_n = \int_a^b \lim_{n \to \infty} f_n =
	\int_a^b f
\end{align*}
\paragraph{Důkaz} (důkaz je technický a opsaný ze skript)
Nejdříve si připravíme několik nerovností. Mějme $\epsilon > 0$. Ze stejnoměrné 
konvergence existuje $n_0$, že $\forall n > n_0 \land \forall x \in [a,b]$:
\begin{align}
f_n(x) - \epsilon < f(x) < f_n(x) + \epsilon
\end{align}
Nechť $D = (a_0, a_1, ..., a_k)$ kde $a = a_0 < a_1 < ... < a_k = b$ je libovolné
dělení intervalu $[a,b]$ a $n > n_0$ je pevné. Potom na intervalech $I_i = [a_i,
a_i+1]$ platí nerovnosti:
\begin{align}
	\label{rov-minmax-int} m_i - \epsilon = \inf_{I_i} f_n - \epsilon \le \inf_{I_i} f \\
	\sup_{I_i} f \le \sup_{I_i} f_n + \epsilon = M_i + \epsilon
\end{align}
Nyní dokážeme, že platí nerovnosti:
\begin{align}
	\label{souc-rov-int} s(f_n, D) - \epsilon < s(f, D) \le S(f, D) < S(f_n, D) + \epsilon
\end{align}
Pro dolní součty odhadneme podle (\ref{rov-minmax-int}) (pro horní součty
analogicky):
\begin{align}
	s(f, D) - \epsilon(b-a) = \sum_{i=0}^{k-1} |I_i| m_i - \epsilon(b-a) =
	\sum_{i=0}^{k-1} |I_i| (m_i - \epsilon ) \\
	\le \sum_{i=0}^{k-1} |I_i| \inf_{I_i} f = s(f, D)
\end{align}
Tedy (\ref{souc-rov-int}) platí $\forall \epsilon > 0$. \\
Nyní již ukážeme, že $f \in \ri[a,b]$. Nechť je dáno $\epsilon > 0$ a $n > n_0$
je libovolné, ale pevné. Protože $f_n \in \ri[a,b]$, můžeme vzít dělení $D_0$
takové, že $0 \le S(f_n, D_0) - s(f_n, D_0) < \epsilon$. Takové dělení
aplikujeme na funkci $f$ (používáme navíc nerovnosti
(\ref{souc-rov-int})):
\begin{align}
	\label{int-rov-int} 0 \le S(f, D_0) - s(f, D_0) \le S(f_n, D_0) + \epsilon - (s(f_n, D_0) -
	\epsilon) < 3 \epsilon
\end{align}
Tedy podle kritéria integrovatelnosti (věta z MA2) má i funkce $f$ integrál na
$[a,b]$. Zbývá tedy dokázat, že jsou si rovny. Nahlédneme, že samotný integrál
je omezen dolním a horním součtem, tedy:
\begin{align}
	\int_a^b f \in  [s(f,D_0), S(f, D_0)] \\
	\int_a^b f_n \in [s(f_n, D_0), S(f, D_0)]
\end{align}
A oba intervaly jsou obsaženy v intervalu $[s(f_n, D_0) - \epsilon, S(f_n, D_) +
\epsilon]$, jehož velikost je $<3\epsilon$ podle (\ref{int-rov-int}). Tedy:
\begin{align}
	\left| \int_a^b f - \int_a^b f_n \right | < \epsilon
\end{align}
Což platí pro každé $\epsilon > 0$ a $n > n_0$, podle definice tedy platí
rovnost:
\begin{align}
	\int_a^b f = \lim_{n\to\infty} \int_a^b f_n
\end{align}


\subsection{Věta o záměně pořadí limity  a derivace}
\setcounter{equation}{0}
\paragraph{Věta}
Nechť $f_n : (a,b) \to \R$ je posloupnost funkcí definovaná na omezeném
otevřeném intervalu a:
\begin{enumerate}
	\item každá funkce $f_n$ má na $(a,b)$ vlastní derivaci
	\item $f_n' \stol g$ na $(a,b)$
	\item posloupnost čísel $(f_n(x_0))$ konverguje pro alespoň jeden bod 
		$x_0	\in (a,b)$
\end{enumerate}
Potom $f_n \stol f$ na $(a,b)$ pro nějakou funkci $f: (a,b) \to \R$ a $f' = g$ na
$(a,b)$.
\paragraph{Důkaz}
(bez důkazu)


\subsection{Věta o záměně pořadí sumace a limity v bodě}
\setcounter{equation}{0}
\paragraph{Věta}
Nechť pro nějaké $\delta > 0$ platí:
\begin{align*}
	f_n : P(x_0, \delta) \to \R \quad \forall n \in \N \\
	\forall n \in \N \exists \lim_{x\to x_0} f_n(x) \text{ vlastní}\\
	\sum_{n=1}^\infty f_n \sto f \text{ na } P(x_0, \delta)
\end{align*}
Pak:
\begin{align*}
	\sum_{n=1}^\infty \left( \lim_{x\to x_0} f_n(x) \right) = \lim_{x\to x_0} 
	\left( \sum_{n=1}^\infty f_n(x) \right)
\end{align*}
\paragraph{Důkaz}
(bez důkazu)



\subsection{Věta o záměně pořadí sumace a integrování}
\setcounter{equation}{0}
\paragraph{Věta}
Nechť $\forall n \in \N$ platí, že $f_n \in\ri[a,b]$ a 
$\sum_{n=1}^\infty f_n \sto$ na $[a,b]$. Potom:
\begin{align*}
	\sum_{n=1}^\infty \left( \int_a^b f_n \right) = \int_a^b \left(
	\sum_{n=1}^\infty f_n \right)
\end{align*}
\paragraph{Důkaz}
(bez důkazu)


\subsection{Věta o záměně pořadí sumance a derivování}
\setcounter{equation}{0}
Nechť $f_n : (a,b) \to \R$ je posloupnost funkcí definovaná na omezeném
otevřeném intervalu a:
\begin{enumerate}
	\item každá funkce $f_n$ má na $(a,b)$ vlastní derivaci
	\item $\sum_{n=1}^\infty f_n' \stol g$ na $(a,b)$
	\item řada čísel $\sum_{n=1}^\infty f_n(x_0)$ konverguje pro alespoň jeden bod 
		$x_0	\in (a,b)$
\end{enumerate}
Potom $\sum_{n=1}^\infty f_n \stol f$ na $(a,b)$ pro nějakou funkci $f:(a,b) \to
\R$ a $f' = g$ na $(a,b)$.
\paragraph{Důkaz}
(bez důkazu)



\subsection{Weierstrassovo kritérium stejnoměrné konvergence řad}
\label{weierstrassovo-kriterium}
\setcounter{equation}{0}
\paragraph{Věta}
Nechť $f_n : M \to \R$ jsou takové funkce, že řada:
\begin{align*}
	\sum_{n=1}^\infty ||f_n||_\infty = \sum_{n=1}^\infty \sup_{x\in M} |f_n(x)|
\end{align*}
konverguje. Potom $\sum f_n$ stejnoměrně konverguje na $M$.
\paragraph{Důkaz}
Protože řada $\sum ||f||_\infty$ konverguje, splňuje Cauchyho podmínku pro
číselné řady a existuje $n_0$ takové, že $\forall n \ge m > n_0$ platí:
\begin{align}
	\sum_{i=m+1}^n \sup_{x\in M} |f_i(x)| < \epsilon
\end{align}
Odhadneme tedy částečný součet původní řady:
\begin{align}
	\left| \sum_{i=m+1}^n f_i(x)  \right| \le \sum_{i=m+1}^n |f_i(x)| \le
	\sum_{i=m+1}^N \sup_{x \in M} |f_i(x)| < \epsilon
\end{align}
Takže původní řada splňuje B-C podmínku (\ref{bc-podminka}) a tedy stejnoměrně konverguje.


\subsection{Abel-Dirichletovo kritérium stejnoměrné konvergence}
\setcounter{equation}{0}
\paragraph{Věta}
Nechť $f_n, g_n: M \to \R$ jsou posloupnosti funkcí. Řada $\sum f_n g_n$
stejnoměrně konverguje na $M$ když je splněna podmínka (A) nebo (D)
	a $\forall x \in M$ je posloupnost $(g_n(x))$ monotónní.
\begin{description}
	\item[(A)] $\sum f_n \sto$ a existuje konstanta $c > 0$ taková, že 
	\begin{align*}
		\forall	n \in \N \quad \forall x \in M: \quad |g_n(x)| < c
	\end{align*}
	\item[(D)] existuje konstanta $c > 0$ taková, že:
	\begin{align*}
		&\forall n \in \N \quad \forall x \in M: \quad |f_1(x) + ... + f_n(x)| <
		c \\
		\land &g_n \sto 0 \text{ na } M
	\end{align*}
\end{description}
\paragraph{Důkaz}
(bez důkazu)



\section{Mocninné řady funkcí}
\setcounter{equation}{0}
\paragraph{Definice}
Mocninná řada s \emph{koeficienty} $a_n \in \R$ a \emph{středem} $x_0$ je nekonečná řada funkcí:
\begin{align*}
	\sum_{n = 0}^\infty  a_n (x - x_0)^n
\end{align*}
Pro jednoduchost v této kapitole uvažujeme řady se středem v 0, tj. ve tvaru $\sum a_n x^n$.


\subsection{Poloměr konvergence}
\label{polomer-konvergence}
\setcounter{equation}{0}
\paragraph{Věta}
Nechť $\sum a_n x^n$ je mocninná řada a $R \in [0, +\infty) \cup \{+\infty\}$ je
definováno vztahem:
\begin{align*}
	R = \frac{1}{\limsup_{n \to \infty} |a_n|^{\frac{1}{n}}}
\end{align*}
Potom pro každé $x \in \R$, když $|x| < R$ mocninná řada absolutně konverguje a
když $|x| > R$ mocninná řada diverguje.
\paragraph{Důkaz}
\begin{enumerate}
	\item Nechť $0 < R < +\infty$. Použijeme Cauchyho odmocninové kritérium
	(věta z MA1) a pro každé $x \in \R$ máme:
	\begin{align}
		\limsup_{n \to \infty} |a_n x^n|^{\frac{1}{n}} = |x|
		\limsup_{n\to\infty} |a_n|^{\frac{1}{n}} = \frac{|x|}{R}
	\end{align}
	Tedy pro $|x| < R$ řada absolutně konverguje a pro $|x| > R$ diverguje.
	\item Pokud $R = +\infty$, máme $\forall x \in \R$:
	\begin{align}
		\limsup_{n \to \infty} |a_n|^{\frac{1}{n}}
	\end{align}
	a mocninná řada tedy na celém $\R$ konverguje.
	\item Analogicky pro $R = 0$, mocninná řada absolutně konverguje pro $x \in
	\R \setminus \{0\}$.
\end{enumerate}


\subsection{Lokálně stejnoměrná konvergence mocninné řady}
\label{lokalne-stejnomerna-konvergence-mr}
\setcounter{equation}{0}
\paragraph{Věta}
Nechť $\sum a_n x^n$ má poloměr konvergence $R > 0$. Potom:
\begin{align*}
	\sum_{n=0}^\infty a_n x^n \stol \text{ na } (-R, R)
\end{align*}
Neboli mocinná řada stejnoměrně konverguje na každém kompaktním podintervalu
konvergence.
\paragraph{Důkaz}
Omezíme se na kompaktní podintervaly, tj. intervaly $[-S,S]$, kde $0 < S < R$.
Na takovém intervalu potom máme:
\begin{align}
	\sum ||a_n x^n||_\infty = \sum |a_n| S^n
\end{align}
Taková řada podle Cauchyho odmocninového kritéria opět konverguje, takže podle
Weierstrassova kritéria (\ref{weierstrassovo-kriterium}) $\sum a_n x^n \sto$ na
$[-S, S]$.



\subsection{Abelova věta o mocinných řadách}
\label{abelova-veta}
\setcounter{equation}{0}
\paragraph{Věta}
Nechť má $\sum a_n x^n$ kladný a konečný poloměr konvergence $R$ a číselná řada
$\sum a_n R^n$ konverguje, čili mocninná řada konverguje pro $x = R$. Potom:
\begin{align*}
	\sum_{n=0}^\infty a_n x^n \sto \text{ na } [0, R] \quad \text{ a } \quad 
	\lim_{x\to R^-} \sum_{n=0}^\infty a_n x^n = \sum_{x=0}^\infty a_n R^n
\end{align*}
\paragraph{Důkaz}
(bez důkazu)


\section{Fourierovy (trigonometrické) řady}
\setcounter{equation}{0}
\paragraph{Definice}
Fourierovou řadou budeme rozumět nekonečnou řadu funkcí tvaru:
\begin{align*}
	\frac{a_0}{2} + \sum_{n=1}^\infty( a_n \cos(nx) + b_n \sin(nx))
\end{align*}
kde $a_0, a_1, ...$ a $b_1, b_2, ...$ jsou pevně dané reálné koeficienty a
proměnná $x$ probíhá $\R$.

\subsection{Skoro-skalární součin}
\setcounter{equation}{0}
\paragraph{Definice}
"Skoro-skalární součin" definujeme jako:
\begin{align*}
	\langle f, g \rangle  := \int_{-\pi}^\pi fg
\end{align*}
Má následující vlastnosti:
\begin{enumerate}
	\item (symetrie) $\langle f_1, f_2 \rangle  = \langle f_2, f_1 \rangle $
	\item (bilinearita) $\langle a_1f_1 + a_2f_2, g \rangle  = a_1\langle f_1, g \rangle  + a_2\langle f_2, g \rangle $
	\item (pozitivní semidefinitnost) $\langle g,g \rangle  \ge 0$
\end{enumerate}


\subsection{Ortogonální systém sinů a cosinů}
\setcounter{equation}{0}
\paragraph{Věta}
Pro každá dvě čísla $m,n \in \N_0$ máme:
\begin{align*}
	\langle \sin(mx), \cos(nx)\rangle = 0	
\end{align*}
Pro každá dvě čísla $m,n \in \N_0$, pokud nejsou současně nulová, máme:
\begin{align*}
	\langle \sin(mx), \sin(nx) \rangle = \langle \cos(mx), \cos(nx) \rangle = 
	\begin{cases}
		\pi \text{ pro } m = n \\
		0 \text{ pro } m \neq 0
	\end{cases}
\end{align*}
Pro $m = n = 0$ pak zvlášť $\langle\sin(0), \sin(0)\rangle = 0$ a
$\langle\cos(0), \cos(0)\rangle = 2\pi$.
\paragraph{Důkaz}
(bez důkazu)





\subsection{Besselova nerovnost a Riemann-Lebesgueovo lemma}
\label{rl-lemma}
\setcounter{equation}{0}
\paragraph{Věta}
Nechť $f \in \ri[-\pi,\pi]$ je funkce a čísla $a_0, a_1, ..., b_1, ...$ jsou
Fourierovy koeficienty funkce $f$. Potom:
\begin{enumerate}
	\item Platí Besselova nerovnost:
	\begin{align*}
		\frac{a_0^2}{2} + \sum_{n=1}^\infty (a_n^2 + b_n^2) \lt \frac{\langle f,
		f \rangle}{\pi} = \frac{1}{\pi} \int_{-\pi}^\pi f^2
	\end{align*}
	Speciálně tedy řada čtverců Fourierových koeficientů konverguje.
	\item Platí Riemann-Lebesgueovo lemma: \\
	Pro $n \to \infty$ platí, že $a_n \to 0$ a $b_n \to 0$, tedy:
	\begin{align*}
		\lim_{n\to\infty} \int_{-\pi}^\pi f(x)\cos(nx) \dx = \lim_{n\to\infty} \int_{-\pi}^\pi f(x)\sin(nx) \dx = 0
	\end{align*}
\end{enumerate}
\paragraph{Důkaz}
\begin{enumerate}
	\item Pro $n \in \N$ označíme částečný součet $s_n = s_n(x)$ Fourierovy řady
	funkce $f$:
	\begin{align}
		s_n = \frac{a_0}{2} + \sum_{k=1}^n(a_k \cos(kx) + b_k\sin(kx)) \\
		\text{kde }\quad  a_k = \frac{\langle f, \cos(kx)\rangle}{\pi}, 
		b_k = \frac{\langle f, \sin(kx)\rangle}{\pi}
	\end{align}
	Díky vlastnostem skalárního součinu snadno získáme nerovnost a
	upravíme ji:
	\begin{align}
		&0 \le \langle f - s_n, f - s_n \rangle = \langle f, f \rangle - 2
		\langle f, s_n \rangle + \langle s_n, s_n \rangle \\
		\label{bess-neq}
		&2 \langle f, s_n \rangle - \langle s_n, s_n \rangle \le \langle f, f
		\rangle
	\end{align}
	Vyjádříme tedy $\langle f, s_n \rangle$ (dosadíme za $s_n$ a pomocí
	bilinearity rozepíšeme, potom podle definice $a_k$ a $b_k$ dosadíme a vytkneme):
	\begin{align}
		\langle f, s_n \rangle &= \frac{a_0}{2} \langle f, \cos(0x) \rangle +
		\sum_{k=1}^n (a_k \langle f, \cos(kx) \rangle + b_k \langle f,
		\sin(kx)\rangle) \\
		\label{bess-fs}
		&= \pi \left( \frac{a_0^2}{2} + \sum_{k=1}^n(a_k^2 + b_k^2) \right)
	\end{align}
	Pro jednodušší značení si zavedeme $a_k'$ a $b_k'$, kde $a_0' = a_0 / 2$,
	$b_0' = 0$ a pro $n > 0$ $a_n' = a_n$ a $b_n' = b_n$. 
	Dále podle bilinearity (\ref{bess-lin}), předchozí věty o ortogonalitě
	(\ref{bess-tvrz}) a dosazením $a_k$ a $b_k$ podle definice (\ref{bess-dos}):
	\begin{align}
		\langle s_n, s_n \rangle &= \left\langle
			\sum_{k=0}^n(a_k' \cos(kx) + b_k' \sin(kx)),
			\sum_{l=0}^n(a_l' \cos(lx) + b_l' \sin(lx))
		\right\rangle \\
		\label{bess-lin}
		&= \sum_{k,l=0}^n(a_k'a_l' \langle\cos(kx), \cos(lx)\rangle +
		2a_k'b_l'\langle\cos(kx),\sin(lx)\rangle +
		b_k'b_l'\langle\sin(kx),\sin(lx)\rangle) \\
		\label{bess-tvrz}
		&= \sum_{k=0}^n\left((a_k')^2\langle\cos(kx),\cos(kx)\rangle +
		(b_k')^2\langle\sin(kx),\sin(kx)\rangle\right) \\
		\label{bess-dos}
		&= \pi \left( \frac{a_0^2}{2} + \sum_{k=1}^n(a_k^2 + b_k^2) \right)
	\end{align}
	Nyní dosadíme výsledky (\ref{bess-fs}), (\ref{bess-dos}) do
	(\ref{bess-neq}):
	\begin{align}
		\pi \left( \frac{a_0^2}{2} + \sum_{k=1}^n(a_k^2 + b_k^2) \right) \le
		\langle f, f \rangle
	\end{align}
	Tedy číslo $\langle f, f\rangle / \pi$ je horním odhadem všech částečných
	součtů, tedy řada nutně konverguje.
	
	\item Z konvergence řady čtverců je vidět, že nutně $\lim (a_n^2 + b_n^2) =
	0$, tedy také $\lim a_n = \lim b_n = 0$.
\end{enumerate}


\subsection{Po částech hladká funkce}
\label{po-castech-hladka-funkce}
\setcounter{equation}{0}
\paragraph{Definice}
Funkce $f:[a,b] \to \R$ je na intervalu $[a,b]$ po \textit{částech hladká}, když
existuje konečná množina $A \supset [a,b]$ taková, že $f$ má na množině $[a,b]
\setminus A$ spojitou první derivaci $f'$ a v každém bodu $a \in A$ má $f$ i
její derivace $f'$ vlastní jednostranné limity. Ty budeme značit $f(a+0)$ a
$f(a-0)$.
Jinak řečeno: $f$ je na $[a,b]$ po částech hladká, když existuje dělení
\begin{align}
	a = a_0 < a_1 < ... < a_k = b
\end{align}
intervalu $[a,b]$ takové, že $f$ má na každém intervalu $(a_i, a_{i+1})$
spojitou první derivaci (sama $f$ je tedy také spojitá) a v dělících bodech
existují vlastní jednostranné limity funkce i derivace.


\subsection{Dirichletova věta o bodové konvergenci Fourierovy řady}
\label{dirichlet}
\setcounter{equation}{0}
\paragraph{Věta}
Nechť funkce $f: \R \to \R$ je $2\pi$-periodická a její zúžení na interval
$[-\pi,\pi]$ je po částech hladké. Její Fourierova řada pak na $\R$ bodově
konverguje k funkci (pro představu -- aritmetický průměr):
\begin{align*}
	\frac{f(x+0) + f(x-0)}{2} \text{, kde~} f'(a_i \pm 0) = \lim_{x \to a_i^\pm} f'(x)
\end{align*}
V každém bodu spojitosti $x$ funkce $f$ její fourierova řada konverguje k číslu
$f(x)$.
\paragraph{Lemma}
o Dirichletově jádře: Nechť $n \in \N$ a 
\begin{align*}
	J_n(x) := \frac{1}{2} + \cos(x) + \cos(2x) + ... + \cos(nx)
\end{align*}
Pak pro každé $x\in\R$, $x \neq 2k\pi$, kde $k \in \Z$:
\begin{align*}
	J_n(x) = \frac{\sin((n+\frac{1}{2})x)}{2\sin(
	\frac{x}{2})}
\end{align*}
a také:
\begin{align*}
	\frac{1}{\pi} \int_{-\pi}^0 J_n(x) \dx = \frac{1}{\pi}\int_0^\pi J_n(x)\dx =
	\frac{1}{2}
\end{align*}
\paragraph{Důkaz lemma}
(bez důkazu)
\paragraph{Důkaz věty}
Nechť $x \in \R$ je pevné. Potom funkci $G: [-\pi, \pi] \to \R$ definujeme jako
$G(0) = 0$ a jinak předpisem:
\begin{align}
	G(u) = \begin{cases}
		\frac{f(x+u)-f(x-0)}{2\sin(\frac{u}{2})} \text{ pro } u \in [-\pi, 0) \\
		\frac{f(x+u)-f(x+0)}{2\sin(\frac{u}{2})} \text{ pro } u \in (0, \pi]
	\end{cases}
\end{align}
Podle l'Hospitalova pravidla spočítáme jednostranné limity:
\begin{align}
	\lim_{u \to 0^-} G(u) &= \lim_{u\to0^-} \frac{f'(x+u)}{\cos(\frac{u}{2})} =
	f'(x-0) \\
	\lim_{u \to 0^+} G(u) &=	f'(x+0)
\end{align}
Nyní ukážeme, že $G(u)$ je omezená funkce: na okolí nuly jsme spočítali vlastní
limity, jinak se jmenovatel ($2\sin(u/2)$) nepřibližuje nule a čitatel je
omezená funkce ($f$ je omezená). Navíc víme, že funkce $G(u)$ má konečně mnoho
bodů nespojitosti (body přechodu z definice po částech hladké funkce, viz
předpoklady, případně nula). Tedy $G(u)$ je podle Lebesgueova kritéria
riemannovsky integrovatelná na intervalu $[a,b]$.\\
Nechť $s_n = s_n(x)$ je $n$-tý částečný součet Fourierovy řady funkce $f$ v daném
bodu $x$:
\begin{align}
	s_n = \frac{a_0}{2} + \sum_{k=1}^n (a_k \cos(kx) + b_k \sin(kx)) \\
	\text{kde } \quad 
		a_k = \frac{\langle f(t), \cos(kt)\rangle}{\pi}, \quad
		b_k = \frac{\langle f(t), \sin(kt)\rangle}{\pi}
\end{align}
Rozdíl $s_n(x)$ a $(f(x+0)+f(x-0))/2$ vyjádříme pomocí funkce $G(u)$. Začneme
vyjádřením $s_n(x)$. Do původního vzorce pro $s_n$ dosadíme $a_k$ a $b_k$ (i za
$a_0$), pro zjednodušení všude vytkneme ($1/\pi$):
\begin{align}
	\frac{1}{\pi} \left( \frac{1}{2} \langle f(t), \cos(0t)\rangle +
	\sum_{k=0}^n \langle \sin(kt) \sin(kx) + \cos(kt)\cos(kx), f(t) \rangle
	\right)
\end{align}
Nyní podle $\langle u, v \rangle + \langle u, w \rangle = \langle u, v + w
\rangle$ sečteme a zjednodušíme:
\begin{align}
	\frac{1}{\pi} \left\langle f(t), \frac{1}{2} + \sum_{k=1}^n ( \cos(kt) \cos(kx) + \sin(kt)
	\sin(kx)) \right\rangle
\end{align}
Použijeme součtový vzorec $\cos(\alpha-\beta) = \cos\alpha \cos\beta +
\sin\alpha\sin\beta$:
\begin{align}
	\frac{1}{\pi} \left \langle f(t), \frac{1}{2} \sum_{k=1}^n \cos(k(t-x))
	\right\rangle
\end{align}
Dosadíme za $t := x + u$ - pozor, nyní se skalární součin integuje podle u!
\begin{align}
	\frac{1}{\pi} \left\langle f(x+u), \frac{1}{2} \sum_{k=1}^n \cos(ku)
	\right\rangle
\end{align}
Kde ale pravá strana skalárního součinu je Dirichletovo integrační jádro, tedy
dosadíme a popíšeme pomocí integrálů:
\begin{align}
	\frac{1}{\pi} \left\langle f(x+u), J_n(u)\right\rangle \\
	= \frac{1}{\pi} \left( \int_{-\pi}^0 f(x+u) J_n(u) \du + \int_0^\pi f(x+u)
	J_n(u)\du \right)
\end{align}
Nyní podle lemmatu přepíšeme: protože $x$ je konstantní, a integrál x
Dirichletova jádra je $1/2$:
\begin{align}
	\frac{f(x-0)}{2} = \frac{1}{\pi} \int_{-\pi}^0 f(x-0) J_n(u) \du
	\text{ a }
	\frac{f(x+0)}{2} = \frac{1}{\pi} \int_{0}^\pi f(x+0) J_n(u) \du
\end{align}
A ukážeme, že rozdíl
\begin{align}
	s_n(x) - \frac{f(x+0)+f(x-0)}{2}
\end{align}
se blíží nule. Vyjádříme tedy pomocí předchozích výpočtů a sloučíme integrály:
\begin{align}
	= \frac{1}{\pi} \left (
		\int_{-\pi}^0 (f(x+u) - f(x-0))J_n(u) \du + \int_0^\pi ( f(x+u) -
		f(x+0))J_n(u) \du
	\right)
\end{align}
Podle lemmatu dosadíme za $J_n$ a jmenovatel zapíšeme jako součást funkce $G(u)$
podle definice:
\begin{align}
	&= \frac{1}{\pi}\left( \int_{-\pi}^0 G(u) \sin((n+1/2)u)\du +
		\int_0^\pi G(u)\sin((n+1/2)u)\du \right) \\
	&= \frac{1}{\pi} \int_{-\pi}^\pi G(u) \sin((n+1/2)u) \du
\end{align}
Což podle definice skalárního součinu:
\begin{align}
	s_n(x) - \frac{f(x+0) + f(x-0)}{2} = \frac{\langle G(u),
	\sin((n+\frac{1}{2})u)\rangle}{\pi}	
\end{align}
Nyní už jenom skalární součin rozložíme podle součtového vzorce
$\sin(\alpha+\beta) = \sin\alpha\cos\beta + \cos\alpha\sin\beta$:
\begin{align}
	= \frac{1}{\pi}\left( \langle G(u)\cos(u/2), \sin(nu)\rangle + 
		\langle G(u) \sin(u/2), \cos(nu)\rangle \right)
\end{align}
Kde podle R-L lemmatu (\ref{rl-lemma}) oba skalární součiny konvergují k $0$ a
věta je dokázána.




\subsection{Věta o stejnoměrné konvergenci Fourierovy řady}
\setcounter{equation}{0}
\paragraph{Věta}
Nechť funkce $f : \R \to \R$ je $2\pi$-periodická a její zúžení na interval
$[\pi,-\pi]$ je po částech hladké. Nechť je navíc $f$ spojitá na $\R$. Pak je f
na $\R$ stejnoměrným součtem své Fourierovy řady.
\paragraph{Důkaz}
(bez důkazu)





\section{Úvod do komplexní analýzy}
\setcounter{equation}{0}

\subsection{Tvrzení o rozkladu přirozených čísel na disjunktní množiny}
\setcounter{equation}{0}
\paragraph{Tvrzení}
Množinu přirozených čísel nelze rozložit na alespoň 2 vzájemě disjunktní aritmetické
posloupnosti s unikátními diferencemi.
\paragraph{Důkaz}
Budeme předpokládat, že máme rozložení na $k$ aritmetických poslupností a
diference jsou různé. Nakonec odvodíme spor. Zaveďme si tedy jednotlivé
posloupnosti $A_i = \{a_i + d_in | n \in \N \}$, kde $a_i$ a $d_i$ jsou první
členy posloupnosti a diference. Vlastnost, že posloupnosti jsou disjunktní
můžeme také vyjádřit pomocí geometrických řad, kde $z \in (-1, 1)$:
\begin{align}
	\sum_{n\in\N} z^n = \sum_{n\in A_1} z^n + ... + \sum_{n\in A_k} z^n
\end{align}
Takové řady jistě konvergují. Přepíšeme tedy pomocí indexů místo množin:
\begin{align}
	z\sum_{n=0}^\infty z^n = z^{a_1} \sum_{n=0}^\infty z^{nd_1} + ... + z^{a_k}
	\sum_{n=0}^\infty z^{nd_k}
\end{align}
A podle vzorce pro součet geometrické řady můžeme sečíst:
\begin{align}
	\frac{z}{1-z} = \frac{z^{a_1}}{1-z^{d_1}} + ... + \frac{z^{a_k}}{1-z^{d_k}} 
\end{align}
Tyto rovnosti nyní převedem do oboru komplexních čísel. Je totiž zřejmé, že
platí i pro libovolné $z  \in \C$, kde $|z| < 1$. Na pomoc si vezmeme také $d_k$-tou
primitivní odmocninu z $1$:
\begin{align}
	\alpha = \cos(2\pi/d_k) + i\sin(2\pi/d_k)
\end{align}
A pro $n\to\infty$ zvolíme posloupnost $z_n$ tak, aby konvergovala k $\alpha$
(což můžeme, protože $\alpha$ leží těsně na hranici kruhu $|z| < 1$) a položíme
$z := z_n$. Jak vidíme, pro dostatečně velké $n$ jde poslední člen pravé rovnosti do nekonečna
($1-\alpha^{d_k}$ se blíží 0 zprava). Upravíme si tedy rovnost pro spor:
\begin{align}
	\frac{z_n^{a_k}}{1-z_n^{d_k}} = \frac{z_n}{1-z_n} - \frac{z_n^{a_1}}{1-z_n^{d_1}} - ...
	- \frac{z_n^{a_{k-1}}}{1-z_n^{d_{k-1}}}
\end{align}
Použijeme absolutní hodnotu a trojúhelníkovou nerovnost:
\begin{align}
	\left| \frac{z_n^{a_k}}{1-z_n^{d_k}} \right| \le
	\left|\frac{z_n}{1-z_n}\right| + \sum_{i=1}^{k-1}
	\left|\frac{z_n^{a_i}}{1-z_n^{d_i}}\right|
\end{align}
Je vidět, že pravá strana pro $n\to\infty$ konverguje ke konečné hodnotě.
Zároveň však víme, že levá strana pro dostatečně velké $n$ konverguje k
$\infty$, což je spor s platností nerovnosti.


\subsection{Holomorfní funkce}
\label{holomorfni-funkce}
\setcounter{equation}{0}
\paragraph{Definice}
Nechť $X \subset \C$ je otevřená množina, $z_0 \in X$ a $f: X \to \C$. Má-li $f$
v $z_0$ derivaci řekneme, že je funkce $f$ \textbf{holomofní} v bodě $z_0$.
Má-li $f$ derivaci v každém bodě množiny $X$, řekneme, že $f$ je holomorfní na množině $X$.
Funkce je \textbf{celá celistvá}, pokud je holomorfní na celém $\C$


\subsection{Komplexní exponenciála a její vlastnosti}
\label{komplexni-exponenciala}
\setcounter{equation}{0}
\paragraph{Definice}
Komplexní exponenciála je definována jako řada:
\begin{align*}
	e^z = \exp(z) := \sum_{n=0}^\infty \frac{z^n}{n!}
\end{align*}
\paragraph{Věta} Funkce $\exp(z)$ má vlastnosti:
\begin{enumerate}
	\item $\forall z \in \C: \quad \exp(z)' = \exp(z)$
	\item $\forall z_1, z_2 \in \C: \quad \exp(z_1 + z_2) = \exp(z_1)\exp(z_2)$
	\item $\forall z \in \C: \quad \exp(z) \neq 0$
	\item $\forall a \in \R: \quad \exp(ai) = \cis(a) = \cos(a) + i\sin(a)$
	\item $\forall z \in \C: \quad |\exp(z)| = \exp(\Re(z))$
	\item $\forall u \in \C, u \neq 0: \quad \exp(z) = u$ má nekonečně mnoho
	řešení lišící se navzájem o násobky $2\pi i$.
\end{enumerate}


\subsection{Analytická funkce}
\label{analyticka-funkce}
\setcounter{equation}{0}
\paragraph{Definice}
Nechť $z_0 \in X \subset \C$, kde $X$ je otevřená a $f: X \to \C$. Pokud
existuje $r > 0$, že $D(z_0, r) \subset X$ a $f$ se dá na disku $D(z_0,r)$
vyjádřit mocninnou řadou se středem v $z_o$ řekneme, že $f$ je \textbf{analytická v okolí
bodu} $z_0$. Když je analytická v každém bodu množiny $X$, je analytická na
množině $X$. Funkce je globálně analytická když je analytická v každém $z_0 \in
X$ a pro každé $r > 0$.


\subsection{Ekvivalence analytičnosti a holomorfismu}
\label{ekvivalence-ah}
\setcounter{equation}{0}
\paragraph{Věta}
Nechť $X \subset \C$ je otevřená množina a $f: X \to \C$. Pak následující
tvrzení jsou ekvivalentní:
\begin{enumerate}
	\item Funkce $f$ je na $X$ holomorfní.
	\item Funkce $f$ je na $X$ analytická.
	\item Funkce $f$ je na $X$ globálně analytická.
\end{enumerate}
\paragraph{Důkaz}
(bez důkazu)


\subsection{Jednoznačnost koeficientů mocninné řady}
\setcounter{equation}{0}
\paragraph{Věta}
Nechť $M = \sum a_nz^n$ a $N = \sum b_n z^n$ jsou mocninné řady a $(z_n)$ prostá
posloupnost komplexních čísel konvergující k 0, která leží v disku konvergence
$M$ i $N$. Pokud $\forall n$: $M(z_n) = N(z_n)$, potom také $a_n = b_n$.
\paragraph{Důkaz}
Můžeme předpokládat, že $z_k \neq 0$. Začneme tedy s prvním členem v $0$:
\begin{align}
	a_0 = M(0) = \lim_{k\to\infty} M(z_k) = \lim_{k\to\infty} N(z_k) = N(0) =
	b_0
\end{align}
Tedy $a_0 = b_0$ a dále dokážeme matematickou indukcí: Nechť jsme již dokázali
rovnost všechn členů po $m-1$ (včetně). Vezmeme funkce:
\begin{align}
	A(z) = \frac{M(z) - \sum_{n=0}^{m-1} a_nz^n}{z^m} \\
	B(z) = \frac{N(z) - \sum_{n=0}^{m-1} a_nz^n}{z^m}
		= \frac{N(z) - \sum_{n=0}^{m-1} b_nz^n}{z^m}
\end{align}
Tyto funkce jsou obě definovány na prstencovém okolí $0$ a $A(z_k) = B(z_k)$ pro
každé $k$. Dále se $A(z)$ na tomto okolí shoduje s funkcí danou mocninnou řadou
$\sum_{n\ge 0} a_{m+n} z^n$, která je spojitá a v $0$ má hodnotu $a_m$.
Analogicky $B(z)$ se shoduje s funkcí danou mocninnou řadou $\sum_{n\ge 0}
b_{m+n} z^n$ a v $0$ má hodnotu $b_m$. Proto platí:
\begin{align}
	a_m = \lim_{z\to0} A(z) = \lim_{k\to\infty} A(z_k) = \lim_{k\to\infty}
	B(z_k) = \lim_{z\to\infty} B(z) = b_m
\end{align}
A krok je dokázán.

\subsection{Holomorfní rozšíření a singularity}
\label{holomorf-roz-a-sing}
\setcounter{equation}{0}
\paragraph{Definice}
Nechť $X \subset Y \subset \C$ jsou dvě otevřené množiny a $f: X \to\C$, $g: Y
\to\C$ holomorfní funkce splňující $f(x) = g(x)$ pro každé $x\in X$. 
Potom je $g$ \textbf{holomorfní rozšíření funkce $f$ na množinu $Y$}.
\paragraph{Definice}
Nechť $f(z) = \sum_{n\ge 0} a_nz^n$ má poloměr konvergence $0 < R <
+\infty$, takže $f: D(0, R) \to \C$ je holomorfní funkce. Řekneme, že bod $u \in
\C$ na konvergenční kružnici (tj. $|u| = R$) je \textbf{singularita funkce $f$},
když neexistuje holomorfní rozšíření $f$ na žádném jeho okolí.



\subsection{Věta o jednoznačnosti holomorfního rozšíření}
\label{veta-o-holomorf-roz}
\setcounter{equation}{0}
\paragraph{Věta}
Holomorfní rozšíření na otevřenou a souvislou množinu je jednoznačné.
\paragraph{Důkaz}
(bez důkazu)



\subsection{Věta o singularitách}
\label{veta-o-singularitach}
\setcounter{equation}{0}
\paragraph{Věta}
Nechť mocninná řada $f(z) = \sum_{n\ge 0} a_nz^n$ má poloměr konvergence $R$
splňující $0 < R < +\infty$.
\begin{enumerate}
	\item Alespoň jeden bod $u\in\C$ s $|u| = R$ je singularitou funkce $f$.
	\item (Pringsheimova věta) Pokud jsou koeficienty $a_n$ reálné a nezáporné,
	je bod $u = R$ singularitou funkce $f$.
\end{enumerate}
\paragraph{Důkaz}
(bez důkazu)






























\newpage
\section{Dodatek A: Požadavky ke zkoušce}
\setcounter{equation}{0}
Požadavky ke zkoušce u Martina Klazara ze zimního semestru roku 2009/2010.
\subsection{Základní pojmy a definice}
\setcounter{equation}{0}

\begin{enumerate}
\item Definujte metrický prostor, otevřené a uzavřené množiny, 
hraniční bod množiny.
\item Definujte limitní bod množiny, izolovaný bod množiny, uzávěr množiny. 
\item Definujte spojité zobrazení mezi metrickými prostory a homeomorfismus. 
\item Podejte obě definice kompaktního metrického prostoru, resp. kompaktní
množiny v metrickém prostoru.
\item Definujte úplný metrický prostor a kontrahující zobrazení mezi metrickými 
prostory.
\item Vysvětlete typy konvergence posloupností a řad funkcí.
\item Definujte mocninnou řadu a poloměr konvergence (v reálném oboru). 
\item Definujte trigonometrickou řadu, Fourierovy koeficienty funkce a Fourierovu 
řadu funkce. 
\item Vysvětlete pojem po částech hladké funkce (\ref{po-castech-hladka-funkce})
\item Definujte holomorfní funkci a analytickou funkci (\ref{holomorfni-funkce},
\ref{analyticka-funkce})
\item Definujte pojem holomorfního rozšíření a singularity
(\ref{holomorf-roz-a-sing})
\end{enumerate}

\subsection{Věty a důsledky bez důkazu}
\setcounter{equation}{0}
\begin{enumerate}
\item Uveďte vlastnosti otevřených a uzavřených množin v metrickém prostoru 
a topologickou charakterizaci spojitosti zobrazení mezi metrickými 
prostory (T.1.1-T.1.3).
\item Uveďte výsledky o  kompaktních množinách v metr. prostoru (T.1.4-V.1.8).
\item Uveďte výsledky o  úplných metr. prostorech (T.1.9 a V.1.10).
\item Uveďte kritéria stejnoměrné konvergence posloupností a řad funkcí 
(T. 2.1-2.2, V. 2.7-2.8). 
\item Uveďte věty o záměně pořadí operace limity s dalšími operacemi pro 
posloupnosti a řady funkcí (V. 2.3-2.5 a jejich verze pro řady).
\item Uveďte výsledky o mocninných řadách (\ref{polomer-konvergence},
		\ref{lokalne-stejnomerna-konvergence-mr}, \ref{abelova-veta}) 
		(orig: V. 2.9, T. 2.10, V. 2.11).
\item Uveďte výsledky o Fourierových řadách (T. 2.12, V. 2.13-2.15).
\item Uveďte vlastnosti komplexní exponenciály (\ref{komplexni-exponenciala})
\item Uveďte hlavní výsledky o holomorfních funkcích (\ref{ekvivalence-ah}, 7 a 8).
\end{enumerate}

\subsection{Věty s důkazy}
\setcounter{equation}{0}
\begin{enumerate}
\item Uveďte a dokažte výsledky o otevřených a uzavřených množinách v 
metr. prostoru. (T.1.1 a T.1.2).
\item Uveďte a dokažte topologickou charakterizaci spojitosti zobrazení 
mezi metr. prostory (T.1.3).
\item Dokažte, že kompaktní množiny v metr. prostoru jsou uzavřené a 
omezené (T.1.5).
\item Uveďte a dokažte vlastnosti spojitých funkcí na kompaktních 
metr. prostorech (V.1.7).
\item Uveďte a dokažte Banachovu větu o pevném bodu (V.1.10). 
\item Dokažte Bolzanovu-Cauchyovu podmínku pro posloupnosti funkcí (T. 2.1).
\item Dokažte Mooreovu-Osgoodovu větu (V 2.3). 
\item Dokažte větu o záměně pořadí limity a integrování (V. 2.4).
\item Dokažte Weierstrassovo kritérium stejnoměrné konvergence (1. část V. 2.7).
\item Dokažte vzorec pro poloměr konvergence mocninné řady (V 2.9).  
\item Dokažte, že mocninná řada konverguje na intervalu konvergence 
lokálně stejnoměrně (T. 2.10).
\item Dokažte Besselovu nerovnost a Riemannovo-Lebesgueovo lemma
(\ref{rl-lemma}) 
\item Dokažte větu o bodové konvergenci Fourierovy řady po částech hladké funkce 
(\ref{dirichlet}, bez lemma). 
\item Dokažte, že množina přirozených čísel není sjednocením alespoň dvou 
disjunktních aritmetických posloupností s různými diferencemi (první Tvrzení
bez čísla).
\item Dokažte tvrzení o jednoznačnosti koeficientů mocninné řady (T. 6).
\end{enumerate}

\end{document}
