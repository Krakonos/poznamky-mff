\documentclass[a4paper,12pt,titlepage]{article}
\usepackage[utf8]{inputenc}
\usepackage{a4wide}
\usepackage[czech]{babel}
\usepackage{amsfonts, amsmath, amsthm, amssymb}
\usepackage{math}

\title{Základy spojité optimalizace}
\author{Ladislav Láska}

\begin{document}

\maketitle
\newpage
\tableofcontents
\newpage


\section{Úvod}
\setcounter{equation}{0}
\subsection{Úloha, cílová funkce, množina řešení}
\setcounter{equation}{0}
\paragraph{Definice}
Úloha matematického programování (optimalizace) rozumíme úlohu
\begin{align*}
	\min_{x\in M} f(x)
\end{align*}
kde $f: \R^n \to \R$.
\paragraph{Definice}
Funkci $f(x)$ nazýváme \textbf{cílovou}, účelovou, kriteriální, objektivní funkcí.
\paragraph{Definice}
Množinu $M$ nazýváme množinou přípustných řešení. Prvek $x \in M$ nazýváme
přípustným řešením optimalizační úlohy. Prvek $x_0 \in M$ nazveme optimálním
řešením.

\subsection{Dělení na jednotlivé disciplíny optimalizace}
\setcounter{equation}{0}
\begin{enumerate}
	\item Volný extrém - $\min_M f(x)$
	\item Vázaný extrém - $\min_M f(x), M \subset \R^n$
	\begin{enumerate}
		\item Lineární programování: \\
			$\min_M cx$, $M = \{ x | A_x
			\{=,\lt,\gt\}
			b\}$, kde $c \in \R^n$, $A \in \R^{m \times n}$, $b \in b^m$
		\item Nelineární programování: \\
			$\min_M f(x), M = \{ x | g_j(x) \lt 0 (j = 1, ..., m) \}, f, g_j:
			\R^n \to \R$
			\begin{enumerate}
				\item Konvexní a zobecněné konvexní programování - $f,g_j$
				konvexní, dále pak kvadratické a hyperbolické programování
				\item Nekonvexní (speciální typy)
			\end{enumerate}
		\item Celočíselné programování: \\
			Lineární/nelineární programování, navíc podmínky pro celočíselnost
			$Ax * b$, aby $x \in \N$.
		\item Parametrické programování: \\
			Lineární/nelineární programování, navíc parametr $\min_{M(U)} c
			(\lambda)^T x$, $c(x) = c + C\lambda$, $M(U) = \{ x | A(U) x * b(U)
			\}$
		\item Vícekriteriální (vektorové) programování: \\
			$\min_M f(x)$, $f(x) = \{f_1(x), .., f_s(x)\}$
		\item Dynamické programování - hledání optimální strategie
		\item Spojité programování (optimalizační procesy)
		\item Teorie her - optimální strategie dvou hráčů
		\item Semiinfinitní programování - nekonečně mnoho podmínek
	\end{enumerate}
\end{enumerate}

\subsection{Motivační úloha}
\setcounter{equation}{0}
\subsubsection{Lineární programování}
$V_1, ..., V_n$ - výrobci vyrábějící výrobek $V$ v množstvích $a_i \gt 0$ \\
$S_1, ..., S_k$ - spotřebitelé požadující výrobek $V$ v množstvích $b_j \gt 0$.\\
Známe cenu za dopravu jednotky výrobku $V$ z $V_i$ do $S_j$ - $c_{i,j} \ge 0$.
\paragraph{Předpoklad:} ceny za dopravu rostou lineárně.
\paragraph{Cíl:} minimalizovat celkové náklady na dopravu.
\paragraph{Hledáme:} množství $x_{i,j} \ge 0$ - kolik výrobce $V_i$ dodá $S_j$.
\paragraph{Cílová funkce}
\begin{align}
	f(x) = \sum_i \sum_j c_{i,j} x_{i,j}
\end{align}
na množině řešení
\begin{align}
	M = \{ x_{i,j} | \sum_{i=1}^m x_{i,j} = b_j \forall j, \sum_{j=1}^m x_{i,j}
	= a_i \forall i, \sum_i a_i = \sum b_j, x_{i,j} \ge 0 \forall i \forall j \}
\end{align}
Všechno je lineární - úloha lineárního programování.
\subsubsection{Celočíslené programování}
Pokud nelze položky libovolně dělit (například lidi), můžeme přidat celočíselnou
podmínku do množiny řešení:
\begin{align}
	x_{i,j} \N_0 \forall i \forall j
\end{align}
\subsubsection{Nelineární programování}
Zrušíme předpoklad lineárního růstu cen (tedy cena závisí na množství)
\begin{align}
	f(x) = \sum_i \sum_j c_{i,j}(x_{i,j}) x_{i,j}
\end{align}
$c_{i,j} = C_{i,j} + c_{i,j} x_{i,j}$
\subsubsection{Parametrické programování}
Produkce není pevná a závisí na parametru: $a_i = \lambda a_i'$. Ostatní vztahy
můžou zůstat třeba jako u lineárního programování.
\subsubsection{Vícekriteriální programování}
Minimalizovat ceny za dopravu, maximalizovat zisky Z.
\begin{align}
	\min \{f(x), g(x)\} \\
	f(x) = \sum_i \sum_j c_{i,j} x_{i,j} \\
	g(x) = - Z(x_{i,j}) 
\end{align}
\subsubsection{Dynamické programování}
Ceny závisí na rozhodnutí.







\end{document}

