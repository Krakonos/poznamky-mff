\documentclass[a4paper,12pt]{article}
\usepackage[utf8]{inputenc}
\usepackage{a4wide}
\usepackage[czech]{babel}
\usepackage{amsfonts, amsmath, amsthm, amssymb}
\usepackage[small,compact]{titlesec}
\usepackage[left=1cm,right=1cm,top=1cm,bottom=1.8cm]{geometry}
\usepackage{multirow}

\newcommand{\F}{\mathbb{F}}
\newcommand{\Z}{\mathbb{Z}}
\renewcommand{\div}{\ \ \mathbf{div}\ }
\newcommand{\wt}{\ \mathbf{wt}}

\title{Samoopravné kódy}
\author{Ladislav Láska}

\begin{document}

\maketitle
\thispagestyle{empty}
\begin{abstract}
Tento text seznámí čtenáře se základními vlastnostmi samoopravných kódů, jejich 
konstrukce a také ukáže část teorie za nimi.
\end{abstract}
\vskip2cm
\tableofcontents
\newpage

\parskip6pt

\section{Úvod}
Představme si, že chceme komunikovat po nějakém nespolehlivém médiu (třeba 
telefonu), nebo na nějaké ukládat data (pevný disk), ale rádi bychom o žádná 
data nepřišli. K takovému účelu nám dobře poslouží \textbf{samoopravné kódy}, 
tedy kódy, které v sobě nesou nějakou redundantní informaci, aby při malém 
poškození bylo možno stále dekódovat původní data.

V této kapitole nastíním některé triviální kódy, o kterých každý čtenář jistě 
slyšel. Nejdříve si však musíme říct, co je to vlastně takový kód a jak ho 
budeme zapisovat. 

\paragraph{Definice}
Data, která chceme poslat budeme nazývat \textbf{datová slova}. Data skutečně 
odeslaná kanálem jsou \textbf{kódová slova}.  \textbf{Parametry kódu} rozumíme 
dvojici $(n,k)$, kde $n$ je délky kódového slova, $k$ délka datového slova.
\paragraph{Konvence}
V dalším textu budeme předpokládat, že přenášíme slova v binárním zápise.

\subsection{Možné chyby v datech}
V datech se mohou dít chyby dvěma způsoby: vynulováním a změnou.  Vynulování je, 
že se každý bit s nějakou pravděpodobností vynuluje (tedy místo poslaných dat se 
objeví samé nuly - nebo jiné konstanty). Změnou (v binárním schématu 
překlopením) se rozumí chyba, kde se bit s nějakou pravděpodobností překlopí.  
Pokud bychom měli jednotlivé bity vícehodnotové (tedy $\F\neq\Z_2$), bylo by 
zapotřebí složitější definice.
\paragraph{Konvence}
V dalším textu budeme předpokládat, že chyby se dějí překlopením.

\subsection{Paritní kód}
Nejjednodušším kódem je kód \textbf{paritní}. Funguje následujícím způsobem: Ke 
kódu doplní jedničku nebo nulu tak, aby byl počet jedniček sudý. Je zřejmé, že 
takový kód není samoopravný, ale spíš samo-chybu-detekující. Umí detekovat právě 
sudý počet chyb, takže jeho použití je velmi limitováno. V dalších kapitolách se 
naučíme zkonstruovat efektivnější kódy s jenom o něco větší přidanou informací.

\subsection{Opakovací kód}
Druhý nejjednodušší kód je \textbf{opakovací}. Jednoduše přepošle datové slovo 
vícekrát za sebou (typicky $3\times$, aby šlo rozpoznat i opravit jednu chybu).  
To samozřejmě funguje docela dobře, nicméně je zcela neefektivní z hlediska 
velikosti přenášených dat.

\section{Lineární kódy}
Lineární kódy jsou takové, že lineární kombinace dvou kódových slova dá další 
kódové slovo.
\paragraph{Definice}
Lineární kód $(n,k)$ kód je podprostor vektorového prostoru $\F_q^n$ dimenze 
$k$.
\paragraph{Konvence}
Protože uvažujeme pouze binární kódy, dále předpokládáme $q=2$.

\subsection{Paritní a opakovací kódy jsou lineární}
\paragraph{Tvrzení}
Paritní a opakovací kódy jsou lineární kódy.
\paragraph{Důkaz}
Mějme nějaká kódová slova $u, v$. Ověříme, že jejich součet je také kódové 
slovo. Z toho okamžitě vyplyne, že i každá jiná lineární kombinace je kódové 
slovo a tedy kód je lineární. \textit{Důkaz provádíme za předpokladu, že 
$\F=\Z_2$. Platí to ale obecně pro jiná tělesa (v případě parity je potřeba kód 
vhodně předefinovat).}
\begin{enumerate}
	\item Paritní kód: Označme si $u_d, v_d$ datové části slova a $p, q$ jejich 
	parity. Je snadné nahlédnout, že počet jedniček ve výsledném slově $w_d$ 
	bude součet jedniček v obou slovech minus počet míst, kde jsou obě slova 
	rovna jedna. Je zřejmé, že pokud podle součtu jedniček v obou slovech by 
	byla parita $r = p + q$ (snadné ověřit z aritmetiky "sudá+lichá=lichá" atp).  
	Také nahlédneme, že pokud byly jedničky na stejném místě v $u$ i $v$, parita 
	se nezmění (jedničky byly v obou, teď je 0), tedy parita výsledného slova má 
	skutečně být $r$.  Výsledné slovo je tedy také kódové slovo.
	\item Opakovací kód: Triviálně, protože jednotlivé kopie slova se 
	neovlivňují a tedy se každá kopie sečte stejně.
\end{enumerate}
Máme tedy dva příklady lineárních kódů.

\subsection{Generující matice}
Z vlastností lineárních kódů si můžeme všimnout, že se každé kódové slovo dá 
vyjádřit jako lineární kombinace bázových slov. To samé platí i pro datová 
slova. Za cvičení si dokažte následující tvrzení:

\paragraph{Tvrzení}
Nechť $u$ je datové slov a $v$ je jemu odpovídající kódové slovo. Vyjádříme-li 
$u=a_1b_1 + ... + a_nb_n$, pak lze vyjádřit $v=a_1b_1' + ... + a_nb_n'$, kde 
$a_i$ jsou nějaké koeficienty a $b_i, b_i'$ jsou navzájem odpovídající si bázové 
vektory ve vektorovém prostoru datových, resp. kódových slov.
\paragraph{Důkaz} (cvičení)
\vskip12pt

Takovou operaci můžeme vyjádřit maticí přechodu mezi bázemi těchto vektorových 
prostorů. 

\paragraph{Definice}
Matici $G$ nazveme \textbf{generujicí} maticí $(n,k)$ kódu, právě tehdy, když 
pro každé datové slovo $u$ a jeho odpovídající kódové slovo $v$ platí: $v = 
u\cdot G$.  Generující matice je ve \textbf{standardním tvaru}, když ji lze psát 
jako $G=(I_k|M)$, kde $I_k$ je jednotková matice $k \times k$ a $M$ je libovolná 
matice $k \times n-k$.

\paragraph{Poznámka}
Je vidět, že definici lze "transponovat", a také se tak někdy používá.  Nenechte 
se proto zmást. Takto mi to ale přijde intuitivnější, o to víc při odvozování 
kódů z polynomů (bude později).

\subsection{Prověrková matice}
\paragraph{Definice}
Matici $H$ nazveme \textbf{prověrkovou} maticí $(n,k)$ kódu, právě tehdy, když 
pro každé kódové slovo $u$ a nekódové slovo $v$ platí: $\mathbf{0} = H \cdot u$ 
a $\mathbf{0} \neq H \cdot v^T$.
\paragraph{Tvrzení}
Pro nějakou generující matici $G$ tvaru $G = (I_k | M)$ řádu $n \times k$, je jí 
odpovídající prověrková matice je tvaru $H = (-M^T | I_{n-k})$.
\paragraph{Důkaz} \textit{(zatím bez důkazu)}

\subsection{Příklady konstrukce}
V následujících pár odstavcích si ukážeme, jak se jednoduše dají vytvořit 
generující matice pro jednoduché kódy z první kapitoly.

Stěžejním pozorováním je, že pokud sestavíme datovou bázi elementární, jednička 
na $i$-té pozici v datovém slově přičte do kódového slova $i$-tý řádek z matice.

\subsubsection{Generující matice parity}
Víme, že paritní matice upravuje počet jedniček na sudý. Chceme tedy, aby se 
elementární vektory báze $(e_i)$ zobrazily na $(e_i | 1)$. Ze znalosti násobení 
matic je tedy zjevné, že generující matice paritního (4,3) kódu je:
\begin{align}
	G = \left( \begin{matrix}
		1 & 0 & 0 & 1\\
		0 & 1 & 0 & 1\\
		0 & 0 & 1 & 1\\
	\end{matrix}
	\right)
\end{align}

\subsubsection{Generující matice opakování}
Podobnou metodou, jako jsme sestavili matici parity, můžeme sestavit matici 
opakování. Vektory elementární báze zobrazíme na 3 po sobě jdoucí, identické 
vektory: $(e_i) \to (e_i|e_i|e_i)$. Matice opakovacího (9,3) kódu bude vypadat:
\begin{align}
	G = \left( \begin{matrix}
		1 & 0 & 0 & 1 & 0 & 0 & 1 & 0 & 0 \\
		0 & 1 & 0 & 0 & 1 & 0 & 0 & 1 & 0 \\
		0 & 0 & 1 & 0 & 0 & 1 & 0 & 0 & 1 \\
	\end{matrix} \right)
\end{align}

\subsection{Hammingova vzdálenost}
Abychom mohli lépe měřit "výkonnost" kódových slov, zavedeme si pojem Hammingovy 
vzdálenosti a ukážeme jeho spojitost se schopnostmi korekce kódu.
\paragraph{Definice}
\textbf{Hammingova vzdálenost} $d$ dvou slov je rovna počtu bitů, ve kterých se 
slova liší.
\paragraph{Věta}
Nechť $C$ je lineární kód. Pak:
\begin{enumerate}
	\item Kód odhaluje $r$ a méně chyb, právě když je hammingova vzdálenost 
	každých dvou kódových slov rovna alespoň $r+1$.
	\item Kód opravuje $r$ a méně chyb, právě když je hammingova vzdálenost 
	každých dvou kódových slov rovna alespoň $2r+1$.
\end{enumerate}
\paragraph{Důkaz}
Tvrzení je zřejmé z faktu, že hammingova vzdálenost tvoří metriku nad prostorem 
kódu $C$, dokonce nad každým jeho nadprostorem $F$. Můžeme si tedy představit 
okolo kódových slov koule poloměru $r$ a chybná slova jako jejich prvky. Pak je 
zřejmé, že pokud žádné kódové slovo nepadne do koule jiného kódového slova, pro 
dané $r$ kód chybu odhalí. Pokud navíc koule namají průniky, lze chybu i 
opravit.

\subsection{Hammingův kód}
Jak vidíme, paritní kód je sice úsporný, ale nedosahuje příliš dobrých výsledků.  
Opakovací kód zase dosahuje dobrých výsledků, ale cena za jeho použití je příliš 
veliká. Představíme si tedy Hammingův kód, mimochodem první ze samoopravných 
kódu.
\subsubsection{Konstrukce hammingova kódu}
Nejdříve si ukážeme, jak takový kód zkonstruovat. Budeme konstruovat obecný 
$(2^n - 1, 2^n - n - 1)$ kód. Mějme tedy kód dlouhý $2^n-1$. Umístíme celkem $n$ 
paritních bitů na místa rovna mocninám dvojky. Každý bit na pozici $2^i-1$ bude 
pokrývat paritou $2^i$ dalších bitů (počínaje jím samým), stejný počet dalších 
bitů vynechá a opět stejný počet pokryje, což se opakuje až do konce slova.  
Názorně tabulka pro $(7,4)$-kód.
\vskip12pt
\begin{center}
\begin{tabular}{|l|c|c|c|c|c|c|c|}
\hline
	Pozice & 0 & 1 & 2 & 3 & 4 & 5 & 6 \\
	Význam & \textit{p0} & \textit{p1} & d0 & \textit{p3} & d1 & d2 & d3 \\
\hline
	& $\bullet$ & & $\bullet$ & & $\bullet$ & & $\bullet$\\
	Pokrývá paritou
	& & $\bullet$ & $\bullet$ & & & $\bullet$ & $\bullet$ \\
	& & & & $\bullet$ & $\bullet$ & $\bullet$ & $\bullet$ \\
	\hline
\end{tabular}
\end{center}
\vskip12pt
Je snadno vidět, že kromě paritních bitů jsou všechny bity pokryty dvoumi 
paritními bity.

\subsubsection{Minimální hammingova vzdálenost hammingova kódu}
\paragraph{Věta}
Hammingův kód má minimální hammingovu vzdálenost 3.
\paragraph{Důkaz}
Sporem: Mějme dvě kódová slova, která se liší v méně než třech bitech. Alespoň 
jeden z bitů musí nutně být datový, protože jinak by nebyla kódová. Pokud se 
slova liší v právě jednom datovém bitu, určitě se musí lišit i ve dvou 
paritních. Pokud se slova liší ve dvou datových bitech, určitě se liší alespoň v 
jednom dalším paritním (je vidět například z obrázku). Pokud se slova liší ve 
třech datových bitech, není co dokazovat.

\subsection{Minimální kódová vzdálenost a prověrková matice}
Nyní ukážeme, jak snadno zjistit minimální hammingovu vzdálenost podle 
prověrkové matice.
\paragraph{Definice}
\textbf{Hammingovu váhu} slova $u$, $\wt(u)$, definujeme jako hammingovu 
vzdálenost od nulového vektoru.
\paragraph{Věta}
Minimální hammingova vzdálenost kódu $d$ je rovna nejmenšímu počtu lineární 
závislých sloupců prověrkové matice $H$.
\paragraph{Důkaz}
Uvědomme si, že při násobení matice s nějakým přijatým vektorem $v$ jednotlivé 
bity ve vektoru volí celé sloupce z původní matice. Přitom víme, že pro nějaké 
korektní kódové slovo $c$ platí:
\begin{align}
	\mathbf{0} = H \cdot c^T = \sum_{i=1}^n H_i c_i^T
\end{align}
Kde $H_i$ je $i$-tý sloupec $H$. Budeme-li uvažovat pouze $c_i \neq 0$, zbylé 
sloupce představují lineárně závislou množinu a tudíž musí být minimální 
vzdálenost $d$ alespoň počet lineárně závislých sloupců.

Uvažujem dále nejmenší množinu lineárně závislých sloupců $\{H_j: j \in S\}$, 
kde $S$ je indexová množina pro sloupce. Potom můžeme psát:
\begin{align}
	\sum_{i=1}^n H_i c_i^T = \sum_{j\in S}^n H_j c_j^T + \sum_{j\notin S} H_j 
	c_j^T = \mathbf{0}
\end{align}
Nyní uvažujem vektor $c'$ takový, že $c_j' = 0 \Leftrightarrow j\notin S$.  
Takový je určitě kódovým slovem, protože výše zmíněná suma pro něj platí. Takže 
určitě $d \le \wt(c')$, což je ale nejmenší počet lineárně závislých sloupců $H$ 
a tvrzení platí.

\textit{TODO: Ověřit, trošku formálněji popsat.}

\subsection{Syndromové dekódování}
Zatím jsme se naučili kódy konstruovat. Také jsme si ukázali generující a 
prověrkovou matici. Nyní si ukážeme, jak efektivně dekódovat chybně přenesená 
slova (pro kód, který korekci umožňuje).

Obecně se nám bude hodit k dekódování tabulka syndromů a možných slov, který jim 
odpovídají. Tu sestavíme následovně: syndrom $\mathbf{0}$ má každé kódové slovo.  
Dále pro každý syndrom a kódové slovo najdeme slovo odpovídající danému syndromu 
od něj odvozené. To uděláme přičtením syndromu ke kódovému slovu (a doplníme 
nulami na stejnou velikost). 

\textit{TODO: Proč? Slovák si myslí, že je to pravda. Já nevím, proč by to 
pravda byla v obecném případě. Nejspíš je to tím, že takové slovo představuje 
například zbytek po dělení v polynomiálním kódu, ale neumím to dokázat.}

\textit{TODO: Příklad.}

Je snadné vypozrovat, že v odečtením libovolných dvou slov na řádku dává kódové 
slovo - důkaz nechám jako snadné cvičení čtenářovi. Každá jednička v těchto 
slovech tedy vyjadřuje, kolik bitů je potřeba změnit ve slově, aby se z něj 
stalo platné kódové slovo. Označme si tedy slova s nejmenším počtem jedniček v 
daném řádku a nazvěme je \textbf{vedoucími reprezentanty}. Při odečtení tohoto 
slova od přijatého slova dostáváme platné kódové slovo a to právě odpovídající 
vyslanému slovu, pokud nedošlo k příliš mnoha chybám.

\section{Polynomiální kódy}
Už při konstrukci hammingova kódu jsme narazili na problém, že popis nemusel být 
zrovna triviální. V dalších odstavcích ukáži, jak se dají samoopravné kódy 
generovat pomocí polynomů. 

\paragraph{Definice}
Nechť $p(x) = a_0 x^0 + ... + a_{n-k}x^{n-k} \in \Z_2[x]$ je polynom a $a_0 \neq 
0 \neq a_{n-k}$. Pak \textbf{polynomiální kód generovaný polynomem} $p(x)$ je 
kód, jehož slova jsou polynomy stupně menšího než $n$ a bezezbytku dělitelné 
$p(x)$.

Zatím zahoďmě stranou, jak dobré takové kódování bude. Podívejme se, jak 
zkonstruujeme kódové slovo z datového. Nejdříve je potřeba si uvědomit, že každé 
datové slovo délky $k$ můžeme zapsat jako polynom stupně maximálně $k$, tedy 
bity slova jsou jeho koeficienty. Takový polynom většinou nebude dělitelný 
$p(x)$, takže ho jím musíme pronásobit. Kódové slovo pro datové slovo $m(x)$ 
tedy bude polynom $v'(x) = p(x)m(x)$. Všimneme si ale, že to má jistou nevýhodu.  
V takovém kódovém slově není přímo obsažena zpráva. Zprávu tedy zakódujeme 
jinak: $v(x) = x^{n-k}m(x) + r(x)$, kde $r(x) := x^{n-k}m(x) \div p(x)$, aby byl 
náš polynom dělitelný. Také lze snadno vypozorovat, že zbytek nebude mít řád 
větší než $n-k$, tedy nižší koeficienty budou představovat "paritu", vyšší 
koeficienty potom samotnou zprávu.

\subsection{Linearita polynomiálních kódů}
\paragraph{Tvrzení}
Každý polynomiální kód je lineární.
\paragraph{Důkaz}
Mějme polynom $p(x)$ generující nějaký kód $C$ a jeho dvě kódová slova $v(x), 
w(x)$. Je zřejmé, že libovolná lineární kombinace $a \cdot v(x) + b\cdot w(x)$ 
je dělitelná $p(x)$, pokud $v(x)$ i $w(x)$ jsou dělitelné $p(x)$, což z 
předpokladu jsou a kód je tedy lineární.

\subsection{Dekódování}
Dekódování je velmi jednoduché, prakticky je to otázka dělení polynomů. Korektně 
přenesená data se poznají tak, že jejich zbytek po dělení příslušným polynomem 
je $0$. Nekorektní data lze dekódovat například syndromovým dekódováním (protože 
každý polynomiální kód je lineární).

\subsection{Od polynomu k maticím}
Protože jsme si ukázali, že polynomiální kódy jsou lineární, musí také pro každý 
polynomiální kód existovat generující a prověrková matice. Jak ji ale zjistit?

Půjdeme na to od lesa. Protože víme, že jsou lineární kódy lineární (uzavřené na 
sčítání), stačí nám zjistit, kam se zobrazí bázové vektory. Těmi jsou $a_ix^i$.  
Nechť tedy podle $v_i$ je sloupcový vektor, na který se zobrazila báze $a_ix^i$, 
tedy po vynásobení $p(x)$. Generující matice je pak sestavena spojením těchto 
vektorů. Prověrkovou matici lze sestavit podle tvrzení o vztahu prověrkové a 
generující matice.

Pozor ale na to, že pokud chceme mít matici ve standardním tvaru, je potřeba 
nejen násobit $p(x)$, ale provést fintu se zbytkem, jako jsme udělali při 
zavádění kódu.

\subsection{Polynom pro paritu}
\paragraph{Pozorování}
Polynom $p(x) = 1 + x$ generuje $(n, n-1)$ kód parity.
\paragraph{Důkaz}
Jak to zjistíme? Buď si odvodíme pravidlo, pro dělitelnost polynomem (což v 
tomto případě lze - polynom $q(x)$ je dělitelný $1 + x$ právě tehdy, když $q(1) 
= 0$), nebo si můžeme zkusit postavit generující matici ve standardním tvaru.  
Podíváme se tedy na generující matici pro $(4,3)$ kód. Musíme tedy zobrazit 
bázové vektory $1, x, x^2$ pomocí našeho polynomu. Také provedeme zmíněnou fintu 
a vynásobíme si bázi $x$, takže parita bude koeficient $a_0$. K tomu je 
zapotřebí vypočítat zbytky po dělení p(x):
\begin{align}
	x &= 1 \mod x + 1\\
	x^2 &= 1 \mod x + 1\\
	x^3 &= 1 \mod x + 1
\end{align}
\textit{Při dělení na papíře vyjde -1. Nezapomínejte na to, že pracujeme v 
$\Z_2$.} Vektory tedy zobrazíme takto (tedy vynásobení $x$ a přičtení zbytku):
\begin{align}
	1 &\sim x + 1 = a(x)\\
	x &\sim x^2 + 1 = b(x)\\
	x^2 &\sim x^3 + 1 = c(x)\\
\end{align}
A přepíšeme do matice. Je samozřejmé, že můžeme sloupce zpřeházet. Většinou jsme 
zvyklí psát paritu na konec (naše schéma jí má jako konstantní člen, takže by 
byla na začátku). U takto jednoduchého kódu ji tedy můžeme přehodit na konec:
\begin{align}
	G = \left(
	\begin{matrix}
	a_1 & a_2 & a_3 & a_0 \\
	b_1 & b_2 & b_3 & b_0 \\
	c_1 & c_2 & c_3 & c_0 \\
	\end{matrix}
	\right) =
	\left(
	\begin{matrix}
	1 & 0 & 0 & 1 \\
	0 & 1 & 0 & 1 \\
	0 & 0 & 1 & 1 \end{matrix}
	\right)
\end{align}
Jak si můžeme všimnout, generující matice je stejná, jako matice paritního kódu 
na začátku textu.



\end{document}
