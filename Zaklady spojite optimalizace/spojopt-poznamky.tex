\documentclass[a4paper,12pt,titlepage]{article}
\usepackage[utf8]{inputenc}
\usepackage{a4wide}
\usepackage[czech]{babel}
\usepackage{amsfonts, amsmath, amsthm, amssymb}
\usepackage{math}

\title{Základy spojité optimalizace}
\author{Ladislav Láska}

\begin{document}

\maketitle
\newpage
\tableofcontents
\newpage


\section{Úvod}
\setcounter{equation}{0}
\subsection{Úloha, cílová funkce, množina řešení}
\setcounter{equation}{0}
\paragraph{Definice}
Úloha matematického programování (optimalizace) rozumíme úlohu
\begin{align*}
	\min_{x\in M} f(x)
\end{align*}
kde $f: \R^n \to \R$.
\paragraph{Definice}
Funkci $f(x)$ nazýváme \textbf{cílovou}, účelovou, kriteriální, objektivní funkcí.
\paragraph{Definice}
Množinu $M$ nazýváme množinou přípustných řešení. Prvek $x \in M$ nazýváme
přípustným řešením optimalizační úlohy. Prvek $x_0 \in M$ nazveme optimálním
řešením.

\subsection{Dělení na jednotlivé disciplíny optimalizace}
\setcounter{equation}{0}
\begin{enumerate}
	\item Volný extrém - $\min_M f(x)$
	\item Vázaný extrém - $\min_M f(x), M \subset \R^n$
	\begin{enumerate}
		\item Lineární programování: \\
			$\min_M cx$, $M = \{ x | A_x
			\{=,\le,\ge\}
			b\}$, kde $c \in \R^n$, $A \in \R^{m \times n}$, $b \in b^m$
		\item Nelineární programování: \\
			$\min_M f(x), M = \{ x | g_j(x) \lt 0 (j = 1, ..., m) \}, f, g_j:
			\R^n \to \R$
			\begin{enumerate}
				\item Konvexní a zobecněné konvexní programování - $f,g_j$
				konvexní, dále pak kvadratické a hyperbolické programování
				\item Nekonvexní (speciální typy)
			\end{enumerate}
		\item Celočíselné programování: \\
			Lineární/nelineární programování, navíc podmínky pro celočíselnost
			$Ax * b$, aby $x \in \N$.
		\item Parametrické programování: \\
			Lineární/nelineární programování, navíc parametr $\min_{M(U)} c
			(\lambda)^T x$, $c(x) = c + C\lambda$, $M(U) = \{ x | A(U) x * b(U)
			\}$
		\item Vícekriteriální (vektorové) programování: \\
			$\min_M f(x)$, $f(x) = \{f_1(x), .., f_s(x)\}$
		\item Dynamické programování - hledání optimální strategie
		\item Spojité programování (optimalizační procesy)
		\item Teorie her - optimální strategie dvou hráčů
		\item Semidefinitní programování - nekonečně mnoho podmínek
	\end{enumerate}
\end{enumerate}

\subsection{Motivační úloha}
\setcounter{equation}{0}
\subsubsection{Lineární programování}
$V_1, ..., V_n$ - výrobci vyrábějící výrobek $V$ v množstvích $a_i \gt 0$ \\
$S_1, ..., S_k$ - spotřebitelé požadující výrobek $V$ v množstvích $b_j \gt 0$.\\
Známe cenu za dopravu jednotky výrobku $V$ z $V_i$ do $S_j$ - $c_{i,j} \ge 0$.
\paragraph{Předpoklad:} ceny za dopravu rostou lineárně.
\paragraph{Cíl:} minimalizovat celkové náklady na dopravu.
\paragraph{Hledáme:} množství $x_{i,j} \ge 0$ - kolik výrobce $V_i$ dodá $S_j$.
\paragraph{Cílová funkce}
\begin{align}
	f(x) = \sum_i \sum_j c_{i,j} x_{i,j}
\end{align}
na množině řešení
\begin{align}
	M = \{ x_{i,j} | \sum_{i=1}^m x_{i,j} = b_j \forall j, \sum_{j=1}^m x_{i,j}
	= a_i \forall i, \sum_i a_i = \sum b_j, x_{i,j} \ge 0 \forall i \forall j \}
\end{align}
Všechno je lineární - úloha lineárního programování.
\subsubsection{Celočíslené programování}
Pokud nelze položky libovolně dělit (například lidi), můžeme přidat celočíselnou
podmínku do množiny řešení:
\begin{align}
	x_{i,j} \N_0 \forall i \forall j
\end{align}
\subsubsection{Nelineární programování}
Zrušíme předpoklad lineárního růstu cen (tedy cena závisí na množství)
\begin{align}
	f(x) = \sum_i \sum_j c_{i,j}(x_{i,j}) x_{i,j}
\end{align}
$c_{i,j} = C_{i,j} + c_{i,j} x_{i,j}$
\subsubsection{Parametrické programování}
Produkce není pevná a závisí na parametru: $a_i = \lambda a_i'$. Ostatní vztahy
můžou zůstat třeba jako u lineárního programování.
\subsubsection{Vícekriteriální programování}
Minimalizovat ceny za dopravu, maximalizovat zisky Z.
\begin{align}
	\min \{f(x), g(x)\} \\
	f(x) = \sum_i \sum_j c_{i,j} x_{i,j} \\
	g(x) = - Z(x_{i,j}) 
\end{align}
\subsubsection{Dynamické programování}
Ceny závisí na rozhodnutí.


\section{Volný extrém}
\setcounter{equation}{0}
Hledání $\min f(x)$.
\subsection{Postačující podmínka}
\setcounter{equation}{0}
Jestliže má funkce $f(x)$ v $x_0 \in \R^n$ spojisté 2. parciální derivace,
$\nabla f(x_0) = 0$, $\nabla^2 f(x_0)$ (Hessova matice) je pozitivně definitní,
potom má reálná funkce $f(x)$ v $x_0$ ostré lokální minimum.
\subsection{Penalizační metody}
\setcounter{equation}{0}
Máme úlohu
\begin{align}
	\min_M f(x), \quad M = \{ x | g_j(x) \le 0, j = 1, ..., n \}
\end{align}
Hledáme penalizační funkci $p(x)$ vyjadřující pokutu za to, že pracujeme s $x \in
M$. $p(x)$ je vytvořená z podmínek $g_j$ a řešíme posloupnost úloh na volný
extrém:
\begin{align}
	\min \{ f(x) + \alpha_k p(x) \}
\end{align}
Po p(x) požadujeme:
\begin{align}
	p(x) > 0: \quad x \nin M \\
	p(x) = 0: \quad x \in M \\
	p(x) \quad \text{spojitá}
\end{align}
Posloupnost takovýchto úloh za jistých okolností konverguje k optimálnímu řešení.
\paragraph{Příklad}
\begin{align}
	\sum_{j=1}^n (g_j^+(x))^2, \quad g_j^+(x) = \max \{ g_j(x), 0 \}
\end{align}
\subsection{Bariérové metody}
\setcounter{equation}{0}
Hledáme bariérovou funkce $b(x)$, která nám zabrání vystoupit z množiny $M$.
Řešíme tedy posloupnost úloh:
\begin{align}
	\min \{ f(x) + \frac{1}{\beta_k} b(x) \}, \quad \beta_k \to \infty
\end{align}
A po funkce b(x) požadujeme:
\begin{align}
	b(x) \le 0: \quad x \in M
	\lim_{x \to \partial M} b(x) = \infty
\end{align}
kde $\partial M$ je hranice množiny.
\paragraph{Příklad}
\begin{align}
	b(x) = - \sum_j \frac{1}{g_j(x)} \\
	b(x) = - \sum_j \log (-g_j(x))
\end{align}
\subsection{Penalizačně-bariérové (SUMT)}
\setcounter{equation}{0}
Rozdělíme množinu \{1 ... m \} na $I_1$ disjunktní $I_2$. Řešíme posloupnost
úloh:
\begin{align}
	\min \{ f(x) + \alpha_k \sum_{j\in I_1}  \varphi(g_j(x)) + \frac{1}{\beta_k}
	\sum_{j\in I_2} \psi (g_j(x)) \}
\end{align}

\section{Metody hledání lokálního minima}
\setcounter{equation}{0}
\subsection{Gradientní}
\setcounter{equation}{0}
Podle věty o přírůstku funkce funkce roste nejvíce ve směru gradientu:
\begin{align}
	f(x) = f(x_0) + \nabla f(x_0 + \theta(x-x_0))^T(x-x_0)
\end{align}
Počítáme:
\begin{align}
	x_{k+1} = x_k - \zeta \nabla f(x_k), \text{ kde }\zeta\text{ je	řešením} \\
	\min_{\zeta \le 0} f(x_k - \zeta \nabla f(x_k))
\end{align}
\subsection{Newtonova metoda}
\setcounter{equation}{0}
\begin{align}
	f(x) \dot= f(x_k) + \nabla f(x_k)(x-x_k) + \frac{1}{2} (x-x_k) \nabla^2
	f(x_k)(x-x_k) \\
	\nabla f(x) = 0 \\
	0 + \nabla f(x_k) + \nabla^2 f(x_k)(x-x_k) - 0 \\
	x_{k+1} = x_k - \nabla f(x_k)(\nabla^2f(x_k))^2 \\
	x_1 = 0
\end{align}
\subsection{Lemkeho metoda}
\setcounter{equation}{0}
Nápad: vezmeme výchozí řešení $x$ a vzdálenost $\rho$, metodou půlení přibližujeme.\\
Požadavky: $\{ x | f(x) \le f(x_1) \}$ je omezená. \\
Problém vhodně  zvolit $x_1$.
\paragraph{Metoda} K $x_1$ sestavíme $x_1 + \rho e^i$ pro $\rho > 0$ a spočítáme funkční
hodnoty $f(x_1 + \rho e_i)$ a porovnáme s funkční hodnotou v $x_1$. Pokud:
\begin{enumerate}
	\item $f((x_1 + \rho e_i) \ge f(x_1) \quad \forall i \quad \Rightarrow \quad
	\rho = \frac{\rho}{2}$
	\item $\exists j$: $y := f(x_1 + \rho e_j) < f(x_1) \quad \Rightarrow \quad$
	opakujeme pro $y$
\end{enumerate}


\section{Lineární programování}
\setcounter{equation}{0}
\paragraph{Definice}
Úlohou pro lineární programování v rovnicovém tvaru rozumíme úlohu:
\begin{align}
	\min_M c^T x, \quad &M = \{ x | Ax = b, x \ge o \},\\ 
	&\text{ kde } \quad A \in \R^{m \times
	n }, b \in \R^m, c \in \R^n, \rank(A) = m, 1 \le m < n, b \ge 0
\end{align}
\paragraph{Definice}
Úlohou lineárního programování normálním tvaru rozumíme
\begin{align}
	\min_M c^T x, &M = \{ x | Ax \le b, x \ge 0 \}, \\
	&\text{ kde } \quad A \in \R^{m \times
	n }, b \in \R^m, c \in \R^n, n,m \ge 1
\end{align}
\subsection{Podprostor}
\setcounter{equation}{0}
\paragraph{Tvrzení}
Množina
\begin{align}
	R = \{ x \in \R^n | a^T x = b, a \neq 0 \}
\end{align}
představuje podprostor dimenze $n-1$ nazývaný \textbf{nadrovinou}.
\paragraph{Důkaz}
Protože $a \neq 0 \Rightarrow \exists a_1 \neq 0 \Rightarrow x_1 = \frac{b}{a_1}
- \sum_{i=1}^n \frac{a_i}{a_1} x_i$. Tedy máme $n-1$ LN vektorů v $R$.
\paragraph{Tvrzení} Množina $R^{n-\alpha} = \{ x \in \R^n | a_i^T x = b_i, i + 1
... \alpha \}$ představuje podprostor $R^n$ dimenze $n - \alpha$.

\subsection{Poloprostor}
\setcounter{equation}{0}
\paragraph{Definice}
Pro libovolnou nadrovinu $R = \{ x \in \R^n | a^T x = b, a \neq b \}$
nazýváme:
\begin{enumerate}
	\item $H^+ = \{ x \in \R^n | a^T x \gt b \}$ otevřeným kladným (pravým)
	poloprostorem $\R^n$ příslušným $R$.
	\item $H^+ = \{ x \in \R^n | a^T x \lt b \}$ otevřeným záporným (levým)
	poloprostorem $\R^n$ příslušným $R$.
	\item $\overline{H^+} = \{ x \in \R^n | a^T x \ge b \}$ uzavřeným kladným (pravým)
	poloprostorem $\R^n$ příslušným $R$.
	\item $\overline{H^+} = \{ x \in \R^n | a^T x \le b \}$ uzavřeným záporným (levým)
	poloprostorem $\R^n$ příslušným $R$.
\end{enumerate}
\paragraph{Tvrzení}
Platí:
\begin{enumerate}
	\item $H^+ \cup H^- \cup R = \R^n$
	\item $\overline{H^+} = H^+ \cup R$, $\overline{H^-} = H^- \cup R$
	\item $H^+ \cap H^- = H^+ \cap R = H^- \cap R = \emptyset$
\end{enumerate}
\paragraph{Tvrzení}
Platí:
\begin{align}
	\dim \mathcal{H}_i^+ = \dim &\overline{\mathcal{H}_i^+} = n \\
	\text{kde } \mathcal{H}_i^+ &= \{ x \in \R^n | x_i > 0 \} \\
	\overline{\mathcal{H}_i^+} &= \{ x \in \R^n | x_i \ge 0 \}
\end{align}
\paragraph{Důkaz}
V každém takovém prostoru leží všechny jednotkové vektory.
\paragraph{Tvrzení}
Platí:
\begin{align}
	\dim \bigcap_{i=1}^n \overline{\mathcal{H}_i^+} = \dim \bigcap_{i=1}^n
	\mathcal{H}_i^+ = n
\end{align}
\paragraph{Důsledek}
Množina přípustných řešení lineárního programování:
\begin{align}
M = \{ x | Ax = b, x \ge 0 \}
\end{align}
kde $b(A) = m$, $a \le m < n$. \\
Taktéž lze zapsat jako:
\begin{align}
	M = \R^{n-m} \cap	\bigcap_{i=1}^n \overline{\mathcal{H}_i^+}
\end{align}
\paragraph{Definice}
Každou množinu $M \subset \R^n$, která se dá popsat jako průnik konečného počtu
nadrovin a uzavřených poloprostorů nazýváme konvexním polyedrem.



=====

====


\subsection{Rozklad konvexního polyedru}
\setcounter{equation}{0}
Označme:
\begin{align}
	R^{n-m} = \{ x \in \R^n | Ax = b \} \\
	R_\alpha = \{ x \in \R^n | x_\alpha = 0 \} \\
	\mathcal{H}_\alpha^+ = \{ x \in \R^n | x_\alpha > 0 \}
	\overline{\mathcal{H}_\alpha^+} = \{ x \in \R^n | x_\alpha \ge 0 \}, \alpha
	= 1, ..., n
\end{align}
Potom:
\begin{align}
M = R^{n-m} \cap \bigcap_{\alpha=1}^n \overline{\mathcal{H}_\alpha^+} =
\R^{n-m} \bigcap_{\alpha=1}^n \mathcal{H}_\alpha^+ \cup R_\alpha = \\
	= \left( R^{n-m} \cap \bigcap_{\alpha=1}^n \mathcal{H}_\alpha^+ \right) \cup
	\left( R^{n-m} \cap R_1 \cap \bigcap_{\alpha=1}^n \mathcal{H}_\alpha^n
	\right) \cup ... ??????????????????????
\end{align}
\paragraph{Definice}
První člen rozkladu $(\R^n \cap \bigcap_{\alpha=1}^n \mathcal{H}_\alpha^+)$
nazýváme \textbf{vnitřkem} konvexního polyedru $M$. Ostatní členy rokladu
nazýváme hraniční, každý prvek hranice, tj. 
\begin{align}
	\left( \R^{n-m} \cap \bigcap_{\alpha \in I_1} \mathcal{H}_\alpha^n \cap
	\bigcap_{\alpha \in I_2} R_\alpha \right), \qquad I_1 \cup I_2 = \{ 1, ...,
	n\}, I_1 \cap I_2 = \emptyset
\end{align}
Pokud je $\neq 0$ a má dimenzi $d$, nazýváme stěnou $M$ dimenze $d$. speciálně stěnu
dimenze $0$ nazýváme vrcholem $M$ a stěnu dimenze $1$ hranice $M$.
\paragraph{Tvrzení}
\begin{align}
	\dim( \R^{n-m} \cap \bigcap_{\alpha=1}^n \mathcal{H}_\alpha^+ ) = n - m 
\end{align}
\begin{enumerate}
	\item $0 \le \dim( \text{ stěny } ) \le n-m-1$
	\item Stěn všech dimenzní je konečný počet.
	\item Průnik libovolných 2 stěn $S_1 \cap S_1 = \emptyset$. 
	\item Uzávěr každé stěny dimenze $d$ je konvexní polyedr dimenze $d$.
\end{enumerate}
\begin{align}
	M = \{ x \in R^n | Ax = b, x \ge 0 \}
\end{align}
Protože $\rank(A) = m \Rightarrow \exists$ konečný počet $\rank(A_B) = m, A_B
\in \R^{m\times m}$
Označíme $A = (A_B A_N)$, kde $A_B$ je regulární. Analogicky rozdělíme x = $(X_B,
X_N)$. Pak můžeme přepsat rovnici $Ax = b \to A_B X_B + A_N X_N = b$, 
tedy 
\begin{align}
	X_B = A_B^{-1}b - A_B^{-1}A_N X_N
\end{align}
Položíme-li všechny $X_N = 0$, získáváme jednoznačně určené $X_B = A_B^{-1} b$
\paragraph{Definice}
Prvek $x = (X_b, X_n) \subset M$ s vlastností $X_B = A_B^{-1} b > 0, X_N = 0$
nazveme nedegenerovaným přípustným bázickým řešením úlohy lineárního
programování.
\paragraph{Poznámka}
Přípustné (nedegenerované) bázické řešení má $n-m$ souřadnic rovných 0.
\paragraph{Definice}
$B$ nazýváme \textbf{bází}, proměnné $x_\alpha, \alpha \in B$ nazýváme \textbf{bázickými}
proměnnými a $x_\alpha, \alpha \in N$ \textbf{nebázickými} proměnnými.
\paragraph{Tvrzení}
Bází $B$ je konečný počet. $|B| = m, |N| = n-m, B \cup N = \{ 1, ..., n \}$
\paragraph{Poznámka}
Protože vrchol je množina 
\begin{align}
	\R^{n-m} \cap \bigcap_{\alpha \in I_1} R_\alpha \cap
	\bigcap_{\alpha \in I_2} \mathcal{H}_\alpha^+
\end{align}
a protože $I_1 \cup I_2 = \{ 1, ..., n \}$ a $I_1 \cap I_2 = \emptyset$, potom
jistě $I_1 = n-m$, protože $R_\alpha = \{ x \in \R^n | x_\alpha = 0 \}$.
\paragraph{Tvrzení}
Geometrickému pojmu \textbf{vrchol} odpovídá algebraický pojem \textbf{přípustné nedegenerované
bázické řešení}.
\paragraph{Tvrzení}
Konvexní polyedr má alespoň jeden vrchol.
\paragraph{Důkaz}
Z rozkladu $M$ můžeme vybrat vždy $|I_1| = n-m$. $R_\alpha = \{x | x_\alpha = 0
\}$, $\alpha\in \{1, ..., n \}$, tedy aby
\begin{align}
	\R^{n-m} \cap \bigcap_{\alpha \in I_1} R^\alpha \cap \bigcap_{\alpha \nin
	I_1} \mathcal{H}_\alpha^+
\end{align}
měla dimenzi 0.


\subsection{1. Základní věta lineárního programování}
\setcounter{equation}{0}
\paragraph{Lemma}
Je-li $M \neq \emptyset$ omezený konvexní polyedr o vrcholech $\vx^i$, $i = \{ 1, ...,
p \}$. Potom můžu psát 
\begin{align}
	M = \{ \vx | A\vx = \vb\} = \{ \vx \in \R^n | \vx = \sum_{i=1}^p x_i \vx^i, x_i \ge 0, \sum_1^p x_i = 1\}
\end{align}
\paragraph{Věta}
Existuje-li řešení úlohy lineárního programování, pak je ho dosaženo alespoň v
jednom vrcholu konvexního polyedru $M$.
\paragraph{Důkaz}
\begin{enumerate}
	\item Uvažujme, že $M$ je omezený. Potom podle lemmatu platí pro $x \in M$
	zápis:
	\begin{align}
		\vx = \sum_{i=1}^p x_i \vx^i, x_i \ge 0, \sum_i x_i = 1
	\end{align}
	pro 
	\begin{align}
		\vc^T \vx = \vc^T \sum_i^p x_i \vx^i = \sum_{i=1}^p x_i \vc^T \vx^i
	\end{align}
	Označíme-li:
	\begin{align}
		\min_{i \in \{1, ..., p\}} \vc^T \vx^i = \vc^T \vx^i
	\end{align}
	Potom můžeme psát
	\begin{align}
		\vc^T \vx \ge \sum_{i=1}^p x_i \vc^T \vx^{i_0} = \vc^T \vx^{i_0}
	\end{align}
	\item Pokud je $M$ neomezený, zvolíme $k >> 0$ a definujme omezený konvexní
	polyedr:
	\begin{align}
		M^* = M \cup \{ \vx \in \R^n | 1^T \vx \le k \}, 1^T = (1, ..., 1)
	\end{align}
	Pro $k$ musí platit (abychom neodřízli vrchol):
	\begin{align}
		k > \max_{i=\{1,...,k\}} \{ 1^T \vx^T \}
	\end{align}
	Nyní podle 1. části důkazu existuje alespoň jeden vrchol $M^*$, který dává
	optimální řešení. Potom
	\begin{enumerate}
		\item Vrchol je v původní množině $M$, je tvrzení dokázáno.
		\item Vrchol není v původní množina $M$, potom musí ležet na hraně
		původního polyedru $M$ (protože jsme přidali jednu lineární rovnici).
		Každá hrana omezeného konvexního polyedru $M^*$ vychází z nějakého
		vrcholu $M$. Z vlastností, že $\vc^T \vx > \vc^T \vx_{opt}$ plyne, že
		pro body této hrany $\{x | \vx^1 + t(\vx^{opt} - \vx^1) \}$ s vlastností $t
		> 1$ platí, že $\vc^T \vx \to -\infty$ a neexistuje řešení dané úlohy.
	\end{enumerate}
\end{enumerate}


\subsection{Simplexová tabulka}
\setcounter{equation}{0}
Známe-li libovolnou bázi B, přepíšeme množinu 
\begin{align}
	M &= \{ \vx | A \vx = \vb, \vx \ge0 \} 
	&= \{ (\vx_0, \vx_N) \in \R^n | \vx_B + A_B^{-1} A_N \vx_N = A_B^{-1} \vb,
	(\vx_B, \vx_N) \ge 0 \}
\end{align}
Přípustné bázické řešení je pak $\vx = (\vx_B, \vx_N) = (A_B^{-1} b, 0)$
a simplexová tabulka zachycující množinu $M$ vypadá: \\
\begin{center}
\begin{tabular}{|c||c|c|cl|}
	\hline
	        & $\vx_B$     & $\vx_N$        &              &     \\
	\hline
	\hline
	$\vx_B$ & $E$         & $A_B^{-1} A_N$ & $A_B^{-1} b$ & $> 0$\\
	\hline
	        & $0, ..., 0$ & $\vc_N - \vz_N$ & $-\vc_0$     & \\
	\hline
\end{tabular} \\
\end{center}
A cílová funkce:
\begin{align}
	\vc = (\vc_B, \vc_N) \\
	\vc^T \vx = \vc_B^T \vx_B + \vc_N^T \vx_N = \vc_B^T( A_B^{-1} \vb -
	A_B^{-1}A_N \vx_N) + \vc_N^T \vx_N
\end{align}
tedy
\begin{align}
	\forall x \in M \vc^T \vx = \vc_0 + (\vc - Z_N) \vx_N
\end{align}
kde $c_0 = \vc^T(A_B^{-1} b, 0)$


\subsection{Simplexová metoda}
\setcounter{equation}{0}
\begin{enumerate}
	\item $\min_M \vc^T \vx$
	\item $M = \{ x \in \R^n | Ax = b, x \ge 0 \}, A \in \R^{m \times n}, h(A) =
	m, 1 \le m < n, b \in \R^n, c \in \R^n$
	\item $\vc^T\vx$ Předpokládejme, že při výpočtu simplexové metody nenastává
	degenerace.
	\item (obecný k-tý krok) Máme simplexovou tabulku. Předpokládejme, že
	bázické řešení je 
	\begin{align}
		x^k = (x_B, x_N) = d_0, o)
	\end{align}
	Množinu $M$ a cílovou funkci přepíšeme:
	\begin{align}
		M = \{ x_0, x_N | x_B + Dx_N = d_0, x_B > 0, x_N \ge 0 \} \\
		c^Tx = c_B^T x_B + c_N^Tx_N = c_0 + (c_N - z_N) x_N
	\end{align}
\end{enumerate}


\subsection{1. věta Simplexové metody}
\setcounter{equation}{0}
\paragraph{Věta}
Platí-li v k-tém kroku simplexové metody
\begin{align}
	c_N - z_N \ge 0
\end{align}
v příslušné simplexové tabulce, potom je příslušné bázické řešení $x^k$
optimální.
\paragraph{Důkaz}
\begin{align}
	\forall x \in M : c_Tx = c_0 + (c_N - z_N) x_n \ge c_0 = c^Tx^k
\end{align}

\subsection{2. věta Simplexové metody}
\setcounter{equation}{0}
\paragraph{Věta}
Existuje-li v k-tém kroku simplexové metody v simplexové tabulce index $h \in N$
tak, že $c_h - z_h < 0$ a příslušný h-tý sloupec matice D splňuje $d_h \le o$,
potom je cílová funkce $c^Tx$ zdola neomezená na množině $M$ a tedy řešení
neexistuje.
\paragraph{Důkaz}
Hrana $M$ vycházející z vrcholu $\vx^k$ má popis $\{ x | x_B = d_0 - td_h, x_j = 0,
j \in N \setminus h, x_k = t \ge 0\}$
\begin{align}
	x\in h: c^Tx = c_0 \underbrace{(c_h-z_B)}_{<0} t \to -\infty
\end{align}

\subsection{3. věta Simplexové metody}
\setcounter{equation}{0}
\paragraph{Věta}
Nejsou-li splněny předpoklady první ani druhé věty simplexové metody, potom
definujeme (r je index v simplexové tabulce):
\begin{align}
	\min_j (c_j - z_j) = c_r - z_r
\end{align}
a hledáme
\begin{align}
	\label{simplex-minimum}
	\min_{d_{ir} > 0} \left\{ \frac{d_{is}}{d_{ir}} \right\} = \frac{d_{so}}{d_{sr}}
\end{align}
a po transformaci tabulky s pivotem $d_{sr}$ získáme nové přípustné bázické řešení
$x^{k+1}$ s vlastností $c^Tx^{k+1} < c^T x^k$
\paragraph{Značení}
$r$-tý sloupec nazýváme klíčovým sloupcem a $s$-tý řádek nazýváme klíčovým řádkem.
\paragraph{Důkaz}
Proměnná $x_s$ se stane nebázickou a nebázická proměnná $x_r$ se stane bázickou,
tedy z $s$-té rovnice vypočítáme $x_r$ a dosadíme za něj do ostatních rovnic i
cílové funkce. (Gaussova eliminace) \\
Nová báze je $B' = (B \setminus \{s\}) \cup \{r\}$ a nebázické proměnné$B' = (B
\setminus \{r\}) \cup \{s\}$ a nové přípustné bázické řešení (plyne z Gausse):
\begin{align}
	d_0': \quad&x_r^{k+1} = \frac{d_{s0}}{d_{sr}} > 0\\
	i \in B \setminus \{s\}: \quad&X_i^{k+1} = d_{i0} - \frac{d_{ir}d_{s0}}{d_{sr}} \\
	&x_s, x_j = 0, j \in N \setminus \{r\}
\end{align}
A ověříme přípustnost takovéhoto řešení:
\begin{enumerate}
	\item Pokud $d_{ir} \le 0 \Rightarrow x_i^{k+1} = d_{i0}' > 0$
	\item Pokud $d_{ir} > 0$, potom ze (\ref{simplex-minimum})
	$\frac{d_{s0}}{d_{sr}} < \frac{d_{i0}}{d_{ir}} \Rightarrow d_{i0}' > 0$ 
\end{enumerate}
A cílová funkce se transformuje:
\begin{align}
	c^Tx^{k+1} = c_0 + (c_r - z_r) \frac{d_{s0}}{d_{sr}}
\end{align}


\subsection{Konečnost simplexové metody}
\setcounter{equation}{0}
\paragraph{Tvrzení}
Simplexová metoda se ukončí po konečném počtu kroků.
\paragraph{Důkaz}
Máme konečný počet bázických řešení a nelze se vracet (protože $\forall k: c^T
x^{k+1} < c^T x^k$).
\paragraph{Poznámka}
V degenerovaném případě může nastat i rovnost a proto teoreticky může nastat
cyklus.


\subsection{Vznik degenerace}
\setcounter{equation}{0}
\begin{enumerate}
	\item Zadáním - $\exists i : \quad b_i = 0$
	\item Pokud klíčový řádek není určen jednoznačeně (je lineární závislý na
	jiném)
\end{enumerate}


\subsection{Odstranění cyklů/degenerace}
\setcounter{equation}{0}
\begin{enumerate}
	\item Odstranění cyklů - \textbf{Blandovo pravidlo}: Volíme minimum ze všech indexů j
	nebázických, pro které je $c_j - z_j < 0$, $r := j$ (takovýto index je určen
	jednoznačně). Také s zvolíme jednoznačně:
	\begin{align}
		s = \{ \min t | \min_{d_{ir} > 0} \{ \frac{d_{i0}}{d_{ir}} \} =
		\frac{d_{t0}}{d_{tr}} \}
	\end{align}

	\item $\epsilon$-modifikace: řešíme novou úlohu pro $\epsilon > 0$.
	\begin{align}
		\min_{M+\epsilon} c^Tx\\
		M+\epsilon= \{x_Bx_N | x_B + Dx_N = a_0 + (ED) \epsilon, x_Bx_N \ge 0 \}
	\end{align}
	kde $\epsilon = (\epsilon, \epsilon^2, ..., \epsilon^n)$.

\end{enumerate}
\paragraph{Důkaz}
\begin{enumerate}
	\item (bez důkazu)
	\item epsilonový vektor zaručuje, že jednotlivé složky nelze poměřit
\end{enumerate}

\subsection{Množina optimálních řešení}
\setcounter{equation}{0}
\paragraph{Definice}
Množinou všech optimálních řešení úlohy nazýváme:
\begin{align}
	M_{opt} = \{ x^i \in M | \min_M c^Tx = c^Tx_i \}
\end{align}

\paragraph{Věta}
Je-li $x^0$ optimální řešení úlohy nalezené simplexovou metodou a $c_N - z_N,
-c_0$ v posldním řádku poslední simplexové tabulky. Potom:
\begin{align}
	M_{opt} = \{ x \in M | x_j = 0, j \in J \}, \\
	\text{ kde } J = \{j \in N | c_j - z_j > 0 \}
\end{align}
\paragraph{Důkaz}
Máme optimální řešení a simplexovou tabulku z posledního kroku.
$x^0 = (x_B, x_N) = (d_0, o)$.
\begin{align}
	\forall x \in M : \quad c^Tx = c_0 + (c_N - z_N) x_N = c_0 + \sum_{j \in J}
	(c_j - z_j) x_j + \sum_{j \nin J} \underbrace{(c_j - z_j)}_{ = 0} x_j  = c_0
\end{align}
Když položíme rovnost $0$, tedy $x_j = 0$, $j \in J$.
\paragraph{Poznámka}
Změna hodnoty cílové funkce po jedné transformaci tabulky
\begin{align}
	-c_0' = -c_0 - \frac{c_r-z_r) \cdot d_{s0}}{d_{sr}}
\end{align}


\subsection{Výpočetní složitost simplexové metody}
\setcounter{equation}{0}
Mějme výpočet simplexovou metodou, kde n a m je počet rovnic a bazických
proměnných.
\begin{enumerate}
	\item Nejhorší odhad počtu kroků je dán počtem bazí, tedy $\binom{n}{m}$.
	\item Na pravděpodobnostních modelech byl ukázán odhad 
	\begin{align}
		P(n,m) \le
		m^{\frac{1}{n-1}} \cdot (n+1)^n \cdot \frac{2}{5} \pi \left( 1 + \frac{e
		\pi}{2}\right)
	\end{align}
	\item Výpočetní zkušenosti ($p$ = počet kroků)
	\begin{enumerate}
		\item Počet kroků nezávisí na $n$ a je mezi $2m-3m$
		\item $p < \frac{3}{2} m$
	\end{enumerate}
\end{enumerate}


\subsection{Jiné metody}
\setcounter{equation}{0}
\subsubsection{Elipsoidová-Chačijanova metoda (1979)}
\setcounter{equation}{0}
Je polynomiální v čase. Krok velmi nepraktický (zmenšují se elipsoidy).
\subsubsection{Karmarkarova projekční metoda}
\setcounter{equation}{0}
Začíná se v libovolném prvku množiny $M$, najdu simplex, vepíšu kouli ve středu
$x_0$, sestavím potenciálovoufunkci, hledám extrém nelineární úlohy.
\subsubsection{Metody konvexního programování}
\setcounter{equation}{0}
Například gradientní metody, metoda vnitřního bodu,...



\subsection{2. základní věty lineárního programování: princip duality}
\setcounter{equation}{0}
Nechť (P) značí úlohu: 
\begin{align}
	\max_{M_1} \vc^T \vx \qquad M_1 = \{\vx\in \R^n | A\vx \le \vb, \vx \ge 0
	\}, \vc \in \R^n, \vb \in \R^m, A \in \R^{m\times n}
\end{align}
a (D) značí úlohu:
\begin{align}
	\min_{M_2} \vb^T \vy \qquad M_2 = \{ \vy\in\R^m | A^T \vy \ge \vc, \vy \ge 0
	\}
\end{align}
\paragraph{Definice}
Úlohu (P) nazýváme primární, a (D) duální úlohou lineárního programování v
normálním tvaru. (dualita však platí i inverzně)
\paragraph{Lemma} (slabá věta o dualitě)
Je-li $M_1$ a $M_2$ neprázdné, potom platí, že
\begin{align}
	\vc^T \vx \le \vb^T\vy \quad \forall x \in M_1, y \in M_2
\end{align}
\paragraph{Důkaz}
Pro libovolné $x \in M_1$ a $\vy \in M_2$ platí:
\begin{align}
	A\vx \le b & A\vy \ge c \\
	x\ge 0 & y \ge 0
\end{align}
Po roznásobení:
\begin{align}
	\vy^T A \vx = \vx^T A^T \vy \le \vb^T \vy \\
	\vx^T A^T \vy \ge \vc^T \vx
\end{align}
tedy $\vc^T\vx \le \vb^T \vy$ a Lemma je dokázáno.
\paragraph{Lemma}
Je-li $M_1, M_2 \neq \emptyset$, potom existuje konečné $\sup_{M_1} \vc^T\vx =
M_1$ a konečné $\inf_{M_2} \vb^T \vy = m_2$.
\paragraph{Důkaz}
Zvolme $\vy^1 \in M_2$, potom z prvního lemmatu 
\begin{align}
	\vc^T \vx \le \vb^T \vy^1 \quad \forall x \in M_1 \Rightarrow \exists m_1
\end{align}
druhý případ analogicky.
\paragraph{Lemma}
Je-li $M_1$ neprázdné a existuje konečné supremum, potom
\begin{align}
	\exists \vy_0 \in M_2 \quad \vb^T\vy_0 \le m_1
\end{align}
Je-li $M_2$ neprázdné a existuje konečné infimum, potom
\begin{align}
	\exists \vx_0 \in M_1 \quad \vc^T\vx_0 \ge m_2
\end{align}
\paragraph{Důkaz} (bez důkazu)
\paragraph{Věta}
Je-li $M_1, M_2 \neq \emptyset$, potom:
\begin{align}
	\exists \max_{M_1} \vc^T \vx = \vc^T \vx^0 = \vb^T\vy^0 = \min_{M_2} \vb^T
	\vy
\end{align}
\paragraph{Důkaz}
$M_1 a M_2 \neq \emptyset$, tedy podle lemma 2 existuje konečné $m_1$, $m_2$ a podle
lemma 3:
\begin{align}
	\exists \vy^0: \vb^T \vy^0 \le m_1
	\exists \vx^0: \vc^T \vx^0 \ge m_2
\end{align}
Odtud podle důsledků a lemma 1:
\begin{align}
	\vc^T\vx^0 \ge m_2 \ge m_1 \ge \vb^T \vy^0 \ge \vc^T\vx^0
\end{align}
Tedy 
\begin{align}
	\vc^T\vx^0 = \max_{M_1} \vc^T\vx = \vb^T \vy^0 = \min_{M_2} \vb^T\vy
\end{align}
\subsection{Důsledky věty o dualitě}
\setcounter{equation}{0}
\begin{enumerate}
	\item Má-li jedna z duálních úloh optimální řešení, pak ho má i druhá a
	platí rovnost funkčních hodnot v optimálních bodech.
	\item Rovnost funkčních hodnot platí právě v optimálních.
	\paragraph{Důkaz}
	Jestliže $x_1$ není optimální řešení $(P) \Rightarrow \exists \overline{\vx}
	\in M_1, \vc^T \vx^1 < \vc^T \overline{\vx}$, což ale platí pro každé $\vy
	\in M_2$.
	\item Souvislost optimálního řešení duálních úloh. 
	\paragraph{Tvrzení}
	Nechť $x_0$ je optimální
	řešení $(P)$ a $x_i^0 > 0$, $i \in I$, $\va_j^T \vx^0 < b_j$, $j \in J_1$.
	Potom celá množina optimálních řešení úlohy duální:
	\begin{align}
		M^2_{opt} = \{ \vy^0 | \vb^T \vy^0 = \min_{M_2} \vb^T \vy \} = 
			\{ y \in M_2 | \va^T_i \vy = \vc_i, i \in I, y_j = 0, j \in J_1 \}
	\end{align}
	\paragraph{Důkaz}
	Z důsledku 2 a vzorce 
	\begin{align}
		\vc^T \vx \le \vx^T A^T \vy = \vy^T A \vx \le \vb^T \vy
	\end{align}
	Pro všechna optimální řešní $\vy$
	\begin{align}
		\underbrace{\vc^T \vx^0 = \vx^{0T} A^T \vy}_{(\ref{dualitad-jedna})}
		= \underbrace{\vy^T A \vx^0 = \vb^T \vy}_{(\ref{dualitad-dva})} \\
		\label{dualitad-jedna}
		0 = (c-A^T\vy)^T \vx^0 = \sum_{i\in I_1} \underbrace{(\vc_i -
		\va_i\vy)^T}_{\le 0} \underbrace{\vx_i^0}_{\ge 0} +
		\underbrace{\sum_{i \nin I_1}(\vc_i - \va_i \vy)^T vx_i^0}_{ = 0}
			\quad\Rightarrow\quad \vc_i = \va_i \vy \\
		\label{dualitad-dva}
		0 = \vy^T(Ax^0 - b) = \sum_{j\in J_1} \underbrace{(\va_j^T \vx^0 -
		\vb_j)}_{<0} \underbrace{\vy_j}_{\ge 0} +
		\underbrace{\sum_{j\nin J_1} (\va_j^T \vx^0 - \vb_j)}_{=0}
			\quad\Rightarrow\quad \vy_j = 0
	\end{align}
	\paragraph{Tvrzení}
	Je-li $\vy^0$ optimální řešení úlohy $(D)$ pro které platí $\vy_j^0 > 0, j \in
	J_2, \va_i^T \vy^0 > \vc_i, i \in I_2$. Potom množina všech optimálních
	řešení úlohy (P) má popis:
	\begin{align}
		M_1^{opt} = \{ \vx \in M_1 | x_i = 0, i \in I_2, \va_j^T \vx = b_j, j
		\in J_2 \}
	\end{align}
	\paragraph{Důkaz}
	Analogicky.

\subsection{Různé typy úloh lineárního programování}
\setcounter{equation}{0}
\begin{enumerate}
	\item $\min_{M_1} \vc^T \vx = - \max_{M_1} (-\vc^T\vx) \quad \Rightarrow
	\quad$\\
	$ (D) - \min_{M_1^*} \vb^T \vy, M_2^* \{ \vy \in \R^m | A^T\vy\ge -\vc,
	\vy\ge 0\}$
	\item $\max_{M_1'} \vc^T \vx M_1' = \{ \vx \in \R^n | A\vx \le \vb \}$ \\
	Zavedeme $\vx^+ \max(\vx, 0) \ge 0$, $\vx^- = \max(-\vx, 0) \ge 0$.
	Přepíšeme novou úlohu pomocí rovnosti $\vx= \vx^+ - \vx^-$ v prostoru
	dvojnásobné dimenze. Ze dvou obrácených nerovností získáme rovnici.
	\item $\max_{M_1''} \vc^T \vx, M_1'' = \{\vx | A\vx = \vb, \vx \ge 0 \}$. \\
	Což můžeme přepsat jako úlohu s dvouma obrácenýma nerovnostma (pro teorii).
\end{enumerate}


\subsection{Duální simplexová metoda}
\setcounter{equation}{0}
Řešíme dvě úlohy naráz. Začneme řešením, kde jsou splněny všechny pomínky (jsou
bázická) až na přípustnost (nezápornost proměnných) a platí: $\vc^T\vx^1 = \vb^T
\vy^1$.
Jediným krokem se dostaneme do přípustného řešení duální úlohy (v primární může
stále být řešení nepřípustné). Následně již v duální úloze nesmíme opustit
přípustná řešení a hledáme dvojce (kde se hodnoty rovnají), dokud i primární
řešení není přípustné. Pokud jsou obě řešení přípustná, nalezli jsme minimum
(podle věty o rovnosti).






\end{enumerate}









\end{document}

