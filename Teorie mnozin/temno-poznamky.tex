\documentclass[a4paper,12pt,titlepage]{article}
\usepackage[utf8]{inputenc}
\usepackage{a4wide}
\usepackage[czech]{babel}
\usepackage{amsfonts, amsmath, amsthm, amssymb}
\usepackage{math}

\title{Teorie množin}
\author{Ladislav Láska}

\begin{document}

\maketitle
\newpage
\tableofcontents
\newpage

\section{Formální jazyk}
\setcounter{equation}{0}
\subsection{Základní součásti jazyka}
\setcounter{equation}{0}
\begin{enumerate}
	\item proměnné
	\item binární predikátový symbol $\in$
	\item binární predikátový symbol $=$
	\item logické spojky $\neg \land \lor \Rightarrow \Leftrightarrow$
	\item kvantifikátory $(\forall x)$, $(\exists x)$
	\item pomocné symboly - závorky
\end{enumerate}
\subsection{Formule}
\setcounter{equation}{0}
\begin{enumerate}
	\item Nechť $x$, $y$ jsou prvky množiny, pak $(x \in y)$ a $(x = y)$ jsou
	atomické formule.
	\item Nechť výrazy $\varphi, \varpsi$ jsou formule, potom:
			$\neg \varphi$, $\varphi \land \varpsi$, $\varphi \lor \vaprsi$,
			$\varphi \Rightarrow \varpsi$, $\varphi \Leftrightarrow \varpsi$
			jsou formule.
\end{enumerate}







\end{document}

