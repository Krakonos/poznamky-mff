\documentclass[a4paper,12pt,titlepage]{article}
\usepackage[utf8]{inputenc}
\usepackage{a4wide}
\usepackage[czech]{babel}
\usepackage{amsfonts, amsmath, amsthm, amssymb}
\usepackage{math}

\title{Teorie množin}
\author{Ladislav Láska}

\begin{document}

\maketitle
\newpage
\tableofcontents
\newpage

\section{Formální jazyk}
\setcounter{equation}{0}
\subsection{Základní součásti jazyka}
\setcounter{equation}{0}
\begin{enumerate}
	\item proměnné
	\item binární predikátový symbol $\in$
	\item binární predikátový symbol $=$
	\item logické spojky $\neg \land \lor \Rightarrow \Leftrightarrow$
	\item kvantifikátory $(\forall x)$, $(\exists x)$
	\item pomocné symboly - závorky
\end{enumerate}
\subsection{Formule}
\setcounter{equation}{0}
\begin{enumerate}
	\item Nechť $x$, $y$ jsou prvky množiny, pak $(x \in y)$ a $(x = y)$ jsou
	atomické formule.
	\item Nechť výrazy $\varphi, \psi$ jsou formule, potom:
			$\neg \varphi$, $\varphi \land \psi$, $\varphi \lor \psi$,
			$\varphi \Rightarrow \psi$, $\varphi \Leftrightarrow \psi$
			jsou formule.
\end{enumerate}

\ \\
===============================================================\\
Tedy toho hodne chybi\\
===============================================================
\section{Axiomy teorie množin}

\subsection{Průnik a rozdíl množin}
\setcounter{equation}{0}
\paragraph{Definice}
Pro množiny a,b po řadě průnikem a rozdílem nazýváme množinu:
\begin{align}
	a \cup b = \{ x : x \in a \land x \in b \} \\
	a \setminus b = \{ x : x \in a \land x \nin b \}
\end{align}

Existuje množina a (Axiom existence), podle vydělení pro formuli $x \neq x$
existuje a podle extenziability je jediná množina $\{ x \in a : a \neq x \}$.


\paragraph{Definice}
$\emptyset$ je jediná množina y splňující:
\begin{align}
	(\forall x) (x \nin y)
\end{align}
A nazýváme jí \textbf{prázdná množina}.

\subsection{Disjunkní množina}
\setcounter{equation}{0}
\paragraph{Definice}
Říkáme, že množina a,b jsou disjunkní, že  je-li $a \cup b = 0$.

\paragraph{Lemma}
\begin{enumerate}
	\item $\neg (\exists y)(y \in \emptyset)$
	\item $(\forall x) (\emptyset \subset x)$
	\item $x \subset \emptyset \Leftrightarrow x = 0$
\end{enumerate}
\paragraph{Lemma}
\begin{align}
	(\forall a) a = \{ x : x \in a \land x = x \}
\end{align}


\subsection{Russelův paradox}
\setcounter{equation}{0}
\paragraph{Věta}
\begin{align}
	\neg ( \exists z) (\forall x) (x \in z)
\end{align}
\paragraph{Důkaz}
Sporem: nechť $z$ je taková množina. Pak mějme formuli $\varphi(x) \quad x \neq
x$. Potom podle axiomu vydělení pro tuto formuli máme $t = \{x \in z : x \nin x
\}$, tedy $t$ je množina. Protože $t$ je množina a $z$ je množina všech množin.
Protože $t \in z$, $t \in t \Leftrightarrow t \nin t$. Tedy neexistuje množina
všech množin.

\subsection{Axiom dvojce}
\setcounter{equation}{0}
\begin{align}
	(\forall a) (\forall b) (\exists z) (\forall x) ( x \in z \Leftrightarrow (x
	= a \lor x = b ))
\end{align}
\paragraph{Definice}
Jsou-li $a, b$ množiny, pak množinu se stávající z prvků $a,b$ nazveme
\textbf{neuspořádanou dvojcí} množin $a,b$ a značíme $\{a,b\}$. Pro $a\neq b$
říkáme, že $\{a,b\}$ dvouprvková, jinak jednoprvková.

\subsubsection{Rovnost množin}
\setcounter{equation}{0}
\paragraph{Lemma}
\begin{enumerate}
	\item $\{x\} = \{y\} \Leftrightarrow x = y$
	\item $\{x\} = \{x,y\} \Leftrightarrow x = y$
	\item $\{x,y\} = \{u,v\} \Leftrightarrow (x = u \land y = v) \lor (x = v
	\land y = u)$
\end{enumerate}

\subsubsection{Uspořádaná dvojce, k-tice}
\setcounter{equation}{0}
\textbf{Uspořádaná dvojce} množin $a,b$ je množina, která má prvky $\{\{a\}, \{a,b\}\}$.
Značíme jí $<a,b>$.
\paragraph{Lemma}
\begin{align}
	<x,y> = <u,v> \Leftrightarrow (x = u \land y = v)
\end{align}
\paragraph{Definice}
Jsou-li dány množiny $a_1, a_2, ..., a_k$, pak uspořádanou $k-tici$ definujeme
jako:
\begin{align}
	&<a_1> = a_1, \text{ a dál indukcí } \\
	&<a_1, a_2, ..., a_k > = <<a_1, ..., a_{k-1}>, a_k>
\end{align}
\paragraph{Lemma}
\begin{align}
	<a_1, ..., a_k> = <b_1, ..., b_k> \\
	\Leftrightarrow \\
	(a_1 = b_1) \land ... \land (a_k = b_k)
\end{align}


\subsection{Axiom sumy}
\setcounter{equation}{0}
\begin{align}
	(\forall a) (\exists z) (\forall x) (x \in z \Leftrightarrow (\exists y)(x
	\in y \land  y \in a))
\end{align}
\paragraph{Značení}
\begin{align}
	\bigcup a = \{ x : (\exists y) (y \in a \land x \in y) \}
\end{align}
\paragraph{Značení}
Nechť $a = \{b,c\}$. Pak $\bigcup a = b \cup c$

\subsubsection{Neuspořádané $k$-tice}
\setcounter{equation}{0}
\paragraph{Značení}
Neuspořádaná $k$-tice je:
\begin{align}
	\{a,b,c\} = \{a,b\} \cup \{c\}
\end{align}
\subsubsection{Průnik}
\setcounter{equation}{0}
\paragraph{Definice}
Pro neprázdnou množinu $a$ lze analogicky definovat
\begin{align}
	\bigcap a = \{ x : (\forall y) (y \in a \Rightarrow x \in y) \}
\end{align}
Pro neprázdnou a existuje $\bigcap a$: 
\begin{align}
	a \neq 0 \quad &(\exists x) x \in a, \quad x = x_0 \\
	a = 0 \quad &\bigcap a \text{ není definovaný }
\end{align}


\subsection{Schéma axiomu nahrazení}
\setcounter{equation}{0}
Je-li $\psi(u,v)$ formule, která neobsahuje volně proměnné $z,w$, potom formule:
\begin{align}
	\label{axiom-nahrazeni-predpoklad}
	(\forall u)(\forall v)(\forall w) ( \psi(u,v) \land \psi(u,w) \Rightarrow
	v=w) \Rightarrow \\
	(\forall a) (\exists z) (\forall v) (v \in z \Leftrightarrow (\exists u)(u
	\in a \land \psi(u,v)))
\end{align}
je axiom teorie množin.
\paragraph{Pozorování}
Pro jedno $u$, $\psi(u,v)$ platí pro nejvýše jedno $v$. To je analogie k funkci.
\paragraph{Definice}
Nechť $a,b$ jsou množiny. \textbf{Kartézský součin} $a \times b$ je množina:
\begin{align}
	a \times b = \left\{ <x,y> : x \in a \land y \in b \right\}
\end{align}
\paragraph{Důkaz}
$a \times b$ je množina. Zvolme a zafixujme $y \in b$ a
nechť $\psi(x,v)$ je formule $v = <x,y>$. Je-li:
\begin{align}
	\psi(x,v) \land \psi(x,w) \Rightarrow v = <x,y> \land w = <x,y> \Rightarrow
	v = w
\end{align}
Tedy je splněn předpoklad axiomu nahrazení (\ref{axiom-nahrazeni-predpoklad})
pro formuli $\psi$.
\begin{align}
	M_y = \{ <x,y> : x \in a\}
\end{align}
je množina podle nahrazení pro $\psi$ pro každé $y$. \\
Nechť navíc $\overline{\psi}(y,v)$ je formule $v = M_y$. Je-li:
\begin{align}
	\overline{\psi}(y,v) \land \overline{\psi}(y,w) \Rightarrow v = M_y \land w = M_y
	\Rightarrow v = w
\end{align}
Tedy je splněn předpoklad axiomu nahrazení (\ref{axiom-nahrazeni-predpoklad})
pro formuli $\overline{\psi}$. Navíc tedy
\begin{align}
	&D = \{ M_y : y \in b \} \text{ je množina } \\
	&\bigcup D = \{<x,y>: x \in a, y \in b \} = a \times b
\end{align}


\subsubsection{Binární relace}
\setcounter{equation}{0}
\paragraph{Definice}
\textbf{Binární relace} je množina $R$, jejímiž prvky jsou uspořádané dvojce.
\begin{align}
	\dom(R) = \{ x : (\exists y) <x,y> \in R \} \text{ je definiční obork }
	\rng(R) = \{ y : (\exists x) <x,y> \in R \} \text{ je obor hodnot }
\end{align}
Protože $R$ je množina, $\dom(R)$ i $\rng(R)$ jsou množiny.
\paragraph{Definice}
Je-li $R$ relace, definujeme:
\begin{align}
	R^{-1} = \{ <x,y>: <y,x> \in R \}
\end{align}
Pro každou relaci $R$, $R^{-1}$ je relace a $(R_{-1})^{-1} = R$.
\paragraph{Definice}
Jsou-li $R, S$ relace, pak
\begin{align}
R \circ S = \{ <x,z> : (\exists y) <x,y> \in R \land <y,z> \in S \}
\end{align}
\paragraph{Definice}
Jsou-li $R, S, T$ relace, pak
\begin{align}
	(T \circ S) \circ R = T \circ (S \circ R)
\end{align}

\subsubsection{Funkce}
\setcounter{equation}{0}
Množina $f$ se nazývá \textbf{funkce}, pokud $f$ je relace a platí:
\begin{align}
	(\forall x \in \dom(f))((y \in \rng(f) \land y \in \rng(f) \land <x,y> \in f
	\land <x,y'> \in f) \Rightarrow y = y')
\end{align}
\paragraph{Značení} $f: A \to B$ znamená: $f$ je funkce, $A = \dom(f)$, $B \supset
\rng(f)$.


Je-li $C \subsetq A$, pak $f \Gamma C = f \cap (C \times B)$ nazýváme $x$ zůžením funkce
$f$ na množinu $C$.
\begin{align}
f'C = \rng(f \Gamma C) = \{ f(x): x \in C \}
\end{align}
\begin{description}
\item Funkce $f: A \to B$ se nazývá \textbf{prostá}, pokud $f^{-1}$ je funkce.
\item Funkce $f: A \to B$ se nazývá \textbf{surjektivní} ("na"), jestliže $B = \rng(f)$
\item Funkce $f$ se nazývá \textbf{bijekce} je-li \textbf{surjektivní} a současně
\textbf{prostá}.
\end{description}


\subsection{Uspořádání}
\setcounter{equation}{0}
\paragraph{Definice}
\textbf{Ostře uspořádaná množina} je uspořádaná dvojce $<a, r>$, kde $a$ je
množina a $r$ je relace, $r \subsetq a \times a $. Přičemž r splňuje:
\begin{align}
	&\forall x, y, z \in a : \quad <x,y> \in r \land <y,z> \in r \Rightarrow
	<x,z> \in r \quad \text{tranzitivita} \\
	&\forall x \in a: \quad \not <x,x> \in r \quad \text{antireflexivita}
\end{align}
Pro zjednodušení místo $<x,y> \in r$ píšeme $x r y$.
\paragraph{Definice}
Ostré uspořádání $r$ nazveme \textbf{lineárním}, pokud 
\begin{align}
	\forall x,y \in a: \quad x = y \lor x r y \lor y r x 
\end{align}
\paragraph{Definice}
Jsou-li $R$, $S$ relace a $a$, $b$ množiny, pak řekneme, že $<a, R>$ je izomorfní
s $<b, S>$, pokud existuje bijekce $f: a\to b$ taková, že
\begin{align}
	\forall x,y \in a: \quad <x,y> \in \R \Leftrightarrow <f(x), f(y)> \in S
\end{align}
a zobrazení $f$ se nazývá \textbf{izomorfismus}.
\paragraph{Definice}
Mějme uspořádanou množinu $<a,r>$. Je-li $m \subset a$, pak řekneme, že $x \in a$
je \textbf{r-nejmenší} prvek množiny $m$, jestliže platí:
\begin{align}
	x \in m \land (\forall Y) (y \in m \Rightarrow (x r y \lor y = x))
\end{align}
Je-li $m \subsetq a$, $x \in a$, řekneme, že $x$ je \textbf{minimální} prvek
množiny $m$,
jestliže platí
\begin{align}
	x \in m \land (\forall y) ( y \in m \Rightarrow \not (y r x))
\end{align}
\paragraph{Definice}
Řekneme, že uspořádání $r$ na množině a je \textbf{dobré} (množina $<a,r>$ je dobře
uspořádaná) jesltiže $r$ je ostré uspořádání množiny $a$ a každá neprázdná
podmnožina $a$ má r-nejmenší prvek.
\paragraph{Pozorování}
Je-li $<a,r>$ dobře uspořádaná, pak je $r$ lineární uspořádání.
$x,y \in a \quad \{x,y\} \subsetq a$ a $\{x,y\}$ má r-nejmenší prvek. Je-li to
$x$,
pak $x r y \lor x = y$. Pokud je to $y$, pak $y r x \lor y = x$.
\paragraph{Značení}
Nechť $<a,r>$ je uspořádaná množina a $x \in a$. Označme $<(\leftarrow, x), r>$
jako:
\begin{align}
	(\leftarrow, x) = \{ y \in a : y r x \}
\end{align}
\paragraph{Lemma 1}
Je-li $<a,r>$ dobře uspořádaná množina, pak pro každé $x \in a \quad <a, r>$ není
izomorfní s $<(\leftarrow, x), r>$
\paragraph{Důkaz}
Sporem: Předpokládejme, že existuje izomorfismus $f: <a,r> \to <(\leftarrow, x),
r>$. Definujme $m = \{ y \in a: f(y) \neq y \}$. $x \neq (\leftarrow, x)$, tedy
$f(x) \neq x\Rightarrow m \neq \emptyset$. $<a,r>$ je tedy dobře uspořádaná,
tedy musí existovat $t$ r-nejmenší prvek množiny $m$. Máme pro všechna $z r t$,
platí že $f(z) = z$.
\begin{enumerate}
	\item $f(t) r t$ ale $f(t) r t$ máme $f(t) \neq t$, $f(f(t)) = f(t)$,
	\textbf{spor}: $f$ není prosté.
	\item $t r f(t)$: kdykoliv $z r t \Rightarrow f(z) r t$, protože $f(z) = z$.
	Navíc kdykoliv $t r z \Rightarrow f(t) r f(z)$ protože $f$ je izomorfismus.
	Tedy $t r f(t)$ , $t \in (\leftarrow, x) \Rightarrow t \neq \rng(f)$, tedy
	$f$ není zobrazení \textbf{na}, což je \textbf{spor}.
\end{enumerate}
\paragraph{Lemma 2}
Jsou-li $<a,r>$, $<b,s>$ dvě dobře uspořádané množiny, které jsou izomorfní, pak
mezi nimi existuje \textbf{jediný} izomorfismus.
\paragraph{Důkaz}
Sporem: Nechť $f, g: a \to b$ jsou dva různé izomorfismy. Tedy existuje nějaké 
$x \in a: f(x) \neq g(x)$. Tedy množina $m = \{ t \in a: f(t) \neq g(t) \}$ je
neprázdná (obsahuje $x$) a $<a,r>$ je dobře uspořádaná, tedy existuje nejmenší
prvek $t$ množiny $m$. Zřejmě platí, že kdykoliv $y r t$, pak $f(y) = g(y)$.
\begin{enumerate}
	\item $f(t) s g(t)$. Pokud $t r z$, protože $g$ je izomorfismus, musí platit,
	že $g(t) s g(z)$. Pokud $z r t$, pak $f(z) = g(z) \Rightarrow f(z) s f(t)
	\Rightarrow g(z) s f(t) \Rightarrow g(t) \neq f(t)$. Tedy $f(t) \nin
	\rng(g)$, tedy není \textbf{na}.
	\item $g(t) s f(t)$ analogicky.
\end{enumerate}
\paragraph{Věta}
Nechť $<a,R>$ a $<b,S>$ dvě dobře uspořádané množiny. Potom nastává právě jedna
z následujícíh možností:
\begin{enumerate}
	\item $<a,R> \cong <b,S>$ (je izomorfní)
	\item $\exists y \in b: \quad <a,R> \cong <(\leftarrow,y), S>$
	\item $\exists x \in a: \quad <(\leftarrow, x), R> \cong <b,S>$
\end{enumerate}
\paragraph{Důkaz}
Položme 
\begin{align}
f = \{ <v,w> : v \in a \land w \in b \land <(\leftarrow,v), R> \cong <(\leftarrow, w), S>\}
\end{align}
\begin{enumerate}
\item
$f$ je zobrazení: nechť $<v,w> \in f$, $<v,w_1> \in f$.
Máme:
\begin{align}
	<(\leftarrow, w), S) \cong <(\leftarrow, v), R> \cong <(\leftarrow, w_1), S>
\end{align}
tedy 
\begin{align}
	<(\leftarrow, w), S> \cong <(\leftarrow, w_1), S>
\end{align}
a podle Lemma 1 $w = w_1$.
\item
$f$ je prosté:
\begin{align}
	<v,w> \in f, <v_1,w> \in f \\
	<(\leftarrow, R> \cong <(\leftarrow, w),S> \cong <(\leftarrow, v), R>
\end{align}
a podle Lemma 1 $v = v_1$
\item
$f$ zachovává uspořádání:
\begin{align}
	<v,w> \in f, \quad <v_1, w_1> \in f
\end{align}
Nechť $v R v_1$. Máme $<(\leftarrow, v_1), R> \cong <(\leftarrow, w_1), S>$.
Nechť $g: <(\leftarrow, v_1), R> \to <(\leftarrow, w_1),S>$ je izomorfismus. 
Je $v R v_1$, $g(v)$ protože $g$ je izomorfismus:
\begin{align}
	<(\leftarrow,v), R> \cong <(\leftarrow, g(v)), S>
\end{align}
z definice $f$. Podle Lemma 2 existuje izomorfismus jediný, ktedy $w = g(v) S
w_1$. \\
Analogicky: pokud $w S w_1$, potom $v R v_1$. \\
Zřejmě platí, že pokud $<v,w> \in f$, pak $f \Gamma (\leftarrow, v)$ je izomorfismus
mezi $<(\leftarrow,v), R>$ a $<(\leftarrow,w), S>$.\\
Položme: 
\begin{align}
	m = \{ v \in a : \forall w \in b \quad <v,w> \nin f \}
	o = \{ w \in b : \forall v \in a \quad <v,w> \nin f \}
\end{align}
Můžou nastat případy:
\begin{enumerate}
	\item $m = o = \emptyset$.  Nastal případ, že $<a, R> \cong <b,S>$ podle $f$.
	\item $m = \emptyset \neq o$. Množina $<b,S>$ je dobře uspořádaná, tedy
	existuje $y \in b$, $y$ je S-nejmenší prvek množiny $o$. V tom případě $f$ je
	izomorfismus mezi $<a,R>$ a $<(\la, y), S>$.
	\item $m \neq \emptyset = o$. Existuje $x$ R-nejmenší prvek množiny $m$ a
	$<(\la, x), R> \cong (b,S)$ a $f$ je hledaný izomorfismus.
	\item $m \neq \emptyset \neq o$, což je ale ve sporu s definicemi $o$ a $m$.
\end{enumerate}
\end{enumerate}
\subsection{Ordinály}
\setcounter{equation}{0}
\paragraph{Definice}
Množina $x$ se nazývá \textbf{tranzitivní}, pokud platí 
\begin{align}
	\forall y: y \in x \Rightarrow y \subsetq x
\end{align}
\paragraph{Definice}
Množina $x$ je \textbf{ordinál}, pokud $x$ je tranzitivní a dobře uspořádaná
relací~$\in$.
\paragraph{Příklad}
0  je ordinál \\
\{ 0, \{0\}, \{\{0\}\}\} je tranzitivní, ale náležení neuspořádává - není
ordinál. \\
\{ 0, \{0\}, \{0, \{0\}, \{0, \{0 \} \} \} \} je ordinál, obvykle se značí 4.

\subsection{Věta o ordinálech}
\setcounter{equation}{0}
\begin{enumerate}
	\item Je-li $x$ ordinál, $y \in x$ a $y \in z$, pak $x \in z$.
	\item Jsou-li $x, y$ ordinály, pak $x \cong y$ právě když $x = y$
	\item Jsou-li $x, y$ ordinály, pak platí právě jedna z možností: $x = y$, $x
	\in y$, $y \in x$.
	\item Jsou-li $x,y, z$ ordinály, $x \in y \land y \in z \Rightarrow x \in z$.
	\item Je-li C neprázdná množina ordinálů, potom $\exists x \in C: \forall y
	\in C: y = x \lor x \in y$
\end{enumerate}








\end{document}

