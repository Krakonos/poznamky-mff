\documentclass[a4paper,12pt,titlepage]{article}
\usepackage[utf8]{inputenc}
\usepackage{a4wide}
\usepackage[czech]{babel}
\usepackage{amsfonts, amsmath, amsthm, amssymb}
\usepackage{math}

\title{Teorie množin}
\author{Ladislav Láska}

\begin{document}

\maketitle
\newpage
\tableofcontents
\newpage

\section{Formální jazyk}
\setcounter{equation}{0}
\subsection{Základní součásti jazyka}
\setcounter{equation}{0}
\begin{enumerate}
	\item proměnné
	\item binární predikátový symbol $\in$
	\item binární predikátový symbol $=$
	\item logické spojky $\neg \land \lor \Rightarrow \Leftrightarrow$
	\item kvantifikátory $(\forall x)$, $(\exists x)$
	\item pomocné symboly - závorky
\end{enumerate}
\subsection{Formule}
\setcounter{equation}{0}
\begin{enumerate}
	\item Nechť $x$, $y$ jsou prvky množiny, pak $(x \in y)$ a $(x = y)$ jsou
	atomické formule.
	\item Nechť výrazy $\varphi, \psi$ jsou formule, potom:
			$\neg \varphi$, $\varphi \land \psi$, $\varphi \lor \psi$,
			$\varphi \Rightarrow \psi$, $\varphi \Leftrightarrow \psi$
			jsou formule.
\end{enumerate}

\ \\
===============================================================\\
Tedy toho hodne chybi\\
===============================================================
\section{Axiomy teorie množin}

\subsection{Průnik a rozdíl množin}
\setcounter{equation}{0}
\paragraph{Definice}
Pro množiny a,b po řadě průnikem a rozdílem nazýváme množinu:
\begin{align}
	a \cup b = \{ x : x \in a \land x \in b \} \\
	a \setminus b = \{ x : x \in a \land x \nin b \}
\end{align}

Existuje množina a (Axiom existence), podle vydělení pro formuli $x \neq x$
existuje a podle extenziability je jediná množina $\{ x \in a : a \neq x \}$.


\paragraph{Definice}
$\emptyset$ je jediná množina y splňující:
\begin{align}
	(\forall x) (x \nin y)
\end{align}
A nazýváme jí \textbf{prázdná množina}.

\subsection{Disjunkní množina}
\setcounter{equation}{0}
\paragraph{Definice}
Říkáme, že množina a,b jsou disjunkní, že  je-li $a \cup b = 0$.

\paragraph{Lemma}
\begin{enumerate}
	\item $\neg (\exists y)(y \in \emptyset)$
	\item $(\forall x) (\emptyset \subset x)$
	\item $x \subset \emptyset \Leftrightarrow x = 0$
\end{enumerate}
\paragraph{Lemma}
\begin{align}
	(\forall a) a = \{ x : x \in a \land x = x \}
\end{align}


\subsection{Russelův paradox}
\setcounter{equation}{0}
\paragraph{Věta}
\begin{align}
	\neg ( \exists z) (\forall x) (x \in z)
\end{align}
\paragraph{Důkaz}
Sporem: nechť $z$ je taková množina. Pak mějme formuli $\varphi(x) \quad x \neq
x$. Potom podle axiomu vydělení pro tuto formuli máme $t = \{x \in z : x \nin x
\}$, tedy $t$ je množina. Protože $t$ je množina a $z$ je množina všech množin.
Protože $t \in z$, $t \in t \Leftrightarrow t \nin t$. Tedy neexistuje množina
všech množin.

\subsection{Axiom dvojce}
\setcounter{equation}{0}
\begin{align}
	(\forall a) (\forall b) (\exists z) (\forall x) ( x \in z \Leftrightarrow (x
	= a \lor x = b ))
\end{align}
\paragraph{Definice}
Jsou-li $a, b$ množiny, pak množinu se stávající z prvků $a,b$ nazveme
\textbf{neuspořádanou dvojcí} množin $a,b$ a značíme $\{a,b\}$. Pro $a\neq b$
říkáme, že $\{a,b\}$ dvouprvková, jinak jednoprvková.

\subsubsection{Rovnost množin}
\setcounter{equation}{0}
\paragraph{Lemma}
\begin{enumerate}
	\item $\{x\} = \{y\} \Leftrightarrow x = y$
	\item $\{x\} = \{x,y\} \Leftrightarrow x = y$
	\item $\{x,y\} = \{u,v\} \Leftrightarrow (x = u \land y = v) \lor (x = v
	\land y = u)$
\end{enumerate}

\subsubsection{Uspořádaná dvojce, k-tice}
\setcounter{equation}{0}
\textbf{Uspořádaná dvojce} množin $a,b$ je množina, která má prvky $\{\{a\}, \{a,b\}\}$.
Značíme jí $<a,b>$.
\paragraph{Lemma}
\begin{align}
	<x,y> = <u,v> \Leftrightarrow (x = u \land y = v)
\end{align}
\paragraph{Definice}
Jsou-li dány množiny $a_1, a_2, ..., a_k$, pak uspořádanou $k-tici$ definujeme
jako:
\begin{align}
	&<a_1> = a_1, \text{ a dál indukcí } \\
	&<a_1, a_2, ..., a_k > = <<a_1, ..., a_{k-1}>, a_k>
\end{align}
\paragraph{Lemma}
\begin{align}
	<a_1, ..., a_k> = <b_1, ..., b_k> \\
	\Leftrightarrow \\
	(a_1 = b_1) \land ... \land (a_k = b_k)
\end{align}


\subsection{Axiom sumy}
\setcounter{equation}{0}
\begin{align}
	(\forall a) (\exists z) (\forall x) (x \in z \Leftrightarrow (\exists y)(x
	\in y \land  y \in a))
\end{align}
\paragraph{Značení}
\begin{align}
	\bigcup a = \{ x : (\exists y) (y \in a \land x \in y) \}
\end{align}
\paragraph{Značení}
Nechť $a = \{b,c\}$. Pak $\bigcup a = b \cup c$

\subsubsection{Neuspořádané $k$-tice}
\setcounter{equation}{0}
\paragraph{Značení}
Neuspořádaná $k$-tice je:
\begin{align}
	\{a,b,c\} = \{a,b\} \cup \{c\}
\end{align}
\subsubsection{Průnik}
\setcounter{equation}{0}
\paragraph{Definice}
Pro neprázdnou množinu $a$ lze analogicky definovat
\begin{align}
	\bigcap a = \{ x : (\forall y) (y \in a \Rightarrow x \in y) \}
\end{align}
Pro neprázdnou a existuje $\bigcap a$: 
\begin{align}
	a \neq 0 \quad &(\exists x) x \in a, \quad x = x_0 \\
	a = 0 \quad &\bigcap a \text{ není definovaný }
\end{align}


\subsection{Schéma axiomu nahrazení}
\setcounter{equation}{0}
Je-li $\psi(u,v)$ formule, která neobsahuje volně proměnné $z,w$, potom formule:
\begin{align}
	\label{axiom-nahrazeni-predpoklad}
	(\forall u)(\forall v)(\forall w) ( \psi(u,v) \land \psi(u,w) \Rightarrow
	v=w) \Rightarrow \\
	(\forall a) (\exists z) (\forall v) (v \in z \Leftrightarrow (\exists u)(u
	\in a \land \psi(u,v)))
\end{align}
je axiom teorie množin.
\paragraph{Pozorování}
Pro jedno $u$, $\psi(u,v)$ platí pro nejvýše jedno $v$. To je analogie k funkci.
\paragraph{Definice}
Nechť $a,b$ jsou množiny. \textbf{Kartézský součin} $a \times b$ je množina:
\begin{align}
	a \times b = \left\{ <x,y> : x \in a \land y \in b \right\}
\end{align}
\paragraph{Důkaz}
$a \times b$ je množina. Zvolme a zafixujme $y \in b$ a
nechť $\psi(x,v)$ je formule $v = <x,y>$. Je-li:
\begin{align}
	\psi(x,v) \land \psi(x,w) \Rightarrow v = <x,y> \land w = <x,y> \Rightarrow
	v = w
\end{align}
Tedy je splněn předpoklad axiomu nahrazení (\ref{axiom-nahrazeni-predpoklad})
pro formuli $\psi$.
\begin{align}
	M_y = \{ <x,y> : x \in a\}
\end{align}
je množina podle nahrazení pro $\psi$ pro každé $y$. \\
Nechť navíc $\overline{\psi}(y,v)$ je formule $v = M_y$. Je-li:
\begin{align}
	\overline{\psi}(y,v) \land \overline{\psi}(y,w) \Rightarrow v = M_y \land w = M_y
	\Rightarrow v = w
\end{align}
Tedy je splněn předpoklad axiomu nahrazení (\ref{axiom-nahrazeni-predpoklad})
pro formuli $\overline{\psi}$. Navíc tedy
\begin{align}
	&D = \{ M_y : y \in b \} \text{ je množina } \\
	&\bigcup D = \{<x,y>: x \in a, y \in b \} = a \times b
\end{align}


\subsubsection{Binární relace}
\setcounter{equation}{0}
\paragraph{Definice}
\textbf{Binární relace} je množina $R$, jejímiž prvky jsou uspořádané dvojce.
\begin{align}
	\dom(R) = \{ x : (\exists y) <x,y> \in R \} \text{ je definiční obork }
	\rng(R) = \{ y : (\exists x) <x,y> \in R \} \text{ je obor hodnot }
\end{align}
Protože $R$ je množina, $\dom(R)$ i $\rng(R)$ jsou množiny.
\paragraph{Definice}
Je-li $R$ relace, definujeme:
\begin{align}
	R^{-1} = \{ <x,y>: <y,x> \in R \}
\end{align}
Pro každou relaci $R$, $R^{-1}$ je relace a $(R_{-1})^{-1} = R$.
\paragraph{Definice}
Jsou-li $R, S$ relace, pak
\begin{align}
R \circ S = \{ <x,z> : (\exists y) <x,y> \in R \land <y,z> \in S \}
\end{align}
\paragraph{Definice}
Jsou-li $R, S, T$ relace, pak
\begin{align}
	(T \circ S) \circ R = T \circ (S \circ R)
\end{align}

\subsubsection{Funkce}
\setcounter{equation}{0}
Množina $f$ se nazývá \textbf{funkce}, pokud $f$ je relace a platí:
\begin{align}
	(\forall x \in \dom(f))((y \in \rng(f) \land y \in \rng(f) \land <x,y> \in f
	\land <x,y'> \in f) \Rightarrow y = y')
\end{align}
\paragraph{Značení} $f: A \to B$ znamená: $f$ je funkce, $A = \dom(f)$, $B \supset
\rng(f)$.


Je-li $C \subsetq A$, pak $f \upharpoonright C = f \cap (C \times B)$ nazýváme $x$ zůžením funkce
$f$ na množinu $C$.
\begin{align}
f'C = \rng(f \upharpoonright C) = \{ f(x): x \in C \}
\end{align}
\begin{description}
\item Funkce $f: A \to B$ se nazývá \textbf{prostá}, pokud $f^{-1}$ je funkce.
\item Funkce $f: A \to B$ se nazývá \textbf{surjektivní} ("na"), jestliže $B = \rng(f)$
\item Funkce $f$ se nazývá \textbf{bijekce} je-li \textbf{surjektivní} a současně
\textbf{prostá}.
\end{description}


\subsection{Uspořádání}
\setcounter{equation}{0}
\paragraph{Definice}
\textbf{Ostře uspořádaná množina} je uspořádaná dvojce $<a, r>$, kde $a$ je
množina a $r$ je relace, $r \subsetq a \times a $. Přičemž r splňuje:
\begin{align}
	&\forall x, y, z \in a : \quad <x,y> \in r \land <y,z> \in r \Rightarrow
	<x,z> \in r \quad \text{tranzitivita} \\
	&\forall x \in a: \quad \not <x,x> \in r \quad \text{antireflexivita}
\end{align}
Pro zjednodušení místo $<x,y> \in r$ píšeme $x r y$.
\paragraph{Definice}
Ostré uspořádání $r$ nazveme \textbf{lineárním}, pokud 
\begin{align}
	\forall x,y \in a: \quad x = y \lor x r y \lor y r x 
\end{align}
\paragraph{Definice}
Jsou-li $R$, $S$ relace a $a$, $b$ množiny, pak řekneme, že $<a, R>$ je izomorfní
s $<b, S>$, pokud existuje bijekce $f: a\to b$ taková, že
\begin{align}
	\forall x,y \in a: \quad <x,y> \in \R \Leftrightarrow <f(x), f(y)> \in S
\end{align}
a zobrazení $f$ se nazývá \textbf{izomorfismus}.
\paragraph{Definice}
Mějme uspořádanou množinu $<a,r>$. Je-li $m \subset a$, pak řekneme, že $x \in a$
je \textbf{r-nejmenší} prvek množiny $m$, jestliže platí:
\begin{align}
	x \in m \land (\forall Y) (y \in m \Rightarrow (x r y \lor y = x))
\end{align}
Je-li $m \subsetq a$, $x \in a$, řekneme, že $x$ je \textbf{minimální} prvek
množiny $m$,
jestliže platí
\begin{align}
	x \in m \land (\forall y) ( y \in m \Rightarrow \neg (y r x))
\end{align}
\paragraph{Definice}
Řekneme, že uspořádání $r$ na množině a je \textbf{dobré} (množina $<a,r>$ je dobře
uspořádaná) jesltiže $r$ je ostré uspořádání množiny $a$ a každá neprázdná
podmnožina $a$ má r-nejmenší prvek.
\paragraph{Pozorování}
Je-li $<a,r>$ dobře uspořádaná, pak je $r$ lineární uspořádání.
$x,y \in a \quad \{x,y\} \subsetq a$ a $\{x,y\}$ má r-nejmenší prvek. Je-li to
$x$,
pak $x r y \lor x = y$. Pokud je to $y$, pak $y r x \lor y = x$.
\paragraph{Značení}
Nechť $<a,r>$ je uspořádaná množina a $x \in a$. Označme $<(\leftarrow, x), r>$
jako:
\begin{align}
	(\leftarrow, x) = \{ y \in a : y r x \}
\end{align}
\paragraph{Lemma 1}
Je-li $<a,r>$ dobře uspořádaná množina, pak pro každé $x \in a \quad <a, r>$ není
izomorfní s $<(\leftarrow, x), r>$
\paragraph{Důkaz}
Sporem: Předpokládejme, že existuje izomorfismus $f: <a,r> \to <(\leftarrow, x),
r>$. Definujme $m = \{ y \in a: f(y) \neq y \}$. $x \neq (\leftarrow, x)$, tedy
$f(x) \neq x\Rightarrow m \neq \emptyset$. $<a,r>$ je tedy dobře uspořádaná,
tedy musí existovat $t$ r-nejmenší prvek množiny $m$. Máme pro všechna $z r t$,
platí že $f(z) = z$.
\begin{enumerate}
	\item $f(t) r t$ ale $f(t) r t$ máme $f(t) \neq t$, $f(f(t)) = f(t)$,
	\textbf{spor}: $f$ není prosté.
	\item $t r f(t)$: kdykoliv $z r t \Rightarrow f(z) r t$, protože $f(z) = z$.
	Navíc kdykoliv $t r z \Rightarrow f(t) r f(z)$ protože $f$ je izomorfismus.
	Tedy $t r f(t)$ , $t \in (\leftarrow, x) \Rightarrow t \neq \rng(f)$, tedy
	$f$ není zobrazení \textbf{na}, což je \textbf{spor}.
\end{enumerate}
\paragraph{Lemma 2}
Jsou-li $<a,r>$, $<b,s>$ dvě dobře uspořádané množiny, které jsou izomorfní, pak
mezi nimi existuje \textbf{jediný} izomorfismus.
\paragraph{Důkaz}
Sporem: Nechť $f, g: a \to b$ jsou dva různé izomorfismy. Tedy existuje nějaké 
$x \in a: f(x) \neq g(x)$. Tedy množina $m = \{ t \in a: f(t) \neq g(t) \}$ je
neprázdná (obsahuje $x$) a $<a,r>$ je dobře uspořádaná, tedy existuje nejmenší
prvek $t$ množiny $m$. Zřejmě platí, že kdykoliv $y r t$, pak $f(y) = g(y)$.
\begin{enumerate}
	\item $f(t) s g(t)$. Pokud $t r z$, protože $g$ je izomorfismus, musí platit,
	že $g(t) s g(z)$. Pokud $z r t$, pak $f(z) = g(z) \Rightarrow f(z) s f(t)
	\Rightarrow g(z) s f(t) \Rightarrow g(t) \neq f(t)$. Tedy $f(t) \nin
	\rng(g)$, tedy není \textbf{na}.
	\item $g(t) s f(t)$ analogicky.
\end{enumerate}
\paragraph{Věta}
Nechť $<a,R>$ a $<b,S>$ dvě dobře uspořádané množiny. Potom nastává právě jedna
z následujícíh možností:
\begin{enumerate}
	\item $<a,R> \cong <b,S>$ (je izomorfní)
	\item $\exists y \in b: \quad <a,R> \cong <(\leftarrow,y), S>$
	\item $\exists x \in a: \quad <(\leftarrow, x), R> \cong <b,S>$
\end{enumerate}
\paragraph{Důkaz}
Položme 
\begin{align}
f = \{ <v,w> : v \in a \land w \in b \land <(\leftarrow,v), R> \cong <(\leftarrow, w), S>\}
\end{align}
\begin{enumerate}
\item
$f$ je zobrazení: nechť $<v,w> \in f$, $<v,w_1> \in f$.
Máme:
\begin{align}
	<(\leftarrow, w), S) \cong <(\leftarrow, v), R> \cong <(\leftarrow, w_1), S>
\end{align}
tedy 
\begin{align}
	<(\leftarrow, w), S> \cong <(\leftarrow, w_1), S>
\end{align}
a podle Lemma 1 $w = w_1$.
\item
$f$ je prosté:
\begin{align}
	<v,w> \in f, <v_1,w> \in f \\
	<(\leftarrow, R> \cong <(\leftarrow, w),S> \cong <(\leftarrow, v), R>
\end{align}
a podle Lemma 1 $v = v_1$
\item
$f$ zachovává uspořádání:
\begin{align}
	<v,w> \in f, \quad <v_1, w_1> \in f
\end{align}
Nechť $v R v_1$. Máme $<(\leftarrow, v_1), R> \cong <(\leftarrow, w_1), S>$.
Nechť $g: <(\leftarrow, v_1), R> \to <(\leftarrow, w_1),S>$ je izomorfismus. 
Je $v R v_1$, $g(v)$ protože $g$ je izomorfismus:
\begin{align}
	<(\leftarrow,v), R> \cong <(\leftarrow, g(v)), S>
\end{align}
z definice $f$. Podle Lemma 2 existuje izomorfismus jediný, ktedy $w = g(v) S
w_1$. \\
Analogicky: pokud $w S w_1$, potom $v R v_1$. \\
Zřejmě platí, že pokud $<v,w> \in f$, pak $f \upharpoonright (\leftarrow, v)$ je izomorfismus
mezi $<(\leftarrow,v), R>$ a $<(\leftarrow,w), S>$.\\
Položme: 
\begin{align}
	m = \{ v \in a : \forall w \in b \quad <v,w> \nin f \}
	o = \{ w \in b : \forall v \in a \quad <v,w> \nin f \}
\end{align}
Můžou nastat případy:
\begin{enumerate}
	\item $m = o = \emptyset$.  Nastal případ, že $<a, R> \cong <b,S>$ podle $f$.
	\item $m = \emptyset \neq o$. Množina $<b,S>$ je dobře uspořádaná, tedy
	existuje $y \in b$, $y$ je S-nejmenší prvek množiny $o$. V tom případě $f$ je
	izomorfismus mezi $<a,R>$ a $<(\la, y), S>$.
	\item $m \neq \emptyset = o$. Existuje $x$ R-nejmenší prvek množiny $m$ a
	$<(\la, x), R> \cong (b,S)$ a $f$ je hledaný izomorfismus.
	\item $m \neq \emptyset \neq o$, což je ale ve sporu s definicemi $o$ a $m$.
\end{enumerate}
\end{enumerate}


\section{Ordinály}
\setcounter{equation}{0}
\paragraph{Definice}
Množina $x$ se nazývá \textbf{tranzitivní}, pokud platí 
\begin{align}
	\forall y: y \in x \Rightarrow y \subsetq x
\end{align}
\paragraph{Definice}
Množina $x$ je \textbf{ordinál}, pokud $x$ je tranzitivní a dobře uspořádaná
relací~$\in$.
\paragraph{Příklad}
0  je ordinál \\
\{ 0, \{0\}, \{\{0\}\}\} je tranzitivní, ale náležení neuspořádává - není
ordinál. \\
\{ 0, \{0\}, \{0, \{0\}, \{0, \{0 \} \} \} \} je ordinál, obvykle se značí 4.

\subsection{Věta o ordinálech}
\setcounter{equation}{0}
\begin{enumerate}
	\item Je-li $x$ ordinál, $y \in x$ a $y \in z$, pak $x \in z$.
	\item Jsou-li $x, y$ ordinály, pak $x \cong y$ právě když $x = y$
	\item Jsou-li $x, y$ ordinály, pak platí právě jedna z možností: $x = y$, $x
	\in y$, $y \in x$.
	\item Jsou-li $x,y, z$ ordinály, $x \in y \land y \in z \Rightarrow x \in z$.
	\item Je-li C neprázdná množina ordinálů, potom $\exists x \in C: \forall y
	\in C: y = x \lor x \in y$
\end{enumerate}
\paragraph{Důkaz}
\begin{enumerate}
	\item Je-li $x$ ordinál, $y \in x$ a $y \in z$, pak y je ordinál a $y =
	<(\la, y), \in> \in x$. \\
	Víme, že: $x$ je ordinál, $x$ je tranzitivní množina, $t \in y$, $y \in x$. Tedy
	$t \in x$. $x$ je tranzitivní množina, $t \in x$, $u \in t$, tedy $u \in x$.
	V množině $x$ máme $u, t, y \in x$, $x$ je uspořádané relací náležení a máme $u
	\in t$, $t \in y$. Tedy $u \in y$. $y$ je tedy tranzitivní množina. \\
	$y$ je relací náležení uspořádaná: Nechť $u, v, w \in y$, $ u \in v \land v
	\in w$. $y \in x$, protože $x$ je tranzitivní množina, $u,v,w \in x$. Přitom
	$u \in v \land v \in w$, $x$ je relací náležení uspořádaná, tedy $u \in w$.
	$y$
	je tedy relací náležení uspořádaná dobře. Nechť $m \subset y$ je neprázdná
	množina, kdykoliv $t \in m$, pak $t \in y$, $x$ je tranzitivní, tedy $m
	\subsetq x, m \neq \emptyset$. Protože $x$ je dobře uspořádaná, existuje $z
	\in m$ nejmenší prvek množiny $m$ v $x$. Ale $m \subsetq y$, tedy $z$ je
	nejmenší prvek i v $y$. \\
	Zbývá dokázat, že $y = <(\la, y), \in>$ v $x$: $t \in y$, protože $y \in x$,
	je $t \in x$ a $t \in y$. Tedy $t \in (\la, y)$. A naopak: $t \in (\la, y)$
	v $x$. Množina $x$ je uspořádaná operací náležení, tedy $t \in y$.
	Dostáváme, že $(\la, y) \subsetq y$.

	\item Jsou-li $x, y$ ordinály, a platí $x \cong y$ pak $x = y$. Nechť $h: (x,
	\in \to (y, \in)$ je izomorfismus. Položme $m = \{ z \in x: h(z) \neq z \}$.
	Pokud $m = \emptyset$, jsme hotovi. Pro spor předokládejme, že $m \neq
	\emptyset$. V tom případě existuje $t \in m$, t nejmenší prvek množiny m.
	Protože h je izomorfismus, platí pro $c,d \in x$:
	\begin{align}
		c \in d \eq h(c) \in h(d)
	\end{align}
	Tedy speciálně
	\begin{align}
		z \in t \eq h(z) \in h(t)
	\end{align}
	Máme ($t$ nejmenší prvek množiny $m$)
	\begin{align}
		z \in t \eq z \in h(t)
	\end{align}
	$t = h(t)$, spor s předpokladem $t \in m$.

	\item Jsou-li $x,y$ ordinály, pak platí právě jedna z následujících
	možností: $x = y$, $x \in y$, $y \in x$. Podle věty o izomorfismu dobrých
	uspořádání buď $x \cong y$ a ale podle 2. $\Rightarrow x = y$, nebo $x \cong
	(\la, z)$ a $x \cong z \in y$, nebo $y \cong (\la, t)$, tedy $x \in y \in x$.


	\item Jsou-li $x, y, z$ ordinály, $x \in y$ a $y \in z$, potom $x \in z$.
	Protože $z$ je tranzitivní množina.
	
	\item Je-li C neprázdná množina ordinálů, pak existuje $x \in C$ tak, že
	\begin{align}
		(\forall y \in C) (x = y) \lor (x \in y)
	\end{align}
	$C \neq \emptyset$, tedy můžeme zvolit  $t \in C$. Pak $t$ je nejmenší prvek
	množiny $C$ a jsme hotovi. V opačném případě existuje $y \in C$, že $y \in
	t$. Tedy $D = \{ y \in C : y \in t \} \neq \emptyset$. $t$ je ordinál: $D \neq
	\emptyset$, $D \subsetq t$, tedy existuje $x \in D$ nejmenší prvek množiny
	$D$. Nechť tedy $y \in t$, tedy $x = y \lor x \in y$; nebo $y = t$, tedy $x
	\in t$; nebo $t \in y$, pak $x \in t$, $t \in y$ dává $x \in y$. Tedy $x$ je nejmenší
	prvek množiny $C$.
\end{enumerate}


\subsection{Neexistence množiny všech ordinálů}
\setcounter{equation}{0}
\paragraph{Věta}
Neexistuje množina všech ordinálů:
\begin{align}
	\neg (\exists z) (\forall x) (x \text{ je ordinál } \Rightarrow x \in z)
\end{align}
Sporem: Nech množina $z$ existuje. Podle vydělení pro formuli "$x$ je ordinál"
existuje $m = \{ x :  x$ je ordinál $ \}$. Podle věty o ordinálch je m:
tranzitivní, ostře uspořádaná relací náležení a to uspořádání je dobré. Podle
definice $m$ je ordinál, máme $m \in m$. Což je spor s bodem 3. věty o
ordinálech.

\subsection{Lemma o tranzitivitě a ordinalitě}
\setcounter{equation}{0}
\paragraph{Lemma}
Je-li $a$ tranzitivní množina ordinálů, pak $a$ je ordinál.
\paragraph{Důkaz}
Stačí ukázat:
\begin{enumerate}
\item 
 náležení je dobré uspořádání na množině $a$. Mějme $x ,y, z \in
a$: $x \in y$, $y \in z$. Ale $x, y, z$ jsou ordinály: podle věty o ordinálech
(bod 4.) $x \in z$.
\item uspořádání je lineární (bod 3.)
\item uspořádání je dobré (bod 5.)
\end{enumerate}

\subsection{}
\setcounter{equation}{0}
\paragraph{Věta}
Je-li $<A, R>$ dobře uspořádaná množina. Pak existuje právě jeden ordinál $c$
tak, že $<A,R> \cong <c, \in>$.
\paragraph{Důkaz}
\begin{enumerate}
	\item Unicita: Nechť $<A, R> \cong <d, \in>$. Dostáváme, že $c \cong d$ a
	podle věty o ordinálech $c = d$.
	\item Existence: Položme $B = \{ a \in a : <(\la, R>\cong x> $ pro nějaký
	ordinál x $\}$. Nechť navíc $f$ je funkce $dom(f) = B$ a splňuje 
	\begin{align}
		(\forall a \in B) f(a) \text{ je ordinál takový, že } < (\la, a), R>
		\cong <f(a), \in>
	\end{align}
	Položme $c = \rng(f)$: $c$ je množina (nahrazení pro formuli "$<(\la, a,
	R>\cong x$"). Podle předchozího lemmatu je $c$ ordinál. Tedy $f$ je
	izomorfismus $B \to c$. Pokud $B = A$, jsme hotovi. Jinak existuje $b \in A
	: B = (\la, b)$ a tedy $f$ je izomorfismus mezi $<B, R>$ a $<c, \in>$. Tedy
	i $f(b)$ je definováno, ačkoli $b \nin B$, což je spor.
\end{enumerate}



\subsection{}
\setcounter{equation}{0}
\paragraph{Definice}
Je-li $<A, R>$ dobře uspořádaná množina, pak typ $<A, R>$ je jediný ordinál $c$, pro
který $<A, R> \cong c$. 
\paragraph{Příklad}
$A = \{ \sqrt{2}, \pi, 6, 7\} \cong 4$
\paragraph{Značení} Malá řecká písmena $\alpha, \beta, \gamma, ...$ je ordinál.
Přičemž nahradíme:
\begin{align}
	\alpha < \beta \text{ za } \alpha \in \beta \\
	\alpha \le \beta \text{ za } (\alpha \in \beta) \lor (\alpha = \beta)
\end{align}
\paragraph{Definice}
Je-li $X$ množina ordinálů, označme:
\begin{align}
	\sup(X) = \bigcup X \\
	\text{ pro } X \neq \emptyset \text{ označme } \min(X) = \bigcap X
\end{align}
\paragraph{Lemma}
\begin{enumerate}
	\item 
	Pro ordinály $\alpha, \beta$ platí
	\begin{align}
		\alpha \le \beta \eq \alpha \subsetq \beta
	\end{align}
	\item 
	Je-li $X$ množina ordinálů, prvek $\sup(X)$ je nejmenší ordinál, který je
	větší nebo roven všem prvkům z $X$ pokud $X \neq \emptyset$. Prvek $\min(X)$
	je nejmenší ordinál v množině X.
\end{enumerate}
\paragraph{Důkaz}
\begin{enumerate}
	\item \ \\
	\item Podle axiomu sumy $X$ je množina, tedy $\bigcup X$ je množina.
	$\bigcup X$ je ordinál:
	\begin{enumerate}
		\item je-li $x \in \bigcup X$, $y \in x$, musí podle axiomu sumy
		existovat $t \in X: x \in t$. Máme $x \in t$, $y \in x$, $t$ je ordinál:
		$y \in t$. Znova podle axiomu sumy $y \in \bigcup X$. Tedy $\bigcup X$
		je tranzitivní množina, je to množina ordinálů podle minulého lemmatu.
		\item $\forall x \in X: \quad x \le \bigcup X$ \\
		Nechť $x \in X$ libovolné, podle věty o ordinálech nastává právě
		jedna z možností
		\begin{align}
			x \in \bigcup X, x = \bigcup X, \bigcup X \in x
		\end{align}
		Pokud
			$\bigcup X \in x$, máme $x = X$, pak $x \in \bigcup X$, 
				tedy $\bigcup x \bigcup X$, spor s ordinalitou $\bigcup X$.


		\item $\bigcup X$ je nejmenší mez. Buď $t < \bigcup X$, tedy $t \in
		\bigcup X$. Stačí ukázat, že t není horní mezí množiny X. Protože $t \in
		\bigcup X$, existuje $y \in X$, že $t \in y$. Pro toto $y$  platí $t <
		y$, tedy $t$
		není horní mezí množiny $X$.
	\end{enumerate}
	Nechť $X \neq \emptyset$ máme dokázat, že $\min(X) = \bigcap X$ je ordinál a
	je nejmenší ze všech ordinálů v $X$. $\bigcap X$ je množina je-li $t \in
	\bigcap X$ a je-li $y \in t$, můžeme zvolit libovolné $x \in X$, je $t \in x$,$y
	\in t$, $x$ ordinál, tedy $y \in x$. Tedy $\bigcap X$ je tranzitivní množina, podle
	předchozího lemmatu je $\bigcap X$ ordinál. \\
	Zbývá dokázat, že $\bigcap X \in X$. X je neprázdná množina ordinálů, podle
	věty o ordinálech (bod 5) existuje nejmenší prvek množiny $x \in X$. Pro
	takové x platí, že kdykoliv $y \in x$, pak $x = y$ nebo $x \in y$.
	\begin{align}
		x = \{t: t \in x \} \subsetq y \text{ pro každé } y \in X \\
		x \le \bigcap X
	\end{align}
	Opačná rovnost $x \ge \bigcap X$ je zřejmá, nebo $\bigcap X \subsetq$ platí
	pro všechna $y \in X$.
\end{enumerate}

\subsection{}
\setcounter{equation}{0}
\paragraph{Definice}
Pro ordinál $\alpha$ je jeho \textbf{ordinální následník} $s(\alpha) = \alpha
\cup \{\alpha\}$.
\paragraph{Lemma}
Pro ordinál $\alpha$ je $s(\alpha)$ též ordinál, $\alpha < s(\alpha)$ a
\begin{align}
	(\forall \beta) (\beta \text{ ordinál } \Rightarrow ( \beta < s(\alpha)
	\eq \beta \le \alpha))
\end{align}
\paragraph{Důkaz}
Je-li $x \in s(\alpha)$, pak buď $x \in \alpha$ nebo $x \in \{\alpha\}$ z
definice. Což je po řadě $x < \alpha$ a $x = \alpha$.
\paragraph{Definice}
Ordinál $\alpha$ se nazývá \textbf{izolovaný}, jestliže $\alpha = \emptyset$ nebo
$\exists \beta$ ordinál a $\alpha = s(\beta)$.
\paragraph{Definice}
Ordinál $\alpha$ se nazývá \textbf{limitní}, jestliže $\alpha \neq \emptyset$ a není
izolovaný.
\paragraph{Definice}
$1 = s(0), 2 = s(1), 3 = s(2), ...$
\paragraph{Definice}
Ordinál $\alpha$ je přirozené číslo, jestliže platí
\begin{align}
	(\forall \beta)(\beta \le \alpha \Rightarrow \beta \text{ je izolovaný
	ordinál })
\end{align}


\subsection{Množina všech přirozených čísel}
\setcounter{equation}{0}
\paragraph{Tvrzení}
Podle axiomu nekonečna:
\begin{align}
	(\exists x) ( \emptyset \in x \land (\forall y) ( y \in x \Rightarrow s(y)
	\in x))
\end{align}
\paragraph{Pozorování}
Množina $x$, zaručená axiomem nekonečna, obsahuje všechna přirozená čísla.
\paragraph{Důkaz}
Sporem: $\exists n$ přirozené číslo takové, že $n \nin x$. Určitě $n \neq 0$
podle axiomu nekonečna. Tedy $\exists m: n = s(m)$. Je $m \in x$? Ne, kdyby bylo
$m \in x$, pak i $n = s(m)$ splňuje $m \in x$, což je ve sporu s předpokladem. 
$n$ je tedy přirozené číslo, tedy
ordinál. Množina $x \setminus n$ je neprázdná, nebo $m \in x\setminus n$. 
Protože $n$ je
dobře uspořádaná a $x \setminus n$ je neprázdná, existuje nejmenší prvek $\tilde{n}
\in x \setminus n$. 
\begin{enumerate}
\item $\tilde{n} = 0$ - spor s axiomem nekonečna.
\item $\tilde{n} \neq 0$, $\exists \tilde{m} \quad \tilde{n} = s(\tilde{m})$.
Protože $\tilde{n} > \tilde{m}$ musí být $\tilde{m} \in x$. Podle axiomu nekonečna
$s(\tilde{m}) = \tilde{n} \in x$. Což je spor.
\end{enumerate}
\paragraph{Definice}
$\omega$ je množina všech přirozených čísel. $\omega$ je ordinál (podle Lemma
3). Všechny menší ordinály než $\omega$ jsou izolované, $\omega$ sama je limitní
(a to dokonce nejmenší).
\paragraph{Poznámka}
Existuje, axiom nekonečna a vydělení pro formuli "n je přirozené číslo".
\paragraph{Věta} (Peanovy axiomy)
\begin{enumerate}
	\item $0 \in \omega$
	\item $(\forall n \in \omega )(s(n) \in \omega)$
	\item $(\forall n,m \in \omega)(n \neq m \impl s(n) \neq s(m))$
	\item (indukce) $\forall X \subsetq \omega$
	\begin{align}
		((0 \in X \land (\forall n \in X)(s(n) \in X))\impl X = \omega).
	\end{align}
\end{enumerate}
\paragraph{Důkaz}
Plyne z věty o ordinálech. Ve 4 předpokládeme ke sporu, že $X \neq \omega$, tedy
$\omega \setminus X$ je neprázdná množina ordinálů. Tedy má nejmenší prvek $n$.
Pokud $n = 0$ - spor, jinak $n \neq 0$, tedy $n = s(m), m \in X$ - spor.
\paragraph{Definice}
Nechť $\alpha$, $\beta$ jsou ordinály.
\begin{align}
	\alpha + \beta = typ<\alpha \times \{0\} \cup \beta \times \{1\}, R>
\end{align}
kde
\begin{align}
	R = \{ << \xi, 0>, <\ni,0>> : \xi < \ni < \alpha \} \cup \\
	\{<<\xi, 1>,<\ni,1>> : \xi < \ni < \beta \} \cup \\
	\{((\alpha\times\{0\}): (\beta\times\{1\})\}
\end{align}
\paragraph{Věta}
Pro libovolné ordinály $\alpha, \beta, \gamma$
\begin{enumerate}
	\item $\alpha + (\beta + \gamma) = (\alpha + \beta) + \gamma$
	\item $\alpha + 0 = \alpha$
	\item $\alpha + 1 = s(\alpha)$
	\item $\alpha + s(\beta) = s(\alpha + \beta)$
	\item Je-li $\beta$ je limitní ordinál, pak $\alpha + \beta = \sup\{\alpha +
	\xi : \xi < \beta \}$
\end{enumerate}
\paragraph{Důkaz}
Triviální z definice.
\paragraph{Poznámka}
Pozor, ordinální sčítání není obecně komutativní.
\paragraph{Definice}
Pro ordinály $\alpha, \beta: \alpha \cdot \beta = typ < \beta \times \alpha,
R>$, kde $R$ je lexikografické uspořádání součinu $\beta\times\alpha$, tedy:
\begin{align}
	<<\xi,\ni>, <\xi',\ni'>> \in R \eq (\xi < \xi') \lor (\xi = \xi' \land \ni <
	\ni')
\end{align}
\paragraph{Věta}
Pro libovolné ordinály $\alpha, \beta, \gamma$ platí:
\begin{enumerate}
	\item $\alpha\cdot(\beta\cdot \gamma) = (\alpha\cdot\beta)\cdot\gamma$
	\item $\alpha \cdot 0 = 0$
	\item $\alpha \cdot 1 = \alpha$
	\item $\alpha\cdot s(\beta) = \alpha\cdot\beta + \alpha$
	\item Je-li $\beta$ limitní ordinál, pak $\alpha\cdot\beta = \sup\{\alpha\cdot
\xi : \xi < \beta \}$
\item $\alpha\cdot(\beta+\gamma) = \alpha\cdot\beta + \alpha\cdot\gamma$
\end{enumerate}
\paragraph{Poznámka}
Pozor: \begin{align}
\omega\cdot 2 = \omega + \omega \\
2\cdot\omega = \omega
\end{align}


\section{Kardinály}
\setcounter{equation}{0}
\paragraph{Definice}
Nechť $a,b$ jsou množiny. 
\begin{enumerate}
	\item Řekneme, že mohutnost množiny $a$ je \textbf{menší nebo rovna} mohutnosti množiny
	$b$ (značíme $a \preceq b$), jestliže existuje zobrazení $f: a \to b$.
	\item Řekneme, že mohutnost množiny $a$ je \textbf{rovna} mohutnosti množiny
	$b$ (značíme $a \approx b$) pokud existuje bijekce $f:a\to b$.
	\item Řekneme, že mohutnost množiny a je \textbf{ostře menší} mohutnosti množiny $b$
	(značíme $a \prec b$)
	právě když $a \preceq b \land \not(a \approx b)$.
\end{enumerate}
\paragraph{Věta}
\begin{enumerate}
	\item $x \approx x$
	\item $x \approx y \impl y \approx x$
	\item $(x \approx y \land y \approx z) \impl x \approx z$
	\item $x \preceq x$
	\item $(x \preceq y \land y \preceq z) \impl x \preceq z$
\end{enumerate}
\paragraph{Věta}(Kantor-Bernstein) Pro množiny $a,b$
\begin{align}
	(a \preceq b \land b \preceq a) \impl a \approx b
\end{align}
\paragraph{Značení}
$g''b = g[b] = \{g(x) : x \in b\}$
\paragraph{Důkaz}
Mějme $f: a \to b$ prosté zobrazení a $g: b \to a$ prosté zobrazení. Pokud f
nebo g bijekce, je věta dokázána - nadále tedy předpokládejme, že $f''a \neq b
\land f''b \neq a$. \\
(sem vlozit obrazek dvou funkci)\\
Pro všechna přirozená čísla definujeme indukcí $a_0 = a$, $b_0 = b$. $a_{n+1} =
g''b_n, b_{n+1} = f''a_n$. Tedy $a_1 = g''b_0 = g''b \subsetneq a = a_0 $.
Analogicky $b_1 = f''a_0 = f''a \subsetneq b = b_0 $.\\
Označme 
\begin{align}
a_\omega = \bigcap\{a_n: n \in \omega\}
b_\omega = \bigcap\{b_n: n \in \omega\} 
\end{align}
Zobrazení $h: a \to b$ definujme předpisem 

$h(x) = f(x)$ pro $x \in
\bigcup_{n\in\omega} a_{2n} \setminus a_{2n+1} \cup a_\omega$ 
$h(x) = t$, kde $t \in b$ a $g(t) = x$ pro 
$x \in \bigcup_{n\in\omega} a_{2n+1} \setminus a_{2n+2}$

$h: a \to b$ je hledaná bijekce. Je zřejmé, že h je funkce a $\dom(h) = a$.

\begin{enumerate}

\item $h$ je prosté: Nechť $x\neq y$, x,y in a. Pokud 
\begin{align}
x, y \in \bigcup_{n\in\omega} a_{2n} \setminus a_{2n+1} \cup a_\omega
\end{align}
mějme $h(x) = f(x)$ , $h(y) = f(y)$, a $f$ je prostá tedy $f(x) \neq f(y)$.

Pokud $x,y \in \bigcup_{n\in\omega} a_{2n+1}\setminus a_{2n+2}$ \\
$h(x) = g^{-1}(x)$, $h(y) = g^{-1}$, $g$ je zobrazení, tedy $h(x) \neq h(y)$.

$x \in \bigcup_{n\in\omega} a_{2n} \setminus a_{2n+1}$, $y \in \bigcup{n\in\omega}
a_{2n+1} \setminus a_{2n+2}$
\begin{align}
	\exists n \quad x \in a_{2n} \setminus a_{2n+1}: h(x) = b_{2n+1} \setminus
		b_{2n+2} \\
	\exists m \quad y \in a_{2m+1} \setminus a_{2m+2}: h(y) = b_{2m} \setminus
		b_{2m+1} \\
	\emptyset = (b_{2n+1} \setminus b_{2n+2}) \cap (b_{2m} \setminus b_{2m+1})
	\\
	x \in a_\omega \quad\lor\quad h(x) \in b_\omega
\end{align}

\item $h$ je surjektivní: $t \in b$. \\
\begin{align}
	t \in b_{2m} \setminus b_{2n+1}
\end{align}
Pak $g(t) \in a_{2n+1} \setminus a_{2n+2}$. Pro $x = g(t)$ máme $h(x) = t$.
Nebo:
\begin{align}
	t \in b_{2n+1} \setminus b_{2n+2}
\end{align}
Pak $b_{2n+1} = f''a_{2n}$ a $\exists  x \in a_{2n} f(x) = t$, $h(x) = t)$.
Nebo:
\begin{align}
	t \in b_\omega \subsetq b_0 = f''a
\end{align}
Existuje takové $x$, že $f(x) = t$. Pro toto $x$ je $x \in a_\omega$. $h(x) = t$.
\end{enumerate}
\paragraph{Definice}
Nechť $A$ je množina. Pokud na $A$ existuje dobré uspořádání, pak položme $|A| =$
nejmenší ordinál $\alpha$, pro který $A \approx \alpha$.
\paragraph{Definice}
Ordinál $\alpha$ se nazývá \textbf{kardinál} pokud $\alpha = |\alpha|$.
Ekvidalentně ordinál $\alpha$ je kardinál, právě když 
\begin{align}
	(\forall \beta)(\beta < \alpha \impl \not (\beta \approx \alpha))
\end{align}
\paragraph{Pozorování}
$\omega$ je kardinál. $\omega + k$ není kardinál (všechny jsou ostře větší než
omega a mezi nima a omegou existuje bijekce).
\paragraph{Lemma}
 Je-li $|\alpha| \le \beta \le \alpha$, pak $|\beta| = |\alpha|$.
\paragraph{Důkaz}
$\beta\subsetq \alpha$, tedy existuje prosté zobrazení $\beta do \alpha$. Máme
$\beta \preceq \alpha$. \\
$\alpha \approx |\alpha|$, konečně $|\alpha| \subsetq \beta$, tedy $|\alpha| \preceq
\beta$. Aplikuji Kantorovu větu.
\paragraph{Lemma}
Je-li n přirozený číslo, potom:
\begin{enumerate}
	\item $n \napprox n+1$
	\item $(\forall \alpha)(\alpha \approx n \impl \alpha = n)$
\end{enumerate}
\paragraph{Důkaz}
\begin{enumerate}
	\item Indukcí: $0 \napprox 1$. Pokud existuje taková $n$, že $n\approx n+1$,
	pak $n \neq 0$ a tedy pro nějaké $m$, $n = m+1$. Tedy:
	\begin{align}
		n = \{0, 1, 2, ..., n\} \\
		n+1 = \{0, 1, 2, ..., m, m+1 \}
	\end{align}
	Je-li $b$ bijekce $b: n \to n+1$, pak existuje $i \in n: b(i) = m+1$.
	Definujme $b': m \to m+1$: Pro $j < i: b'(j) = b(j)$, pro $j > i: b'(j) =
	b(j-1)$. Tedy $b'$ je bijekce $m\to m+1$, což je spor s minimalitou n. (ten
	předpis je asi špatně, chce to promakat)
\end{enumerate}
\paragraph{Důsledek}
Všechna přirozená čísla jsou kardinály a $\omega$ je kardinál.
\paragraph{Definice}
Množina $A$ je \textbf{konečná} pokud $|A| \lt \omega$. Množina $A$ je
\textbf{spočetná}, pokud $|A| \le \omega$. Množina $A$ se nazývá nespočetná, pokud
není spočetná (tj. je velká, nebo jí nelze dobře uspořádat).
\subsection{Sčítání a násobení}
\setcounter{equation}{0}
\paragraph{Definice}
Jsou-li $\kappa, \lambda$ kardinály, pak:
\begin{enumerate}
	\item $\kappa \oplus \lambda = |\kappa \times\{0\} \cup \lambda\times\{1\}|$
	\item $\kappa \otimes \lambda = |\kappa \times \lambda|$
\end{enumerate}
\paragraph{Poznámka}
Oproti ordinálnímu sčítání a násobení jsou kardinální operace komutativní.
\paragraph{Lemma}
Pro $n,m \in \omega$ :
\begin{align}
	n \oplus m = n + m < \omega
	n \otimes m = n \cdot m < \omega
\end{align}
\paragraph{Důkaz}
Stačí ukázat, že $n+m < \omega$ a že $n \cdot m < \omega$. Zbytek je aplikace
posledního lemmatu. \\
Indukcí pro sčítání:
\begin{enumerate}
	\item $n+0 = n < \omega$
	\item $n + s(m) = s(\underbrace{n+m}_{<\omega}) < \omega$ 
\end{enumerate}
Stejnětak pro násobení:
\begin{enumerate}
	\item $n\times 0 = 0 < \omega$
	\item $n \times s(m) = \underbrace{n \cdot m}_{<\omega} + n < \omega$
\end{enumerate}
\paragraph{Věta}
Každý nekonečný kardinál je limitní ordinál.
\paragraph{Důkaz}
Sporem: buď $\kappa$ kardinál a $\kappa = \alpha + 1$. Jenomže $\alpha \ge
\omega$, tedy $1 + \alpha = \alpha$. Tedy $\kappa = |\kappa| = |1+\alpha| =
|\alpha| < \kappa$, což je spor.


=====================================================
TADY NECO CHYBI
=====================================================


\paragraph{Definice}
Buď $a$ je množina, $\le$ uspořádání na množině $a$. Množina $c \subsetq a$ se nazývá
řetězcem, jestliže $(c, \le)$ je uspořádána lineárně.
\paragraph{Definice}
Buď $(a, \le)$ uspořádaná množina, $b \subsetq a$. Prvek $x \in a$ se nazývá horní
mezí množiny $b$, jestliže 
\begin{align}
	(\forall y \in b) y \le x
\end{align}
a maximálním prvkem množiny b, jestliže
\begin{align}
	(x \in b) \land ((\forall y \in b) \neg y > x)
\end{align}

\subsection{Princip maximality}
\setcounter{equation}{0}
(také Zornovo lemma, Zorn-Kuratowského lemma)
\paragraph{Věta}
Nechť $(a, \le)$ je uspořádaná množina a nechť každý řetězec v a má horní mez.
Pak:
\begin{align}
	\forall x \in a \exists m \in a: \quad m \text{ je maximálním prvkem } a
	\land m \ge x
\end{align}
\paragraph{Důsledek}
Nechť platí princip maximality. Jsou-li M a N libovolné množiny, pak buď $M
\precsim N$
%nebo $N \subsim M$
\paragraph{Důkaz důsledku}
Uvážíme $a = \{ f: f$ je prosté zobrazení$, \dom f \subsetq M, \rng f \subsetq N
\}$. Uspořádáme $(a, \subsetq)$. Je-li $c \subsetq a$ řetězec, potom $\bigcup c$
je opět prostá funkce, přičemž je to horní mez řetězce $c$. Tedy existuje
maximální prvek $g$ množiny $(a,\subsetq)$. Platí buď $\dom(g) = M$ nebo $\rng(g) =
N$ (protože pokud existuje $x \in M \setminus \dom(g)$ a současně $y \in N \setminus
\rng(g)$, potom $g \cup <x,y>$ je prostá funkce a obsahuje $g$, což je spor s
maximalitou). Je-li $\dom(g) = M$, pak $M \precsim N$, neboť $g$ je prosté
zobrazení $M \to N$. Pokud $\rng(g) = N$, potom $N \precsim M$, $g^{-1}: N \to
M$ je prosté.


\subsection{Princip dobrého uspořádání}
\setcounter{equation}{0}
\paragraph{Tvrzení}
Pro každou množinu a existuje $R \subsetq a \times a$, takové, že $(a, R)$ je dobré uspořádání.

\subsection{Ekvivalence axiomu výběru, p. maximality a dobrého uspořádání}
\setcounter{equation}{0}
\paragraph{Věta}
Následující výroky jsou ekvivalentní:
\begin{enumerate}
	\item axiom výběru
	\item princip maximality
	\item princip dobrého uspořádání
\end{enumerate}
\paragraph{Důkaz}
\begin{enumerate}
	\item ($1. \impl 3.$) Nechť $a \neq 0$. Podle axiomu výběru na $g(a)
	\setminus \{0\}$ existuje selektor výběru $f$. Trans. indukcí definujeme
	zobrazení $g: Or \to a$ následujícím způsobem:
	g(0) = f(a). Je-li $\alpha \in Or$ a $(\forall \beta < \alpha) g(\beta)$ je
	definováno, definujeme $\xi = a \setminus \{g(\beta) : \beta < \alpha \}$.
	Pokud je tato množina neprázdná, definujeme $g(\alpha) = f (\xi)$ a indukce
	končí. Pokud je prázdná, pak $g: \alpha \to a$ je prosté zobrazení na a
	tedy definuje dobré uspořádání a. Zbývá ukázat, že indukce skončí: pokud by
	se nezastavila, získáme prosté zobrazení $g: Or \to a$, $\rng(g)$ jemnožina
	(protože a je množina). Ordinály jsou vlastní třída - spor.
	\item ($3. \impl 2.$) Máme $(a, \le)$, každý řetězec v a má horní mez a $x \in
	a$. Podle 3. existuje dobré uspořádání a. Nechť $c \subset a$ je řetězec.
	Budeme říkat, že $c$ splňuje (*), jestliže:
	\begin{align}
		x \in c \land (\forall t \in c) t \ge x \\
		\land (\forall y) (y \in c \land x < y), \text{ pak } y \\
		\text{ je } \prec \text{ nejmenší horní mez rětězce } \{ t \in c: t < y
		\}
	\end{align}
	Víme, že existuje alespoň jeden řetězec splňující (*), totiž řetězec
	$\{x\}$. Položme $b = \bigcup \{ c : c \subsetq$ a je řetězec splňující (*)
	\}. \textbf{$b$ je řetězec}: pro spor předpokládejme, že existují $z,t \in c$ takové,
	že nejsou porovnatelné. Tedy existuje $y_0 \in b$ je $\prec$-nejmenší prvek 
	splňující $(\exists w \in b) y_0 a w$ jsou
	$\le$-neporovnatelné. Existuje tedy $y_1 \in b$, že $y_1$ je $\prec-n$ejmenší prvek řetězce
	b, že $y_0$ a $y_1$ jsou $\le$-neporovnatelné. Protože $y_0$ a $y_1 \in b$
	existují řetězce $c_0$ a $c_1$, splňující (*), že $y_0 \in c_0$ a $y_1 \in
	c_1$. Nechť $z \in c_1$, $z < y_1$. Pokud x < z, pak $z \prec y_1: c_1$
	splňuje (*), z je $\prec$-nejmenší horní mez množiny \{t: t je horní mezí $\{v
	\in c: x \le v \le z \}$\}. Tvrdím, že $z < y_0$: z = $y_0$ není možné, oba jsou
	v řetězci $c_1$ a tedy porovnatelné, ale $y_0$ a $y_1$ nejsou. Pokud by $z >
	y_0$, pak $y_0 < z < y$, ale $y_0$ a $y_1$ jsou $\le$-neporovnatelné. Pokud z, $y_0$
	jsou $\le$-neporovnatelné: $z \prec y_1$ (víme - z i $y_1$ jsou horními mezemi $\{t
	\in c_1: x \le t \le z \}$), tedy ale z a $y_0$ musí být porovnatelné - spor.
	Dostáváme: $\{ t: x \le t < y \} = \{ t: x \le t < y_0 \}$. Protože $y_0
	\prec y_1$, $c_1$ nesplňuje (*), což je spor. Tedy $b$ je řetězec. \\
	Podle předpokladu principu maximality existuje $m$ horní mez $b$. Musí
	platit, že $m \in b$. Kdyby to neplatilo: vezmeme celé b a $M = \{ t: t \text{je ostrá
	horní mez} b \}$, která je neprázdná (obsahuje alespoň $m$). Taková množina má
	nejmenší prvek $z \in M$. $b \cup \{z\}$ je opět řetězec splňující (*) a $b
	\cup \{z\} \subsetq b = \bigcup \{ c$: $c$ je řetězec splňující (*) \}, což je
	spor ($z \nin b$). Tedy m je největší prvek b a kdykoliv $y \in a$ tak buď $y
	\le m$ nebo jsou neporovnatelné.
	\item ($2. \impl 1.$) Nechť $m$ je množina, na které hledáme selektor. Nechť a
	je množina $\{ f: \dom(f) \to \cup m: \dom(f) \subsetq m, kdykoliv x \in
	\dom(f), x \neq 0, pak f(x) \in x \}$. Uspořádáme $(a, \subsetq)$. Je-li $c \subsetq a$
	řetězec. $\bigcup c$ je funkce, $\forall f \in c$, $f \subsetq \bigcup c$. $(a,
	\subsetq)$ splňuje předpoklady principu maximality a podle 2. existuje v a
	maximální prvek g. Tvrdím, že $g$ je hledaný selektor: kdyby existovalo $x \in
	m$, $x \nin \dom(g)$, $x \neq 0$, zvolme: $t \in x$, $g \cup <x,t> \neq g$ - spor s
	maximalitou.
\end{enumerate}
\paragraph{Důsledky}
\begin{enumerate}
	\item (AC) $\impl$ pro každou množinu $a$, |a| existuje. (plyne ihned z principu
	dobrého uspořádání)
	\item Pro každou nekonečnou množinu a, $A \approx A \times A \approx A \times
	\{ 0, 1 \}$. 
	\item Každou nekonečnou množinu lze rozdělit na nekonečně mnoho nekonečných
	částí. Podle axiomu výběru víme, že $|A| = \kappa \ge \omega$ a $\kappa
	\approx \kappa \times \kappa$.
	\item Je-li $\kappa \ge \omega$ a pro každé $\alpha \in \kappa$ je
	$X_\alpha$	množina, $|X_\alpha|\le \kappa$, pak
	\begin{align}
		\left|\bigcup_{\alpha\in\kappa} X_\alpha \right| \le \kappa
	\end{align}
	\item Jsou-li $X$, $Y$ množiny a existuje $f: X \to Y$ surjektivní
	zobrazení, pak $|Y| \le |X|$. Dk: kartézský součin $X \times Y \subsetq \{
	<y,x> : y = f(x) \} = r$. Je-li $g \subsetq r$ funkce, pak $\dom(g) = Y$, $g: Y
	\to X$ prostě, protože $f$ je funkce.
\end{enumerate}


\subsection{}
\setcounter{equation}{0}
\paragraph{Definice}
Buďte A a B množiny. $^A B = \{f: f \text{je funkce}, f: A \to B \}$
\paragraph{Lemma}
Jsou-li B, C disjunktní množiny a A množina, pak:
\begin{align}
	{}^{(B \cup C)} A \approx {}^B A \times {}^C A
	{}^C ( {}^B A ) \approx {}^{C\times B} A
\end{align}
\paragraph{Důkaz}
Máme-li $f: B \cup C \to A$. Definujme $F(f) = < f \upharpoonleft B$, $f
\upharpoonleft C >$ \\
$F$ je prosté zobrazení. $G: {}^BA \times {}^CA \to {}^{B\cup C}A \\
G(<f_1, f_2>) = f_1 \cup f_2 \\
G = F^{-1}$

Je-li $f \in ^C(^B A)$, $f$ je funkce, $f : C \to ^B A$. Tedy pro každé $t \in c$ je
$f(t)$ funkce z $B \to A$, pro každé $t \in C$, $v \in B f(t)(v) \in A$. Položme $F:
^C(^B A) \to ^{C \times B} A$ pro funkce $F(f) = g$, kde $g \in ^{C \times B} A$ a je
definována předpisem $g(<t, v>) = f(t)(v)$. $F$ je bijekce.




\subsection{}
\setcounter{equation}{0}
\paragraph{Definice}
Jsou-li $\kappa$ a $\lambda$ kardinály, potom $\kappa^\lambda = \left| ^\lambda
\kappa \right|$.

\paragraph{Lemma}
Jsou-li $\kappa$ a $\lambda$ kardinály, $\kappa \ge 2$ a $\lambda \ge \omega$,
$\kappa \le \lambda$ pak
$2^\lambda = \kappa^\lambda = |\mathcal{P}(\lambda)|$
\paragraph{Důkaz}
\begin{align}
	\Phi: ^\lambda 2 \to \mathcal{P}(\lambda)  & \Phi(f) = \{ \xi < \lambda :
	f(\xi) = 1 \}\\
	\Phi^{-1}  : \mathcal{P} \to ^\lambda 2 & \Phi^{-1}  ( x) = \chi x \quad \\
	\chi_x(\xi) = \{ \xi \in x: 1, \xi \nin x: 0 \\
\end{align}
Zřejmě $\Phi$ je bijekce:
\begin{align}
	^\lambda 2 \approx \mathcal{P}(\lambda) \\
	^\lambda 2 \subsetq ^\lambda \kappa \subsetq \subsetq \mathcal{P}(\lambda
	\times \lambda \approx \mathcal{P} (\lambda) = ^\lambda 2
\end{align}

\paragraph{Lemma}
Jsou-li $\kappa$, $\lambda$ kardinály, pak 
\begin{align}
	\kappa^{\lambda \oplus \mu} = \kappa^\lambda \otimes \kappa^\mu \\
	(\kappa^\lambda)^\mu = \kappa^{\lambda \times \mu} 
\end{align}
(bez dukazu)

\paragraph{Definice}
Jsou-li $\alpha, \beta$ ordinály a $f: \alpha \to \beta$, řekneme, že f
zobrazuje $\alpha$ do $\beta$ \textbf{kofinálně} je-li $\rng(f)$ neomezená
množina v $\beta$.
\paragraph{Definice}
Kofinalita ordinálu $\beta$ je nejmenší ordinál $\alpha$ takový, že existuje
konfinální zobrazení z $\alpha$ do $\beta$. Značíme $\alpha = cf(b)$
\paragraph{Pozorování}
Určitě víme, že 
\begin{align}
	cf(\beta) \le \beta
\end{align}
protože existuje identita. Také víme, že pokud $\beta$ je ordinální následník:
$cf(\beta) = 1$. (kofinální zobrazení $f: 1 \to \alpha + 1$ je definované $f(0) =
\alpha$

\paragraph{Lemma}
Pokud je $\beta$ limitní, pak existuje konfinální zobrazení $f: cf(\beta) \to
\beta$, které je ostře rostoucí.
\paragraph{Důkaz}
Buď $g$ je kofinální zobrazení z $cf(\beta) \to \beta$. Definujme indukcí $f(0) =
g(0)$ a je-li $\alpha < \beta$ a z náme $f(\gamma) \forall \gamma < \alpha$,
položme $f(\alpha) = \sup\{g(\gamma) : \gamma \le \alpha \} \cup \{ f(\gamma:
\gamma < \alpha \}$. Přímo z definice plyne, že $f$ je ostře rostoucí a přitom $f
\ge g$ a tedy kofinální.
\paragraph{Lemma}
Je-li $\alpha$ limitní ordinál a $f : \alpha \to \beta$ ostře rostoucí kofinální
zobrazení, potom $cf(\alpha) = cf(\beta)$.
\paragraph{Důkaz}
$f: \alpha \to \beta$ je ostře rostoucí. $h: cf(\alpha) \to \alpha$ zvolme kofinální
a ostře rostoucí (podle předchozího lemma) a máme
\begin{align}
	g = f \circ h \qquad \qquad g: cf(\alpha) \to \beta
\end{align}
a přitom je kofinální: Je-li $\gamma < \beta$, určitě existuje $\Delta < \alpha$, že
$f(\Delta) \ge \gamma$. Ale $h$ je také kofinální: $\exists \eta < cf(\alpha)$:
$h(\eta) \ge f(\Delta)$ a $f$ je ostře rostoucí: $f(h(\eta) \ge \gamma$. Potom
\begin{align}
	cf(\beta) \le cf(\alpha)
\end{align}
Nechť $g: cf(\beta) \to \beta$ je kofinální a ostře rostoucí zobrazení.
Definujme $h: cf(\beta) \to \alpha$ předpisem $h(\xi) =$ minimální $\gamma \in
\alpha$ takové, že $f(\gamma) > g(\gamma)$. Všimněme si, že $f: cf(\beta) \to
\gamma$ a protože $f$ je ostře rostoucí, pak $h$ je kofinální zobrazení. Tedy 
\begin{align}
	cf(\alpha) \le cf(\beta)
\end{align}
\paragraph{Poznámka}
Kofinalita reální přímky je $\omega$, protože každý reálné číslo je menší než
nějaké celé.
\paragraph{Důsledek}
Kofinalita ordinálu $cf(cf(\beta)) = cf(\beta)$.
\paragraph{Definice}
Ordinál $\beta$ je regulární pokud $\beta$ je limitní ordinál a $\beta =
cf(\beta)$.
\paragraph{Lemma}
Je-li ordinál $\beta$ regulární, pak $\beta$ je kardinál.
\paragraph{Důkaz} Sporem: Nechť $\beta$ není kardinál. Tedy $|\beta| < \beta$.
Avšak máme bijekci $b: |\beta| \to \beta$, což je kofinální zobrazení (z
regularity). Tedy $cf(\beta) \le |\beta| < \beta$, což je ve sporu s regularitou
$\beta$.
\paragraph{Lemma}
$\omega$ je regulární kardinál.
\paragraph{Důkaz}
(bez důkazu)
\paragraph{Lemma}
Je-li $\kappa$ kardinál, pak $\kappa^+$ je regulární.
\paragraph{Důkaz}
Sporem: Buď $\alpha < \kappa^+$, $f: \alpha \to \kappa^+$ kofinální zobrazení.
Víme, že
	$|\alpha| \le \kappa$.
Kdykoliv $\xi < \alpha$, pak $f(\xi) < \kappa^+$, tedy $|f(\xi)| \le \kappa$.
Tedy:
\begin{align}
	\kappa^+ = \bigcup \left\{f(\xi) : \xi < \alpha \right\} \\
	|\cup \{f(\xi: \xi < \alpha \}| < \kappa \oplus\kappa = \kappa
\end{align}
což je spor.
\paragraph{Lemma}
Je-li $\alpha$ limitní ordinál, pak $cf(\omega_\alpha) = cf(\alpha)$.
\paragraph{Důkaz}
\begin{align}
	\omega_\alpha = \sup\{\omega_\beta : \beta < \alpha \}
\end{align}
Ihned plyne z jednoho z předchozích lemmat.
\paragraph{Lemma (Königovo)}
Předpokládejme axiom výberu. Je-li $\kappa$ nekonečný kardinál a $cf(\kappa) \le
\lambda$, $cf(\kappa) > 1$, pak $\kappa^\lambda > \kappa$.
\paragraph{Důkaz}
Stačí dokázat pro $\lambda = cf(\kappa)$. 
Nechť $g : \kappa \to ^\lambda \kappa$. Máme ukázat, že $g$ není surjektivní.
Zvolme kofinální zobrazení $f: \lambda \to \kappa$. Definujme h: $\lambda \to
\kappa$ takto: 
\begin{align}
	h(0) = 0 \\
	\text{ pro } \alpha < \lambda h(\alpha) = \min \left( \kappa \setminus
	\{g(\mu)(\alpha) : \mu < f(\alpha) \} \right)
\end{align}
Pro takto definovou funkci $h$, $h \nin \rng(g)$. Kdyby $h = g(\mu)$ pro nějaké
$\mu <\kappa$, pak existuje nějaké $\alpha <\lambda$, takže $f(\alpha) >\mu$.
Tedy funkce $h(\alpha) \nin \{g(\mu)(\alpha) : \mu < f(\alpha)\}$.
\paragraph{Důsledek}
Předpokládáme axiom výběru. Je-li $\lambda \ge \omega$, pak $cf(2^\lambda) >
\lambda$. Položme $\kappa = 2^\lambda$. Máme $\kappa^\lambda = (2^\lambda)^\lambda
= 2^{\lambda \otimes\lambda} = 2^ = \kappa$. Kdyby $cf(\kappa) \le \lambda$,
podle Königova lemmatu by platilo, že $\kappa^\lambda > \kappa$.
\paragraph{Definice}
Zobecněná hypotéza kontinua (GCH) je tvrzení, že 
\begin{align}
	(\forall \alpha) 2^{\aleph_\alpha} = \aleph_{\alpha + 1}
\end{align}
Ekvivalentně pro každý nekonečný kardinál $\kappa$, $2^\kappa = \kappa^+$. \\
Hypotéza kontinua (CH) je tvrzení, že $2^\omega = \omega_1$.
\paragraph{Lemma}
Předpokládejme zobecněnou hypotézu kontinua. Nechť $\kappa, \lambda \ge 2$ jsou
kardinály a alespoň jeden z nich je nekonečný. Pak platí:
\begin{enumerate}
	\item $\kappa \le \lambda \impl \kappa^\lambda = \lambda^+ $
	\item $\kappa > \lambda \ge cf(\kappa) \impl \kappa^\lambda = \kappa^+ $
	\item $\lambda < cf(\kappa) \impl \kappa^\lambda = \kappa$
\end{enumerate}
\paragraph{Pozorování}
$\kappa \le \lambda$ pak $2^\lambda \approx \kappa^\lambda \approx
\mathcal{P}(\lambda)$.
\paragraph{Důkaz}
\begin{enumerate}
	\item Z pozorování a GCH: $\kappa \le \lambda \quad \kappa^\lambda =
	2^\lambda =^{GCH} \lambda^+$
	\item Nechť $\kappa > \lambda \ge cf(\kappa)$. Podle Königova lemmatu
	$\kappa^\lambda > \kappa$. \\
	$\kappa > \lambda \qquad \kappa^\lambda = 2^\kappa = \kappa^+$. Tedy pro
	všechna $\lambda$, $cf(\kappa) \le \lambda \le \kappa$ platí $\kappa^\lambda =
	\kappa^+.$
	\item $\lambda < cf(\kappa)$: \\
	$^\lambda \kappa = \cup \{\ ^\lambda \alpha: \lambda < \kappa \}$ protože
	pro každou $f: \lambda \to \kappa$ existuje $\alpha < \kappa$ že $\rng(f)
	\subsetq \alpha$. f nemůže být kofinální zobrazení. Kdykoliv $\alpha <
	\kappa$, pak
	\begin{align}
		| ^\lambda\alpha| \le | \max\{alpha, \lambda\}, \max\{\alpha, \lambda\}
		| \le^{GCH} \max\{\alpha, \lambda\}^+ \le \kappa
	\end{align}
	Tedy $\kappa \le | ^\lambda\kappa| \le \kappa \otimes \kappa = \kappa$.
\end{enumerate}
\paragraph{Věta (Hausdorffova formule)}
Předpokládáme axiom výběru. Jsou-li $\kappa, \lambda$ nekonečné kardinály, potom
$(\kappa^+)^\lambda = \kappa^+ \otimes \kappa^\lambda$.
\paragraph{Důkaz}
Zřejmě $\kappa^+ \otimes \kappa^\lambda \le (\kappa^+)^\lambda \otimes
\kappa^\lambda \le (\kappa^+)^\lambda \otimes (\kappa^+)^\lambda =
(\kappa^+)^\lambda$. Zbývá dokázat nerovnost $(\kappa^+)^\lambda \le \kappa^+
\otimes \kappa^\lambda$.
\begin{enumerate}
	\item  $\lambda \ge \kappa^+: (\kappa^+)^\lambda \le \kappa^\lambda =
	2^\lambda = \kappa^\lambda = \kappa^\lambda \otimes \kappa^=$
	\item $\lambda < \kappa^+: $ protože $\kappa^+$ je regulární kardinál, potom
	pro každou $f: \lambda \to \kappa^+$ existuje nějaké $\alpha < \kappa^+$, že
	$\rng(f) \subsetq \alpha$. Tedy 
	\begin{align}
		(\kappa^+)^\lambda = | ^\lambda \kappa^+| = | \cup \{ ^\lambda \alpha :
		\alpha < \kappa^+ \} | \\
		\le \kappa^+ \otimes | ^\lambda \kappa| = \kappa^+ \otimes
		\kappa^\lambda
	\end{align}
\end{enumerate}
\paragraph{Definice}
Předpokládejme axiom výběru. Nechť $I \neq \emptyset$ a $\forall i \in I$ buď
$\kappa_i$ kardinální číslo. Definujme 
\begin{align}
	\sum_{i\in I} \kappa_i = \left| \bigcup \{\kappa_i x \{i\}: i \in I
	\}\right| \\
	\prod_{i\in I} \kappa_i = \left| \prod_{i \in I} \{\kappa_i: i \in I\} \right |
\end{align}
\paragraph{Věta (Königova nerovnost)}
Předpokládejme axiom výběru. Je-li $I \neq \emptyset$, pro každé $i \in I
\kappa_i$, $\lambda_i$ jsou kardinální čísla, přičemž $\kappa_i <\lambda_i$, pak 
\begin{align}
	\sum_{i\in I} \kappa_i < \prod_{i\in I} \lambda_i
\end{align}
\paragraph{Důkaz}
Z lemmatu a obrázkem (dopsat!)
%TODO: DOPSAT A DOMALOVAT OBRAZEK



===================================================
TADY ZASE NECO CHYBI
===================================================

\subsection{$\Delta$-systém}
\setcounter{equation}{0}
\paragraph{Definice}
Soubor $\mathcal{A}$ množin se nazývá $\Delta$-systém, pokud existuje množina K (jádro
systému), takže:
\begin{align}
	(\forall A \in \mathcal{A}) K \subsetq A \\
	\{ A \setminus K : A \in \mathcal{A} \} \text{ je disjunktní systém }
\end{align}
\subsection{Věta o $\Delta$-systému}
\setcounter{equation}{0}
\paragraph{Věta}
Nechť $\kappa > \omega$ regulérní kardinál. Je-li $< A_\alpha : a \in \kappa>$
systém konečných množin, pak existuje $I \subsetq \kappa$:
$|I| = K$,  tak, že  $<A_\alpha : \alpha \in I>$ tvoří $\Delta$-systém.
\paragraph{Důkaz}
Protože všechny množiny $A_\alpha$ jsou konečné, a $\kappa$ nespočetný regulérní ,
tak existuje $n \in \omega$ ... (chybí)
Indukcí podle $\mathbf{n}$.
\begin{enumerate}
	\item n = 1: $\forall \alpha \in I_0 \quad A_\alpha = \{x_\alpha\}$.
	\begin{align}
		\exists x \in \bigcup_{\alpha \in I_0} A_\alpha \text{ tak, že }
		|\{\alpha \in I_0 : x = x_\alpha \}| = \kappa \\
		I = \{ \alpha \in I_0 : x = x_\alpha \}
	\end{align}
	a $\{\{x_\alpha\}: x \in I\}$ tvoří $\Delta$-systém.
	Druhá možnost: (chybí)
	\item Indukční krok: předpokládejme platnost pro $|A_\alpha| = n$. Dokážeme
	\begin{align}
		\forall \alpha \in I_0 \quad |A_\alpha| = n + 1
	\end{align}
	Pro každé $\alpha \in I_0$ zvolme bod $x_0 \in A_\alpha$ a zvolíme $B_\alpha
	= A_\alpha \setminus \{x_\alpha\}$  množinu menší mohutnosti. Podle
	předpoklu indukce víme, že existuje $I_1 \subset I_0$, $|I_1| = \kappa$, tak,
	že $\{B_\alpha : \alpha \in I_1\}$ tvoří $\Delta$-systém s jádrem K.
	Zbávající body $x_\alpha : \alpha \in I_1$. \\
	Dvě možnosti:
	\begin{enumerate}
		\item $\exists x \quad x_\alpha = x $ pro $\kappa$ indexů z množiny I.
		Položíme $I = \{ \alpha \in I_1: x_\alpha = x\}$, v tomto případě
		$<A_\alpha: \alpha \in I>$ tvoří $\Delta$-systém s jádrem $K\cup\{x\}$.
		\item $I \subsetq I_1: |I_1| = \kappa$, tak pro $\alpha, \beta \in I,
		\alpha \neq \beta$ je $x_\alpha \neq \alpha, \beta$. $<A_\alpha : \alpha \in
		I>$ je $\Delta$-systém s jádrem K.
	\end{enumerate}
\end{enumerate}
\paragraph{Důsledek}
Uvažujme $\mathcal{F}$ = všechny funkce, které mají konečný definiční obor
$\subsetq \omega_1$ a obor hodnot $\subsetq \omega$. Kdykoliv $M \subsetq
\mathcal{F}$, $|M| = \omega_1$, pak existuje $\varphi, \psi \subsetq M$, pak
$\varphi \cup \psi$ je opět funkce.
\paragraph{Důkaz}
$\{\dom(\varphi) : \varphi \in M\}$ jsou konečné podmnožiny a je jich
$\omega_1$, tedy obsahují nespočetný $\Delta$-systém s jádrem K. Na takovéto
množině je však pouze spočetně mnoho funkcí (protože obor hodnot jsou přirozená
čísla) - tedy se některé funkce na jádře K shodují a můžeme je sjednotit.


\section{Stacionární množiny}
\setcounter{equation}{0}
\paragraph{Definice}
Nechť $\delta$ je limitní ordinál.
\begin{enumerate}
	\item Říkáme, že množina $A \subsetq \delta$ je neomezená (v $\delta$), pokud
	pro každé $\alpha < \delta$ existuje $\beta \in A: \alpha < \beta$.
	\item Říkáme, že množina $A \subsetq \delta$ je uzavřená (v $\delta$),
	jestliže pro každé limitní $\alpha < \delta$ platí, že $\sup A \cap \alpha =
	\alpha \impl \alpha \in A$
	\item Říkáme, že množina $A \subsetq \delta$ je uzavřená neomezená (v
	$\delta$) je-li $A$ uzavřená a neomezená.
\end{enumerate}


\paragraph{Lemma}
Nechť $\delta$  je limitní ordinál a $cf(\delta) > \omega$. Pak je-li $\tau <
cf(\delta)$ a $\{ C_\xi: \xi \in \tau \}$ soubor uzavřených neomezených množin v
$\delta$, pak 
\begin{align}
	\bigcap \{C_\xi: \xi < \tau \}
\end{align}
je uzavřené neomezené v $\delta$.
\paragraph{Důkaz}
Pložme 
\begin{align}
	C = \bigcap \{C_\xi: \xi < \tau \}
\end{align}
\begin{enumerate}
	\item $C$ je uzavřená v $\delta$: Nechť $\alpha < \delta$ je limitní ordinál, pro
	který platí, že $\alpha = \sup(C \cap \alpha)$. Pro každé $\xi \in \tau:
	C_\xi \supsetq C$. Tedy $C_\xi \cap \alpha \supsetq C \cap \alpha$ a tedy
	$\sup(C_\xi \cap \alpha) = \alpha$. $C_xi$ je uzavřená. Proto $\alpha \in
	C_\xi$.
	Tedy $\alpha \in \bigcap_{\xi < \tau} C_\xi = C$.
	\item C je neomezená: Zvolme libovolné $\alpha < \delta$. Položme $\alpha_0
	= \alpha$. Pak
	$\forall \xi < \tau: C_\xi$ neomezená, tedy existuje nějaké $\beta_\xi \in
	C_\xi$, že $\beta_xi < \alpha_0$ a máme množinu $\{ \beta_xi : \xi < \tau
	\}$,
	což má horní mez $\alpha_1$, protože $cf(\delta) >$ jejich počet. Dál indukcí
	známe $\alpha_0 < \alpha_1 < ... < \alpha_n$. $C_\xi$ neomezená v $\delta$,
	existuje $\beta_\xi^n \in C_\xi$, $\alpha_n < \beta_\xi^n$. Ale $\{ \beta_\xi^n :
	\xi < \tau \}$ není kofinální v $\delta$, tedy existuje $\alpha_{n+1} >
	\beta_\xi^n$, protože $\xi < \tau$. \\
	(a tady se to nějak okecá s obrázkem)
\end{enumerate}

\subsection{Stacionární množina}
\setcounter{equation}{0}
\paragraph{Definice}
Nechť $\delta$ je ordinál a $cf(\delta) > \omega$. $S \subsetq \delta$. Říkáme, že
množina $S$ je \textbf{stacionární} v $\delta$, jestliže pro každou uzavřenou,
neomezenou množinu $C$ je $S \cap C \neq 0$.
\paragraph{Příklad}
\begin{enumerate}
	\item Všechny uzavřené a neomezené množiny jsou stacionární.
	\item $\{\alpha < \omega_2: cf(\alpha) = \omega\}$ je stacionární množina v
	$\omega_2$,	která není uzavřená.
\end{enumerate}

\paragraph{Definice}
Nechť $\kappa$ je kardinál a $<A_\alpha : \alpha \in \kappa>$ je soubor podmnožin
kardinálu $\kappa$. Následující množina 
\begin{align}
	\triangle_{\alpha \in \kappa} A_\alpha = \left\{ \gamma \in \kappa: (\forall
	\alpha \in \gamma) \gamma \in A_\alpha \right\}
\end{align}
se nazývá \textbf{diagonálním průnikem} množin $A_\alpha \alpha \in \kappa$.
\paragraph{Lemma}
\begin{align}
	\triangle A_\alpha = \bigcap \{ A_\alpha \cap (\alpha+1): \alpha < \kappa \}
\end{align}
\paragraph{Důkaz}
Buď $\gamma \in \triangle A_\alpha$. Kdykoliv $\alpha < \gamma$, pak $\gamma \in
A_\alpha \subsetq A_\alpha \cup (\alpha + 1)$. Kdykoliv $\alpha \ge \gamma$, pak
$\gamma \in \alpha+1 \subsetq A_\alpha \cup (\alpha+1)$. Tedy $\gamma \in
\bigcap_{\alpha \in \kappa} A_\alpha (\alpha+1)$
Buď $\gamma \in \bigcap A_\alpha \cap (\alpha+1)$.
\begin{enumerate}
	\item Je-li $\gamma \le \alpha$, pak $\gamma \in \alpha + 1$
	\item Je-li $\gamma > \alpha$, pak $\gamma \in A_\alpha$ 
\end{enumerate}
Tedy $\gamma \in \triangle_{\alpha \in \kappa} A_\alpha$
\paragraph{Lemma}
Nechť $\kappa > \omega$ je regulární kardinál. $<A_\alpha: \alpha \in\kappa>$ je
soubor uzavřených neomezených podmnožin. Pak $\triangle_{\alpha\in\kappa} A_\alpha$ je uzavřená
neomezená.
\paragraph{Důkaz}
Položme
\begin{align}
	C = \triangle_{\alpha \in \kappa} A_\alpha
\end{align}
\begin{enumerate}
	\item C je uzavřená: Buď $\gamma \in \kappa$, $\gamma$ limitní, $\gamma = \sup(C \cap
	\gamma)$. Potom pro všechny $\alpha \in \kappa$ máme $\gamma = \sup((A_\alpha \cup
	(\alpha+1)) \cap \gamma)$. $A_\alpha$ uzavřená a neomezená v $\kappa$. $A_\alpha
	\cup (\alpha+1)$ je uzavřená a neomezená také. Tedy $\gamma \in A_\alpha \cup
	(\alpha+1)$ pro všechna $\alpha \in \kappa$. Tedy $\gamma \in C$ a C je uzavřená.
	\item C je neomezená: Buď $\xi_0 < \kappa$ libovolný ordinál.
	\begin{align}
		\bigcap_{\alpha \le \xi_0} A_\alpha \text{ je uzavřená neomezená
		podmnožina} \kappa \text{ podle minulého lemma }
	\end{align}
	Tedy existuje $\xi_1 > \xi_0, \xi_1 \in \bigcap_{\alpha\le\xi_0} A_\alpha$.
	Dál indukcí: známe $\xi_0 < \xi_1 < ... <\xi_n$. \\
	$\xi_n < \kappa$ - podíváme na $\bigcap_{\alpha \le \xi_m}
	A_\alpha$, tedy máme $\xi = \sup \{\xi_n : n \in \omega\} \in \kappa$.
	Buď $\alpha < \xi$ libovolný ordinál. Pak určitě existuje nějaké $n \in
	\omega$, kde $\alpha < \xi_n$. Z definice $\xi_{n+1}, \xi_{n+2}, ...$
	dostáváme, že $\xi_{n+1}, \xi_{n+2}, ...  \in A_\alpha$. $A_\alpha$ je uzavřená.
	Tedy $\{ \xi_i: n+1 \le i < \omega \} \subsetq A_\alpha$. $\xi \in A_\alpha$ pro
	všechna $\alpha < \xi$. Tedy $\xi \in \triangle_{\alpha\in\kappa} A_\alpha$.
\end{enumerate}


\paragraph{Definice}
Buď $A$ je množina ordinálních čísel a funkce $f: A \to Or$ se nazývá regresivní
na množině $A$, jestliže 
\begin{align}
	(\forall \alpha \in A) (\alpha > 0 \impl f(\alpha) < \alpha)
\end{align}

\subsection{Fodorova věta (Pressing-down lemma)}
\setcounter{equation}{0}
\paragraph{Věta}
Nechť $\kappa > \omega$ je regulární kardinál. Nechť $E \subsetq \kappa$. Pak
následující podmínky jsou ekvivalentní:
\begin{enumerate}
	\item $E$ je stacionární
	\item Je-li $f: E \to \kappa$ regresivní, pak existuje $\alpha < \kappa$, že
	$f^{-1}$ neomezená v $\kappa$.
	\item Je-li $f: E \to \kappa$ regresivní, pak existuje $\alpha < \kappa$
	tak, že $f^{-1} (\{\alpha\})$ je stacionární v $\kappa$.
\end{enumerate}











\end{document}

