\label{LP:celociselnost}
\df (Celočíselný LP) Lineární program nazveme celočíselným, pokud k němu přidáme
podmínku, aby všechny proměnné byly celočíselné. Pro celočíselné programy budeme
používat zkratku ILP (Integer Linear Program).

Je zřejmé, že ILP nemusí jít řešit polynomiálně, například protože je snadné
najít redukci ze SATu na ILP. Nadruhou stranu ale není důvod, proč by jednodušší
programy nemohly jít řešit polynomiálně. 

\tv Pro LP \eqref{LP:max-bipartitni-parovani} řešící perfektní párování v bipartitním grafu platí, že pro každé
optimální řešení existuje alespoň jedno stejně dobré řešení celočíselné.

\begin{align}
\label{LP:max-bipartitni-parovani}
&\max \sum_{e \in E} w_e x_e: \quad \sum_{e \in E: v \in e} x_e = 1 \forall v
	\in V, \quad x_e \in \{0, 1\} \\
	\text{relaxace:}\qquad	&\max \sum_{e \in E} w_e x_e: \quad \sum_{e \in E: v \in e} x_e = 1 \forall v
	\in V, \quad 0 \leq x_e \leq 1
\end{align}

\dk Nechť máme nějaké řešení, které má neceločíselné
složky. Začněme tedy s nějakou neceločíselnou hranou a najdeme cyklus (sudý,
jiné v bipartitním grafu nejsou) neceločíselných hran. Takový existuje, protože
suma každého vrcholu je 1 a vstoupili jsme z neceločíselného. Na cyklu poté pro
nějaké $\epsilon$ změníme váhy, pro lichou hranu $+$, pro sudou $-$. To
zachovává optimalitu, zvolíme tedy $\epsilon$ největší, aby zachovávalo také
přípustnost. Protože ale suma na každém vrcholu zůstává stejná, znamená to, že
některá hrana narazila na hranici $0$ nebo $1$. Máme tedy optimální řešení,
které má méně neceločíselných hodnota. Takto můžeme pokračovat, dokud žádné
takové nejsou. \qed

Existuje ale nějaké obecné tvrzení, které by nám zaručovalo podobnou hezkou
vlastnost? Odpovědí jsou totálně unimodulární matice.

\df (Totálně unimodulární matice) Matice je totálně unimodulární, pokud jsou
determinanty všech jejích čtvercových podmatic rovny $0$, $1$ nebo $-1$.

\poz Pokud je $A$ totálně unimodulární, potom také $A^T$ je totálně
unimodulární.

\vt (O celočíselnosti LP) Nechť $\max c^Tx : Ax = b$ je lineární program, $A$ je totálně
unimodulární matice a $b$ je celočíselné. Potom všechna bazická řešení tohoto
programu jsou celočíselná.

\dk Pro nějaké řešení existuje báze $B$. Vezmeme si matici $A_B$ a víme, že pro
nějaký vektor $x'$ platí $A_Bx_B = b_B$. Pokud je determinant $A_B$ nenulový
(což je, jinak by to nebyla báze), můžeme použít Cramerovo pravidlo na výpočet
$x_i = \det((A_B)_i) / \det_(A_B)$. Tady $(A_B)_i$ je matice, kde $i$-tý
sloupeček byl nahrazen vektorem $b_i$. Nyní stačí říct, že jmenovatel je $\pm1$
z totální unimodularity a čitatel můžeme spočítat třeba rozvojem podle sloupečku
$i$, a opět z totální unimodularity máme v každém kroku celočíselné hodnoty.
\qed

Takže totálně unimodulární matice nám dávají záruku, že všechny bazické
řešení (tedy takové, které vydá simplexová metoda) jsou celočíselné. Jak ale
poznáme, že je matice totálně unimodulární?

\vt Matice $A$ je totálně unimodulární právě tehdy, když pro každou podmnožinu
řádků matice $X$ lze rozdělit tyto řádky na skupiny $\cdot 1$ a $\cdot -1$ tak,
aby součet každého sloupečku byl $0$ nebo $\pm 1$.

\poz Matice incidence bipartitního grafu je totálně unimodulární: Nechť řádky
jsou vrcholy a sloupečky hrany. Pak každý sloupeček má součet právě $2$. V
libovolném obarvení nám tedy stačí, když řádky budou obarveny dle partity a
součet bude $0$; pokud některý řádek hraně bude chybět, součet bude $\pm1$.

\poz Matice incidence orientovaného grafu je totálně unimodulární: Všechny řádky
můžeme obarvit stejně, protože součet sloupečku je vždy $0$ a skládá se právě z
jedné $1$ a jedné $-1$.
