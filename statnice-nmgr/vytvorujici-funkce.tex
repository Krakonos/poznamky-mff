\section{Vytvořující funkce}
\label{sec:vytvorujici-funkce}

\df Nechť $(a_n)$ je posloupnost reálných čísel. Vytvořující řadou této
posloupnosti rozumíme mocninnou řadu:
$$a_0 + a_1x + a_2x^2 + \dots = \sum_{n=0}^\infty a_nx^n$$

Je-li tato řada konvergentní pro nějaké $x \neq 0$, nazveme tuto řadu
vytvořující funkcí\footnote{Pro odlišení od ostatních druhů vytvořujících
funkcí se někdy nazývá obyčejná vytvořující funkce.} posloupnosti $(a_n)$ a
budeme ji značit $a(x)$.

Podobně můžeme definovat vytvořující řadu, resp. vytvořující funkci
dvourozměrné (nebo i více\-rozměrné) posloupnosti $(a_{m,n})$:
$$a(x,y) = \sum_{m,n=0}^\infty a_{m,n}x^my^n$$

\subsection{Vlastnosti vytvořujících funkcí}
Najít vytvořující funkci posloupnosti umíme snadno z definice. Když chceme
naopak získat z vytvořující funkce posloupnost, můžeme užít následující vzorec:

$$a_n = {a^{(n)}(0)\over n!}$$

kde $a^{(n)}(0)$ značí $n$-tou derivaci funkce $a$ v bodě 0. Tato vlastnost
plyne z matematické analýzy, konkrétně z Taylorova rozvoje.

Z definice je také jasné, že ne každá posloupnost reálných čísel má vytvořující
funkci (příslušná vytvořující řada nemusí konvergovat). Stejně tak ne každá
funkce odpovídá nějaké posloupnosti reálných čísel (nemusí mít definované
derivace).


\subsection{Operace s vytvořujícími funkcemi}

\begin{description*}
\item[Součet] Posloupnost $(a_0+b_0,a_1+b_1,\dots)$ má vytvořující funkci $a(x)+b(x)$.
\item[Násobení konstantou] Posloupnost $(\alpha a_0, \alpha a_1, \dots)$ má vytvořující funkci $\alpha a(x)$.
\item[Posun vpravo] Posloupnost $(\underbrace{0,0,\dots,0}_{n\times},a_0,a_1,\dots)$ má vytvořující funkci $x^na(x)$.
\item[Posun vlevo] Posloupnost $(a_k,a_{k+1},\dots)$ má vytvořující funkci $(a(x)-\sum_{i=0}^{k-1}a_ix^i)/x^k$. Tady musíme nejen vydělit $x^k$, ale navíc odečíst prvních $k$ členů. Např. posun o 3: $(a(x)-a_0-a_1x-a_2x^2)/x^3$.
\item[Dosazení $\alpha x$ a $x$] Vytvořující funkce $a(\alpha x)$ odpovídá posloupnosti $(a_0,\alpha a_1,\alpha^2 a_2,\dots)$.
\item[Dosazení $x^n$ za $x$] Vytvořující funkce $a(x^n)$ odpovídá posloupnosti $(a_0,\underbrace{0,\dots,0}_{n-1\times},a_1,\underbrace{0,\dots,0}_{n-1\times},a_2,\dots)$.
\item[Derivace] Vytvořující funkce $a'(x)$ odpovídá posloupnosti $(a_1,2a_2,3a_3,\dots)$.
\item[Násobení] Násobením vytvořujících funkcí $a(x)b(x)$ dostáváme vytvořující funkci $c(x)$, která odpovídá posloupnosti $(c_n); c_n = \sum_{k=0}^n a_kb_{n-k}$.
\end{description*}

\subsection{Použití vytvořujících funkcí}

Vytvořující funkce mohou být využity při odvozování explicitních vzorců pro rekurentní posloupnosti a kombinatorické počítání (tam se ale jedná především o exponenciální vytvořující funkce).

Příkladem použití je odvození explicitního vzorce pro $n$-tý člen Fibonacciho posloupnosti (důkaz v sekci \ref{sec:rekurence}).

\subsection{Exponenciální vytvořující funkce}
\df Exponenciální vytvořující řadou posloupnosti $(a_n)$ je mocninná řada:
$$a_0 + a_1x + {a_2x^2 \over 2!} + \dots = \sum_{n=0}^\infty a_n{x^n\over n!}$$
Pokud tato řada konverguje pro $x \neq 0$, pak ji nazveme exponenciální vytvořující funkcí posloupnosti $(a_n)$ a označíme ji $A(x)$.

Chceme-li z exponenciální vytvořující funkce odvodit odpovídající posloupnost, můžeme spočítat $a_n = A^{(n)}(0)$, kde $A^{(n)}(0)$ značí $n$-tou derivaci funkce $A(x)$ v bodě 0.

{\bf Poznámka:} Některé posloupnosti nemají odpovídající obyčejnou vytvořující funkci, ale mají exponenciální vytvořující funkci. Příkladem je posloupnost $(a_n); a_n = n!$.

S exponenciálními vytvořujícími funkcemi můžeme stejně jako u obyčejných vytvořujících funkcí používat operace sčítání a násobení konstantou. Další operace můžeme používat s úpravami. Speciálně zajímavé je násobení exp. vytvořujících funkcí:
$$A(x)B(x) = C(x),\qquad c_n = \sum_{k=0}^n{n\choose k}a_kb_{n-k}$$

Exponenciální vytvořující funkce se používají především v kombinatorickém počítání.

\todo Použití exp. vytv. fcí v kombinatorickém počítání.
