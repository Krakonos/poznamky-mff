\documentclass[landscape,a4paper]{article}
\usepackage[utf8]{inputenc}
\usepackage[czech]{babel}
\usepackage[left=1cm,right=1cm,top=1cm,bottom=1cm]{geometry}
\usepackage[usenames,dvipsnames]{color}

\newcommand{\card}[1]{\makebox{\raisebox{4.2cm}{\raisebox{-2.1cm}{\parbox{6.9cm}{\begin{center}{\bf\large#1}\end{center}}}}}}
\newcommand{\whiteout}[1]{{\color{Gray}#1}}

\newcommand{\cardx}[2]{\framebox{{{%
\parbox[t][4.2cm]{6.65cm}{\begin{center}{\bf\large#1}\end{center}
\begin{itemize}
#2
\end{itemize}
}}}}}

\parindent=0pt
\begin{document}
\pagestyle{empty}

\noindent
\cardx{Barevnost grafů}{
	\item Perfektní grafy
	\item Barevnost a maximální stupeň
	\item Barevnost a girth
	\item Vybíravost
}
\cardx{Regulární grafy}{
	\item Silně regulární grafy
	\item Moorovy grafy
	\item Regularita a Hamiltonovskost
}
\cardx{Souvislost grafů}{
	\item Souvislost orientovaných grafů
	\item Nash-Williams a arboricita
	\item Linkovanost
}
\cardx{Speciální vlastnosti orientovaných grafů}{
	\item Algoritmus pro hledáíní SSK
	\item Turnaje a hamiltonovské cesty
}
\cardx{Algebraické vlastnosti grafů}{
	\item Vlastní čísla grafu
	\item Expandéry
	\item Konstrukce expandérů
}
\cardx{Teorie párování}{
	\item Párování na množinách
	\item Párování v grafech
	\item Velikost maximálního párování
	\item Matching polytope
}
\cardx{Ramseyova teorie}{
	\item Turánova věta
	\item Ramseyovy věty
	\item Hales-Jewett
	\item Van der Waerden
}
\cardx{Nekonečná kombinatorika}{
	\item Königovo nekonečné lemma
	\item Věta o barevnosti nekonečných grafů
	\item Hallova věta
}
\cardx{Strukturální vlastnosti množinových systémů}{
	\item Sunflower lemma
	\item Discrepancy
	\item Bollobásova a Spernerova věta
}
\cardx{Kombinatorické počítání}{
	\item Princip inkluze a exkluze
	\item Problém šatnářky
	\item Cayleyho formule
	\item Fibonacciho a Catalanova čísla
}
\cardx{Vytvořující funkce}{
	\item Vlastnosti VF
	\item Operace s VF
	\item Použití VF (rekurence)
	\item Exponenciální VF
}
\cardx{Rekurence}{
	\item Věta o řešení lin. rekurencí
	\item Příklady použití (Fib. čísla)
	\item Master theorem
}
\cardx{Základní pravděpodobnostní modely}{
	\item Linearita střední hodnoty, Rozptyl
	\item Metoda alternace
	\item Černovova nerovnost, \\ Lovászovo lokální lemma
}
\cardx{Asymptotické odhady funkcí}{
	\item ?
}
\cardx{Pravděpodobnostní konstrukce a algoritmy}{
	\item Snižování potřebné náhody
	\item Min-Cut, Quick Sort
	\item Pravděpodobnostní hierarchie
	\item PCP věta
}
\cardx{Grafové algoritmy}{
	\item Nejkratší cesta
	\item Maximální tok (3 indové)
	\item Maximální párování
}

\newpage
\card{Speciální vlastnosti orientovaných grafů}
\card{Souvislost grafů}
\card{Regulární grafy}
\card{Barevnost grafů}
\\
\card{Nekonečná kombinatorika}
\card{Ramseyova teorie}
\card{Teorie párování}
\card{Algebraické vlastnosti grafů}
\\
\card{Rekurence}{
\card{Vytvořující funkce}{
\card{Kombinatorické počítání}{
\card{Strukturální vlastnosti množinových systémů}
\\
\card{Grafové algoritmy}
\card{Pravděpodobnostní konstrukce a algoritmy}{
\card{Asymptotické odhady funkcí}{
\card{Základní pravděpodobnostní modely}{

\newpage
\cardx{Algebraické a aritmetické algoritmy}{
	\item Strassen, Euklides, Eratosthenes
	\item Prvočísla
}
\cardx{Teorie mnohostěnů}{
	\item Roviny, nadroviny, mnohostěny
	\item Lineární programování
	\item Farkas lemma
	\item Věta o dualitě
}
\cardx{Problém obchodního cestujícího}{
	\item Obtížnost TSP
	\item Aproximační alg. na TSP
}
\cardx{Speciální matice}{
	\item Pozitivně definitní \\ a semidefinitní matice
	\item Hadamardovy matice
	\item Totálně unimodulární matice
}
\cardx{Celočíselnost}{
	\item Obtížnost
	\item Relaxace
	\item Totální unimodularita
}
\cardx{Teorie matroidů}{
	\item Definice přes nezávislé množiny, \\ báze, kružnice, rankovou funkci
	\item Přehled vlastností, operace
	\item Hladový algoritmus na matroidu
	\item Matroid intersection theorem
}
\cardx{Elipsoidová metoda}{
	\item Idea metody
}
\cardx{---}{
	\item ---
}
\newpage
\card{Speciální matice}
\card{Problém obchodního cestujícího}
\card{Teorie mnohostěnů}
\card{Algebraické a aritmetické algoritmy}
\\
\card{}
\card{Elipsoidová metoda}
\card{Teorie matroidů}
\card{Celočíselnost}

\end{document}
