\section{Základní pravděpodobnostní modely}
\df (Pravděpodobnostní prostor) Pravděpodobnostní prostor je trojce $(\Omega, 
\Sigma, P)$, kde $\Omega$ je množina elementárních jevů, $\Sigma \subseteq 
2^\Omega$ a $P: \sigma \to [ 0, 1 ]$ je pravděpodobnostní distribuce (tedy 
funkce udávající pravděpodobnost každého jevu ze $\Sigma$).  $\Sigma$ je 
$\sigma$-algebra. Pravděpodobnost elementárního jevu $a$ budeme značit jako 
$p_a$, pokud je toto označení jednoznačné.

%\df ($\sigma$-algebra) Podmnožina $\Sigma \subseteq 2^X$ množiny $X$ je 
%$\sigma$-algebra, pokud splňuje:
%\begin{enumerate}
%	\item $X \in \Sigma$
%	\item Je uzavřená na doplněk: $A \in \Sigma \Rightarrow X \setminus A \in 
%	\Sigma$
%	\item Je uzavřená na spočetné sjednocení: $A_1, A_2, \dots \in \Sigma 
%	\Rightarrow \bigcup A_i \in \Sigma$.
%\end{enumerate}

\df (Uniformní distribuce) Distribuce $P$ je uniformní, pokud $P(a) = 1 / 
|\Omega|$ pro každé $a \in \Omega$.

Chceme-li získat náhodný graf, potřebujeme k tomu mít dobrý způsob. Obecně není 
vhodné vybrat náhodný graf ze všech možných grafů, ale spíše vybra náhodný graf 
na $n$ vrcholech. Existuje několik možných modelů:
\begin{enumerate}
	\item (Gilbert) $G(n,p)$ je náhodný graf na $n$ vrcholech, kde každá 
		potenciální hrana je do grafu přidána s pravděpodobností $p$.
	\item (Erdös-Rényi) $G_e(n,m)$ je náhodný graf vybraný ze všech grafů na $n$ 
		vrcholech obsahujících $m$ hran.
	\item (Regular) $G_r(n,r)$ je náhodný graf vybraný ze všech $r$-regulárních 
		grafů na $n$ vrcholech.
\end{enumerate}
Gilbertův model je nejpoužívanější a nejjednodušší na používání, včetně 
algoritmického výběru. Například vybra regulární graf není vůbec jednoduché.  
Jeden příklad konstrukce je vzít $nr$ vrcholů, rozdělit je do $n$ skupin po $r$ 
vrcholech, zvolit náhodné párování a kontrahovat všechny skupiny. Pokud vyjde 
graf bez smyček a multihran, máme výsledek, jinak musíme opakovat...


\tv (Union Bound) Pro spočetně mnoho jevů platí: $P(\bigcup A_i) \leq \sum 
P(A_i)$.

\df (Podmíněná pravděpodobnost) Pro jevy $A$ a $B$ platí:
\begin{align}
	P[A|B] = {P[A \cdot B] \over P[B]}
\end{align}
Tedy pravděpodobnost, že $A$ nastane za předpokladu, že $B$ nastal, je rovna 
pravděpodobnosti, že nastanou oba najednou, normalizovaná pravděpodobností jevu 
$B$.

\df (Náhodná proměnná) Náhodná proměnná je funkce $X: \Sigma \to Y$, tedy na 
základě náhodného experimentu nabývá nějaké hodnoty (například: počet hran 
náhodného grafu, barevnost, velikost největší kliky,...).

\df (Indikátor) Indikátor (nějakého jevu $A$) je náhodná proměnná, která nabývá 
hodnot $1$ nebo $0$, podle toho, zda při náhodném experimentu jev $A$ nastal, 
nebo ne.

\df (Nezávislost) Jevy $X$ a $Y$ jsou nezávislé právě tehdy, když $P(A\cap B) = 
P(A) \cdot P(B)$.

\df (Nezávislost více jevů) Spočetně jevů $A_i$ je navzájem nezávislých právě 
tehdy, pokud ($I$ je indexová množina představující všechny možné podmnožiny $A 
= \bigcup A_i$):
\begin{align}
	P\left(\bigcap_{i \in I} A_i\right) = \prod_{i\in I} P(A_i) \qquad \forall I 
\end{align}

\pzn Pozor! Po dvou nezávislé jevy nezaručují vzájemnou nezávislost!

\subsection{Linearita střední hodnoty, rozptyl}

\df (Střední hodnota) Pro náhodnou proměnnou $X$ je střední hodnota definována 
jako:
\begin{align}
	\E[X] = \sum_{a \in \Omega} X(a) \cdot p_a
\end{align}

\vt (Princip průměru) Vždy existuje alespoň jeden jev, který nabývající stejné 
hodnoty, nebo větší, než střední hodnota, a naopak.

\vt (Linearita střední hodnoty) Pro náhodné proměnné $X$, $Y$ a libovolnou 
konstantu $c$ platí:
\begin{align}
	\E[X+Y] = \E[X] + \E[Y] \\
	\E[c \cdot X] = c \cdot \E[X]
\end{align}
Pokud jsou navíc $X$ a $Y$ nezávislé, potom:
\begin{align}
	\E[X \cdot Y] = \E[X] \cdot \E[Y]
\end{align}

\poz (Střední hodnota indikátoru) Pro jev $A$ a jeho indikátor $I_A$ platí: 
$\E[I_A] = \sum_{\omega \in \Omega} I_A(\omega) \cdot p_\omega = \sum_{\omega 
\in \Omega} 1 \cdot p_\omega = P(A)$.

\vt (Markovova nerovnost) Nechť $X \geq 0$ je náhodná proměnná a $a > 0$. Potom 
platí:
\begin{align}
	P(X \geq a) \leq {\E X \over a}
\end{align}

\df (Rozptyl) Pro náhodnou proměnnou $X$ definujeme rozptyl jako:
\begin{align}
	\Var(X) = \E[(X - \E X)^2] ( = \E(X^2) - (\E X)^2 )
\end{align}

\vt (Čebyševova nerovnost) Nechť $X$ je náhodná proměnná, potom $\forall t > 0$ 
platí:
\begin{align}
	P(|X - \E X| \geq t) \leq {\Var(X) \over t^2}
\end{align}

\pzn Hodí-li se to, můžeme nervonosti otočit (doplněk pravděpodobnosti), vnitřní
nerovnost se však stane ostrou!

%\tv Pro $m \geq 1$ platí: $\binom{2m}{m} \geq 2^{2m} / (4 \sqrt m + 2)$.
%\dk Díváme se na kombinační číslo jako volbu z množiny. Zvolme si $X_1, \dots, 
%X_{2m}$ proměnné nabývající $0$ nebo $1$ obojí s pravděpodobností $1/2$ a $X = 
%\sum X_i$.  Potom $\E X = m$ a $\Var(X) = m/2$. Podívejme se na pra

%\vt (Erdös-Ko-Rado) Nechť $X$ je množina a $\F\subseteq 2^X$ je protínající se 
%systém množin a každá množina tohoto systému je alespoň $k$-prvková, kde $|X| = 
%n \geq 2k$, potom $|\F| \leq \binom{n-1}{k-1}$.

\subsection{Metoda alternace}

\vt (Slabá Turánova) Nechť $G$ je graf na $n$ vrcholech s $m$ hranami a $d := 
2/mn$ průměrný stupeň. Potom $\alpha(G) \geq n/2d$.

\dk Hledáme velkou nezáváslou množinu: Zvolme si náhodnou podmnožinu $U$, každý 
vrchol do ní vložme s pravděpodobností $p$ (tu určíme později) a mějme náhodné 
proměnné $X := |U|$ a $Y := |E(G[U])|$. Spočítejme střední hodnotu:
\begin{align}
	\E X = np \qquad \E Y = p^2 m = p^2 \cdot {dn \over 2 }
\end{align}
Pokud bychom neměli žádné hrany, máme docela velkou nezávislou množinu. My ale 
hrany máme. Odstraníme tedy vrchol od každé hrany a podíváme se, kolik zbyde:
\begin{align}
	\E[X - Y] = (pn - p^2 \cdot {dn \over 2}) = pn \cdot (1-p \cdot {d \over 2})
\end{align}
Stačí zvolit $ p := 1/d$ a díky principu průměru máme větu dokázánu.

\vt (Vysoká barevnost bez malých cyklů) Pro každé $k,l > 0$ existuje graf 
takový, že $\chi(G) > k$ a $\girth(G) > l$.

\dk Zvolíme náhodný graf $G(n,p)$. Počet cyklů dlouhých $i$ je 
$1/2(i-1)!\binom{n}{i}\leq n^i$ (zvolím $i$ prvků, jedním začnu, zpermutuju 
jejich pořadí a každý cyklus jsem započítal dvakrát, proto $/2$). Množina hran 
je cyklus, pokud tam jsou všechny hrany, tedy počet cyklů menších než $l$ je:
\begin{align}
	\E(X) \leq \sum_{i=3}^l n^ip^i
\end{align}
Stačí zvolit $n$ a $p$ šikovně, aby $\E(X) \leq n/4$ a z Markovovy nerovnosti 
získáme, že $P(X \geq n/2) < 1/2$. Je tedy snadné získat graf, ve kterém stačí 
odebrat nanejvýš $n/2$ vrcholů, abychom se zbavili krátkých cyklů.

Dále odhadneme barevnost. Každá barevnostní třída je nezávislá množina, tedy 
$\chi(G) \geq |V| / \alpha(G)$. Nechť $a$ je dolní odhad na velikost nezávislé 
množiny. Potom z Markovovy nervonosti máme:
\begin{align}
	P(\alpha(G) \geq a) \leq \binom{n}{a}(1-p)^{\binom{a}{2}}
\end{align}

Což pro vhodně zvolené parametry $p$ jde k nule a tedy speciálně pro nějaké 
dostatečně velké $n$ je pravděpodobnost $< 1/2$.

Protože obě nerovnosti jsou ostré, existuje graf, který splňuje obojí pro 
libovolné $a$ a $l$. Stačí tedy zvolit graf dostatečně velký, aby i po odebrání 
$n/2$ vrcholů, potřebné ke zrušení krátkých cyklů, zbyla dostatečně velká 
nezávislá množina (teoreticky mohu odebírat pouze z jedné nezávislé množiny, a 
to té největší).

A jaké jsou ty šikovné parametry? Určitě $p$ musí být závislé na $n$, aby se 
pravé strany snižovaly, například stačí $p := n^{1/2l -1}$.

\subsection{Černovova nerovnost, Lovászovo Lokální Lemma}

Je typické, že umíme snadno charakterizovat události, které jsou špatné -- 
například pokud barvíme graf, špatná událost je, že vrchol má stejně barevného 
souseda. Ačkoliv můžou být pravděpodobnosti jednotlivých událostí libovolně 
velké (blízké 1), pokud jsou nezávislé, stále stačí ukázat,že:
\begin{align}
	\overline{P\left(\bigcap_i A_i\right)} = P\left(\overline{\bigcap_i 
	A_i}\right) = \prod_i P(\overline{A_i})> 0
\end{align}
Bohužel v hodně případech nejsou jevy nezávislé. Definujme si tedy úroveň, jakou
jsou na sobě závislé:


\df (Nezávilost s více událostmi) Událost $A$ je nezávislá s událostmi {$B_1, 
\dots, B_k$} právě tehdy, když:
\begin{align}
	\forall J \subseteq [k] \qquad P\left(A\cap\bigcap_{j\in J} B_j\right) = 
	P(A) \cdot P\left(\bigcap_{j \in J} B_j\right)
\end{align}

\df (Závislostní graf) Nechť $A_i$ jsou (špatné) události. Závislostní graf $D =
(\{A_i\}, E)$ je orientovaný graf, který má jako vrcholy události a vrchol $A_i$ 
je nezávislý (ve smyslu předchozí definice) se všemi událostmi, do kterých z něj 
{\bf nevede hrana}. 

\poz Závislostní graf nemusí být definovaný unikátně, speciálně lze beztrestně 
přidávat hrany a úplný graf je vždy závislostní (není ale příliš užitečný).

\lm (Lovászovo Lokální) Nechť $A_1, \dots, A_n$ jsou události omezené 
pravděpodobností $p$ ($\forall i: P(A_i) \leq p$) a existuje pro ně závislostní 
graf s maximálním výstupním stupněm $d$. Potom pokud je $ep(d+1) \leq 1$, platí:
\begin{align}
	P\left(\bigcap_{i=1}^n \overline{A_i}\right) > 0
\end{align}

\vt (Černovova nerovnost) Nechť $X_1, \dots, X_n$ jsou nezávislé náhodné 
proměnné, nabývající hodnot $+1$ a $-1$ s pravděpodobností $1/2$. Potom pro $X 
:= \sum X_i$, $\sigma^2 := \Var(X)$ a $t \geq 0$ platí:
\begin{align}
	P(X \geq t) &< \exp\left({-t^2 \over 2 \sigma^2}\right) \\
	P(X \leq -t) &< \exp\left({-t^2 \over 2 \sigma^2}\right) \\
	P(|X| \geq t) & < 2\exp\left({-t^2 \over 2 \sigma^2}\right)
\end{align}

