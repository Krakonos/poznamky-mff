\df (Souvislost) Graf je hranově (vrcholově) $k$-souvislý právě tehdy, když je
souvislý a po odebrání $k-1$ hran (vrcholů) zůstává souvislý.

\df (Souvislost orientovaných grafů) Graf $G$ je slabě souvislý, pokud je jeho
dezorientovaný ekvivalent souvislý. Graf je silně souvislý, pokud je každá uspořádaná dvojice vrcholů $(u,v)$ spojená orientovanou cestou z $u$ do $v$.

{\it Více o silné souvislosti se nachází v sekci \ref{silna-souvislost}, včetně
algoritmu na hledání komponent silné souvislosti.}

\vt (Ford-Fulkerson) Pro každý graf $G$ platí, že je hranově $k$-souvislý právě
tehdy, když mezi každou dvojicí vrcholů existuje alespoň $k$ hranově
disjunktních cest.

\vt (Menger) Pro každý graf $G$ platí, že je vrcholově $k$-souvislý právě
tehdy, když mezi každými dvěma různými vrcholy existuje alespoň $k$ cest, které
jsou vrcholově disjunktní (až na koncové vrcholy).

\vt (Souvislost a stupeň) Nechť $G$ má průměrný stupeň alespoň $4k$. Potom má
$G$ $k$-souvislý podgraf.

\vt (Minimální souvislý graf) Algoritmicky najít nejmenší (co do počtu hran)
podgraf, který je $k$-souvislý je $\NP$-těžké.

%\df (Arboricita) Arboricita grafu $\A(G)$ je nejmenší počet lesů, které stačí k
%pokrytí celého grafu.
%
%\poz (Dolní odhad na Arboricitu) Pro libovolný graf $\A(G) \geq m/(n-1)$,
%protože každá kostra má nanejvýš $n-1$ hran.
%
%\vt (Nash-Williams) Pro graf $G=(V,E)$ platí: $\A(G) = \max_{S \subseteq V}
%\lceil m_S / (|S| -1 ) \rceil$, kde $m_S$ je počet hran grafu indukovaného na
%$S$.

\vt (Nash-Williams) Multigraf obsahuje $k$ hranově disjunktních koster právě
tehdy, když pro každé rozdělení vrcholů na $r$ partit, má $G$ alespoň $k(r-1)$
hran mezi partitami.

\vt Každý hranově $2k$-souvislý multigraf má $k$ hranově disjunktních koster.

\dk Pomocí Nash-Williamse: každá partita je s ostatními ze souvislosti spojena
alespoň $k$ hranově disjunktními cestami, z každé cesty jedna hrana bude mezi
partitami, což splňuje podmínky pro použití věty.\qed

\subsection{Linkovanost}

\df (Linkovanost) Graf $G$ je $k$-linkovaný, pokud pro libovolné množiny
vrcholů $s_i, t_i, i := 1,\dots, k$ existují hranově disjunktní cesty z $s_i$
do $t_i$.

\pzn Linkovanost je silnější varianta souvislosti, speciálně $k$-linkovaný graf
je určitě $k$-souvislý, naopak to neplatí; Mengerova věta totiž nezaručuje (a
ani nemůže) propojení správných vrcholů.

\vt (Souvislost a linkovanost) Nechť $G$ je $2k$-souvislý a obsahuje minor
$K_{4k}$. Potom je $G$ $k$-linkovaný.

\vt (Velká souvislost a linkovanost) Nechť $G$ je $2^{4k+1}$-souvislý, potom je
$G$ $k$-linkovaný.



