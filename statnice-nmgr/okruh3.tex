\documentclass[a4paper,12pt,titlepage]{article}
\usepackage[utf8]{inputenc}
\usepackage[czech]{babel}
\usepackage{amsfonts, amsmath, amsthm, amssymb}
\usepackage[small,compact]{titlesec}
\usepackage{anyfontsize}
\usepackage{rotating}
\usepackage{mdwlist}
\usepackage[left=1.5cm,right=1.5cm,top=1.5cm,bottom=2cm]{geometry}
\usepackage{wrapfig}
\usepackage{subcaption}
\usepackage{hyperref}
\usepackage{color}
\usepackage{tikz}
\newcommand*\circled[1]{\tikz[baseline=(char.base)]{
  \node[shape=circle,draw,inner sep=1pt] (char) {#1};}}

\usepackage{titlesec}
\titleformat*{\section}{\removelastskip\bigskip\LARGE\bfseries}
\titleformat*{\subsection}{\large\bfseries}
\titleformat*{\subsubsection}{\large\bfseries}

\newcommand{\shn}{\Theta}
\newcommand{\lm}{\smallskip\noindent\bf Lemma\rm{} }
\newcommand{\dk}{\smallskip\noindent\bf Důkaz\rm{} }
\newcommand{\df}{\smallskip\noindent\bf Definice\rm{} }
\newcommand{\vt}{\smallskip\noindent\bf Věta\rm{} }
\newcommand{\poz}{\smallskip\noindent\bf Pozorování\rm{} }
\newcommand{\pzn}{\smallskip\noindent\bf Poznámka\rm{} }
\newcommand{\dsl}{\smallskip\noindent\bf Důsledek\rm{} }
\newcommand{\tv}{\smallskip\noindent\bf Tvrzení\rm{} }
\newcommand{\app}{\smallskip\noindent\bf Aplikace\rm{} }
\newcommand{\alg}{\smallskip\noindent\bf Algoritmus\rm{} }
\newcommand{\F}{\mathcal{F}}
\newcommand{\B}{\mathcal{B}}
\newcommand{\A}{\mathcal{A}}
\renewcommand{\c}{\mathcal{C}}
\renewcommand{\L}{\mathcal{L}}
\renewcommand{\O}{\mathcal{O}}
\newcommand{\E}{\mathbb{E}}
\newcommand{\Z}{\mathbb{Z}}
\newcommand{\C}{\mathbb{C}}
\newcommand{\NP}{\mathcal{NP}}
\newcommand{\R}{\mathbb{R}}
\newcommand{\Q}{\mathbb{Q}}
\newcommand{\N}{\mathbb{N}}
\newcommand{\Ft}{\mathbb{F}}
\newcommand{\xttt}{{\chi_T^\bot}^T}
\newcommand{\todo}[1]{\bf TODO: \rm#1}
\newcommand{\set}[1]{\{#1\}}
\renewcommand{\L}{\mathcal{L}}
\newcommand{\I}{{\bf I}}
\newcommand{\In}{{\bf I}_n}
%\newcommand{\qed}{\hfill QED}
\DeclareMathOperator{\rank}{rank}
\DeclareMathOperator{\disc}{disc}
\DeclareMathOperator{\zz}{\circled{z}}
\DeclareMathOperator{\Sp}{Sp}
\DeclareMathOperator{\Tr}{Tr}
\DeclareMathOperator{\Ker}{Ker}
\newcommand\bigzero{\makebox(0,0){\text{\huge0}}}
\newcommand\bigone{\makebox(0,0){\text{\huge1}}}
\newcommand{\bigddots}[1]{\makebox(0,0){\rotatebox{-35}{\text{\xleaders\hbox{$\cdot$\hskip4pt}\hskip#1\kern0pt}}}}
\newcommand{\sk}[1]{{\langle #1\rangle}}
\newcommand{\diagdots}[3][-25]{%
  \rotatebox{#1}{\makebox[0pt]{\makebox[#2]{\xleaders\hbox{$\cdot$\hskip#3}\hfill\kern0pt}}}%
}

\title{Státnice -- Informatika -- I4\\ Diskrétní modely a algoritmy\\ ~\\ 
Kombinatorická optimalizace}
\author{Ladislav Láska\\ Jan Musílek}

\begin{document}

\maketitle
\newpage
\tableofcontents
\newpage

\section{Grafové algoritmy}
\section{Algebraické a aritmetické algoritmy}
\subsection{Jednoduché algoritmy}
\begin{itemize}
	\item Strassenův algoritmus na násobení matic
	\item Euklidův algoritmus
	\item Eratosthenovo síto
\end{itemize}
\subsection{Prvočísla}

\vt (Lagrangeova) Pokud má konečná grupa $G$ podgrupu $H$, pak je řád $G$
dělitelný řádem $H$.
\vt (Malá Fermatova) Je-li $p$ prvočíslo a $\gcd(x,y) = 1$, potom platí $x^{p-1}
\equiv_y 1$.

\df (Eulerova funkce) Pro číslo $n$ definujeme $\varphi(n)$ počet čísel mezi $1$
a $n-1$ nesoudělných s $n$ (tj. $\gcd(n,i) = 1$.

\poz Některé vlastnosti Eulerovy funkce jsou:
\begin{itemize}
	\item $\varphi(n) = |\Z_n^*|$ (není samozřejmé)
	\item $\varphi(p) = p-1$ (z definice)
	\item $\varphi(ab) = \varphi(a) \cdot \varphi(b)$ pokud jsou $a$ a $b$
		nesoudělná
\end{itemize}
Speciálně, známe-li rozklad na prvočísla, je triviální podle posledních dvou
bodů vypočítat hodnotu funkce $\varphi$.

\vt (Eulerova) Pro libovolné $n > 0$ a $x$ s ním nesoudělné platí
$x^{\varphi(n)} \equiv_n 1$.

\poz Malá fermatova věta je speciální případ Eulerovy věty, kdy $n := p$.

\vt (O diskrétní odmocnině) Pokud je $p$ liché prvočíslo a $x \in \Z_p^*$, potom
má $x$ buďto právě dvě odmocniny, nebo nemá žádnou.

\vt (Čínská věta o zbytcích) Nechť $a_1, \dots, a_t$ jsou navzájem nesoudělná
čísla a $n = \prod a_i$, potom pro každé číslo $0 < x < n$ existuje právě jedna
$b$-tice zbytků $b_i = n \mod b_i$, tedy číslo je jednoznačně určené zbytky po
dělení $a_i$.

\alg (Fermatův test) Pro $n$ zvolíme $x$ uniformě náhodně mezi $2$ a $n-1$.
Spočteme $x^{n-1} \mod n$ a pokud vyjde $1$, prohlásíme $n$ za prvočíslo, jinak
za číslo složené.

\dk Pokud algoritmus odpoví, že je číslo složené, číslo $x$ je fermatův svědek,
který pomocí Malé fermatovy věty dokazuje složenost čísla. Abychom mohli Malou
Fermatovu větu použít, musí být $x$ a $n$ nesoudělné; pokud však soudělné jsou,
náš algoritmus stále odpoví správně a navíc výsledek výpočtu $x^{n-1}$ je
dělitelný společným dělitelem. \qed

\poz Bohužel existují čísla, říká se jim Carmichaelova, která Fermatova svědka
nemají a Fermatův test je nikdy neodhalí.

\vt Pokud $n$ není prvočíslo ani Carmichaelovo číslo, platí $x^{n-1} \equiv_n
1$ pro nanejvýš $n/2$ různých $x$.

\dk Označme všechna taková čísla jako $H$. Všechna jsou (z definice)
invertibilní, navíc jejich součin je taktéž v $H$; $H$ je tedy podgrupa $Z_n^*$,
a protože není číslo Carmichaelovo, musí $|H| \cdot k = \varphi(n) = |Z_n^*|$
pro nějaké $k \geq 2$ z Lagrangeho věty. \qed

\alg (Rabin-Miller) Nechť na vstupu je $n$, potom následující algoritmus na
prvočísla odpovídá správně a na složená čísla odpoví špatně s pravděpodobností
nanejvýš $1/4$.
\begin{enumerate}
	\item Vygeneruj náhodné $x: 1 < x < n$.
	\item Pokud $\gcd(x,n) \neq 1$, odpověz, že je číslo {\bf složené}.
	\item Najdi $t$ a liché $m$, t. ž. $n-1 = 2^t \cdot m$.
	\item Spočti $b_0 := x^m$ a $b_{i+1} := b_i^2$ pro $i\in \{1, \dots, t\}$.
	\item Pokud $b_t \equiv 1$, odpovíme, že je číslo {\bf složené} ($x$ je
		Fermatův svědek).
	\item Pokud jsou všechna $b_i \equiv 1$, odpovíme, že je číslo {\bf
		prvočíslo}.
	\item Jinak pro nejvyšší $i$ takové, že $b_i \not\equiv 1$, pokud je $b_i
		\equiv -1$ odpovíme, že je číslo {\bf prvočíslo}, jinak {\bf složené}.
\end{enumerate}

\pzn (Derandomizace) Pokud by se ukázalo, že platí Riemannova hypotéza (Grupa je
generována svými $\O(\log^2 n)$ nejmenšími prvky), stačilo by testovat tyto
generátory (protože všechna $x$, která o prvočíslu lžou, jsou v generátoru -- to
ale není triviální ukázat). Tedy pokud ještě určíme, jaká je konstanta v $\O$,
což by ale mělo být $2$.

\alg (RSA) Nechť $p,q$ jsou dvě velká prvočísla, $n := p \cdot q$, $y = x^{-1}$
náhodný reverzibilní prvek a jeho inverzi modulo $\varphi(n)$. Potom lze
šifrovat pomocí $a^x \mod n$ a dešifrovat pomocí $a^y \mod n$.

\dk (Pouze pro $\gcd(a,n) = 1$) Počítejme: $a^{xy} = a^{1 + k\varphi(n)}$,
protože $xy \equiv_{\varphi(n)} 1$. Ale to je rovno po vytknutí $1$ z exponentu
$a \cdot a^{k\varphi(n)} = a \cdot (a^{\varphi(n)})^k = a$ podle Eulerovy věty.

\section{Teorie mnohostěnů}
\section{Problém obchodního cestujícího}

Problém obchodního cestujícího budeme zkracovat na TSP, z anglického názvu
Travelling salesman problem. Úkolem je najít nejkratší cestu, která navštíví
všechny města a vrátí se zpět do prvního, za předpokladu, že mezi každými dvěma
městy vede silnice. V řeči teorie grafů můžeme problém definovat následovně.

\medskip
\noindent\textbf{Optimalizační verze:} V daném ohodnoceném úplném grafu najděte
nejkratší Hamiltonovskou kružnici. 

\noindent\textbf{Rozhodovací verze:} Existuje v daném ohodnoceném úplném grafu
Hamiltonovská kružnice kratší než $k$?
\medskip

Obecně nemusí platit trojúhelníková nerovnost, ale předpokládáme, že váhy
jednotlivých hran jsou nezáporné.

\subsection{Obtížnost TSP}

\vt TSP je $\NP$-úplný.

\dk Převodem na Hamiltonovskou kružnici v grafu $G = (V,E)$. Hranám úplného
grafu, které jsou v $E$ přiřadíme váhu 1, hranám které nejsou v $E$ váhu 2 a
ptáme se, zda existuje řešení TSP s váhou $|V|$. TSP $\in \NP$ dokážeme tak, že
použijeme výsledné pořadí vrcholů jako certifikát.\footnote{Řešíme rozhodovací
verzi, takže stačí ověřit, že je to kružnice s délkou nejvýše $k$ -- nemusíme
ověřovat, že je nejkratší, to by samozřejmě bylo těžší.}

Z toho, že TSP je $\NP$-úplný je jasné, že neznáme žádný polynomiální
deterministický algoritmus, který by TSP řešil. Známe ale několik aproximačních
algoritmů, které najdou přijatelně dobré (i když ne nejlepší) řešení.

\subsection{Aproximační algoritmy na TSP}

Všechny aproximační algoritmy pro TSP předpokládají, že platí trojúhelníková
nerovnost.

\alg (Aproximace přes kostry) Nejprve najdeme minimální kostru grafu $G$. Ta je
dolním odhadem na řešení TSP.\footnote{Kdyby bylo řešení TSP menší, než
minimální kostra, pak bychom dostali menší kostru vypuštěním libovolné hrany z
řešení TSP.} Začneme z libovolného vrcholu a určíme pořadí vrcholů tak, že
půjdeme okolo rovinného nakreslení minimální kostry po směru hodinových ručiček.
Toto pořadí vrcholů vydáme jako výsledek.

\vt Aproximace přes kostry je polynomiální 2-aproximační algoritmus pro TSP s
trojúhelníkovou nerovností.

\alg (Christofides)
\begin{enumerate*}
\item Nalezneme minimální kostru $T$ grafu $G$.
\item Označme $O$ vrcholy s lichým stupněm v $T$. Nalezneme perfektní párování
$M$ na vrcholech $O$ s minimální váhou.
\item Sjednotíme $T$ a $M$ a získáme multigraf $H$.
\item Najdeme Eulerovský tah v $H$ ($H$ je souvislý a má stupně všech vrcholů sudé).
\item Vytvoříme Hamiltonovskou kružnici tak, že půjdeme po Eulerovském tahu a
budeme přeskakovat již navštívené vrcholy.
\end{enumerate*}

\vt Christofides je polynomiální $3\over 2$-aproximační algoritmus pro TSP s
trojúhelníkovou nerovností.

\dk Všimněme si nejprve, že přeskakováním již navštívených vrcholů nikdy
nedostaneme delší cestu, což plyne z trojúhelníkové nerovnosti.

Označme $A$ množinu hran, která je optimálním řešením TSP v úplném grafu $G =
(V,w)$. Protože graf $(V,A)$ je souvislý, obsahuje nějakou kostru $T$ a zjevně
$w(A) \ge w(T)$. Označme $B$ množinu hran, která je optimálním řešením TSP v
úplném grafu $(O,w)$. Protože platí trojúhelníková nerovnost, musí být $w(A) \ge
w(B)$.

Ukážeme, že existuje perfektní párování na $O$, které váží méně než $w(B)/2 \le
w(A)/2$. $O$ jistě obsahuje sudý počet vrcholů, tedy existuje maximální
párování. Vezmeme graf $(O,B)$, který sestává právě z jedné Hamiltonovské
kružnice. V této kružnici vezmeme buď všechny liché hrany, nebo všechny sudé
hrany a prohlásíme je za párování. Protože váha celé kružnice je $w(B)$, pak
váha lichých nebo sudých hran musí být nejvýš $w(B)/2$. Tedy, $w(M) + w(T) \le
w(A) + w(A)/2$ a algoritmus je $3\over 2$-aproximační. 
\qed





\section{Speciální matice}
\section{Celočíselnost}
\section{Párování a toky v sítích}
\section{Teorie matroidů}

Matroid je struktura v kombinatorice, která zobecňuje koncept
\uv{nezávislosti}, jehož konkrétním příkladem je lineární nezávistlost ve
vektorových prostorech. Existuje mnoho ekvivalentních způsobů jak zavést
matroidy, nejvýznamnějšími jsou nezávislé množiny, báze, kružnice a ranková
funkce. Teorie matroidů si často vypůjčuje teminologii z lineární algebry a
teorie grafů.

\subsection{Definice přes nezávislé množiny}
\df Matroid $M$ je dvojice $(S,I)$, kde $S$ je konečná množina (nazýváme ji
\emph{nosná množina} a $I \subset 2^S$ je množina podmnožin $S$ (nazýváme je
nezávislé množiny), splňující následující vlastnosti:
\begin{enumerate*}
\item Prázdná množina je nezávislá, tedy $\emptyset \in I$.
\item Každá podmnožina nezávislé množiny je nezávislá, tedy pro každé $A' \subseteq A \subseteq S$ platí $A \in I \Rightarrow A' \in I$. Tato vlastnost se nazývá \emph{dědičnost}.
\item Pokud $A$ a $B$ jsou dvě nezávislé množiny z $I$, $|A| = |B| + 1$, pak existuje prvek $x\in A\setminus B$ tž. $B\cup\{x\}$ je nezávislá. Tato vlastnost se nazývá \emph{výměnná vlastnost}.
\end{enumerate*}

\df Podmnožina $S$, která není v $I$ se nazývá \emph{závislá}. Maximální nezávislá
množina (po přidání libovolného prvku se stane závislou) se nazývá \emph{báze}.
Naopak minimální závislou množinu (po odebrání libovolného prvku se stane
nezávislou) nazýváme \emph{kružnice}.

\pzn Grafová analogie: $S$ je množina hran grafu, $I$ je množina všech lesů na
hranách $S$. Báze odpovídá kostře grafu $G$, kružnice cyklu v grafu $G$.

\pzn Vektorová analogie: $S$ je množina sloupců matice. Nezávislé množiny jsou
pak právě lineárně nezávislé množiny vektorů z $S$. Báze matroidu odpovídají
bázím vektorového prostoru. 

\subsection{Definice přes báze}
\df Matroid $M$ je dvojice $(S,\B)$, kde $S$ je konečná množina a $\B$ je
množina podmnožin $S$ (tyto podmnožiny nazýváme báze), splňující následující
vlastnosti:
\begin{enumerate*}
\item $\B$ je neprázdná.
\item Když $A,B \in \B$ jsou různé a $a \in A\setminus B$, pak existuje $b \in B\setminus A$ tž. $A\setminus\{a\}\cup\{b\} \in \B$.
\end{enumerate*}

\poz Tato definice je ekvivalentní s definicí přes nezávislé množiny. Nezávislé
množiny jsou právě podmnožiny bází.

\subsection{Definice přes kružnice}
\df Matroid $M$ je dvojice $(S,\c)$, kde $S$ je konečná množina a $\c$ je
množina podmnožin $S$ (tyto podmnožiny nazýváme kružnice), splňující
následující vlastnosti: 
\begin{enumerate*}
\item $\emptyset \notin \c$
\item Pokud $A$ i $B$ jsou kružnice, pak $A\subseteq B \Rightarrow A = B$.
\item Když $A$ i $B$ jsou kružnice a $e\in A\cap B$, pak existuje kružnice v $A\cup B$, která neobsahuje $e$.
\end{enumerate*}

\poz Tato definice je ekvivalentní s definicí přes nezávislé množiny. Nezávislé
množiny jsou právě ty, které neobsahují žádnou kružnici.

\subsection{Definice přes rankovou funkci}
\df Ranková funkce je zobrazení $r: 2^S \rightarrow \N$, která každé podmnožině
$S$ přiřadí velikost její největší nezávislé podmnožiny.

\df Matroid $M$ je dvojice $(S,r)$, kde $S$ je konečná množina a $r: 2^S
\rightarrow \N$ je ranková funkce, splňující následující vlastnosti:
\begin{enumerate*}
\item $0 \le r(X) \le |X|$
\item $X\subseteq Y \Rightarrow r(X) \le r(Y)$
\item $r(X\cup Y) + r(X\cap Y) \le r(X) + r(Y)$
\end{enumerate*}

\poz Tato definice je ekvivalentní s definicí přes nezávislé množiny. Nezávislé množiny jsou právě ty, kde $r(X) = |X|$.

\subsection{Přehled jednoduchých vlastností}
\tv Všechny báze matroidu $M$ mají stejnou velikost $r(M)$. Toto číslo
označujeme jako rank matroidu $M$. V grafech je zjevně $r(G) = n-k$, kde $n$ je
počet vrcholů a $k$ počet komponent.

\df Duální matroid $M^*$ definujeme tak, že jeho báze jsou doplňky bází $M$.
Pak zjevně platí $M^{**} = M$.

\df Smyčka je prvek $S$, který není obsažen v žádné bázi. Co-smyčka je prvek
$S$ obsažený v každé bázi.

\df Uniformní matroid $U_{k,n}$ na $n$ prvcích definujeme tak, že nezávislými
podmnožinami jsou právě ty, jejihž velikost je nejvýše $k$. V $U_{0,n}$ je
každý prvek smyčka, v $U_{n,n}$ je každý prvek co-smyčka.

\subsection{Operace s matroidy}
\df Nechť $M = (S_1,I_1)$ a $N = (S_2,I_2)$ jsou matroidy. \emph{Direktní
součet} matroidů $M$ a $N$ je matroid $(S_1 \sqcup S_2, I_1 \sqcup I_2)$, kde
$A\sqcup B$ značí disjuktní sjednocení množin $A$ a $B$.

\df Nechť $M = (S_1,I_1)$ a $N = (S_2,I_2)$ jsou matroidy. \emph{Sjednocení
matroidů} $M$ a $N$ je matroid $(S_1\cup S_2, \{i_1\cup i_2 ~|~ \forall i_1\in
I_1, \forall i_2\in I_2\})$.

\subsection{Hladový algoritmus}
\df \emph{Systém nezávislých množin} je dvojice $(S,I)$, kde $S$ je konečná
množina a $I \subset 2^S$ je množina podmnožin $S$ (nazýváme je nezávislé
množiny), splňující, že $\emptyset\in I$ a každá podmnožina nezávislé množiny
je též nezávislá.

\vt Mějme $M = (X,J)$ systém nezávislých množin a orákulum, které nám pro každou
množinu $Y\subseteq X$ určí, zda je nezávislá. Definujeme hladový algoritmus
tak, že začne s prázdnou množinou $S$ a poté do ní v každém kroku přidá prvek z
$X$ s nejnižším ohodnocením, který neporuší nezávislost $S$. Potom hladový
algoritmus vygeneruje maximální nezávislou množinu $S$ s minimálním součtem
ohodnocení pro každou nezápornou ohodnocovací funkci $w: X \rightarrow \N$ právě
tehdy, když $M$ je matroid.

\dk Rozdělíme tvrzení na dvě implikace a dokážeme každou zvlášť.

\noindent \textbf{1) $M$ je matroid $\Rightarrow$ hladový algoritmus funguje pro
každé ohodnocení $w$}

Je zjevné, že hladový algoritmus nalezne maximální nezávislou množinu. Vezměmě
výstup hladového algoritmu $\{x_1,\dots,x_r\}$ tž. $w(x_1) \le \dots \le
w(x_r)$. Pro spor předpokládejme, že existuje nezávislá množina
$\{y_1,\dots,y_r\}$ tž. $w(y_1) \le \dots \le w(y_r)$ a $w(x_k) > w(y_k)$
pro nějaké $k$. Pokud by takové $k$ neexistovalo, je $\{x_1,\dots,x_r\}$
maximální nezávislá množina s minimálním ohodnocením.

Předpokládejme, že $k$ je nejmenší takové, že $w(x_k) > w(y_k)$. Zjevně $k > 1$.
Potom mějme $S = \{x_1,\dots,x_{k-1}\}$ a $T = \{y_1,\dots,y_k\}$. Platí, že
$|T| = |S|+1$ a tedy podle výměnné vlastnosti matroidů existuje $j \in
\{1,\dots,k\}$ tž. $S\cup \{y_j\}$ je nezávislá množina (a $y_j$ není v $S$).
Protože $w(y_j) \le w(y_k) < w(x_k)$, hladový algoritmus by vybral $y_j$ dříve,
než $x_k$, což je spor.

\noindent \textbf{2) hladový algoritmus funguje pro každé ohodnocení $w$
$\Rightarrow$ $M$ je matroid}

Předpokládejme, že máme systém nezávislých množin, který nesplňuje výměnnou
vlastnost. Zkonstruujeme ohodnocovací funkci $w$ takovou, že na ní hladový
algoritmus selže. Mějme tedy $A,B \subset X$ nezávislé množiny takové, že $|A| =
|B| + 1$, ale pro žádné $a \in A\setminus B$ není $B\cup \set a$ nezávislá
množina. Pak definujme:
\begin{align*}
w(x) = \left\{
	\begin{array}{lr}
		w_1\quad\text{pokud}~x \in A\cap B \\
		w_2\quad\text{pokud}~x \in B \setminus A \\
		w_3\quad\text{pokud}~x \in A \setminus B \\
		w_4\quad\text{jinak}
	\end{array}
	\right.
\end{align*}

Dále definujme, $w_1 < w_2 < w_3 < w_4$. Všimněme si, že hladový algoritmus
nejprve vybere všechny prvky z $A\cap B$, poté všechny prvky z $B\setminus A$ a
pak (podle předpokladu) už nebude moci vybrat žádný prvek z $A\setminus B$.
Protože $B$ není maximální, budeme k němu muset přidat prvek z
$X\setminus\set{A\cup B}$, který má váhu $w_4$.

Nyní nám stačí nastavit konstanty $w_1,\dots,w_4$ tak, aby byl výsledek
hladového algoritmu těžší, než kterákoliv nezávislá množina obsahující $A$
(taková zjevně musí existovat). Mějme tedy $m$ velikost maximální nezávislé
množiny. Chceme:
\begin{align*}
|A\cap B|w_1 + |B-A|w_2 + (m-|B|)w_4 > |A\cap B|w_1 + |A-B|w_3 + (m-|A|)w_4
\end{align*}
Protože $|A| = |B|+1$, tak:
\begin{align}
\label{rov:hungry-w4}
w_4 > |A-B|w_3 - |B-A|w_2
\end{align}
Hladový algoritmus vybere o jeden prvek s váhou $w_4$ více, než kdyby nejdříve
pobral celou množinu $A$. Musíme tedy nastavit $w_4$ dostatečně velké. Jeden ze
způsobů, jak jej zvolit (pro libovolné $\varepsilon\in (0,1)$):
\begin{align*}
w_1 &= \varepsilon /|X| \\
w_2 &= \varepsilon /|B-A| \\
w_3 &= (1+\varepsilon) /|A-B| \\
w_4 &= 2
\end{align*}

Pak už stačí jen dokázat, že takto zvolená $w$ splňuje $w_1 < w_2 < w_3 < w_4$ a
že platí rovnice \ref{rov:hungry-w4}.
\qed

\subsection{Matroid intersection theorem}
\todo



\section{Elipsoidová metoda}

\end{document}
