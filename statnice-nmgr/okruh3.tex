\documentclass[a4paper,12pt,titlepage]{article}
\usepackage[utf8]{inputenc}
\usepackage[czech]{babel}
\usepackage{amsfonts, amsmath, amsthm, amssymb}
\usepackage[small,compact]{titlesec}
\usepackage{anyfontsize}
\usepackage{rotating}
\usepackage{mdwlist}
\usepackage[left=1.5cm,right=1.5cm,top=1.5cm,bottom=2cm]{geometry}
\usepackage{wrapfig}
\usepackage{subcaption}
\usepackage{hyperref}
\usepackage{color}
\usepackage{tikz}
\newcommand*\circled[1]{\tikz[baseline=(char.base)]{
  \node[shape=circle,draw,inner sep=1pt] (char) {#1};}}

\usepackage{titlesec}
\titleformat*{\section}{\removelastskip\bigskip\LARGE\bfseries}
\titleformat*{\subsection}{\large\bfseries}
\titleformat*{\subsubsection}{\large\bfseries}

\newcommand{\shn}{\Theta}
\newcommand{\lm}{\smallskip\noindent\bf Lemma\rm{} }
\newcommand{\dk}{\smallskip\noindent\bf Důkaz\rm{} }
\newcommand{\df}{\smallskip\noindent\bf Definice\rm{} }
\newcommand{\vt}{\smallskip\noindent\bf Věta\rm{} }
\newcommand{\poz}{\smallskip\noindent\bf Pozorování\rm{} }
\newcommand{\pzn}{\smallskip\noindent\bf Poznámka\rm{} }
\newcommand{\dsl}{\smallskip\noindent\bf Důsledek\rm{} }
\newcommand{\tv}{\smallskip\noindent\bf Tvrzení\rm{} }
\newcommand{\app}{\smallskip\noindent\bf Aplikace\rm{} }
\newcommand{\alg}{\smallskip\noindent\bf Algoritmus\rm{} }
\newcommand{\F}{\mathcal{F}}
\newcommand{\B}{\mathcal{B}}
\newcommand{\A}{\mathcal{A}}
\renewcommand{\c}{\mathcal{C}}
\renewcommand{\L}{\mathcal{L}}
\renewcommand{\O}{\mathcal{O}}
\newcommand{\E}{\mathbb{E}}
\newcommand{\Z}{\mathbb{Z}}
\newcommand{\C}{\mathbb{C}}
\newcommand{\NP}{\mathcal{NP}}
\newcommand{\R}{\mathbb{R}}
\newcommand{\Q}{\mathbb{Q}}
\newcommand{\N}{\mathbb{N}}
\newcommand{\Ft}{\mathbb{F}}
\newcommand{\xttt}{{\chi_T^\bot}^T}
\newcommand{\todo}[1]{\bf TODO: \rm#1}
\newcommand{\set}[1]{\{#1\}}
\renewcommand{\L}{\mathcal{L}}
\newcommand{\I}{{\bf I}}
\newcommand{\In}{{\bf I}_n}
%\newcommand{\qed}{\hfill QED}
\DeclareMathOperator{\rank}{rank}
\DeclareMathOperator{\disc}{disc}
\DeclareMathOperator{\zz}{\circled{z}}
\DeclareMathOperator{\Sp}{Sp}
\DeclareMathOperator{\Tr}{Tr}
\DeclareMathOperator{\Ker}{Ker}
\newcommand\bigzero{\makebox(0,0){\text{\huge0}}}
\newcommand\bigone{\makebox(0,0){\text{\huge1}}}
\newcommand{\bigddots}[1]{\makebox(0,0){\rotatebox{-35}{\text{\xleaders\hbox{$\cdot$\hskip4pt}\hskip#1\kern0pt}}}}
\newcommand{\sk}[1]{{\langle #1\rangle}}
\newcommand{\diagdots}[3][-25]{%
  \rotatebox{#1}{\makebox[0pt]{\makebox[#2]{\xleaders\hbox{$\cdot$\hskip#3}\hfill\kern0pt}}}%
}

\title{Státnice -- Informatika -- I4\\ Diskrétní modely a algoritmy\\ ~\\ 
Kombinatorická optimalizace}
\author{Ladislav Láska\\ Jan Musílek}

\begin{document}

\maketitle
\newpage
\tableofcontents
\newpage

\section{Grafové algoritmy}
\section{Algebraické a aritmetické algoritmy}
\section{Teorie mnohostěnů}
\section{Problém obchodního cestujícího}
\section{Speciální matice}
\section{Celočíselnost}
\section{Párování a toky v sítích}
\section{Teorie matroidů}

Matroid je struktura v kombinatorice, která zobecňuje koncept
\uv{nezávislosti}, jehož konkrétním příkladem je lineární nezávistlost ve
vektorových prostorech. Existuje mnoho ekvivalentních způsobů jak zavést
matroidy, nejvýznamnějšími jsou nezávislé množiny, báze, kružnice a ranková
funkce. Teorie matroidů si často vypůjčuje teminologii z lineární algebry a
teorie grafů.

\subsection{Definice přes nezávislé množiny}
\df Matroid $M$ je dvojice $(S,I)$, kde $S$ je konečná množina (nazýváme ji
\emph{nosná množina} a $I \subset 2^S$ je množina podmnožin $S$ (nazýváme je
nezávislé množiny), splňující následující vlastnosti:
\begin{enumerate*}
\item Prázdná množina je nezávislá, tedy $\emptyset \in I$.
\item Každá podmnožina nezávislé množiny je nezávislá, tedy pro každé $A' \subseteq A \subseteq S$ platí $A \in I \Rightarrow A' \in I$. Tato vlastnost se nazývá \emph{dědičnost}.
\item Pokud $A$ a $B$ jsou dvě nezávislé množiny z $I$, $|A| = |B| + 1$, pak existuje prvek $x\in A\setminus B$ tž. $B\cup\{x\}$ je nezávislá. Tato vlastnost se nazývá \emph{výměnná vlastnost}.
\end{enumerate*}

\df Podmnožina $S$, která není v $I$ se nazývá \emph{závislá}. Maximální nezávislá
množina (po přidání libovolného prvku se stane závislou) se nazývá \emph{báze}.
Naopak minimální závislou množinu (po odebrání libovolného prvku se stane
nezávislou) nazýváme \emph{kružnice}.

\pzn Grafová analogie: $S$ je množina hran grafu, $I$ je množina všech lesů na
hranách $S$. Báze odpovídá kostře grafu $G$, kružnice cyklu v grafu $G$.

\pzn Vektorová analogie: $S$ je množina sloupců matice. Nezávislé množiny jsou
pak právě lineárně nezávislé množiny vektorů z $S$. Báze matroidu odpovídají
bázím vektorového prostoru. 

\subsection{Definice přes báze}
\df Matroid $M$ je dvojice $(S,\B)$, kde $S$ je konečná množina a $\B$ je
množina podmnožin $S$ (tyto podmnožiny nazýváme báze), splňující následující
vlastnosti:
\begin{enumerate*}
\item $\B$ je neprázdná.
\item Když $A,B \in \B$ jsou různé a $a \in A\setminus B$, pak existuje $b \in B\setminus A$ tž. $A\setminus\{a\}\cup\{b\} \in \B$.
\end{enumerate*}

\poz Tato definice je ekvivalentní s definicí přes nezávislé množiny. Nezávislé
množiny jsou právě podmnožiny bází.

\subsection{Definice přes kružnice}
\df Matroid $M$ je dvojice $(S,\c)$, kde $S$ je konečná množina a $\c$ je
množina podmnožin $S$ (tyto podmnožiny nazýváme kružnice), splňující
následující vlastnosti: 
\begin{enumerate*}
\item $\emptyset \notin \c$
\item Pokud $A$ i $B$ jsou kružnice, pak $A\subseteq B \Rightarrow A = B$.
\item Když $A$ i $B$ jsou kružnice a $e\in A\cap B$, pak existuje kružnice v $A\cup B$, která neobsahuje $e$.
\end{enumerate*}

\poz Tato definice je ekvivalentní s definicí přes nezávislé množiny. Nezávislé
množiny jsou právě ty, které neobsahují žádnou kružnici.

\subsection{Definice přes rankovou funkci}
\df Ranková funkce je zobrazení $r: 2^S \rightarrow \N$, která každé podmnožině
$S$ přiřadí velikost její největší nezávislé podmnožiny.

\df Matroid $M$ je dvojice $(S,r)$, kde $S$ je konečná množina a $r: 2^S
\rightarrow \N$ je ranková funkce, splňující následující vlastnosti:
\begin{enumerate*}
\item $0 \le r(X) \le |X|$
\item $X\subseteq Y \Rightarrow r(X) \le r(Y)$
\item $r(X\cup Y) + r(X\cap Y) \le r(X) + r(Y)$
\end{enumerate*}

\poz Tato definice je ekvivalentní s definicí přes nezávislé množiny. Nezávislé množiny jsou právě ty, kde $r(X) = |X|$.

\subsection{Přehled jednoduchých vlastností}
\tv Všechny báze matroidu $M$ mají stejnou velikost $r(M)$. Toto číslo
označujeme jako rank matroidu $M$. V grafech je zjevně $r(G) = n-k$, kde $n$ je
počet vrcholů a $k$ počet komponent.

\df Duální matroid $M^*$ definujeme tak, že jeho báze jsou doplňky bází $M$.
Pak zjevně platí $M^{**} = M$.

\df Smyčka je prvek $S$, který není obsažen v žádné bázi. Co-smyčka je prvek
$S$ obsažený v každé bázi.

\df Uniformní matroid $U_{k,n}$ na $n$ prvcích definujeme tak, že nezávislými
podmnožinami jsou právě ty, jejihž velikost je nejvýše $k$. V $U_{0,n}$ je
každý prvek smyčka, v $U_{n,n}$ je každý prvek co-smyčka.

\subsection{Operace s matroidy}
\df Nechť $M = (S_1,I_1)$ a $N = (S_2,I_2)$ jsou matroidy. \emph{Direktní
součet} matroidů $M$ a $N$ je matroid $(S_1 \sqcup S_2, I_1 \sqcup I_2)$, kde
$A\sqcup B$ značí disjuktní sjednocení množin $A$ a $B$.

\df Nechť $M = (S_1,I_1)$ a $N = (S_2,I_2)$ jsou matroidy. \emph{Sjednocení
matroidů} $M$ a $N$ je matroid $(S_1\cup S_2, \{i_1\cup i_2 ~|~ \forall i_1\in
I_1, \forall i_2\in I_2\})$.

\subsection{Hladový algoritmus}
\df \emph{Systém nezávislých množin} je dvojice $(S,I)$, kde $S$ je konečná
množina a $I \subset 2^S$ je množina podmnožin $S$ (nazýváme je nezávislé
množiny), splňující, že $\emptyset\in I$ a každá podmnožina nezávislé množiny
je též nezávislá.

\vt Mějme $M = (X,J)$ systém nezávislých množin a orákulum, které nám pro každou
množinu $Y\subseteq X$ určí, zda je nezávislá. Definujeme hladový algoritmus
tak, že začne s prázdnou množinou $S$ a poté do ní v každém kroku přidá prvek z
$X$ s nejnižším ohodnocením, který neporuší nezávislost $S$. Potom hladový
algoritmus vygeneruje maximální nezávislou množinu $S$ s minimálním součtem
ohodnocení pro každou nezápornou ohodnocovací funkci $w: X \rightarrow \N$ právě
tehdy, když $M$ je matroid.

\dk Rozdělíme tvrzení na dvě implikace a dokážeme každou zvlášť.

\noindent \textbf{1) $M$ je matroid $\Rightarrow$ hladový algoritmus funguje pro
každé ohodnocení $w$}

Je zjevné, že hladový algoritmus nalezne maximální nezávislou množinu. Vezměmě
výstup hladového algoritmu $\{x_1,\dots,x_r\}$ tž. $w(x_1) \le \dots \le
w(x_r)$. Pro spor předpokládejme, že existuje nezávislá množina
$\{y_1,\dots,y_r\}$ tž. $w(y_1) \le \dots \le w(y_r)$ a $w(x_k) > w(y_k)$
pro nějaké $k$. Pokud by takové $k$ neexistovalo, je $\{x_1,\dots,x_r\}$
maximální nezávislá množina s minimálním ohodnocením.

Předpokládejme, že $k$ je nejmenší takové, že $w(x_k) > w(y_k)$. Zjevně $k > 1$.
Potom mějme $S = \{x_1,\dots,x_{k-1}\}$ a $T = \{y_1,\dots,y_k\}$. Platí, že
$|T| = |S|+1$ a tedy podle výměnné vlastnosti matroidů existuje $j \in
\{1,\dots,k\}$ tž. $S\cup \{y_j\}$ je nezávislá množina (a $y_j$ není v $S$).
Protože $w(y_j) \le w(y_k) < w(x_k)$, hladový algoritmus by vybral $y_j$ dříve,
než $x_k$, což je spor.

\noindent \textbf{2) hladový algoritmus funguje pro každé ohodnocení $w$
$\Rightarrow$ $M$ je matroid}

Předpokládejme, že máme systém nezávislých množin, který nesplňuje výměnnou
vlastnost. Zkonstruujeme ohodnocovací funkci $w$ takovou, že na ní hladový
algoritmus selže. Mějme tedy $A,B \subset X$ nezávislé množiny takové, že $|A| =
|B| + 1$, ale pro žádné $a \in A\setminus B$ není $B\cup \set a$ nezávislá
množina. Pak definujme:
\begin{align*}
w(x) = \left\{
	\begin{array}{lr}
		w_1\quad\text{pokud}~x \in A\cap B \\
		w_2\quad\text{pokud}~x \in B \setminus A \\
		w_3\quad\text{pokud}~x \in A \setminus B \\
		w_4\quad\text{jinak}
	\end{array}
	\right.
\end{align*}

Dále definujme, $w_1 < w_2 < w_3 < w_4$. Všimněme si, že hladový algoritmus
nejprve vybere všechny prvky z $A\cap B$, poté všechny prvky z $B\setminus A$ a
pak (podle předpokladu) už nebude moci vybrat žádný prvek z $A\setminus B$.
Protože $B$ není maximální, budeme k němu muset přidat prvek z
$X\setminus\set{A\cup B}$, který má váhu $w_4$.

Nyní nám stačí nastavit konstanty $w_1,\dots,w_4$ tak, aby byl výsledek
hladového algoritmu těžší, než kterákoliv nezávislá množina obsahující $A$
(taková zjevně musí existovat). Mějme tedy $m$ velikost maximální nezávislé
množiny. Chceme:
\begin{align*}
|A\cap B|w_1 + |B-A|w_2 + (m-|B|)w_4 > |A\cap B|w_1 + |A-B|w_3 + (m-|A|)w_4
\end{align*}
Protože $|A| = |B|+1$, tak:
\begin{align}
\label{rov:hungry-w4}
w_4 > |A-B|w_3 - |B-A|w_2
\end{align}
Hladový algoritmus vybere o jeden prvek s váhou $w_4$ více, než kdyby nejdříve
pobral celou množinu $A$. Musíme tedy nastavit $w_4$ dostatečně velké. Jeden ze
způsobů, jak jej zvolit (pro libovolné $\varepsilon\in (0,1)$):
\begin{align*}
w_1 &= \varepsilon /|X| \\
w_2 &= \varepsilon /|B-A| \\
w_3 &= (1+\varepsilon) /|A-B| \\
w_4 &= 2
\end{align*}

Pak už stačí jen dokázat, že takto zvolená $w$ splňuje $w_1 < w_2 < w_3 < w_4$ a
že platí rovnice \ref{rov:hungry-w4}.
\qed

\subsection{Matroid intersection theorem}
\todo



\section{Elipsoidová metoda}

\end{document}
