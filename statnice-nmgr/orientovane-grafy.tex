
\subsection{Turnaje a hamiltonovské cesty}

\df (Turnaj) Turnaj je orientace úplného grafu.

\vt (O počtu Hamiltonovských cest) Existuje turnaj na $n$ vrcholech, který 
obsahuje alespoň $n!/2^{n-1}$ hamiltonovských cest.

\dk Zorientujme graf nezávisle uniformě náhodně. Nechť máme permutaci $\pi$.  
Jaká je pravděpodobnost, že daná permutace (tedy pořadí vrcholů) tvoří 
orientovanou cestu? Máme $n-1$ hran na cestě, všechny mají permutací 
předepsanou orientaci. Tedy $P(X_\pi)= 1/2^{n-1}$. Nechť $I_\pi$ je indikátor 
tohoto jevu, pak $\E[I_\pi] = P(X_\pi) = 1/2^{n-1}$.

Spočítáme-li střední hodnotu počtu hamiltonovských cest, tedy součet indikátorů 
přes všechny permutace, máme $n!/2^{n-1}$. Z vlastností střední hodnoty víme, 
že existuje alespoň jeden graf, kolik má právě tolik (nebo více) 
hamiltonovských cest.
