\section{Speciální vlastnosti orientovaných grafů}

\label{silna-souvislost}

V orientovaných grafech se množiny hran skládají z uspořádaných dvojic vrcholů.
Oproti neorientovaným grafům rozlišujeme dva druhy souvislosti. Graf je slabě
souvislý, pokud je souvislá jeho symetrizace. Graf $G$ je silně souvislý, pokud
$\forall u,v \in V(G), u\neq v$ existuje orientovaná cesta z $u$ do $v$.
Analogicky definujeme silně souvislé komponenty (SSK) grafu $G$.

\subsection{Algoritmus pro hledání SSK}

Než popíšeme algoritmus, všimneme si nejprve několika vlastností orientovaných
grafů. Graf SSK je orientovaný acyklický graf. Proto v něm musí existovat
alespoň jedna zdrojová komponenta (do níž nevedou žádné hrany) a alespoň jedna
stoková komponenta (z níž nevedou žádné hrany). Kdybychom měli vrchol ze stokové
komponenty, mohli bychom z něj pustit DFS, který najde právě všechny vrcholy v
této SSK. Po odebrání této komponenty z grafu bychom našli novou stokovou
komponentu a pokračovali, dokud by nebyl celý graf rozdělen.

Nevíme sice, jak najít vrchol ze stokové komponenty, ale je jednoduché najít vrchol ze zdrojové komponenty -- stačí na $G$ spustit DFS (opakovaně, pokud neprošel všechny vrcholy). Vrchol, který opustíme jako poslední\footnote{Při zavedení DFS si obvykle zadefinujeme pořadí podle opouštění vrcholu, značíme $out(v)$.}, je určitě ve zdrojové komponentě. Je zjevné, že když otočíme orientaci všech hran v $G$, SSK se nezmění, pouze se ze stokových stanou zdrojové a naopak.

\lm {\it V grafu SSK $G$ vede hrana z komponenty $C_1$ do $C_2$, pak 
$\max_{x\in C_1} out(x) > \max_{y\in C_2} out(y)$}

Z výše uvedeného již vyplývá jakým způsobem algoritmus pracuje: Nejprve obrátí orientaci všech hran v $G$ a najde pořadí $out(v)$ pro všechny vrcholy $G$. Poté opakovaně spouští DFS z ještě neprozkoumaných vrcholů, v pořadí od nejvyššího nalezeného $out(v)$.\footnote{Zápis algoritmu v pseudokódu je přejat z \url{http://mj.ucw.cz/vyuka/ads/25-grafy.pdf}.}

\begin{enumerate*}
\item Sestrojíme graf $G^T$ s obrácenými hranami
\item $Z \leftarrow$ prázdný zásobník
\item Pro všechny vrcholy $v$ nastavíme $komp(v) = 0$
\item Spouštíme DFS v $G^T$ opakovaně, než prozkoumáme všechny vrcholy. Kdykoliv přitom opouštíme vrchol, vložíme ho do $Z$. Vrcholy v zásobníku jsou tedy setříděné podle $out(v)$.
\item Postupně odebíráme vrcholy ze zásobníku $Z$ a pro každý vrchol $v$:
\item $\qquad$Pokud $komp(v) = 0$
\leftmargin=6cm
\indent
\item $\qquad\qquad$ Spustíme DFS($v$) v $G$, přičemž vstupujeme pouze do vrcholů s $komp(x) = 0$ a tuto hodnotu přepisujeme na $v$.
\end{enumerate*}

\dk Důkaz korektnosti algoritmu je zjevný z lemmatu a pozorování, která mu předchází. Algoritmus pouze spouští DFS a žádný vrchol neprojde vícekrát, takže se určitě zastaví. Časová i prostorová složitost jsou $\Theta(n+m)$.
\qed

Existuje i tzv. Tarjanův algoritmus pro hledání SSK, který je složitější a jeho časová i prostorová složitost je asymptoticky stejná (musí být, protože také volá DFS). Nemusí ovšem konstruovat graf s obrácenými hranami.

\subsection{Turnaje a hamiltonovské cesty}

\df (Turnaj) Turnaj je orientace úplného grafu.

\vt (O počtu Hamiltonovských cest) Existuje turnaj na $n$ vrcholech, který 
obsahuje alespoň $n!/2^{n-1}$ hamiltonovských cest.

\dk Zorientujme graf nezávisle uniformě náhodně. Nechť máme permutaci $\pi$.  
Jaká je pravděpodobnost, že daná permutace (tedy pořadí vrcholů) tvoří 
orientovanou cestu? Máme $n-1$ hran na cestě, všechny mají permutací 
předepsanou orientaci. Tedy $P(X_\pi)= 1/2^{n-1}$. Nechť $I_\pi$ je indikátor 
tohoto jevu, pak $\E[I_\pi] = P(X_\pi) = 1/2^{n-1}$.

Spočítáme-li střední hodnotu počtu hamiltonovských cest, tedy součet indikátorů 
přes všechny permutace, máme $n!/2^{n-1}$. Z vlastností střední hodnoty víme, 
že existuje alespoň jeden graf, který má právě tolik (nebo více) hamiltonovských
cest.
