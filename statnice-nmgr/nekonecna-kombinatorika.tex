\section{Nekonečná kombinatorika}

\vt (Cantor-Bernstein) {\it Nechť $A$ a $B$ jsou dvě množiny, pro které existují prostá zobrazení $f: A \rightarrow B$ a $g: B \rightarrow A$. Potom existuje bijekce $h: A \rightarrow B$.

\dk Sestrojíme bipartitní graf a ukážeme, že ve všech jeho komponentách existuje párování. Hezky popsáno v \texttt{http://mj.ucw.cz/papers/kg1.pdf}

\lm (König) V každém zakořeněném stromu, který má nekonečně mnoho vrcholů, ale pouze konečné stupně, existuje nekonečná cesta začínající v kořeni.

\dk Triv. -- začnu v kořeni a zanořím se vždy do nekonečného podstromu, takový musí existovat.

Königovo lemma lze využít v důkazu věty o kompaktnosti výrokové logiky.\footnote{\it Spočetná množina formulí $\Psi$ je splnitelná právě tehdy, když je splnitelná každá její konečná podmnožina.}

\subsection{Hallova věta}

\df Systém různých reprezentantů (SRR) {\it množinového systému $(X,S)$ je prostá funkce $f:S\rightarrow X$ tž. $\forall A \in S: f(A) \in A$.}

\vt (spočetná Hallova) {\it Buď $(X,S)$ množinový systém obsahující spočetně mnoho konečných množin. Pak $(X,S)$ má SRR právě tehdy, když pro libovolný podsystém $T \subseteq S$ s konečně mnoha podmnožinami platí $|\bigcup T| \ge |T|$} (Hallova podmínka)

