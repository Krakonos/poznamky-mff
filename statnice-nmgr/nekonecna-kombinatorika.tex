\section{Nekonečná kombinatorika}

\vt (Cantor-Bernstein) {\it Nechť $A$ a $B$ jsou dvě množiny, pro které existují prostá zobrazení $f: A \rightarrow B$ a $g: B \rightarrow A$. Potom existuje bijekce $h: A \rightarrow B$.

\dk Sestrojíme bipartitní graf a ukážeme, že ve všech jeho komponentách existuje párování. Hezky popsáno v \texttt{http://mj.ucw.cz/papers/kg1.pdf}

\subsection{Königovo nekonečné lemma}

\lm (Königovo nekonečné lemma) Nechť $V_0,V_1,\dots$ je nekonečná posloupnost disjunktních neprázdných konečných množin a $G$ je graf jejich sjednocení. Předpokládejme, že každý vrchol $v \in V_n$ má souseda $f(v) \in V_{n-1}$. Potom $G$ obsahuje nekonečnou cestu $v_0v_1\dots$, kde $\forall n\in \N: v_n\in V_n$.

\medskip\noindent {\bf Důsledek} V každém zakořeněném stromu, který má nekonečně mnoho vrcholů, ale pouze konečné stupně, existuje nekonečná cesta začínající v kořeni.

\dk Vezmeme všechny konečné cesty končící ve $V_0$. Těch je nekonečně, takže z principu holubníku existuje $v_0\in V_0$ tž. jím prochází nekonečně mnoho konečných cest. Některým z jeho sousedů $v_1\in V_1$ prochází také nekonečně mnoho konečných cest. Tímto argumentem můžeme pokračovat neomezeně a vygenerovat tak hledanou nekonečnou cestu $v_0v_1\dots$
\qed

Königovo lemma lze využít v důkazu věty o kompaktnosti výrokové logiky.\footnote{\it Spočetná množina formulí $\Psi$ je splnitelná právě tehdy, když je splnitelná každá její konečná podmnožina.}

\subsection{Věta o barevnosti nekonečných grafů}

\vt (Bruijn \& Erdös, 1951) {\it Nechť $G$ je graf a $k\in \N$. Pokud je každý konečný podgraf $G$ obarvitelný nejvýše $k$ barvami, pak je nejvýše $k$ barvami obarvitelný i $G$.}

\dk Mějme očíslování vrcholů $G$ $v_0, v_1, \dots$ Definujeme $G_n = G[v_0,\dots, v_n]$. Množinu všech $k$-obarvení grafu $G_n$ označíme $C_n$. Vytvoříme graf, jehož vrcholy tvoří jednotlivá obarvení $\bigcup_{n\in\N} C_n$ a hrany $cc'$ tž. $c\in C_n$ a $c'\in C_{n-1}$ (tedy hrany vedou mezi obarvením grafu s $n-1$ vrcholy, do všech rozšíření tohoto obarvení na graf s $n$ vrcholy). V tomto grafu dle Königova nekonečného lemma najdeme nekonečnou cestu $c_0c_1\dots$ tž. $\forall n\in\N: c_n \in C_n$. Potom $c := \bigcup{n\in\N}c_n$ je obarvení grafu $G$ $k$ barvami.
\qed

\subsection{Hallova věta}
\label{nekonecna-kombinatorika:hall}

\df Systém různých reprezentantů (SRR) {\it množinového systému $(X,S)$ je prostá funkce $f:S\rightarrow X$ tž. $\forall A \in S: f(A) \in A$.}

\vt (spočetná Hallova) {\it Buď $(X,S)$ množinový systém obsahující spočetně mnoho konečných množin. Pak $(X,S)$ má SRR právě tehdy, když pro libovolný podsystém $T \subseteq S$ s konečně mnoha podmnožinami platí $|\bigcup T| \ge |T|$} (Hallova podmínka)

\dk Technický. Pro každý prvek $x \in X$ zavedeme výrokovou proměnnou $a_x$ a nadefinujeme si výrokové formule tak, aby vyjadřovaly podmínky, že každá množina má reprezentatna a dvě různé množiny nemají stejného reprezentanta. Potom použijeme větu o kompaktnosti výrokové logiky.

