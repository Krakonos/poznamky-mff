\section{Kombinatorické počítání}
Do této kapitoly patří princip inkluze a exkluze, problém šatnářky, počítání dvěma
způsoby, různé odhady faktoriálu a binomických koeficientů apod.

\subsection{Princip inkluze a exkluze}

\vt \emph{(Princip inkluze a exkluze)} Pro každý soubor $A_1,A_2,\dots,A_n$
konečných množin platí

$$\left|\bigcup_{i=1}^n A_i\right|
= \sum_{k=1}^n \left( (-1)^{k-1} \sum_{I\in{\{1,\dots,n\}\choose k}} \left|\bigcap_{i\in I} A_i\right| \right)
= \sum_{\emptyset\neq I\subseteq \{1,\dots,n\}} (-1)^{|I|-1}\left|\bigcap_{i\in I}A_i\right|$$

V případě, že neznáme velikosti všech průniků, ale známe velikosti všech alespoň
$m$-násobných průniků, pak nám \emph{Bonferroniho nerovnost} říká, že chyba
vzniklá zanedbáním průniků všech více než $m$-násobných průniků má stejné
znaménko, jako první vynechaný průnik.

\subsection{Problém šatnářky}

\noindent\textbf{Pohádka:} Přijde $n$ pánů do divadla a nechají své kloubouky v
šatně. Šatnářka, neb je roztržitá, vydá při odchodu každému pánovi náhodně
jeden z klobouků. Jaké je pravděpodobnost, že žádný pán nedostane svůj klobouk?

\smallskip
\noindent\textbf{Zadání:} Jaký je počet permutací množiny $\{1,\dots,n\}$ bez
pevného bodu?

Problém můžeme vyřešit pomocí principu inkluze a exkluze. Označme $A_i$ množinu
všech permutací, ve kterých je $i$ pevným bodem. Počet všech permutací s alespoň
jedním pevným bodem je pak zjevně $\left|\bigcup_{i=1}^n A_i\right|$. Přitom
velikost každé $A_i$ je zjevně $(n-1)!$ (prvek $i$ je zafixovaný a ostatní mohou
být zpermutovány libovolně) a velikost průniku $k$ různých $A_i$ je $|A_{i_1}
\cap A_{i_2} \cap \dots \cap A_{i_k}| = (n-k)!$. Z toho dosadíme do principu
inkluze a exkluze.

$$\left|\bigcup_{i=1}^n A_i\right| = \sum_{k=1}^n (-1)^{k-1}{n\choose k}(n-k)! =
\sum_{k=1}^n (-1)^{k-1}{n!\over k!}$$

Tedy počet permutací bez pevného bodu je:

$$n! - \sum_{k=1}^n (-1)^{k-1}{n!\over k!} = n!\left( 1-{1\over 1!}+{1\over
2!}-\cdots +(-1)^n{1\over n!} \right) \approx {n!\over e}$$


\subsection{Cayleyho formule}

\vt \emph{(Cayleyho formule)} $\forall n\ge 2$ je $\kappa(K_n)$, tj. počet
stromů na daných $n$ vrcholech, roven $n^{n-2}$.

\dk Budeme počítat tzv. \emph{povykosy}. Povykos\footnote{Postup výroby kořenového
stromu.} je strom, v němž je jeden vrchol označen jako kořen a hrany jsou
očíslovány od 1 do $m$. Kořen vybírám z $n$ možností, hrany mohu očíslovat
$(n-1)!$ způsoby a počet stromů na $n$ vrcholech je $\kappa(K_n)$. Počet
povykosů na $n$ vrcholech je tedy zjevně $n\cdot(n-1)!\cdot\kappa(K_n)$.

Nyní spočítáme povykosy druhým způsobem. Budeme stavět rovnou stromy orientované
směrem ke kořeni. Pravidla pro přidávání orientovaných hran jsou následující:
\begin{enumerate*}
\item Nesmíme vytvořit (ani neorientovanou) kružnici, tedy každá hrana musí
spojovat dvě různé komponenty již vytvořenho grafu.
\item V každém vrcholu až na jeden bude začínat právě jedna hrana. V kořeni
nezačíná žádná hrana. Navíc je vidět, že v každé komponentě existuje právě jeden
vrchol, do kterého nevede žádná hrana (tedy každá komponenta má svůj kořen).
\end{enumerate*}
Z těchto pravidel už je jasné jak budeme postupovat. Po přidání $k$ hran má graf
$n-k$ komponent. V dalším kroku vybereme jeden libovolný vrchol (z množiny všech
vrcholů), do kterého hrana povede a počáteční vrchol hrany vybereme z ostatních
komponent -- těch je $n-k-1$. Z toho dostáváme počet povykosů $\prod_{k=0}^{n-2}
n(n-k-1)$. Počítali jsme dvěma způsoby, teď mezi ně položíme rovnítko:

\begin{align*}
\prod_{k=0}^{n-2} n(n-k-1) &= n\cdot(n-1)!\cdot\kappa(K_n) \\
n^{n-1}\cdot (n-1)! &= n\cdot(n-1)!\cdot\kappa(K_n) \\
\kappa(K_n) &= n^{n-2}
\end{align*}
\qed

\subsection{Fibonacciho čísla}

V kapitole \ref{sec:rekurence} jsou Fibonacciho čísla zadefinována a je
proveden výpočet explicitního vzorce pro $n$-tý člen pomocí vytvořujících
funkcí.

\subsection{Catalanova čísla}

Catalanova čísla jsou udána rekurencí
$$C_0 = 1\qquad C_{n+1} = \sum_{i=0}^n C_iC_{n-i} = {2(2n+1)\over n+2}C_n$$
Explicitní vzorec pro $n$-té Catalanovo číslo je 
$$C_n = {1\over n+1}{2n\choose n}$$
Některé aplikace Catalánových čísel:
\begin{itemize*}
\item Počet zakořeněných binárních stromů s $n$ listy je $C_{n-1}$.
\item Počet korektních uzávorkování $2n$ závorek je $C_n$.
\item Počet triangulací konvexního $n$-úhelníku je $C_{n-2}$.
\item Počet způsobů, jak se v mřížce $n\times n$ dostat $2n$ kroky z levého dolního do pravého horního rohu, aniž bychom překročili diagonálu je $C_n$.
\end{itemize*}


