\subsection{Roviny, nadroviny, mnohostěny}

\df (Nadrovina) V $\R^n$ je nadrovina množinou všech $x$ takových vektorů, kde
se zafixuje jedna (nebo více) složek vektoru. Matematicky tedy: $\{x \in \R^n:
a^Tx = b\}$ pro nějaké $a \in \R^n$, kde $a \neq 0$ a $b \in R$ (nulový vektor by generoval
celý prostor, třeba napíšeme jedničku do skožky $a$, kterou chceme zafixovat,
a její požadovanou hodnotu do $b$).

\poz Všimneme si, že pokud chceme fixovat více souřadnic, tak to neznamená, že
se jejich hodnoty zafixují, ale spíše jejich součet (podle hodnoty $a$). To ale
ve skutečnosti chceme, tím se totiž dá vyrobit nadrovina, která má jiný
\uv{sklon}, než kolmý na jednu osu.

\df (Poloprostor) V $\R^n$ je poloprostor množina všech $x$ takových vektorů,
které jsou \uv{na jedné straně nadroviny}, matematicky tedy:  $\{x \in \R^n:
a^Tx \leq b\}$ pro nějaké $a \in \R^n, a \neq 0, b \in \R$.

\df (Konvexní mnohostěn) Konvexní mnohostěn v $\R^n$ je množina, kterou lez
vyjádřit jako průnik konečně mnoha poloprostorů.

\poz Může se snadno stát, že mnohostěn sice bude konvexní, ale nebude omezený.

\vt (Oddělovací) Nechť množiny $C,D \subseteq \R^n$ jsou uzavřené, konvexní a disjunktní
a $C$ navíc omezená. Pak existuje nadrovina, která silně odděluje $C$ a $D$, tj.
$C \subseteq \{x | a^Tx < b\}$ a $D \subseteq \{x | a^T x > b\}$.

\dk Seženeme si $c \in C$ a $d \in D$ takové, že jejich vzdálenost je minimální.
Jediný problém může být, když nebude $D$ omezená. V takovém případě můžeme ale
vybrat nějaký bod $d'\in D$, nejvzdálenější bod k němu v $c'\in C$ a potom
udělat kružnici se středeme v $c'$ protínající $d'$; průnik této kružnice s $D$
nám výběr potřebně omezní a z výběru $c'$ a $d'$ je zřejmé, že správný bod v ní
bude ležet (jinak jsem špatně vybral nejvzdálenější $c'$).

Pak stačí vzít vektor $(c,d)$ jako normálu naší nadroviny, a ukotvit ji třeba na
půli cesty mezi $c$ a $d$:
\begin{align}
	a &:= d - c \\
	b &:= a^T\left({c+d \over 2}\right)
\end{align}

\df (Kužel) Nechť $a_i\in R^n$ jsou vektory. Potom definujeme kužel $C$ jako
konvexní obal polopřímek vycházejících z počátku a procházejících body $a_i$.

\lm (Farkas) Nechť $a_1, \dots, a_n, b \in \R^m$. Pak nástane jedna z možností:
\begin{enumerate}
	\item Bod $b$ leží v kuželu definovaným $a_i$.
	\item Existuje nadrovina $h$ procházející počátkem, tvaru $h = \{x \in \R^m
	: y^Tx = 0\}$, oddělující bod $b$ od kuželu definovaného $a_i$.
\end{enumerate}
\dk Pokud je $b$ v kuželu, je hotoov. Jinak nadrovinu můžeme najít pomocí
oddělovací věty aplikované na $b$ a kužel. Ta nám sice nedá nadrovinu nutně
procházející počátkem, ale protože je kužel konvexní a nadrovina $b$ odděluje
od kuželu, můžeme ji do počátku posunout a kužel stále protínat nebude. \qed

Dále se nám bude hodit ale více algebraické vyjádření:

\lm (Farkas) Soustava nerovnic $Ax \leq b$ má nezáporné řešení $x$ právě, když
každé nezáporné $y$, pro než $y^TA \geq 0^T$ splňuje také $y^Tb \geq 0$.

\subsection{Lineární programování}

\df (Standardní tvar) Úloha LP je ve standardním tvaru, pokud je tvaru:
\begin{align}
	\max_x c^Tx : \quad Ax \leq b
\end{align}

\df (Rovnicový tvar) Úloha LP je v rovnicovém tvaru, pokud je tvaru:
\begin{align}
	\max_x c^Tx : \quad Ax = b, \quad x \geq 0
\end{align}

Tedy jediná nerovnice, je podmínka na nezápornost vektoru $x$. Máme-li obecnou
úlohu LP, můžeme ji celkem snadno do tohoto tvaru převést, nerovnice totiž lze
nahradit přidáním dalších proměnných a konstant tak, aby stačila nezápornost.

Pro simplexovou metodu chceme rovnicový tvar, pro geometrickou představu nám
bude ale milejší stadardní tvar, protože svou podobou každá nerovnice přímo
vyjadřuje nějakou nadrovinu, a tedy jejich průnik nám tvoří mnohostěn!

\df (Bázické řešení) Vektor $x \in \R^n$ je bázické přípustné řešení úlohy LP v
rovnicovém tvaru, pokud je $x$ přípustné řešení a existuje $B \subseteq \{1,
\dots, n\}$ taková, že čtvercová matice $A_B$ (sloupečky indexované $B$) je
regulární a $x_j = 0$ kdykoliv $j \notin B$.

\tv (Unikátnost bázických řešení) Pro každou množinu $B$ existuje nanejvýš jedno
přípustné bázické řešení $x \in \R^n$.

\dk Kdykoliv máme $B$, matice $A_B$ je regulární a tedy soustava $A_B x_B = b_B$
právě jedno řešení. Podmínka pro přípustné bázické řešení navíc říká, že ostatní
prvky $x_{\overline{B}}$ musí být nulové. \qed

\df (Báze) Množině $B$ budeme říkat báze. Pokud navíc určuje přípustné řešení,
budeme jí říkat {\it přípustná báze}.

\vt (O existenci optimálního řešení) Pro program v rovnicovém tvaru platí:
\begin{enumerate}
	\item Pokud existuje alespoň jedno přípustné řešení a účelová funkce je na
	množině všech přípustných řešení shora omezená (pro minimalizační zdola),
	pak existuje také optimální řešení.
	\item Existuje-li optimální řešení, potom i některé z bázickcýh přípustných
	řešení je optimální.
\end{enumerate}
\dk Trošku technický, stačí ale dokázat, že pokud je účelová funkce omezená, pro
libovolné řešení $x_0$ existuje lepší bázické řešení $x_b$, které je alespoň tak
dobré.

To nám samo o sobě dává informaci, jak řešit lineární programy: stačí projít
všechny báze, vyřešit pro každou z nich soustavu rovnic, a hledat
nejlepší řešení. Bohužel, bazí je exponenciálně mnoho, abychom to dělali hloupě.

{\it Simplexová metoda} je chytrý způsob, jak řešit lineární programy. Typicky
začíná s nějaký bázickým řešením a s vědomostí, že existuje optimální bázické řešení,
se k němu snaží dospět postupným vylepšováním účelové funkce. Ačkoli je to
metoda v praxi rychlá, nikde není záruka, že běží v polynomiálním čase --
bohužel se může stát, že před nalezením optima projde většinu bazických řešení.


\vt (O Dualitě LP) Pro dvě úlohy LP:
\begin{align}
\tag P
\label{dualita:P}
	\max c^Tx&: \quad Ax \leq b, \quad x \geq 0 \\
\tag D
\label{dualita:D}
	\min b^Ty&: \quad A^T y \geq c, \quad y \geq 0
\end{align}
nastane jedna z následujících možností:
\begin{enumerate}
	\item Ani jedna z úloh nemá přípustné řešení.
	\item Jedna z úloha je neomezená, druhá nemá přípustné řešení.
	\item Obě úlohy mají přípustné řešení, a navíc pro optimální řešení $x^*$ a
	$y^*$ platí $c^T x^* = b^Ty^*$.
\end{enumerate}

\dk Použijeme Farkasovo Lemma. Nechť $\gamma = c^Tx^*$ je hodnota účelové funkce
pro nějaké optimální řešení \eqref{dualita:P}. Pro $\epsilon \geq 0$ má tedy z
definice optima systém $Ax \leq b, c^Tx \geq \gamma + \epsilon$ přípustné řešení právě pro
$\epsilon = 0$. Farkasovo lemma ale neví nic o účelové funkci. Nevadí, přidáme
ji jako řádek do naší matice $A$ a optimalitu vyjádříme jako $\gamma + \epsilon$
v jedné složce pravé strany!
\begin{align}
	A' = \left(\begin{array}{c} A \\ -c^T \end{array}\right), \quad b' =
	\left(\begin{array}{c} b \\ -\gamma - \epsilon\end{array}\right)
\end{align}
A to již můžeme poslat do lemmatu a protože víme, že pro $\epsilon > 0$ žádné
nezáporné řešení neexistuje. Musí tedy existovat nějaký nezáporný vektor $y'$, že
$y'^TA' \geq 0$, ale $y'^T b' < 0$. To ale můžeme dát dohromady a levou stranu
transponovat:
\begin{align}
	y^TA' \geq 0 > y^T b' \qquad \Rightarrow \qquad
	A'^Ty \geq b'^T y
\end{align}
Co to znamená? Pamatujme, že cílovou funkci máme zakódovanou ve vektoru
$y'$, dává nám to tedy nějaké řešení duálního programu, které je jenom o epsilon
menší, než optimum primáru. Protože to můžeme udělat pro libovolně malé
$\epsilon$, musí si být rovny. 
\qed

