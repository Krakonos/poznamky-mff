\section{Rekurence}
\label{sec:rekurence}

Základním stavebním kamenem při počítání rekurencí jsou vytvořující funkce (viz
sekci \ref{sec:vytvorujici-funkce}). Nejprve si uvedeme větu pro řešení
lineárních rekurencí a pak ukážeme její použití pro nalezení explicitního
vzorce pro $n$-tý člen Fibonacciho posloupnosti.

\subsection{Věta o řešení lineárních rekurencí}

\vt (kuchařka na řešení lineárních rekurencí) Buď $A_{n+k} = c_0A_n + \dots +
c_{k-1}A_{n+k-1}$ lineární rekurence s počátečními podmínkami
$A_0,\dots,A_{k-1}$. Dále buď $R(x) = x^k - c_{k-1}x^{k-1} - \dots - c_0x^0$
její charakteristický polynom a $\lambda_1,\dots,\lambda_z$ navzájem různé
kořeny tohoto polynomu s násobnostmi po řadě $k_1,\dots,k_z$. Potom existují
konstanty $C_{ij} \in \C$ takové, že
$$A_n = \sum_{i=1}^z\sum_{j=0}^{k_i-1}\left(C_{ij}\cdot{n+j\choose j}\cdot\lambda_i^n\right)$$
Pokud polynom $R$ nemá násobné kořeny, vztah lze zapsat jednodušeji:
$$A_n = \sum_{i=1}^kC_i\lambda_i^n$$

\dk Konstrukcí\footnote{Částečně přejato z \url{http://mj.ucw.cz/papers/linrec.pdf}.}. Mějme zadanou lineární rekurenci, kde pro každé $n\ge k$ platí:
$$A_n = c_0A_{n-k} + c_1A_{n-k+1} + \dots + c_{k-1}A_{n-1}$$

Označíme-li $G$ hledanou vytvořující funkci, platí $A_n = [x^n]G(x)$, kde
operátor $[x^n]$ značí koeficient u $x^n$ v mocninné řadě pro funkci $G$. Pak
také $A_{n-i} = [x^n](G(x)\cdot x^i)$. Přepíšeme tedy rekurentní vztah novou
notací:

\begin{align}
[x^n]G(x) &= c_0[x^n](G(x)\cdot x^k) + c_1[x^n](G(x)\cdot x^{k-1}) + \dots + c_{k-1}[x^n](G(x)\cdot x^1) \\
&= [x^n]\left(G(x)\cdot (c_0x^k + c_1x^{k-1} + \dots + c_{k-1}x^1)\right)\end{align}

Pokud by se koeficienty u $x^n$ rovnaly pro všechna $n$, byly by stejné i
vytvořující funkce. Pokud se nerovnají pro $n<k$, pak se funkce na levé a pravé
straně liší o polynom $P(x)$ stupně menšího než $k$:

\begin{align}
G(x) &= G(x)\cdot(c_0x^k + \dots + c_{k-1}x^1) - P(x) \\
G(x) &= {P(x)\over c_0x^k+c_1x^{k-1}+\dots +c_{k-1}x^1-1}
\end{align}

Nalezneme kořeny $\alpha_1, \dots, \alpha_k$ polynomu ve jmenovateli zlomku,
tj. polynomu $Q(x) = c_0x^k+c_1x^{k-1}+\dots +c_{k-1}x^1-1$. Předpokládejme,
že jsou navzájem různé. Pak můžeme provést rozklad na parciální zlomky:
$$G(x) = {C_1\over x-\alpha_1} + \dots + {C_k\over x-\alpha_k}$$

Když zjistíme, jakou posloupnost generuje $C_i/(x-\alpha_i)$, máme
vyhráno. Vydělíme proto čitatele i jmenovatele $-\alpha_i$ a dostaneme
ekvivalentní tvar:
$${-C_i/\alpha_i\over1-{1\over\alpha_i}x} = {-C_i\lambda_i\over 1-\lambda_ix},\qquad\text{kde}~\lambda_i = 1/\alpha_i$$

Pomocí pravidla o dosazení $\lambda_ix$ za $x$ a násobení $\lambda_i$
zjistíme, že vytvořující funkce odpovídá polynomu, jehož koeficient u $x^n$
je $D_i\lambda_i^n$ pro nějakou konstantu $D_i$. Když sečteme všechny parciální zlomky, dostaneme explicitní vzorec:

$$A_n = [x^n]G(x) = D_1\lambda_1^n + \dots + D_k\lambda_k^n$$

Důkaz pro vícenásobné kořeny je podobný, jen se musíme vypořádat s tím, že nám
v rozkladu na parciální zlomky vzniknou členy ve tvaru $C_i/(x-\alpha_i)^j$. K
tomu nám pomůže zobecněná binomická věta.
\qed

Konstanty $D_i$ spočteme z počátečních podmínek rekurence (vyřešením soustavy
lineárních rovnic). Pro praktické počítání si všimněme, že $\lambda_i$ jsou
kořeny zrcadlového polynomu ke $Q(x)$ -- můžeme tedy počítat rovnou kořeny
tohoto zrcadlového polynomu $Q^*(x) = x^k-c_{k-1}x^{k-1}-\dots-c_0x^0$ a ušetřit
si práci. Tento polynom se při studiu rekurencí často používá a říká se mu
charakteristický polynom rekurence.

\subsection{Příklady použití}
\noindent\textbf{Příklad} (odvození explicitního vzorce pro $n$-tý člen Fibonacciho posloupnosti)
$$F_0 = 0\qquad\qquad F_1 = 1\qquad\qquad F_n = F_{n-1} + F_{n-2}$$

Tedy vytvořující funkce $F(x)$ bude vypadat následovně:
$$F(x) = {-x\over x^2+x-1} = {x\over 1-x-x^2}$$

K tomu zrcadlový polynom jmenovatele:
$$Q^*(x) = x^2-x-1\qquad\qquad\lambda_{1,2} = {1\pm\sqrt 5\over 2}$$

Tedy explicitní vzorec bude (zatím s neznámými konstantami $D_i$):
$$F_n = D_1\left({1+\sqrt 5\over 2}\right)^n + D_2\left({1-\sqrt 5\over 2}\right)^n$$

Konstanty $D_1$ a $D_2$ odvodíme z rovnic pro $F_0$ a $F_1$, tedy:
$$F_0 = 0 = D1 + D2$$
$$F_1 = 1 = D_1\left({1+\sqrt 5\over 2}\right) + D_2\left({1-\sqrt 5\over 2}\right)$$

Vyřešíme soustavu lineárních rovnic a dostaneme $D_1 = 1/\sqrt 5$, $D_2 =
-1/\sqrt 5$. Platí tedy explicitní vzorec pro $n$-té Fibonacciho číslo:

$$F_n = {1\over\sqrt 5}\left(\left({1+\sqrt 5\over 2}\right)^n - \left({1-\sqrt 5\over 2}\right)^n\right)$$
\qed

\subsection{Master theorem}

Master theorem nachází využití v analýze časové složitosti rekurzivních
algoritmů. Máme-li algoritmus, který v každém kroku rozdělí problém na $a$
podproblémů velikosti $N/b$ a zároveň tímto dělením a naopak sestavováním
celků v odpovědi stráví čas $\Theta(N^c)$, můžeme použít následující.

\vt \emph{(Master theorem)} Rekurentní rovnice $T(N) = a\cdot T(N/b) + \Theta(N^c), T(1) = 1$ má pro konstanty $a \ge 1, b > 1, c\ge 0$ řešení:
\begin{itemize*}
\item $T(N) = \Theta(N^c \log N)$, pokud $a/b^c = 1$
\item $T(N) = \Theta(N^c)$, pokud $a/b^c < 1$
\item $T(N) = \Theta(N^{\log_b a})$, pokud $a/b^c > 1$
\end{itemize*}

\dk Uvažujme rekurenci:\footnote{Důkaz je částečně převzat z \url{mj.ucw.cz/vyuka/ads/29-dac.pdf}}
\begin{align*}
T(1) &= 1 \\
T(N) &= a\cdot T(N/b) + \Theta(N^c)
\end{align*}

Bez újmy na obecnosti předpokládejme, že $N$ je mocninou $b$. Pokud by nebylo,
spočítáme horní odhad přes nejbližší vyšší mocninu $b$ a uvědomíme si, že je
vyšší jen konstantakrát. Představíme si strom rekurze. Každý vnitřní vrchol
stromu má přesně $a$ synů, takže na $i$-té hladině se nachází $a^i$ vrcholů.
Problémy na $i$-té hladině mají velikost $N/b^i$ a celý strom rekurze má
hloubku $\log_b N$.

V jednom vrcholu $i$-té hladiny strávíme čas $\Theta((N/b^i)^c)$, na celé
hladině $\Theta(a^i\cdot (N/b^i)^c)$. Pro snadnější počítání tento čas upravíme do
tvaru $\Theta(N^c\cdot (a/b^c)^i)$. Potom v součtu na všech hladinách strávíme:
\begin{align*}
\Theta(N^c\cdot [(a/b^c)^0 + (a/b^c)^1 + \dots + (a/b^c)^{\log_b N} ])
\end{align*}

Výraz v hranatých závorkách je geometrická řada s kvocientem $q=a/b^c$. Její součet závisí na velikosti kvocientu.
\begin{itemize*}
\item $q = 1$: Všechny členy řady jsou rovny jedné, součet je tedy $\log_b N +
	1$. Tomu odpovídá časová složitost $T(N) = \Theta(N^c \cdot \log N)$
\item $q < 1$: Odhadneme součet shora součtem nekonečné geometrické
	posloupnosti s kvocientem $q$, tedy $1/(1-q)$, což je konstanta. Proto
	$T(N) = \Theta(N^c)$.
\item $q > 1$: Řadu sečteme\footnote{Podle vzorce pro součet geometrické posloupnosti $s_n = a_1\cdot(q^n-1)/(q-1)$ pro $q\neq 1$.} na $(q^{1+\log_b N} - 1) / (q-1) = \Theta(q^{log_b N})$. Nejvíce času v tomto případě trávíme v listech. Výraz chytře upravíme:
\begin{align*}
q^{\log_b N} = \left(a\over b^c\right)^{\log_b N} = {b^{\log_b a\cdot\log_b N} \over b^{c\cdot\log_b N}} = {(b^{\log_b N})^{\log_b a} \over (b^{\log_b N})^c} = {N^{\log_b a}\over N^c}
\end{align*}
Vyjde nám $T(N) = \Theta(N^c\cdot N^{\log_b a} / N^c) = \Theta(N^{\log_b a})$.
\end{itemize*}
\qed

