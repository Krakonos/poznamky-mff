\section{Speciální matice}

\subsection{Pozitivně definitní matice}

\df \emph{Pozitivně definitní matice} je čtvercová matice $M$, pro kterou platí
$$\mathbf{x} \neq 0 \Rightarrow \mathbf{x}^TM\mathbf{x} > 0$$
Pozitivně definitní matice má vždy kladná vlastní čísla. Naopak, každá
symetrická matice, jejíž vlastní čísla jsou kladná, je pozitivně definitní.
Obecněji můžeme definovat pozitivně definitní matici v oboru komplexních čísel.

\df \emph{Pozitivně definitní matice} je hermitovská\footnote{V hermitovské
matici platí, že $M_{ij} = \overline{M_{ji}}$, tedy že na symetrickém místě se
nachází číslo komplexně sdružené. Stejně tak zobecněná transpozice $\mathbf{z}^*
= \mathbf{\bar z}^T$.} matice $M$, pro kterou platí
$$\mathbf{z} \neq 0 \Rightarrow \mathbf{z}^*M\mathbf{z} > 0$$

Analogicky můžeme definovat i negativně definitní matice.

\subsection{Pozitivně semidefinitní matice}
\df \emph{Pozitivně semidefinitní matice} je čtvercová matice $M$, pro kterou platí
$$\forall \mathbf{x} \in \C^n:\quad \mathbf{x}^*M\mathbf{x} \ge 0$$

Analogicky můžeme definovat i negativně semidefinitní matice.

\todo Láďa: formulace semidefinitního programování + příklad MAXCUT, souvislost
s elipsoidovou metodou.

\subsection{Hadamardovy matice}

\df Hadamardova matice je čtvercová matice s hodnotami pouze $+1$ a $-1$, jejíž
řádky jsou po dvou ortogonální.

Mezi jejich zajímavé vlastnosti Hadamardovy matice patří:

\begin{itemize*}
\item $HH^T = n\In$
\item $\det(H) = \pm n^{n\over 2}$ Navíc má Hadamardova matice v absolutní
hodnotě největší determinant ze všech matic, jejichž prvky nabývají hodnot z
intervalu $[-1, 1]$.
\item Každé dva řádky mají přesně polovinu hodnot v odpovídajících sloupcích stejných a polovinu opačných.
\end{itemize*}

\tv Řád Hadamardovy matice musí být 1, 2, nebo násobek 4.

\alg (Silvestrova konstrukce) Mějme Hadamardovu matici $H$ řádu $n$. Potom 
$$\begin{pmatrix}
H & H \\
H & -H
\end{pmatrix}$$
je Hadamardova matice řádu $2n$.

Zobecněním této konstrukce můžeme dojít k odvození, že mám-li Hadamardovy matice
řádu $n$ a $m$, dokážeme zkonstruovat Hadamardovu matici řádu $nm$. Ani to nám
ale ještě nestačí na všechny násobky 4.

\conj \emph{(Hadamard conjencture)} Pro každé $k \in \Z^+$ existuje Hadamardova
matice řádu $4k$.

\smallskip
Nejmenší Hadamardova matice, o níž se zatím neví zda existuje, je řádu 668.

\smallskip
Hadamardovy matice se používají v samoopravných kódech, v tzv. Hadamardově kódu.
Ten se používá všude tam, kde hrozí velký šum a nespolehlivost přenosového
kanálu. Byl např. použit v roce 1971 pro přenos fotografií Marsu zpět na zem ze
sondy Mariner 9.

\smallskip\noindent\textbf{Hadamardův kód} mapuje zprávy délky $k$ v binární
abecedě na kódová slova délky $2^k$, přičemž minimální vzdálenost kódu je
$2^k/2$. Kódová slova získáme jako řádky $H_n$ a $-H_n$ a nahrazením všech
$1$ za $0$ a $-1$ za $1$. Tak získáme $2n$ kódových slov délky $n = 2^k$.

Existuje i varianta Hadamardova kódu s kódovými slovy délky $2^{k-1}$ a
minimální vzdálenosti $2^{k-2}$, který je v praxi používán častěji, neboť při
stejné relativní vzdálenosti používá kratší kódová slova.

\tv Pro délku bloku zprávy $k \le 7$ jsou Hadamardovy kódy optimální vzhledem k
minimální vzdálenosti kódu.

\subsection{Totálně unimodulární matice}

\df (Totálně unimodulární matice) Matice je totálně unimodulární, pokud jsou
determinanty všech jejích čtvercových podmatic rovny $0$, $1$ nebo $-1$.

Více o aplikacích v sekci \ref{LP:celociselnost}, v rámci celočíselnosti LP.
