{\it Trošku širší úvod lze nalézt ve skriptíčkách J. Matouška: Lineární
programování a lineární algebra pro informatiky, ITI série 2006-311.}

Elipsoidová metoda je výrazná proto, že jako první dala polynomiální odhad na
řešení lineárních programů. Její praktičnost ale mírně pokulhává, za chvíli bude
jasné proč.

\alg (Elipsoidová metoda) Pro LP $\max c^Tx: Ax \leq b$, $R$ a $\varepsilon$
definujeme algoritmus, který vrátí nějaký bod polyedru $P$ definovaného rovnicí $Ax \leq b$,
pokud tento polyedr obsahuje kouli velikosti alespoň $\varepsilon$.
\begin{enumerate}
	\item Začni s koulí $E_0 := B(0, R)$ (předpokládáme, že $P \subseteq E_0$).
	\item V kroku $k$ nechť $s_k$ je střed elipsoidu $E_k$. Pokud $s_k \in P$,
	vrať $s_k$ jako výsledek a skonči.
	\item Jinak zvol nějakou porušenou nerovnost, označ $H_k$ jako půlelipsoid
	$E_k$ vyhovující porušené nerovnosti a najdi elipsoid $E_{k+1}$, který má
	nejmenší objem takový, že obsahuje $H_k$.
	\item Je-li objem $E_{k+1}$ menší, než objem koule o poloměru $\varepsilon$,
	oznam že úloha nemá řešení a skonči. Jinak pokračuj krokem 2.
\end{enumerate}
Tento algoritmus provede zhruba $(\O n^2\log (R/\varepsilon))$ iterací.

Když pomineme spoustu chlupatých problémů, které je potřeba v algoritmu vyřešit
(například jak se poprat s elipsoidama, které nevypadají, že by měly racionální
parametry), zbývá ještě vyřešit pár problémů.

Například algoritmus jak je popsaný vůbec nebere v úvahu účelovou funkci. To lze
napravit tak, že ji vyjádříme jako další nerovnost do soustavy, parametrizujeme
a nějak odhadneme. Elipsoidovou metodou pak zjistíme, zda existuje nějaké řešení
vyhovující našeho parametru a půlením intervalu můžeme najít optimum (s
libovolnou přesností).

Zajímavé ale je, že elipsoidová metoda nepotřebuje znát celou soustavu rovnic,
stáčí jí oddělovací orákulum, které v kroku 3. ukáže na nějakou nerovnost, podle
které se má změnit elipsoid. Pomocí tohoto triku lze například řešit soustavy,
které mají exponencielně mnoho nerovností, ale je jednéduché (polynomiální)
určit, zda je nějaké řešení přípustné, a případně poukázat na přerušenou
nerovnost (typicky například MAXCUT, kde nalezení cesty ukazuje na porušenou
nerovnost).
