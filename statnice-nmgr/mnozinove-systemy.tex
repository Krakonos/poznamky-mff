\subsection{Extrémální vlastnosti}

\df Slunečnice (někdy také $\Delta$-systém) s $k$ lístky a jádrem $Y$ je systém 
množin $S_1, \dots, S_k$ takových, že $\forall i\neq j: S_i \cap S_j = Y$ a 
navíc $S_i \setminus Y \neq \emptyset$.

\vt (Slunečnicové lemma) Nechť $\F$ je množinový systém s množinami velikosti 
$s$.  Pokud $|\F| > s!  (k-1)^s$, potom $\F$ obsahuje slunečnici s $k$ lístky.

\dk Indukcí podle $s$. Pro $s=1$ máme alespoň $k-1$ jednoprvkových množin, a 
všechny tvoří slunečnici s malými lístky a prázdným jádrem. Dále nechť $s\geq 
2$: $\A := \{A_1, \dots, A_t\}$ označme maximální systém po dvou disjunktních 
množin z $\F$. Pokud $t \geq k$, tvoří slunečnici s prázným jádrem a máme 
hotovo. Jinak vezměme $B := \bigcup_i A_i$; pro tu platí $|B| \leq s\cdot (k-1)$ 
a nějaký prvek, říkejme mu $x$, se vyskytuje v hodně množinách, konkrétně můžeme 
odhadnout četnost jednoho prvku holubníkovým principem (pro každý prvek $B$ 
zavedeme přihrádku a dáme do ní množiny, které tento prvek obsahují):
\begin{align*}
	{|\F| \over |B|} > (s-1)!(k-1)^{s-1}
\end{align*}
Takových množin je dokonce tolik, že samy o sobě splňují nutnou podmínku pro 
slunečnici velikosti $k$ v množinách o $1$ menších. Aplikujeme tedy indukci na 
systém $\F_x := \{S - x: S \in \F, x \in S\}$ a do výsledné slunečnice prvek 
$x$ vrátíme.

\vt (Slunečnicová spekulace) Ve slunečnicovém lemmatu stačí dokonce $|\F| \geq 
c_k^s$, kde $c_k$ je nějaká konstanta.

\pzn Ačkoliv Slunečnicové lemma zahřeje zevnitř, jeho aplikace se příliš 
nezmiňují. Zde je jedna:

Je otevřeným problémem, zda existuje algoritmus na násobení matic v čase 
$n^{2+\epsilon}$ pro každé $\epsilon$. Postup Coppersmitha a Winograda nabízí 
jednu z možností, jak takový algoritmus sestrojit; speciálně potřebují získat 
speciální Abelovskou grupu splňující hromádku požadavků; lze ukázat, že 
existence takové grupy je ve sporu se Slunečnicovou spekulací. To ale 
neznamená, že i kdyby Slunečnicová spekulace byla pravdivá, že neexistuje 
rychlý algoritmus na násobení matic, jsou další postupy, jak toho dosáhnout 
(speciálně Cohn-Umans).


\vt (Erdös-Ko-Rado) Nechť $n \geq 2k$, potom každý průsečný systém 
$k$-prvkových podmnožin $n$-prvkové množiny má nanejvýš $\binom{n-1}{k-1}$ 
množin.

\df (Discrepancy) Pro zjištění, jak je nějaké řešení dobré, můžeme definovat 
discrepanci $\disc(\F,f)$ pro množinový systém $\F$ a funkci $f: X \to \dots$ 
jako:
\begin{align}
	f(S) &= \sum_{x\in S} f(x) \\
	\disc(\F,f) &= \max_{S\in \F} |f(S)| \\
	\disc(\F) &= \min_f \disc(\F,f)
\end{align}
Diskrepence tedy vyjadřuje, jak dobré ohodnocení vydala funkce $f$ pro nějaký 
náš cíl, který chceme ohodnotit.

\app Mějme $X$ $n$-prvkovou množinu a systém $\F$ jejích podmnožin. Chtěli 
bychom obarvit prvky $X$ dvěmi barvami tak, aby v každé množině $\F$ byl počet 
barev co možná nejvyváženější. Zvolíme tedy $f: X \to \{-1, 1\}$, vyjadřující 
dvě barvy (pokud jich je v množině stejně, sečtou se na 0, což je dobré).  
Množinový systém $\F = 2^X$ by nedopadlo dobře, speciálně $\disc(\F) = n/2$ 
(nejlepší strategie je obarvit půl na půl, kdykoliv se obarví více, množina co 
obsahuje právě všechny stejně barevné prvky dosáhne horšího skóre). Co ale, 
kdybychom uvažovali menší skupinu množin?

\tv (Černovova nerovnost) Nechť $X_1, \dots, X_n$ jsou nezávislé náhodné 
proměnné nabývající hodnot $+1$ a $-1$ s pravděpodobností $1/2$ a $X = \sum_i 
X_i$. Potom pro každé $t \geq 0$ platí:
\begin{align}
	P[X \geq t] &< e^{-t^2/2\sigma^2} \\
	P[X \leq -t] &< e^{-t^2/2\sigma^2}\\
	\label{cernov:abs} P[|X| \leq t] &< 2\cdot e^{-t^2/2\sigma^2}
\end{align}
(v tomto případě $\sigma = \sqrt{n}$ díky jednoduchým proměnným).


\tv Nechť $|X|=n$ a $|\F| = m$ a největší množina má velikost $s$.  Potom 
$\disc(\F) \leq \sqrt{2s\ln(2m)}$.

\dk Nechť $f$ je zvolena uniformě nezávisle náhodně. Podívejme se tedy na nějaké
$S \in \F$, jaké má skóre. Použijeme Černovovu nerovnost (\ref{cernov:abs}):
\begin{align}
	P[|f(X)| \geq t] < 2 \cdot e^{-t^2/2s^2}
\end{align}
Stačí zvolit $t$ tak, aby vyšla pravá strana dostatečně malá. Máme-li $m$ 
podmnožin, stačí ostře menší, než $1/m$ (po použití Union Boundu vyjde $< 1$).  
Zvolme například $t := \sqrt{2s\ln(2m)}$, pro které to zrovna platí. Víme tedy, 
že převděpodobnost, že existuje prvek se skóre $|f(S)| \geq t$ je ostře menší 
než 1; existuje tedy $f$, pro kterou věta platí.

\vt (Bollobás) Nechť $A_1, \dots, A_m$ a $B_1, \dots, B_m$ jsou posloupnosti 
množin takové, že $A_i \cap B_j = \emptyset$ právě tehdy, když $i = j$. Potom 
pokud $a_i := |A_i|$ a $b_i := |B_i|$, platí:
\begin{align}
	\sum_{i=1}^m\binom{a_i + b_i}{a_i}^{-1} \leq 1
\end{align}

\dk Nechť $X = \bigcup_i A_i \cup B_i$. Postupujme indukcí podle velikosti $X$, 
kde pro $|X| = 1$ je tvrzení platné triviálně. Uvažujme následující systémy:
\begin{align}
	\F_x := \{ (A_i, B_i - x) : x \notin A_i\}
\end{align}
Protože v každém z těchto systému chybí jeden prvek, lze na ně aplikovat indukci 
a sečíst výsledky. První člen odpovídá množinám, kde $x\notin A_i \cup B_i$, 
druhý těm, kde $x \in B_i$:
\begin{align}
	\sum_{i=1^m}(n-a_i-b_i) \binom{a_i+b_i}{a_i} + b_i{a_i+b_i-1}{a_i}^{-1} \leq 
	n
\end{align}
Protože $\binom{k-1}{l} = {k-l\over k}\binom{k}{l}$, můžeme upravit druhý člen 
na:
\begin{align}
	b_i \cdot \left({(a_i+b_i)-a_i \over a_i+b_i} 
	\binom{a_i+b_i}{a_i}\right)^{-1} = (a_i + b_i) \cdot 
	\binom{a_i+b_i}{a_i}^{-1}
\end{align}
A celou rovnici vydělit $n$, což dokončuje důkaz. \qed

\df (Řetězec a antiřetězec) Systém částěčně uspořádaných množin $\F$ je řetězec,
pokud jsou každé dva prvky porovnatelné (operací \uv{podmnožina}), antiřetězec, 
pokud žádné dva prvky nejsou porovnatelné.

\vt (Sperner) Nechť $\F$ je systém podmnožin $X$, $|X| = n$. Potom pokud $\F$ je
antiřetězec, tak $|\F| \leq \binom{n}{\lfloor n/2 \rfloor}$.

\dk Nechť $A_1, \dots, A_m$ je antiřetězec. Definujme $B_i := X \setminus A_i$.  
To určitě splňuje předpoklady pro Bollobásovu větu ($A_i \cap B_i = \emptyset$, 
protože to jsou doplňky, jinde existuje jistě nějaký prvek, protože jde o 
antiřetězec):
\begin{align}
	|F| \cdot \binom{n}{\lfloor n/2 \rfloor}^{-1} \leq 
	\sum_{i=1}^m\binom{n}{|A_i|}^{-1} \leq \sum_{i=1}^m \binom{a_i + 
	b_i}{a_i}^{-1} \leq 1
\end{align}
\qed
