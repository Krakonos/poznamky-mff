\renewcommand{\arraystretch}{1.25}
\section{Grafové algoritmy}
\subsection{Nejkratší cesta}

Nejkratší cesty z jednoho vrcholu do všech ostatních (orientované grafy s
nezápornými váhami hran):

\begin{center}
\begin{tabular}{ l l p{8cm} }
	\hline
	\bf Algoritmus & \bf Čas & \bf Poznámka \\
	\hline
	Bellman-Ford & $\O(nm)$ & Funguje i v grafu se zápornými váhami. \\
	Dijkstra & $\O(n^2)$ & Funguje i v neorientovaných grafech. \\
	Dijkstra + FH & $\O(m + n\log n)$ & FH = Fibonacciho halda \\
	\hline
\end{tabular}
\end{center}

\noindent Nejkratší cesty mezi všemi dvojicemi vrcholů (orientované grafy bez záporných cyklů):

\begin{center}
\begin{tabular}{ l l p{8cm} }
	\hline
	\bf Algoritmus & \bf Čas & \bf Poznámka \\
	\hline
	Floyd-Warshall & $\O(n^3)$ & Funguje i v neorientovaných grafech. \\
	\hline
\end{tabular}
\end{center}

\subsection{Maximální tok}

Tj. algoritmy na hledání maximálního toku v síti ze zadaného vrcholu $s$
(zdroj), do zadaného vrcholu $t$ (stok). Velikost maximálního toku je navíc
rovna velikosti minimálního řezu.

\begin{center}
\begin{tabular}{ l l }
	\hline
	\bf Algoritmus & \bf Čas \\
	\hline
	Ford-Fulkerson & $\O(E\cdot \max_f)$\footnote{$\max_f$ je velikost maximálního toku} \\
    Edmonds-Karp & $\O(VE^2)$ \\
    Dinic & $\O(V^2E)$ \\
	Tři indové & $\O(V^3)$ \\
	Orlin + King, Rao, Tarjan & $\O(VE)$ \\
    \hline
\end{tabular}
\end{center}

\begin{itemize*}
\item \textbf{Ford-Fulkerson} hledá libovolnou zlepšující cestu.
\item \textbf{Edmonds-Karp} je Ford-Fulkerson, který hledá nejkratší
zlepšující cestu (pomocí BFS).
\item \textbf{Dinic} si staví vrstevnatou síť rezerv a v každém kroku jednu
hranu nasytí.
\item \textbf{Tři indové} počítají potenciál u vrcholů a v každém kroku
jeden vrchol nasytí.
\item \textbf{Orlin + King, Rao, Tarjan} je hodně složitý, dělal se celý
semestr na Semináři z grafových algoritmů.
\end{itemize*}

\subsection{Maximální párování}

\begin{center}
\begin{tabular}{ l l p{10cm} }
	\hline
	\bf Algoritmus & \bf Čas & \bf Poznámka \\
	\hline
	Edmonds & $\O(n^4)$ & Tzv. květinkový algoritmus. Později byla časová složitost zlepšena až na $\O(m\sqrt n)$ (Micali, Vazirani). \\
	Mucha, Sankowski & $\O(n^{2.376})$ & Randomizovaný, složitost je odvozena z násobení matic. \\
	\hline
\end{tabular}
\end{center}


\renewcommand{\arraystretch}{1.0}
