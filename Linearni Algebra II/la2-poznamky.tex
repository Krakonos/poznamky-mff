\documentclass[a4paper,10pt]{article}
\usepackage[utf8]{inputenc}
\usepackage{a4wide}
\usepackage[czech]{babel}
%\usepackage{bbm}
\usepackage{amsfonts, amsmath, amsthm, amssymb}
\usepackage{math}
\usepackage{enumerate}

\title{Lineární algebra II}
\author{Ladislav Láska}

\begin{document}

\maketitle
\newpage
\tableofcontents
\newpage


\section{Determinant}

\paragraph{Definice}
Nechť $A$ je čtvercová matice řádu $n$ nad $\K$, potom \textbf{determinant} matice
$A$ ($\det(A)$) je zobrazení z $\K^{n \times n} \to \K$ dáno vztahem:
\begin{align*}
	\det(A) = \sum_{p \in S_n} \sgn (p) \prod_{i=1}^n A_{i,p(i)}
\end{align*}

\subsection{Vlastnosti determinantu}
\setcounter{equation}{0}
\paragraph{Pozorování}
$\det(A^T) = det(A)$

\paragraph{Důkaz (neformální)}
Všimnene si, že determinant je součet podle permutací, které pokryjí právě
všechny kombinace řádků a sloupců. Je tedy zřejmé, že přehozením řádků a sloupců
(tj. transpozicí) se determinant nezmění. 

\paragraph{Důkaz}
Formálně podle definice determinantu (\ref{det-vl-1}), transpozice
(\ref{det-vl-2}), a použitím inverzní permutace (\ref{det-vl-3}) ukážeme:
\begin{align} 
	\sum_{p \in S_n} &\sgn (p) \prod_{i=1}^n (A^T)_{i,p(i)} \label{det-vl-1} \\
	= & \sum_{p \in S_n} \sgn (p) \prod_{i=1}^n A_{p(i),i} \label{det-vl-2}\\
	= & \sum_{p \in S_n} \sgn (p^{-1}) \prod_{i=1}^n A_{i,p^{-1}(i)} \label{det-vl-3}
\end{align}
Což je ale $\det(A)$, protože součet je přes všechny permutace.

\subsection{Úpravy matic a jejich vliv na determinant}
\setcounter{equation}{0}

\subsubsection{Přerovnání řádků/sloupců}
\setcounter{equation}{0}
\paragraph{Pozorování}
Přerovnání sloupců (pro řádky analogicky) můžeme zapsat jako permutaci $q$. Při
výpočtu determinantu to znamená, že budeme počítat se složením permutací -
vzhledem k tomu, že determinant je součet přes všechny permutace, budou nás
zajímat pouze znaménka.

\paragraph{Tvrzení}
Nechť $A$ je čtvercová matice a $B$ je matice odvozená od $A$ přerovnáním sloupců (s
řádky analogicky) podle permutace $q$. Potom platí:
\begin{align*} 
	\det(B) = \sgn (q) \det (A)
\end{align*}

\paragraph{Důkaz}
Podle definice determinantu (\ref{det-prerovnani-1}), přerovnáním permutací zpět
na A (\ref{det-prerovnani-2}), přidáním identity (\ref{det-prerovnani-2}) a
úpravou (\ref{det-prerovnani-3}, \ref{det-prerovnani-4}) získáme:
\begin{align} 
	\det(B) &= \sum_{p \in S_n} \sgn (p) \prod_{i=1}^n B_{i,p(i)}
							\label{det-prerovnani-1}\\
	&= \sum_{p \in S_n} \sgn (p) \prod_{i=1}^n A_{i,q^{-1}(p(i))}
							\label{det-prerovnani-2} \\
	&= \sum_{p \in S_n} \underbrace{\sgn (p) \sgn(q^{-1})}_{sgn(q^{-1} \circ p)} 
		\sgn(q) \prod_{i=1}^n A_{i,q^{-1}(p(i))}
							\label{det-prerovnani-3} \\
	&= \sgn (q) \sum_{p \in S_n} \sgn (q^{-1} \circ p) \prod_{i=1}^n A_{i,
	(q^{-1} \circ p)(i)} 
							\label{det-prerovnani-4} 
\end{align} 
Což je podle definice determinantu $\sgn (q) \det (A)$.

\paragraph{Důsledek}
Záměna dvou řádků mění znaménko.

\subsubsection{Linearita násobku řádku matice vůči determinantu}
\setcounter{equation}{0}
\paragraph{Věta}
Nechť $A'$ je odvozená matice od $A$ vynásobením $k$-tého řádku
konstantou $t$, potom platí:
\begin{align*} 
	\det (A') = t \cdot \det (A)
\end{align*} 

\paragraph{Důkaz}
Všimneme si, že každý součin si do každého řádku "šáhne" právě jednou. Můžeme
tedy z každého takovéhou součinu vytknout konstantu před sumu.

\subsubsection{Linearita součtu matice vůči determinantu}
\setcounter{equation}{0}
\paragraph{Věta}
Mějme matici $A$ takovou, že platí (všechny řádky jsou stejné, jenom $i$-tý v
$A$ je součtem $i$-tého v $B$ a $C$):
\begin{align*}
	\forall k \neq i \quad \forall j: \quad &B_{k,j} = C_{k,j} = A_{k,j}\\
					  			 & A_{i,j} = B_{i,j} + C_{i,j}
\end{align*}
Potom platí:
\begin{align*}
	\det(A) = \det(B) + \det(C)
\end{align*}


\paragraph{Důkaz}
Rozepíšeme podle definice determinant $A$ a $i$-tý řádek vyjádříme jako součet 
$i$-tého řádku z $B$ a $C$:
\begin{align} 
	\det(A) = \sum_{p \in S_n} sgn(p) \cdot A_{1,p(1)} \cdot ... \cdot
	(B_{i,p(i)} + C_{i, p(i)}) \cdot ... \cdot a_{n, p(n)}
\end{align}
Závorkou ale sumu můžeme roznásobit:
\begin{align} 
	&\sum_{p \in S_n} sgn(p) \cdot A_{1,p(1)} \cdot ... \cdot
	B_{i,p(i)} \cdot ... \cdot a_{n, p(n)} \\
	+&\sum_{p \in S_n} sgn(p) \cdot A_{1,p(1)} \cdot ... \cdot
	C_{i, p(i)} \cdot ... \cdot a_{n, p(n)}
\end{align}
Což podle definice determinantu je:
\begin{align}
	\det(B) + \det(C) = \det(A)
\end{align}	

\paragraph{Důsledek}
Přičtení $t$-násobku $i$-tého řádku k $j$-tému řádku determinant nemění: lze rozložit
na dvě matice kde v jedné bude původní řádek, v druhé přičítaný $t$-násobek řádku.
Druhá matice bude mít však nulový řádek, tj. nulový determinant.

\subsection{Výpočet determinantu}
\setcounter{equation}{0}

\subsubsection{Determinant součinu}
\setcounter{equation}{0}
\paragraph{Věta}
Nechť $A$ a $B$ jsou čtvercové matice stejného řádu nad tělesem $\K$, potom
platí:
\begin{align*}
	\det (A \cdot B) = \det (A) \cdot \det (B)
\end{align*}


\paragraph{Pozorování}
Všimneme si, že pokud je $A$ nebo $B$ singulární, součin je také singulární, tj.
nulový determinant.
\paragraph{Pozorování}
Je-li $A$ regulární, existuje $R$ regulární taková, že $R \cdot A = I_n$.

\paragraph{Důkaz}
$A$ si rozložíme na součin elementárních matic:
\begin{align}
	A = E_1 \cdot E_2 \cdot ...
\end{align}
Potom dosazením platí (\ref{det-soucin-1}) a z asociativity násobení matic
(\ref{det-soucin-2}):
\begin{align}
	\det (A \cdot B) &= \det ( (E_1 \cdot E_2 \cdot ...) \cdot (B)) = 
											\label{det-soucin-1}\\
					 &= \det ( E_1 \cdot (E_2 \cdot ... \cdot (B)))
					 						\label{det-soucin-2}
\end{align}
Protože však $E_i$ je elementární matice, můžeme rozlišit následující případy:
\begin{enumerate}
	\item $E_i$ odpovídá vynásobení $i$-tého řádku konstantou $t \neq 0$ 
	$\Rightarrow$ determinant $t$-krát. Taková matice je jednotková matice s $t$
	na místě $i,i$. Její determinant je $t$, násobením touto maticí se tedy
	determinant $t$-krát zvýší.
	\item $E_i$ odpovídá přičtení $j$-tého řádku k $i$-tému. Taková matice je
	jednotková s $1$ (nebo $t$, pokud přičítáme násobek) nad nebo pod hlavní
	diagonálou. Její determinant je $1$ (každá parmutace obsahující přidaný
	prvek musí obsahovat i jednu nulu, tj. na $t$ nezáleží). Násobením touto
	maticí se tedy determinant nezmění.
\end{enumerate}
Pokud víme, jak dané matice mění determinant, můžeme je postupně vytknout:
\begin{align}
		\det (E_1) \cdot \det ( E_2 \cdot ... \cdot (B)) 
\end{align}
Takto vytkneme všechny elementární matice. Nyní je však ale můžeme opět spojit
pod jeden determinant:
\begin{align}
		\det (E_1) \cdot ... \cdot \det (E_n) \cdot (B) = 
		\det (A) \cdot \det(B)
\end{align}
Což jsme chtěli dokázat.

\paragraph{Důsledek}
Matice je regulární právě když má nenulový determinant.
\paragraph{Důsledek}
Determinant matice a inverzní matice jsou si rovny.

\subsubsection{Rozvoj determinantu podle i-tého řádku/sloupce}
\setcounter{equation}{0}
\paragraph{Značení} Maticí $A^{i,j}$ značíme matici vzniklou z $A$ vynecháním $i$-tého
řádku a $j$-tého sloupce.

\paragraph{Tvrzení}
Pro libovolnou matici řádu $n \ge 2$ a libovolné $i$ od $1 .. n$ platí:
\begin{align*}
	\det (A) = \sum_{j=1}^n a_{i,j} (-1)^{i+j} \cdot \det (A^{i,j})
\end{align*}

\paragraph{Důkaz}
Užijeme linearitu determinantu: Zapíšeme i-tý řádek jako vhodnou lineární
kombinaci:
\begin{align}
	a_{i,1}  ... a_{i,n} = a_1 \cdot (1, 0, ... , 0 ) + a_{i,2} (0, 1, 0, ... , 0 ) +
	... + a_{i,n} (0, ... , 0, 1)
\end{align}
Rozložíme matici podle i-tého řádku na jednotlivé elementární matice. Po
vytknutí konstanty získáme jeden řádek s nulami a jedničkou na jednom místě:
\begin{align}
	\left( \begin{matrix}
		 \\
		\hline 0 & \cdots & 1 & \cdots & 0 \\
		\hline  \\
	\end{matrix} \right)
\end{align}
Takovouto jedničku přesuneme do prvního sloupce cyklem délky $i$, ten se dá
složit z $i-1$ transpozic. Analogicky celý řádek prohodím na první:

\begin{align}
	(-1)^{i+j}
	\left( \begin{array}{c|ccc}
		1 & 0 & \cdots & 0 \\
		\hline \vdots  \\
		\vdots   \\
	\end{array} \right)
\end{align}
Nyní je však zřejmé, že každá permutace, která použije $j$-tý sloupec $j\neq1$ bude
nulová. Pokud $j=1$, jednička hodnotu nezmění. Můžeme tedy první řádek a sloupec
vynechat.

\subsection{Adjungovaná matice}
\setcounter{equation}{0}
\paragraph{Definice}
Pro čtvercovou matici $A$ definujeme \textbf{adjungovanou} matici $\adj(A)$
předpisem:
\begin{align*}
	(\adj(A))_{i,j} = (-1)^{i+j} \cdot \det (A^{j,i})
\end{align*}

\subsection{Vztah adjungované matice a determinantu}
\setcounter{equation}{0}
\paragraph{Věta}
Pro každou regulerní matici platí:
\begin{align*}
	A^{-1} = \frac{1}{\det(A)} \cdot \adj (A)
\end{align*}
\paragraph{Důkaz}
Všimneme si, že pokud násobíme matici $A$ její adjungovanou matici, vznikne nám:
\begin{align}
	A \cdot \adj(A) = \left( \begin{matrix}
		\det(A) &  & 0 \\
		 & \ddots &  \\
		0 &  & \det(A) 
	\end{matrix}\right) = \det(A) \cdot I_n
\end{align}
Což je zřejmé, pokud si uvědomíme, že každý z determinantů ve výsledné matici
vznikl podle rozvoje $i$-tého řádku (koeficienty v $A$, znaménka a subdeterminanty
jsou v adjungované matici). \\
Potom je již odvození triviální:
\begin{align}
	A \cdot \adj(A) \cdot \frac{1}{\det(A)} = I_n \\
	A^{-1} = \adj(A) \cdot \frac{1}{\det(A)}
\end{align}

\paragraph{Důsledek}
Pokud jsou $A$ a $A^{-1}$ celočíselné, $\det(A) = \pm 1$

\subsection{Cramerovo pravidlo}
\setcounter{equation}{0}
\paragraph{Věta}
Nechť matice $A$ je regulární matice soustavy. Potom řešení každé soustavy $Ax =
b$ lze spočítat po složkách 
\begin{align*}
	x_i = \frac{\det(A_{i\to b})}{\det(A)}
\end{align*}
kde $A_{i\to b}$ je matice odvozená z $A$ nahrazením $i$-tého sloupce vektorem
pravých stran $b$.
\paragraph{Důkaz}
Podíváme se na součin $A^{-1} \cdot A \cdot x$ a upravíme:
\begin{align}
	I x = A^{-1} \cdot A \cdot x = A^{-1} b = \frac{1}{\det(A)} \cdot \adj(A) \cdot b \\
\end{align}
Nyní můžeme vyjádřit $x_i$:
\begin{align}
	x_i = \frac{1}{\det(A)} \cdot (\adj(A) \cdot b)_i 
\end{align}
A rozložíme závorku jako maticový součin (matice, vektor):
\begin{align}
	\frac{1}{\det(A)} \cdot \sum_{j=1}^n \adj(A)_{i,j} b_j
\end{align}
Kde ale suma vyjadřuje rozvoj podle $i$-tého sloupce $A_{i\to b}$:
\begin{align}
	\frac{1}{\det(A)} \cdot \det(A_{i \to b})
\end{align}

\subsection{Objem rovnoběžnostěnu}
\setcounter{equation}{0}
\paragraph{Věta}
Nechť $x_1, ..., x_n$ jsou vektory tvořící rovnoběžnostěn. Potom objem
$V=|\det(A)|$.
\paragraph{Důkaz}
Malůvkou.

\newpage
\section{Polynomy}
\setcounter{equation}{0}
\subsection{Polynom}
\setcounter{equation}{0}
\paragraph{Definice}
Polynomem $P$ stupně $n$ proměnné $x$ nad tělesem $\K$ nazveme výraz:
\begin{align*}
	\sum_{i=0}^n a_i x^i \text{ , kde } a_n \in \K \quad \land \quad a_n \neq 0
\end{align*}
\paragraph{Definice}
Na polynomu definujeme oprace:
\begin{description}
	\item[součet po složkách]
		\begin{align*}
			P(x)^n + Q(x)^m = \sum_{i=0}^{\max\{m,n\}} (p_i + q_i) x^i
		\end{align*}
	\item[násobení]
		\begin{align*}
			(p \cdot q)(x) = \sum_{i=0}^{m+n} c_i x^i \quad \text{ kde }  \quad
			c_i = \sum_{j=\max\{0, i-m\}}^{\min\{i, n \}} a_j b_{i+j}
		\end{align*}
	\item[dělení se zbytkem]
		\begin{align*}
			\forall p, q \quad \exists r,t \in K(x): \quad \deg t \lt \deg q \quad 
			\land \quad p = r \cdot q + t
		\end{align*}
\end{description}

\subsection{Algebraicky uzavřené těleso, kořen polynomu}
\setcounter{equation}{0}
\paragraph{Definice}
Kořen polnomu je prvek $k$ pro který platí, že $P(k) = 0$.
\paragraph{Definice}
Těleso, kde všechny polynomy mají alespoň $1$ kořen, nazveme \textbf{algebraicky
uzavřené}.

\subsection{Základní věta algebry (bez důkazu)}
\setcounter{equation}{0}
\paragraph{Věta}
Každý polynom stupně alespoň $1$ má nad $\C$ kořen.
(bez důkazu)
\paragraph{Důsledek}
Každý polynom stupně alespoň $1$ lze rozložit na součin kořenových součinitelů.

\subsection{Reprezentace polynomu}
\setcounter{equation}{0}
Polynom stupně $n$ můžeme reprezentovat jako:
\begin{description}
	\item[koeficienty] $a_0, ..., a_n \in \K$
	\item[kořeny a $a_0$] na algebraicky uzavřeném tělese
	\item[hodnotami $P(x)$] v n+1 různých bodech
\end{description}

\subsection{Vandermondova matice}
\setcounter{equation}{0}
\paragraph{Definice}
Nechť $x_1, ..., x_n$ jsou body z tělesa $\K$. Potom \textbf{Vandrmondovou
maticí} nazveme matici:
\begin{align*}
	\left(
	\begin{matrix}
		x_1^0 & x_1 & x^2_1 & \cdots & x^{n-1}_1 \\
		x_2^0 & x_2 & x^2_2 & \cdots & x^{n-1}_2 \\
		\vdots & \vdots & \vdots & \ddots & \vdots \\
		x_n^0 & x_n & x^2_n & \cdots & x^{n-1}_n
	\end{matrix}
	\right)
\end{align*}

\subsection{Věta o regulárnosti Vandermondovy matice}
\setcounter{equation}{0}
\paragraph{Věta}
Vandermondova matice je regulární právě tehdy, pokud hodnoty $x_1, ..., x_{n+1}$
jsou různé.
\paragraph{Důkaz}
Spočítáme determinant matice: odečteme první řádek od všech ostatních.
\begin{align}
	A = 
	\left|
	\begin{matrix}
		x_1^0 & x_1 & x^2_1 & \cdots & x^{n-1}_1 \\
		x_2^0 & x_2 & x^2_2 & \cdots & x^{n-1}_2 \\
		\vdots & \vdots & \vdots & \ddots & \vdots \\
		x_n^0 & x_n & x^2_n & \cdots & x^{n-1}_n
	\end{matrix}
	\right| =
	\left|
	\begin{matrix}
		1 & x_1       & x^2_1        & \cdots & x^{n-1}_1 \\
		0 & x_2 - x_1 & x^2_2 -x^2_1 & \cdots & x^{n-1}_2 -x^{n-1}_1\\
		\vdots & \vdots & \vdots & \ddots & \vdots \\
		0 & x_n - x_1 & x^2_n -x^2_1 & \cdots & x^{n-1}_n - x^{n-1}_1
	\end{matrix}
	\right|
\end{align}
Nyní je již vidět, že můžeme rekurentně spočítat jako rozvoj podle prvního
sloupce. Tedy $\det(A)$ je roven součinu rozdílů každé dvojice $x_i$.


\subsection{Lagrangeova interpolace (bez důkazu)}
\setcounter{equation}{0}
\paragraph{Věta}
Nechť $[x_i, y_i]$, $i = 1 ... n+1$ jsou body ve kterých známe funkční hodnotu.
Nadefinujeme pomocnou funkci $P_i$:
\begin{align*}
	P_i(x) = \frac{(x-x_1)(x-x_2) ... (x-x_{i-1})(x-x_{i+1}) ...
	(x-x_{n+1})}{(x_i - x_1)(x_i - x_2) ... (x_i - x_{i-1})(x_i - x_{i+1}) ...
	(x_i - x_{n+1})}
\end{align*}
Je snadné vidět, že $P_i(x_i) = 1$, zatímco $P_i(x_j) = 0 \quad \forall j \neq
i$. Lagrangeův polynom tedy vypočítáme pomocí vztahu:
\begin{align*}
	P_n(x) = \sum_{i=1}^{n+1} y_i P_i(x)
\end{align*}
(bez důkazu)


\section{Vlastní čísla a vektory}
\setcounter{equation}{0}
\subsection{Vlastní číslo, vlastní vektor}
\setcounter{equation}{0}
\paragraph{Definice}
Nechť $V$ je vektorový prostor nad $\K$ a $f: V \to V$ je lineární zobrazení.
Potom \textbf{vlastním číslem} zobrazení $f$ rozumíme takové $\lambda \in \K$,
že $f(u) = \lambda \cdot u$, vlastní vektor vlastnímu číslu $\lambda$ je každý,
pro který platí $u \in V \quad \Rightarrow \quad  f(u) = \lambda \cdot u$.
\paragraph{Definice}
Je-li vektorový prostor V konečně generovaný a $\dim(V) = n$, potom můžeme $f$
reprezentovat maticí zobrazení $A=[f]_X$ a rozšířit definici na matice, tj.
vlastní číslo matice A je libovolné $\lambda \in \K$, takové že $Ax = \lambda x$
pro $x \in \K^n$, $x \neq 0$. Vlastní vektor příslušný $\lambda$ je takový $x$, že
$Ax = \lambda x$

\subsection{Spektrum matice}
\setcounter{equation}{0}
\paragraph{Definice}
Množina všech vlastních čísel matice se nazývá \textbf{Spektrum matice}.


\subsection{Charakteristický polynom}
\setcounter{equation}{0}
\paragraph{Definice}
Charakteristickým polynomem matice $A$ nazveme polynom $P_A(t)$ určený výrazem
$P_A(t) = \det(A - t \cdot I)$.

\subsection{Věta o vlastním čísle a charakteristickém polynomu}
\setcounter{equation}{0}
\paragraph{Věta}
Pro matici $A^{n\times n}$ platí, že $\lambda$ je vlastní číslo matice $A$ právě
když je $\lambda$ kořen $P_A(t)$.
\paragraph{Důkaz}
Platí triviálně z definice:
\begin{align}
	Ax &= \lambda x  \\
	(A - \lambda I) x &= 0 \qquad (x \neq 0 \Rightarrow)\\
	A - \lambda I & \text{ singulární }
\end{align}

\subsection{Věta o zachování podobnosti součinů matic}
\setcounter{equation}{0}
\paragraph{Věta}
Nechť $A$, $B$ jsou čtvercové matice, potom $A\cdot B$ a $B \cdot A$ mají stejná
vlastní čísla.
\paragraph{Poznámka}
Pro násobení blokových matic platí:
\begin{align*}
	\left( \begin{matrix}
		I & J \\ K & L
	\end{matrix}\right)
	\cdot
	\left( \begin{matrix}
		P & Q \\ R & S
	\end{matrix}\right)
	=
	\left( \begin{matrix}
		IP + JR & IQ +JS \\ KP + LR & KQ + LS
	\end{matrix}\right)
\end{align*}
\paragraph{Důkaz}
Mějme matice $C$, $R$, $D$ nadefinované a rozepsané podle součinu blokových matic:
\begin{align}
	\overbrace{
	\left( \begin{matrix}
		AB & 0 \\ B & 0
	\end{matrix}\right)}^C
	\cdot
	\overbrace{
	\left( \begin{matrix}
		I & A \\ 0 & I
	\end{matrix}\right)}^R
	=
	\left( \begin{matrix}
		AB & ABA \\ B & BA
	\end{matrix}\right)
	=
	\overbrace{
	\left( \begin{matrix}
		I & A \\ 0 & I
	\end{matrix}\right)}^R
	\cdot
	\overbrace{
	\left( \begin{matrix}
		0 & 0 \\ B & BA
	\end{matrix}\right)}^D
\end{align}
Po rozepsání determinantů matic $C$ a $D$ vidíme, že jsou to po řadě z definice
$P_{AB}(t)$, $P_{BA}(t)$ - dokážeme tedy rovnost. Vyjádříme $C = R \cdot D \cdot R^{-1}$ a upravíme
$P_C$:
\begin{align}
	P_C &= \det(C - tI) \\
	&= \det( R \cdot D \cdot R^{-1} - tI) \\
	&= \det( R \cdot D \cdot R^{-1} - R \cdot tI \cdot R^{-1} ) \\
	&= \det( R\cdot (D - tI)\cdot R^{-1}) \\
	&= \det(R) \det( D-tI) \det R^{-1} \\
	&= P_D(t)
\end{align}

\subsection{Cayley-Hamilton}
\setcounter{equation}{0}
\paragraph{Věta}
Nechť $P_A(t)$ je charakteristický polynom matice $A$. Potom platí, že:
\begin{align*}
	P_A(A) = 0
\end{align*}
\paragraph{Důkaz}
Nechť $M :=  A - tI$. Spočítáme $\adj(M)$ a uvědomíme si, že adjungovaná matice
má v každé buňce polynom $t$ stupně nejvýše $n-1$. Rozložíme tedy a zavedeme
$B_i$ koeficienty $t^i$:
\begin{align}
	adj(A - tI) = t^{n-1} B_{n-1} + ... + t^0 B_0
\end{align}
Nyní podle pravidel o adjungované matici ($I \cdot \det(A) = A \cdot \adj(A)$) můžeme zapsat $P_A(t)$ jako:
\begin{align}
	P_A(t) \cdot I = (A - tI)(t^{n-1} B_{n-1} + ... + t^0 B_0) = a_n \cdot t^n
	\cdot I + ... + a_0 I
\end{align}
Nyní porovnáme koeficienty u $t^i$:
\begin{align}
	&t^n &-I \cdot B_{n-1} &= a_n I \\
	&t^i &A \cdot B_i - I \cdot B_{i-1} &= a_i I \\
	&t^0 &A \cdot B_0 &= a_0 I
\end{align}
Pokud vynásobíme $i$-tou rovnici $A^i$ zleva a soustavu sečteme, získáme:
\begin{align}
	&P = P_A(A) \\
	&L = \underbrace{-A^n B_{n-1} + A^{n-1}(AB_{n-1}}_0 - \underbrace{B_{n-2}) +
	... }_0 = 0
\end{align}
Tedy $P_A(A) = 0$.

\subsection{Věta o nezávislosti vlastních vektorů}
\setcounter{equation}{0}
\paragraph{Věta}
Nechť $x_1, ..., x_n$ jsou vlastní vektory přístlušící různým vlastním číslům
$\lambda_1, ..., \lambda_n$ zobrazení $f$. Potom $x_1, ..., x_n$ jsou lineárně
nezávislé.
\paragraph{Důkaz}
Sporem a indukcí: 
\begin{align}
	\exists a_1, ..., a_k \neq 0  \quad \sum_{i=1}^k a_i x_i = 0
\end{align}
Vyjádříme 0:
\begin{align}
	&0 = f(0) = f\left(\sum_{i=1}^k a_i x_i\right) = \sum_{i=1}^k a_i f(x_i) =
	\sum_{i=1}^k a_i \lambda_i x_i \\
	&0 = \lambda 0 = \lambda \sum_{i=1}^k a_i x_i = \sum_{i=1}^k a_i \lambda
	x_i\\
	&0 = 0 - 0 = \sum_{i=1}^k a_i \lambda_i x_i - \sum_{i=1}^k a_i \lambda_k x_i 
	= \sum_{i=1}^{k-1} \underbrace{a_i x_i}_{\neq 0} \underbrace{( \lambda_i -
	\lambda_k)}_{\neq 0}
\end{align}
Tedy existují dva lineárně závislé vektory.

\subsection{Podobné matice}
\setcounter{equation}{0}
\paragraph{Definice}
Řekneme, že čtvercové matice $A$ a $B$ jsou podobné pokud existuje regulární matice
$R$ taková, že $A = R \cdot B \cdot R^{-1}$.

\subsection{Věta o vlastních číslech podobných matic}
\setcounter{equation}{0}
\paragraph{Věta}
Nechť $A$ a $B$ jsou podobné matice a $\lambda, x$ jsou vlastní číslo a jeho vlastní
vektor matice $A$. Potom $y = R \cdot x$ je vlastní vektor $B$ příslušící $\lambda$.
\paragraph{Důkaz}
Vyjádříme $B$:
\begin{align}
	A = R \cdot B \cdot R \quad \Rightarrow \quad B = R \cdot A \cdot R^{-1}
\end{align}
Rozepíšeme $B \cdot y$:
\begin{align}
	B \cdot y = (R \cdot A \cdot R^-1)(R \cdot x) = R\cdot A \cdot x = R \cdot
	\lambda \cdot x = \lambda \cdot y
\end{align}
Tedy $y$ je vlastní vektor $B$ příslušící vlastnímu číslu $\lambda$.

\subsection{Diagonalizovatelná matice}
\setcounter{equation}{0}
\paragraph{Definice}
Matice je \textbf{diagonalizovatelná}, pokud je podobná nějaké diagonální matici.

\subsection{Věta o diagonalizovatelnosti matic a vlastních vektorech}
\setcounter{equation}{0}
\paragraph{Věta}
Nechť je matice $A$ řádu $n$. Potom je diagonalizovatelná právě když má $n$ vlastních
vektorů.
\paragraph{Důkaz}
Z definice diagonalizovatelnosti jasně plyne, že sloupce matice jednoznačně
odpovídají vlastním vektorům.
\paragraph{Důsledek}
Pokud má matice řádu $n$ vlastních čísel, je diagonalizovatelná.
\paragraph{Důsledek}
Matice $A \in \C^{n \times n}$ má vlastní čísla $\lambda_1, ..., \lambda_n$ násobnosti
$r_1, ..., r_k$ a navíc:
\begin{align*}
	\forall i \in \{1, ..., n\} \quad \rank(A - \lambda_i I) = n - r_i
\end{align*}
pak právě tehdy je A diagonalizovatelná.

\subsection{Matice v Jordanově normálním tvaru}
\setcounter{equation}{0}
\paragraph{Definice}
Matice v Jordanově normálním tvaru je diagonálně bloková matice, kde každý blok má na hlavní
diagonále stejné číslo a nad hlavní diagonálou $1$ nebo $0$.

\subsection{Věta o podobnosti Jordanově matici (bez důkazu)}
\setcounter{equation}{0}
\paragraph{Věta}
Každá komplexní čtvercová matice je podobná matici v Jordanově normálním tvaru.
(bez důkazu)

\subsection{Věta o diagonalizaci symetrických matic}
\setcounter{equation}{0}
\paragraph{Věta}
Každá reálné symetrická matice je diagonalizovatelná.

\subsection{Hermitovská matice, unitární matice}
\setcounter{equation}{0}
\paragraph{Definice}
Komplexní čtvercová matice A je \textbf{hermitovská} pokud platí:
\begin{align*}
	a_{i,j} = \overline{a_{j,i}}
\end{align*}
Hermitovská \textbf{transpozice} matice $A$ je $A^H$, kde $(A^H)_{i,j} =
\overline{a_{j,i}}$.
\paragraph{Definice}
Komplexní čtvercová matice se nazývá unitární pokud $A^H A = I$
\paragraph{Pozorování}
Součin unitárních matic je unitární matice:
\begin{align*}
	A^H A = I, B^H B = I \quad \Rightarrow \quad (AB)^H \cdot AB = B^H A^H A B =
	I
\end{align*}


\subsection{Reálnost vlastních čísel hermitovské matice}
\setcounter{equation}{0}
\paragraph{Věta}
Každá hermitovská matice má všechna vlastní čísla reálná a je
diagonalizovatelná.
\paragraph{Důkaz}
\begin{enumerate}
	\item Nechť $v$ je vlastní vektor matice $A$ příslušící vlastnímu číslu
	$\lambda$. Mějme výraz: $v^H A v$. Ten upravíme dvojím způsobem:
	\begin{align}
		\nonumber v^H A &v =\\
		&= v^H (Av) = v^H (\lambda v) = \lambda (v^H v) \\
		&= (v^H A)v = (v^HA^H) v = (vA)^H v = (\lambda v)^Hv
	\end{align}
	výrazy (1) a (2) se musí rovnat. Získáme tedy vztah:
	\begin{align}
		\lambda(v^H v) = (\lambda v)^Hv
	\end{align}
	Tedy $\lambda = \overline{\lambda} \Rightarrow \lambda \in \R$.
	\item $A$ je hermitovská matice. Podle definice tedy musí platit $AR = RD$.
	Ukážeme na obrázku:
	\begin{align}
	\left(
		\begin{matrix}
		A&& \\
		&\ddots& \\
		&&\ddots
		\end{matrix}
	\right)
	\left(
		\begin{matrix}
			\vdots & \vdots & \vdots \\
			x_1 & x_i & x_n \\
			\vdots & \vdots & \vdots
		\end{matrix}
	\right) 
	= 
	\left(
		\begin{matrix}
			\vdots & \vdots & \vdots \\
			x_1 & x_i & x_n \\
			\vdots & \vdots & \vdots
		\end{matrix}
	\right) 
	\left(
		\begin{matrix}
			\ddots & \cdots & 0 \\
			\vdots & D & \vdots \\
			0 & \cdots & \ddots 
		\end{matrix}
	\right)
	\end{align}
	\item $R$ je unitární: z předpokladů víme, že $A$ je hermitovská. Proto
	platí $\forall i,j \quad a_{i,j} = \overline{ a_{i,j} }$, tedy $a_{i,j} \in
	\R$. Pro unitárnost chcem, aby $R^H R = I$. 
\end{enumerate}


\subsection{Binet-Cauchyho věta (bez důkazu)}
\setcounter{equation}{0}
\paragraph{Věta}
Nechť $A$ a $B$ jsou obdélníkové matice řádu $m \times n$. Potom:
\begin{align*}
	|A^T \cdot B| = \sum_{I \in \binom{n}{m}} |A^T_I \cdot B_I|
\end{align*}


\subsection{Laplaceova matice}
\setcounter{equation}{0}
\paragraph{Definice}
Nechť $G = (V,E)$ je graf na $n$ vrcholech $v_1, ..., v_n$. Potom definujeme Laplaceovu matici
$Q$ předpisem:
\begin{align*}
	&q_{i,i} = \deg(v_i) \\
	&q_{i,j} =
		\begin{cases}
			-1 \quad \Leftrightarrow (v_i,v_j) \in E \\
			0
		\end{cases}
\end{align*}

\subsection{Počet koster grafu}
\setcounter{equation}{0}
\paragraph{Věta}
Nechť $G$ je graf. Potom platí:
\begin{align*}
	\kappa(G) = \det(Q^{1,1})
\end{align*}
\paragraph{Důkaz}
Zavedeme libovolnou orientaci grafu G a zaznamenáme do matice incidence $D$:
\begin{align}
	d_{i,j} = \begin{cases}
		1 \Leftrightarrow e_j = (v_i, v) \\
		-1 \Leftrightarrow e_j = (v, v_i) \\
		0
	\end{cases}
\end{align}
Všimneme si, že $D \cdot D^T = Q$. Potom rozepíšeme a podle B-C věty:
\begin{align}
	\det(Q^{1,1}) = \det(D^1 \cdot (D^1)^T) = \sum_{I=\binom{n}{n-1}} |D_I^1
	{D_I^1}^T| = \sum_{I=\binom{n}{n-1}} |D_I^1|^2
\end{align}
Pokud tedy dokážeme, že $D_I^1 = \pm 1$ právě když $v_i:\ i \in I$ indukují
strom, jinak 0, věta je dokázána.
\paragraph{Lemma 1}
Pokud hrany $\{e_i | i \in I\}$ indukují strom, potom $|D_I^1| = \pm 1$.
\paragraph{Důkaz}
Uspořádáme vrcholy $w_1, ..., w_k$ tak, aby $w_i$ byl list na vrcholech $w_{i+1},
...$. Potom můžeme uspořádat sloupce v matici $D_I^1$ podle $i$. Vznikne nám:
\begin{align}
	\pm
	\left|
	\begin{matrix}
		\pm 1 & 0 & \cdots & 0 \\
		0 & \pm 1 & & \vdots \\
		\vdots & & \ddots & 0 \\
		0 & \cdots & & \pm 1 
	\end{matrix}
	\right| = \pm 1
\end{align}
\paragraph{Lemma 2} Pokud hrany $\{e_i | i \in I\}$ neindukují strom, potom
$|D_I^1| = 0$.
\paragraph{Důkaz}
Pokud hrany neindukují strom, existuje cyklus $C$. Mějme tedy vrcholy $w_1, ...,
w_c \in C$ takové, že $(w_i, w_{i+1 \mod c}) \in E_G$. Pokud přičteme řádky
matice příslušící $w_1, ..., w_{c-1}$ k řádku $w_c$, 
získáme nulový řádek - matice je tedy singulární.


\newpage

\section{Vektorové prostory se skalárním součinem}
\setcounter{equation}{0}
\subsection{Skalární součin}
\setcounter{equation}{0}
\paragraph{Definice}
Nechť $V$ je vektorový prostor nad $\C$. Zobrazení, které dvěma vektorům $u, v
\in V$ přiřadí číslo $\left<u|v\right> \in \C$ se nazývá \textbf{skalární součin} pokud splňuje
axiomy: \\\ \\
\begin{tabular}{rl}
	\textbf{(0)} & $\forall u \in V \quad \left<u|u\right> = 0 \quad \Leftrightarrow u = 0$ \\
	\textbf{(LN)} & $\forall a \in \C \quad \forall u,v \in V \quad \left<au|v\right> = a\left<u|v\right>$ \\
	\textbf{(LS)} & $\forall u,v,w \in V \quad \left<u+v|w\right> = \left<u|w\right> + \left<v|w\right>$ \\
	\textbf{(KS)} & $\forall u,v \in V \quad \left<v|u\right> = \overline{\left<u|v\right>}$ \\
	\textbf{(P)} & $\forall u \in V \quad \left<u|u\right> \ge 0$
\end{tabular}
\ \\
\paragraph{Pozorování}
$\left<u|av\right> = \overline{\left<av|u\right>} = \overline{a}\left<u|v\right>$
\paragraph{Pozorování}
Skalární součin lze vyjádřit regulární matici A: $\left<u|v\right> = u^T A^T A v$

\subsection{Norma}
\setcounter{equation}{0}
\paragraph{Definice}
Nechť $V$ je vektorový prostor se skalárním součinem. Potom norma určená tímto
součinem je zobrazení $V \to \R$ dané předpisem $||u|| = \sqrt{\left<u|u\right>}$.
\paragraph{Poznámka}
V $\R^n$ můžeme definovat úhel sevřený přímkami pomocí vztahu: $\left<u|v\right> = ||u||
\cdot ||v|| \cdot \cos \varphi$
\paragraph{Poznámka}
Z předchozí vztahu je možné dokázat Kosinovou větu.

\subsection{Cauchy-Schwarzova nerovnost}
\setcounter{equation}{0}
\paragraph{Věta}
Nechť $V$ je vektorový prostor se skalárním součinem a normou z něj odvozenou.
Potom platí:
\begin{align*}
	\forall u,v \in V \quad |\left<u|v\right>| \le ||u|| \cdot ||v||
\end{align*}
\paragraph{Důkaz}
(Pokud $u = 0 \lor v = 0$ platí triviálně.) \\
Zaveďme parametr $a \in C$ a dokážeme $||u + av|| \ge 0$:
\begin{align}
	0 \le ||u+av||^2 &= \left<u + av|u+av\right> \\
	&= \left<u|u\right> + a\left<v|u\right> + \overline{a}\left<u|v\right> +
	a\overline{a}\left<v|v\right>
\end{align}
Zvolíme $a = - \frac{\left<u|v\right>}{\left<v|v\right>}$, dosadíme a získáme:
\begin{align}
	0 \le  \quad \left<u|u\right> - \frac{\overline{\left<u|v\right>}\left<u|v\right>}{\left<v|v\right>}
\end{align}
Což upravíme a odmocníme:
\begin{align}
	|\left<u|v\right>|^2  \quad &\le  \quad \left<u|u\right>\left<v|v\right> \\
	|\left<u|v\right>| \quad  &\le \quad ||u|| \cdot ||v||
\end{align}

\paragraph{Důsledek}
Norma odvozená ze skalárního součinu
\paragraph{Důkaz} (trojúhelníková nerovnost)
\begin{align}
	||u+v|| &\le ||u|| + ||v|| \\
	||u+v|| &= \sqrt{\left<u+v|u+v\right>}\\
	&=\sqrt{\left<u|u\right> + \left<u|v\right> + \left<v|u\right> + \left<v|v\right>}
\end{align}
Protože $2 \R $e$(a) \le 2|a|$ můžeme upravit:
\begin{align}
	\le \sqrt{||u||^2 + ||v||^2 + 2|\left<u|v\right>|}
\end{align}
A podle C-S:
\begin{align}
	\le \sqrt{||u||^2 + 2||u|| \cdot ||v|| + ||v||^2} = ||u|| \cdot ||v||
\end{align}


\subsection{Norma prostoru zobrazení}
\setcounter{equation}{0}
\paragraph{Definice}
Norma prostoru zobrazení $V \to \R$ splňuje axiomy:
\begin{enumerate}
	\item $\forall u \in V \quad ||u|| \gt 0$
	\item $\forall u \in v \quad ||u|| = 0 \Leftrightarrow u = 0$
	\item $\forall u \in V \ \forall a \in \C \quad ||au|| = |a| ||u||$
	\item $\forall u,v \in V \quad ||u+v|| \le ||u|| + ||v||$
\end{enumerate}



\subsection{Ortogonální vektory}
\setcounter{equation}{0}
\paragraph{Definice}
Nechť V je vektorový prostor se skalárním součinem. Dva vektory $u, v \in V$
jsou navzájem ortogonální pokud platí $\left<u|v\right> = 0$. Značíme $u\bot v$.
\paragraph{Pozorování}
Každý systém vzájemně ortogonálních vektorů je lineárně nezávislý.

\subsection{Ortonormální báze}
\setcounter{equation}{0}
\paragraph{Definice}
Nechť $V$ je vektorový prostor se skalárním součinem a $Z$ je báze taková, že:
\begin{enumerate}
	\item $\forall v \in Z \quad ||v|| = 1$
	\item $\forall v, w \in Z \quad v \neq w \Rightarrow v \bot w$
\end{enumerate}

\subsection{Fourierovy koeficienty}
\setcounter{equation}{0}
\paragraph{Tvrzení}
Nechť $Z=(v_1, ..., v_n)$ je báze vektorový prostor se skalárním součinem $V$.
Potom:
vektor $u$ vyjádříme jako lineární kombinaci vektorů báze $Z$:
\begin{align*}
	\forall u \in V \quad u = \sum_{i=1}^n \left<u|v_i\right>v_i
\end{align*}
\paragraph{Důkaz}
Vyjádříme lineární kombinaci a rozepíšeme:
\begin{align}
	u = \sum_{i=1}^n a_i v_i \qquad \text{chceme: } \  a_i = \left<u|v_i\right> \\
	\left<u|v_i\right> = \left< \sum_{j=1}^n a_jv_j|v_i \right> = \sum_{j=1}^n a_j \left<v_j|v_i\right>
\end{align}
Vidíme, že $\left<v_j|v_i\right>$ je rovný $0$ pokud $i \neq j$, jinak $1$. Můžeme tedy
upravit na:
\begin{align}
	= a_i \left<v_i|v_i\right> = a_i 
\end{align}
\paragraph{Definice}
Koeficientům $\left<u|v_i\right>$ se říká Fourierovy koeficienty.

\subsection{Parsevalova rovnost}
\setcounter{equation}{0}
\paragraph{Tvrzení}
Nechť $Z=(v_1, ..., v_n)$ je báze vektorového prostoru se skalárním součinem $V$.
Potom:
\begin{align*}
	\forall u,v \in V \quad \left<u|w\right> = [w]_Z^H[u]_Z
\end{align*}
\paragraph{Důkaz}
Vyjádříme pomocí Fourierových koeficientů:
\begin{align}
	\left<u|w\right> = \left< \sum_{i=1}^n\left<u|v_i\right> \Big| \sum_{j=1}^n \left<w|v_j\right>v_j \right>
\end{align}
A rozepíšeme:
\begin{align}
	\sum_{i=1}^n \sum_{j=1}^n
	\left<u|v_i\right>\overline{\left<w|v_j\right>}\left<v_i|v_j\right> 
	= \sum_{i=1}^n \left<u|w_i\right> \overline{\left<w|v_j\right>} =
	[w]_Z^H[u]_Z
\end{align}


\subsection{Unitární zobrazení}
\setcounter{equation}{0}
\paragraph{Definice}
Lineární zobrazení $f:V \to W$ mezi vektorovými prostory se skalárním součinem
se nazývá unitární, pokud zachovává skalární součin:
\begin{align*}
	\forall u, v \in V \quad \left<u|w\right> = \left< f(u) | f(w) \right >
\end{align*}

\subsection{Podmínka unitárního zobrazení}
\setcounter{equation}{0}
\paragraph{Věta}
Zobrazení $f: V \to W$ je unitární právě když pro normy odvozené ze skalárního
součinu platí, že:
\begin{align*}
	\forall u \in V \quad ||u|| = ||f(u)||
\end{align*}
\paragraph{Důkaz}
\begin{description}
	\item[$\Rightarrow$] triviální
	\item[$\Leftarrow$] Rozepíšeme podle definice lineárního zobrazení a normy
	(podobně jako u C-S nerovnosti)
		\begin{align}
			||u+aw||^2      &=  ||u||^2 + a \left< w | u\right> +
				\overline{a} \left<u | w \right> + a \overline{a} 	||w||^2 \\
			||f(u + aw)||^2 &= 	||f(u)||^2 + a \left< f(w) | f(u)\right> +
				\overline{a} \left< f(u) | f(w) \right> + a \overline{a} ||f(w)||^2
		\end{align}
		Levé strany se z předpokladu rovnají, porovnáme tedy pravé strany.
		Můžeme přitom zanedbat členy zapsané jako norma, protože ty jsou z
		definice taktéž rovny.
		Zároveň zvolme $a = 1$ (\ref{puz-1}) a $a = i$ (\ref{puz-2}):
		\begin{align}
			\label{puz-1} 
				\left< w | u \right> + \left< u | w \right> &= \left<
				f(w)|f(u)\right> + \left< f(u)|f(w)\right> \\
			\label{puz-2}
				\left< w | u \right> - \left< u | w \right> &= \left<
				f(w)|f(u)\right> - \left< f(u)|f(w)\right>
		\end{align}
		Rovnice sečteme a vydělíme $2$. Získáme tedy:
		\begin{align}
			\left< w | u \right> = \left< f(w) | f(u) \right>
		\end{align}
\end{description}


\subsection{Isometrie}
\setcounter{equation}{0}
\paragraph{Definice}
Unitární izomorfismus prostoru se skalárním součinem se nazývá \textbf{isometrie}.


\subsection{Matice zobrazení isometrie}
\setcounter{equation}{0}
\paragraph{Věta}
Nechť $V$ a $W$ jsou vektorové prostory s ortonormálními bázemi $X$ a $Y$ stejné konečné
dimenze. Potom:
\begin{align*}
	f: V \to W \text{ je isometrie} \quad \Leftrightarrow \quad [f]_{XY} \text{ je
	unitární}
\end{align*}
\paragraph{Důkaz}
X je ortonormální, platí tedy:
\begin{align}
	\label{mzi-1}\left< u | w \right> = [w]_X^H[u]_X
\end{align}
Y je ortonormální, platí tedy:
\begin{align}
	\label{mzi-2}\left< f(u) | f(w) \right> = [f(w)]_Y^H[f(u)]_Y =
	([f]_{XY}[w]_X)^H [f]_{XY} [u]_X = [w]_X^H \underbrace{[f]_{XY}^H[f]_{XY}}_A[u]_X
\end{align}
Rovnost (\ref{mzi-1}) = (\ref{mzi-2}) platí právě tehdy když $A = I_n$, tedy
$[f]_{XY}$ je z definice unitární.

\subsection{Ortogonální projekce}
\setcounter{equation}{0}
\paragraph{Definice}
Nechť $W$ je vektorový prostor se skalárním součinem, $V \subsetq W$ a $Z = (v_1,
..., v_n)$ je ortonormální báze $V$. Potom zobrazení $p: W \to V$ definované
předpisem:
\begin{align*}
	p(u) = \sum_{i=1}^n \left< u | v_i \right> v_i
\end{align*}
se nazývá \textbf{ortogonální projekcí} prostoru $W$ do $V$.

\subsection{Lemma o ortogonální projekci}
\setcounter{equation}{0}
\paragraph{Lemma}
Nechť $W$ je vektorový prostor se skalárním součinem, $V \subsetq W$ a $Z = (v_1,
..., v_n)$ je ortonormální báze $V$. Nechť $p$ je ortogonální projekcí $W \to V$, potom 
\begin{align*}
	u - p(u) \bot v_i \quad \forall v_i \in Z
\end{align*}
\paragraph{Důkaz}
Rozepíšeme a ověříme podle definice ortogonality:
\begin{align}
	\left< u - p(u) | v_i \right> = \left< u - \sum_{j=1}^n \left<u|v_j\right>
	v_j \Big| v_i \right> = \left<u | v_i\right> - \sum_{j=1}^n
	\left<u|v_j\right> \underbrace{\left<v_j | v_i \right>}_{=0 \text{ pro }
	j\neq i} = 0
\end{align}

\subsection{Gram-schmidtova ortonormalizace}
\setcounter{equation}{0}
\paragraph{Algoritmus}
\begin{description}
	\item[vstup] báze $U$
	\item[výstup] ortonormální báze $V$
	\item[činnost]
		$\forall i = 1 ... |U|:$
		\begin{enumerate}
			\item $w_i = u_i - \sum_{j=1}^{i-1}\left<u_i|v_i\right>v_i$
			\item $v_i = \frac{w_i}{||w_i||}$
		\end{enumerate}
\end{description}
\paragraph{Korektnost}
\begin{enumerate}
	\item $w_i \bot v_j \quad \forall j < i \quad \Rightarrow \quad v_i \bot
	v_j$ platí podle lemma o ortogonální projekci
	\item triviální
	\item Lineární obal báze je stejný podle Lemma o výměně
\end{enumerate}
\paragraph{Důsledek}
Pro každý vektorový prostor konečné dimenze existuje ortonormální báze.

\subsection{Metoda nejmenších čtverců}
\setcounter{equation}{0}
\paragraph{Tvrzení}
$p(u)$ je nejbližší bod k $u$ v prostoru $V$.
\paragraph{Důkaz}
Nechť $a = u - p(u)$, $b = w - p(u)$. Potom vypočítáme jejich rozdíl:
\begin{align}
	||a - b|| = \sqrt{\left<a-b|a-b\right>} = ... = \sqrt{\left<a|a\right> +
	\left<b|b\right>} \gt ||a||
\end{align}
\paragraph{Aproximace nejmenšími čtverci}
Použijeme kolmou projekci a aproximujeme tak řešení s minimální chybou.

\subsection{Ortogonální doplněk}
\setcounter{equation}{0}
\paragraph{Definice}
Nechť V je množina vektorů ve vektorovém prostoru W se skalárním součinem. Pak
\textbf{ortogonální doplněk} množiny V je množina
\begin{align*}
	V^\bot := \{u \in W | u \bot v_i \quad \forall v_i \in V\}
\end{align*}
\paragraph{Pozorování}
$U \subsetq V \Rightarrow U^\bot \ge V^\bot$
\paragraph{Důkaz}
Triviálně podle definice:
$u \in V^\bot \Rightarrow u \bot v \quad \forall v \in V \Rightarrow u \bot v
\forall v \in U \Leftrightarrow u \in U^T$


\subsection{Vlastnosti ortogonálního doplňku}
\setcounter{equation}{0}
\paragraph{Věta}
Nechť $V$ je podprostor $W$ se skalárním součinem. Potom platí:
\begin{enumerate}[a)]
	\item $V^\bot$ je podprostor $W$
	\item $V \cap V^\bot = \{ 0 \}$
	\item $\dim(V) + \dim(V^\bot) = \dim(W)$
	\item $(V^\bot)^\bot = V$
\end{enumerate}
\paragraph{Důkaz}
\begin{enumerate}[a)]
	\item Nechť $u, v \in V^\bot$. Potom $\forall w \in V$:
		\begin{align}
			\left< u + v | w \right> = \left< u | w \right> + \left< v | w
			\right> = 0 \quad \Rightarrow \quad u+v \in V^\bot
		\end{align}
	\item Sporem: nechť $u \in V\cap V^\bot$, $u \neq 0$. Potom axiomu platí:
		\begin{align}
			0 < \left< u | u \right>
		\end{align}
		Zároveň však z definice ortogonálního doplňku $\forall u \in V \forall v \in V^\bot: u \bot v$,
		potom také pro $u = v$ platí $u \bot u$. Neboli z definice ortogonality:
		\begin{align}
			\left< u | u \right> = 0
		\end{align}
		Což je spor.
\end{enumerate}
Pro body c a d budeme potřebovat následující Lemma:
\paragraph{Lemma} Bázi X vektorového prostoru V lze doplnit na ortonormální bázi
Z vektorového prostoru W. Tedy:
\begin{align}
	Y := Z \setminus X \qquad &X = \{x_1, ..., x_k\} \\
		& Y = \{y_1, ..., y_l\}
\end{align}
Potom $V^\bot = L(Y)$.
\paragraph{Důkaz lemmatu}
\begin{enumerate}
	\item $L(Y) \subsetq V^\bot$: \\
		Dokážeme, že $Y \subsetq V^\bot$.
		\begin{align}
			\forall x_i \in X \quad \forall y_i, y_j \in Y \quad x_i \bot y_j
			\quad \Rightarrow \quad y_j \bot \sum \alpha_i x_i \quad \Rightarrow
			\quad Y \subsetq V^\bot
		\end{align}
		Ukážeme, že platí ortogonalita pro libovolné $w \in L(Y)$, $z \in V$.
		Rozepsáním definice:
		\begin{align}
			\left< w | z \right> = \left < \sum \beta_i y_i \Big| \sum \alpha_i
			x_i \right> = \sum \sum \beta_j \alpha_i \left< y_j | x_i\right> = 0
		\end{align}
		Tedy $L(Y) \subsetq V^\bot$.
	\item Nechť $w \in V^\bot$. Potom je $w$ ortogonální k libovolnému $x_i$:
		\begin{align}
			\left< w | x_i \right> = 0
		\end{align}
		Vyjádříme $w$ jako lineární kombinaci $x_i$	a $y_i$
		\begin{align}
			w = \sum \underbrace{\alpha_i}_{=\left<w|x_i\right>} x_i + \sum \beta_i y_i
		\end{align}
		Zde je vidět, že první suma je nulová, tedy $w \in L(Y)$ a proto 
		$V^\bot	\subsetq L(Y)$
\end{enumerate}
\begin{enumerate}[c)]
	\item Podle lemma je již zřejmé, že $\dim(V) = |X|$ a $\dim(V^\bot) = |Y|$,
	proto:
	\begin{align}
		|X| + |Y| = |W| = \dim(W)
	\end{align}
	\item $(V^\bot)^\bot = L(Z \setminus Y) = L(X) = V$
\end{enumerate}

\section{Pozitivně definitní matice}
\setcounter{equation}{0}
\subsection{Pozitivně definitní matice}
\setcounter{equation}{0}
\paragraph{Definice}
Hermitovská matice $A$ řádu $n$ se nazývá pozitivně definitní pokud 
\begin{align*}
	\forall x \in \C^n \setminus \{ 0 \} \quad x^H A x \gt 0
	\end{align*}


\subsection{Věta o matici skalárního součinu}
\setcounter{equation}{0}
\paragraph{Věta}
Nechť $V$ je vektorový prostor se skalárním součinem a $X = (x_1, ..., x_n)$ je
jeho báze. Potom pro matici $A$ definovanou:
\begin{align*}
	a_{i,j} = \left< x_i | x_j \right> 
\end{align*}
platí, že:
\begin{align*}
	\forall u,v \in V \quad \left< u|w \right> = [w]_X^H A [u]_X
\end{align*}
\paragraph{Důkaz}
Vyjádříme jako lineární kombinaci vektorů báze:
\begin{align}
	[u]_X = (\alpha_1, ..., \alpha_n) \quad \to \quad u = \sum \alpha_i x_i \\
	[w]_X = (\beta_1, ..., \beta_n) \quad \to \quad u = \sum \beta_i x_i
\end{align}
Ověříme rozpisem podle definice:
\begin{align}
	\left< u | w \right> \left< \sum \alpha_i x_i \Big| \sum \beta_i x_i \right>
	= \sum \sum \alpha_i \overline{\beta_i} \left< x_i | x_j \right> = [w]_X^H A
	[u]_X
\end{align}
Kde poslední krok si lze představit jako maticové násobení.
\paragraph{Důsledek}
Z vlastnosti skalárního součinu je taková matice hermitovská:
$\left<x_j|x_i\right> = \overline{\left<x_i | x_j\right>}$

\subsection{Ekvivalentní podmínky hermitovské matice}
\setcounter{equation}{0}
\paragraph{Věta}
Nechť $A$ je hermitovská matice řádu $n$. Potom jsou následující podmínky
ekvivalentní:
\begin{enumerate}[a)]
	\item $A$ je pozitivně definitní
	\item $A$ má všechny vlastní čísla kladná
	\item existuje regulární matice $U$ taková, že $A = U^H U$
\end{enumerate}
\paragraph{Důkaz}
\begin{description}
	\item[$a \Rightarrow b)$] $A$ je z předpokladu hermitovská, má tedy vlastní
	čísla reálná. Mějme tedy vlastní číslo $\lambda \in \R$ a k němu příslušný
	vlastní vektor $x$. Potom z definice:
	\begin{align}
		0 < x^HAx = \lambda x^H x
	\end{align}
	Z maticového násobení je zřejmé, že $x^Hx > 0$, tedy také $\lambda > 0$.
	\item[$b \Rightarrow c)$] $A$ je z předpokladu hermitovská, existuje tedy
	unitární $R$: $A = R^H D R$, kde $D$ je diagonální. Zvolme tedy:
	\begin{align}
		D': d_{i,j}' = \sqrt{d_{i,j}}
	\end{align}
	Tedy platí, že $A =  R^{H} D'^H D' R$. Zvolme tedy $U=D'R$.
	\item[$c \Rightarrow a)$] Podle definice pozitivně definitní matice ověříme:
	\begin{align}
		x^H A x = x^H U^H U x = (Ux)^H \cdot (Ux)
	\end{align}
	Kde je zřejmé, že daný součin bude kladný.
\end{description}


\subsection{Choleského rozklad}
\setcounter{equation}{0}
\paragraph{Věta}
Pro pozitivně definitní matici existuje jednoznačně určená trojúhelníková matice
$U$ s kladnými prvky na diagonál taková, že $A = U^HU$.
\paragraph{Algoritmus}
\begin{description}
	\item[Vstup] hermitovská matice A
	\item[Výstup] trojúhelníková matice U nebo A není pozitivně definitní
	\item[Postup] \ \\
	Pro $i := 1, ..., n$
	\begin{align}
		u_{i,i} = \sqrt{ a_{i,i} - \sum_{k=1}^{i-1} \overline{u_{k,i}} u_{k,i}}
	\end{align}
	Pro $j := i+1, ..., n$
	\begin{align}
		u_{i,j} = \frac{1}{u_{i,i}} \left( a_{i,j} - \sum_{k=1}^{i-1}
		\overline{u_{k,i}} u_{k,j} \right)
	\end{align}
\end{description}

\subsection{Podmínka na pozitivně definitní matice podle determinantu}
\setcounter{equation}{0}
\paragraph{Věta}
Nechť $A$ je bloková matice tvaru:
\begin{align*}
	A=\left(
	\begin{matrix}
	\alpha & \cdots & a^H & \cdots \\
	\vdots & && \\
	a & & \tilde{A} & \\
	\vdots &&&
	\end{matrix}
	\right)
\end{align*}
Pak je matice $A$ pozitivně definitní právě když:
\begin{align*}
	\alpha > 0 \quad \land \quad a \left( \tilde{A} - \frac{1}{\alpha} a a^H
	\right) \text{ je pozitivně definitní }
\end{align*}
\paragraph{Poznámka}
Gaussovou eliminací můžeme matici $A$ upravit na tvar
\begin{align*}
	A=\left(
	\begin{matrix}
	\alpha & \cdots \quad a^H \quad  \cdots \\
	\vdots & & \\
	0 &  \tilde{A} - \frac{1}{\alpha} a a^H \\
	\vdots &
	\end{matrix}
	\right)
\end{align*}
\paragraph{Důkaz}
\begin{description}
\item[$\Leftarrow$]
Mějme libovolný vektor $x \in \C^n \neq 0$ zapsaný ve tvaru:
\begin{align}
	x = \left( \begin{matrix} x_1 \\ \tilde{x} \end{matrix}\right) \qquad \tilde{x} \in
	\C^{n-1}, x_1 \in \C
\end{align}
Ověříme podmínku pro pozitivně definitní matici: (aplikujeme maticové násobení)
\begin{align}
	x^HAx &= \left(\overline{x_1}, \tilde{x}^H\right) \cdot \left( \begin{matrix} \alpha &
	a^H \\ a & \tilde{A} \end{matrix} \right) \cdot \left( \begin{matrix} x_1 \\
	\tilde{x} \end{matrix} \right) = \\
	&=\left( \tilde{x}_1 \alpha + \tilde{x}^H a, \quad \overline{x}_1 a^H + \tilde{x}^H
	\tilde{A}\right) \cdot \left( \begin{matrix} x_1 \\ \tilde{x} \end{matrix} \right)
	 \\
	&=\tilde{x}_1\alpha x_1 + \tilde{x}^H a x_1 + \overline{x}_1 a^H \tilde{x} +
	\tilde{x}^H \tilde{A} \tilde{x} 
\end{align}
Přičteme $0$:
\begin{align}
	\tilde{x}_1\alpha x_1 + \tilde{x}^H a x_1 + \overline{x}_1 a^H \tilde{x} +
	\tilde{x}^H \tilde{A} \tilde{x} +
    \frac{1}{\alpha} \tilde{x}^H a a^H \tilde{x}
	- \frac{1}{\alpha} \tilde{x}^H a a^H \tilde{x}
\end{align}
Povytýkáme:
\begin{align}
	\underbrace{\tilde{x}^H\left( \tilde{A} - \frac{1}{\alpha} a a^H \right)
	\tilde{x}}_{\ge 0} +
	\underbrace{\left( \sqrt{\alpha} \overline{x}_1 + \frac{1}{\sqrt{\alpha}} \tilde{x}^Ha
	\right) \left(\sqrt{\alpha} x_1 + \frac{1}{\sqrt{\alpha}}
	a^H\tilde{x} \right)}_{\text{komplexně sdružené} \Rightarrow \ge 0}
\end{align}
\item[$\Rightarrow$] \ \\
\begin{enumerate}
	\item $\alpha = e_1^H A e_1 > 0$ - platí
	\item Pro libovolné $\tilde{x} \in \C^{n-1}$ zvolíme $x_1 := -
	\frac{1}{\alpha}a^H \tilde{x}$ a položím $x = \left( x_1 \\ \tilde{x}
	\right)^T$. Potom tedy z předpokladu ověříme:
	\begin{align}
		0 < x^H A x = \tilde{x}\left( \tilde{A} - \frac{1}{\alpha} a a^H \right)
		+ 0
	\end{align}
	Kde díky volbě $x$  se nám zbytek členů odečte.
\end{enumerate}
\end{description}
\paragraph{Důsledek}
Pozitivně definitní matice lze rozeznat Gaussovo eliminací
\paragraph{Důsledek}
Jaccobiho podmínka: Hermitovská matice $A$ řádu n je pozitivně definitní právě
tehdy, pokud mají matice $A_1, ..., A_n$ kladný determinant (kde $A_i$ vznikne z
$A$ vymazáním posledních $i$ řádků a sloupců).
\paragraph{Důkaz}
Rekurentně aplikujeme větu na matici $A$ v odstupňovaném tvaru.


\section{Kvadratické a bilineární formy}
\setcounter{equation}{0}
\subsection{Bilineární forma}
\setcounter{equation}{0}
\paragraph{Definice}
Nechť $V$ je vektorový prostor nad $\K$ a $f: V \times V \to \K$ zobrazení splňující:
\begin{enumerate}
	\item $\forall u,v,w \in V \quad f(u+v, w) = f(u, w) + f(v,w)$
	\item $\forall u,v \in V \quad \forall \alpha \in \K \quad f(\alpha u, v) =
	\alpha f(u,v)$
	\item $\forall u,v,w \in V \quad f(u,v+ w) = f(u, v) + f(u,w)$
	\item $\forall u,v \in V \quad \forall \alpha \in \K \quad f(u, \alpha v) =
	\alpha f(u,v)$
\end{enumerate}
Potom $f$ se nazývá \textbf{bilineární formou} na $V$.
\paragraph{Definice}
Pokud navíc platí:
\begin{align*}
	\forall u,v \in V \quad f(u,v) = f(v, u)
\end{align*}
je bilineární forma \textbf{symetrická}.

\subsection{Kvadratická forma}
\setcounter{equation}{0}
\paragraph{Definice}
Zobrazení $g: V \to \K$ se nazývá kvadratická forma pokud existuje bilineární
forma $f$ taková, že:
\begin{align*}
	\forall u \in V g(u) = f(u, u)
\end{align*}
\paragraph{Pozorování}
$g(\alpha u) = f(\alpha u, \alpha u) = \alpha^2 g(u)$


\subsection{Matice lineární a kvadratická forma}
\setcounter{equation}{0}
\paragraph{Definice}
Nechť $V$ je vektorový prostor a $X=(v_1, ..., v_n)$ je jeho báze. Potom definujeme:
\begin{description}
	\item[Matici lineární formy] $f$ vůči bázi $X$ jako matici $B$ kde platí $b_{i,j}
	= f(v_i, v_j)$
	\item[Matici kvadratické formy] jako matici symetrické bilineární formy
	která ji vytvořuje.
\end{description}

\subsection{Analytické vyjádření}
\setcounter{equation}{0}
\paragraph{Definice}
Nechť $f$ je bilineární formou nad $\K^n$ a $B$ je její matice. Její analytické vyjádření polynomem:
\begin{align*}
	f((x_1, ..., x_n)^T, (y_1, ..., y_n)^T) = \sum_{i=1}^n \sum_{j=1}^n x_i y_i
	b_{i,j}
\end{align*}

\subsection{Matice formy a matice přechodu}
\setcounter{equation}{0}
\paragraph{Pozorování}
Nechť $B$ je maticí formy $f$ vůdci bázi $Y$. Potom matice $[id]_{XY}^T B [id]_{XY}$
je maticí téže formy vůči $X$.
\paragraph{Důkaz}
Rozepíšeme podle definice:
\begin{align}
	[u]_Y = [id]_{XY} [u]_X \\
	f(u,v) = [u]_Y^T B[v]_Y = [u]_X \underbrace{[id]_{XY}^T B [id]_{XY}}_{B_X} [w]_X
\end{align}

\subsection{Silvestrův zákon setrvačnosti kvadratických forem}
\setcounter{equation}{0}
\paragraph{Věta}
Nechť $f: V \to \R$ je kvadratická forma. Potom existuje báze $X$ prostoru $V$ taková,
že matice $f$ vůči $X$ je diagonálí a prvky na diagonále jsou nulové ($n_0$), kladné
($n_+$), záporné ($n_-$) a \textbf{signatura formy} je trojce $(n_0, n_+, n_-)$.
Silvestrův zákon setrvačnosti pak říká, že signatura formy je neměnná na volbě
báze a je pro všechny vhodné báze stejná. Taková vhodná báze se nazývá \textbf{polární}.
\paragraph{Důkaz}
\begin{enumerate}
	\item Dokážeme, že taková vhodná báze existuje: Mějme libovolnou bázi $Y$,
	poté sestavíme matici $B'$ formy $f$ vůči $Y$. Víme, že $B'$ je reálná
	symetrická matice (z definice). Podle věty o diagonalizaci symetrických
	matic existuje regulární matice $R$ taková, že $D$ je diagonální:
	\begin{align}
		R^{-1} \cdot B' \cdot R = D
	\end{align}
	Taktéž je vidět, jak vyjádříme matici $B$: ta je vůči bázi $X$, $B'$ vůči
	$Y$. Použijeme matici přechodu, kde B bude diagonální.
	\begin{align}
		B = [id]_{XY}^T B' [id]_{YX}
	\end{align}
	Nyní si stačí uvědomit, že sloupce $[id]_{XY}$ jsou vektory hledané báze $X$
	vůči $Y$.
	\item
	Nechť $V$ a $V'$ jsou vhodné báze, kde má forma $f$ diagonální matici $D$,
	$D'$ uspořádanou tak, že $d_{i_0,i_0} \gt 0$ pro jisté $i_0$. Pro spor
	předpokládejme, že $d'_{i_0,i_0} \le 0$ pro to samé $i_0$ (ostatní případy
	podobně). Nechť tedy existuje $j_0 < i_0$ pro které platí $d'_{j_0,j_0} \le
	0$ a
	nechť $L(\{v_1, ..., v_{i_0}\})$ a $L(\{v'_{j_0}, ..., v'_n\})$ jsou podprostory.
	Aby byly dimenze příslušných prostorů alespoň dimenze $L(V)$, požadujeme
	netriviální průnik. Mějme tedy $w$:
	\begin{align}
		0 \neq w = (w_1, ..., w_k) = \sum_{i=0}^{i_0} a_i v_i = \sum_{j=j_0}^n a'_j v'_j
	\end{align}
	Potom ale z předpokladů víme, že (v analytickém vyjádření; také je třeba si
	uvědomit, že matice d je diagonální a její prvky jsou pouze kladné/záporné)
	\begin{align}
		f(w) = \sum_i w_i^2 d_{i,i} \gt 0\\
		f(w) = \sum_i w_i^2 d'_{i,i} \le 0
	\end{align}
	Což je spor.
\end{enumerate}


\subsection{O přímkách svírajících úhel v $\R^n$}
\setcounter{equation}{0}
\paragraph{Věta}
Ne více než $\binom{n+1}{2}$ přímek v $\R^n$ může svírat stejný úhel.
\paragraph{Důkaz}
Je dáno $n$ přímek svírající stejný úhel udané vektory $v_1, ..., v_n\in \R^d$, kde
$||v_i||= 1$. Potom určíme úhel:
\begin{align}
	\left|\left< v_i | v_j \right>\right| = \begin{cases}
		1 \Leftrightarrow i = j \\
		\cos \alpha \Leftrightarrow i \neq j
	\end{cases}
\end{align}
Uvažme:
\begin{align}
	a_1 v_1 v_1^T + ... + a_n v_n v_n^T = 0
\end{align}
Pro každé $j$ vynásobím $v_j^T$ zleva, $v_j$ zprava:
\begin{align}
	a_1 v_j^T v_1 v_1^T v_j + ... + a_n v_j^T v_n v_n^T v_j = \\
	= \sum_{j=1}^n a_i \left< v_j | v_i \right>^2 = a_j + \sum_{i\neq j} a_i
	\cdot
	\cos^2 \varphi
\end{align}
Což můžeme maticově zapsat jako reálnou symetricou matici vynásobenou (zprava)
sloupcovým vektorem $(a_i)$. Taková soustava má však trivální řešení, proto
$v_1, ..., v_n$ jsou lineárně nezávislé a tudíž $n \le \binom{d+1}{2}$.





















\end{document}
