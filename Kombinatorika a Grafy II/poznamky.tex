\documentclass{article}
\usepackage[utf8]{inputenc}
\usepackage[czech]{babel}
\usepackage[left=1cm,right=1cm,top=1cm,bottom=1.5cm]{geometry}
\usepackage{graphics}
\usepackage{wrapfig}
\usepackage{float}

\usepackage{../math}

\title{Kombinatorika a Grafy II}
\author{Ladislav Láska}

\renewcommand{\paragraph}[1]{\ \\\smallskip\noindent\textbf{#1}\ }

\begin{document}

\maketitle
\tableofcontents
\newpage

\section{Perfektní párování}
Perfektní párování (taktéž {\it 1-faktor}) grafu $G$ je $M \subset E_G$ takový, 
že $\forall v \in V_g: \deg(v) = 1$.
\subsection{Hallova věta}
\subsection{Tutteho věta}
\paragraph{Pozorování}
Pokud v grafu, který má perfektní párování, vezmeme libovolnou množinu $S$, pak 
nebude v grafu $G\setminus S$ více lichých komponent, než je vrcholů v $S$.  
Proč? V perfektním párování totiž z každé liché komponenty musí vést párovací 
hrana do $S$.
\paragraph{Věta} (Tutte) Graf má perfektní párování \textbf{právě tehdy}, když 
platí Tutteho podmínka:
\begin{align}
	q(G\setminus S) \leq |S|
\end{align}
kde $q(G)$ je počet lichých komponent $G$.
\paragraph{Důkaz}
Předpokládejme, že graf $G$ je maximální bez perfektního párování, splňující 
Tutteho podmínku (zde maximální ve smyslu, že přidáním libovolné hrany již 
perfektní párování mít bude). To můžeme, protože jestli $G$ vznikl z $G'$ 
přidáváním hran a $G$ má špatnou množinu $S$ porušující Tutteho podmínku, je 
tato množina špatná i pro $G'$ (přidáváním hran nezvyšujeme počet lichých 
komponent).

Nyní se podíváme, jak takový graf vypadá, pokud by obsahoval špatnou množinu 
$S$:

\bigskip

\centerline{\it $(*)$ Všechny komponenty $G\setminus S$ jsou úplné a každý 
vrchol z $S$ má hranu do každého vrcholu $G\setminus S$.}

\bigskip

Také si všimneme, že pokud taková množina $S$ existuje v grafu $G$ bez 
perfektního párování, tak je buď špatná a nebo je špatná prázdná množina: pokud 
by $S$ byla dobrá, můžeme spárovat všechny komponenty a párovací hrany vést do 
$S$.  Protože ale $G$ nemá perfektní párování, musí být velikost $G$ lichá a 
tedy $\emptyset$ je špatná množina porušující Tutteho podmínku.

Nyní pro spor předpokládejme, že $G$ nemá perfektní porování a žádná množina $S$ 
nemá vlastnost $(*)$. Vezměme tedy nějakou množinu $S$ největší mezi takovými, 
že v ní každý vrchol má hranu do každého vrcholu $G\setminus S$. Protože 
nesplňuje vlastnost $(*)$, existují v nějaké komponentě $G\setminus S$ vrcholy 
$a,a'$ nespojené hranou. Označme $a,b,c$ první tři vrcholy na nejkratší cestě 
mezi $a,a'$. Protože navíc $b \notin S$, existuje $d \in V_G$, že $bd\in E_G$ 
(odpusťmě si triviální případ, kde $aa'$ byla jediná neobsažená hrana v 
komponentě a komponenta byla jediná).

Z maximality $G$ určitě $G+ac$ má párování $M_{ac}$ a $G+bd$ párování $M_{bd}$ 
(navíc tato párování přidané hrany obsahují). Nyní si všimneme, že $M_{ac} 
\oplus M_{bd}$ se sestává z sudých cyklů střídající hrany z obou párování.  
Označme $C_{ac}$ a $C_{bd}$ cykly obsahující hrany $ac$ a $bd$. Rozlišme dva 
příklady:

\begin{enumerate}
	\item $C_{ac} \neq C_{bd}$, pak prohozením párovacích a nepárovacích hran na 
	cyklu $C_{ac}$ v párování $M_{ac}$ získáme perfektní párování, což je spor s 
	tím, že ho $G$ nemá.
	\item $C_{ac} = C_{bd}$, pak vytvořme nový cyklus $C'$ tak, že v něm uděláme 
	zkratku z $a$ nebo $c$ přímo do $b$. Protože nová hrana neleží v párování 
	$M_{bd}$, stenětak jako hrana $ac$. Potom prohodíme párovací a nepárovací 
	hrany $C'$ v párování $M_{bd}$ a máme perfektní párování grafu $G$, což je 
	opět spor.
\end{enumerate}
\qed

\subsection{Petersenova věta}
\paragraph{Věta} (Petersen)
Každý 3-regulární graf bez mostů má perfektní párování.
\paragraph{Důkaz}
Ukážeme, že každá $S$ splňuje Tutteho podmínku. Mějme libovolnou množinu $S$ a 
uvažujme $C$ lichou komponentu $G\setminus S$. Nahlédneme, že graf $G$ má sudý 
součet stupňů, stejnětak jako komponenta $C$. Ale graf $C$ přispívá do $G$ 
lichým součtem (lícho vrcholů, každý 3 hrany) -- musí tedy existovat lichý počet 
$S-C$ hran. Protože je navíc graf bez mostů, musí být takové alespoň 3.
Počet hran mezi $S$ a $G\setminus S$ je tedy alespoň $3q(G \setminus S)$.  
Zároveň je ale omezen součtem stupňů vrcholů v $S$, tedy nejvíce $3|S|$. Tedy 
$q(G\setminus S) \leq |S|$ a věta je dokázána.
\qed

\subsection{Maximální párování a vylepšující cesty}
\paragraph{Definice} Vrchol je nespárovaný (volný) v párování $M$ právě tehdy 
když není incidentní s žádnou hranou z $M$.
\paragraph{Definice} Cesta se v párování $M$ nazývá střídavá, právě tehdy když 
střídá hrany z párování $M$ a které nejsou.
\paragraph{Pozorování} Existují tři druhy střídavých cest: cesta mezi dvěma 
spárovanými vrcholy, mezi jedním nespárovaným a jedním spárovaným a mezi dvěma 
nespárovanými. Také se můžeme setkat se střídavým cyklem (ten je vždy sudý).
\paragraph{Definice} Vylepšující cesta je střídavá cesta mezi dvěma 
nespárovanými vrcholy.
\paragraph{Věta} Párování $M$ je maximální (co do počtu hran), právě tehdy když 
neexistuje vylepšující cesta.
\paragraph{Důkaz} Pokud je párování maximální, tak neexistuje vylepšující cesta 
(jinak bychom cestu převrátili a měli větší párování, což by byl spor). Dokažme 
tedy opačnou implikaci:

Nechť $M$ je párování, které není maximální a $N$ je nějaké maximální párování.  
Podívejme se na symetrickou diferenci $D = M \oplus N$. Protože $M$ i $N$ jsou 
párování, každý vrchol v $D$ sousedí nanejvýš s jednou hranou v $M$ a jednou v 
$N$. Všimneme si, že že $D$ skládá ze střídavých cest (střídají hrany z $M$ a 
$N$) a sudých cyklů. Protože navíc $M$ není maximální, $|N| > |M|$. V grafu $D$ 
nejsou hrany společné pro obě párování, to ale nerovnost nerozbije. Najdeme 
komponentu v $D$ takovou, že má je v ní více hran z $N$ než $M$. Taková ze 
zmíněné nerovnosti musí existovat. Nahlédneme ale, že sudý cyklus má hran 
stejně, cesta z vrcholu párovaného v $M$ do vrcholu párovaného v $N$ také, cesta 
z i do vrcholů párovaných v $M$ má hran z $N$ dokonce méně a tudíž musí jít o 
cestu z a do vrcholů párovaných v $N$. Taková cesta je ale vylepšující cesta pro 
$M$, což jsme chtěli dokázat.
\qed

Pomocí tohoto výsledku můžeme zkonstruovat jednoduchý algoritmus:

\paragraph{Algoritmus} (Maximální párování)
Začni s nějakým (třeba prázdným párováním M):
\begin{enumerate}
	\item Najdi vylepšující cestu, pokud existuje a převrať její hrany. Pokud 
	jsi našel, opakuj 1 s novým párováním.
	\item Vrať párování jako maximální.
\end{enumerate}
\paragraph{Důkaz} (přímo plyne z předchozí věty o maximálním párování a 
vylepšující cesty)
\qed

Ukazuje se ale, že hledání vylepšujících cest není v obecném případně snadné, 
protože když se do cesty vpletou liché cykly, přestane stačit jednoduché 
prohledávání do šířky. Ukážeme si nyní vylepšený algoritmus, který toto řeší.

\subsubsection{Edmondsův květinkový algoritmus}
\paragraph{Definice} Květinkou rozumíme lichý cyklus, ze kterého vede alespoň 
jedna hrana. Tu budeme nazývat stonek a budeme si přát, aby byl vždy spárovaný a 
na ostatní hrany na květince tvořily střídavou cestu.
\paragraph{Pozorování} Všimneme si, že pokud máme květinku a daný stonek, pak 
každý vrchol je ze stonku dosažitelný lichou i sudou střídavou cestou.

\begin{wrapfigure}{r}{0.2\textwidth}
\centering
\includegraphics{kvetinka.pdf}
\caption{Květinka}
\label{edmonds:kvetinka}
\end{wrapfigure}

\paragraph{Algoritmus} (Edmondsův květinkový)\\
\indent Vstup: (G, M) - graf a nějaké párování na něm \\
\indent Výstup: P vylepšující cesta (prázdná, pokud neexistuje)
\begin{enumerate}
	\item Pro každý nespárovaný vrchol vytvoř strom v $F$, sestávající se pouze 
	z daného vrcholu.
	\item Pro každý vrchol $v$ se sudou vzdáleností od kořene a pro jeho každou 
	hranu $e = vw$ učiň jedno z (a už se do daného vrcholu a hrany nevracej):
	\begin{enumerate}
		\item Pokud $w$ není v $F$, přidej do stromu hrany $vw$ a $wx$, kde $x$ 
		je vrchol spárovaný s $w$.
		\item Pokud má $w$ lichou vzdálenost od kořene, nedělej nic.
		\item Pokud vede hrana $e$ do jiného podstromu, ohlaš vylepšující cestu 
		mezi kořeny těchto stromů přes $e$.
		\item Jinak kontrahuj kytičku tvořenou hranou $e$ a cestou mezi $v,w$, 
		zarekurzi se a vrať odpověď z rekurze.
	\end{enumerate}
	\item Odpověz prázdnou cestou.
\end{enumerate}

Jinými slovy: budujeme zakořeněný les z volných vrcholů takový, že z vrcholů na 
lichých hladinách vždy vede párovací hrana a ze sudých nepárovací hrana.  

\paragraph{Důkaz}
Spojování stromů je celkem přímočaré, zbývá uvěřit, že kontrakcí kytiček 
nezměníme existenci vylepšující cesty. Nyní předpokládejme, že vždy vylepšující 
cesta vede přes stonek. Ukážeme, že existence takové vylepšující cesty je 
ekvivalentní a také jak ji sestrojit.

\begin{enumerate}
	\item Pokud existuje v $G$ vylepšující cesta, existuje i v $G'$. Protože je 
	stonek spárovaný, z hrana na vylepšující cestě opouštějící květinku nesmí 
	být spárovaná.  Po kontrakci hrana bude sousedit přímo s vrcholem $v$ a 
	tudíž přes květinku také povede vylepšující cesta (ačkoli kratší).
	\item Pokud existuje v $'G$ vylepšující cesta, existuje i v $G$. Označme 
	vrchol, kterému náleží hrana vedoucí z $v$ v $G'$ ve vylepšující cestě $u$.  
	Podívejme se na tento vrchol v květince -- sousedí s právě jednou spárovanou 
	hranou. Vydáme se po této hraně nejkratší cestou po květince do vrcholu $v$.  
	Taková cesta nám ale dává vylepšující cestu vedoucí přes květinku v grafu 
	$G$. Pokud bychom tuto operaci dělali vícekrát na jedné květince, stačí 
	uvažovat, že ji pokaždé překořeníme do jiného vrcholu -- to ale vůbec 
	nevadí.
\end{enumerate}
\paragraph{Složitost}
Je zřejmé, že vnitřní cyklus vždy přidá vrchol do lesa, nalezne zlepšující 
cestu, nebo zkontrahuje květinku. Tedy jedna iterace trvá nejvýše $\O(n^2)$.  
Pokud budeme algoritmus iterovat pro nalezení maximálního párování, pustíme ho 
maximálně $\O(m)$ (při každém vylepšení párování použijeme o hranu navíc).  
Celková složitost je tedy $\O(n^2m)$
\qed

\paragraph{Pozorování}
Kontrakce kytičky obecně existenci vylepšující cesty nezachovává, když cesta 
neprochází stonkem (protipříklad: trojcyklus s vyčnívající hranou z každého 
vrcholu).
\subsection{Počet perfektních párování}
\paragraph{Poznámka} V následujícím textu předpokládáme grafy s násobnými 
hranami, ale bez smyček (ty nám kazí bipartitní grafy).
\paragraph{Lemma} Nechť $G$ je bipartitní 3-regulární graf na 2n vrcholech 
takový, že pro každou hranu $e \in E_G$ má $G$ alespoň $N$ perfektních párování 
neobsahujících $e$. Pak má $G$ alespoň $3N/2$ perfektních párování.
\paragraph{Důkaz} Nechť $v$ je vrchol $G$ a $e_1, e_2, e_3$ jeho hrany. Označme 
$P_i$ množinu všech perfektních párování grafu $G$, které obsahují hranu $e_i$ a 
nahlédněme, že $\sum |P_i|$ je počet všech perfektních párování grafu $G$. Dále 
nahlédneme, že $|P_1| + |P_2|$ je počet perfektních párování neobsahujících 
hranu $e_3$ a analogicky pro ostatní hrany. Po sečtení získáme $2(|P_1| + |P_2| 
+ |P_3|) \geq 3N$, což tvrzení dokazuje.
\qed

\paragraph{Věta} (Voorhoeve)
Nechť $G$ je bipartitní 3-regulární graf na $2n$ vrcholech a $e$ je hrana $G$.  
Pak má $G$ alespoň $N = \frac{3}{2} \left( \frac{4}{3} \right)^n$ perfektních 
párování neobsahujících hranu $e$.
\paragraph{Důkaz} Indukcí: pro $n=1$ se jedná o dva vrcholy s trojnásobnou 
hranou a věta platí.
Dále pro $n>1$:

\begin{enumerate}
	\item $G$ není souvislý: Označme $G_e$ komponentu souvislosti obsahující $e$ 
	a $G' := G - G_e$. Z indukčního předpokladu plyne, že $G_e$ má alespoň 
	$\frac{3}{2} \left( \frac{4}{3} \right)^{|V_{G_e}|/2}$ perfektních párování 
	nebosahujících $e$. Pro libovolnou hranu $f$ z $G'$ analogicky, přičemž 
	každá kombinace párování je platná. To nám tedy dává: 
	$\frac{9}{4}\left(\frac{4}{3}\right)^{|V_{G_e}|/2 + |V_{G'}|/2} > N$, takže 
	tvrzení platí.
	\item $G$ je souvislý. Označme $e = uv$ a $u,w_1,w_2$ sousedy $v$. Jelikož 
	je $G$ souvislý a má alespoň 4 vrcholy, můžeme předpokládat, že $w_2 \neq 
	u$. Rozlišme několik případů:
	\begin{enumerate}
		\item $w_1 = u$: Označme pak $z$ souseda $u$ různého od $v$ a $H$ buď 
		graf získaný z $G \setminus \{u,v\}$ přidáním hrany $e' = w_2z$, viz 
		Obrázek \ref{parovani:paralelni}. Snadno nahlédneme, že výsledný graf je 
		bipartitní a 3-regulární na $2(n-1)$ vrcholech, tedy podle indukčních 
		předpokladů a Lemmatu má graf $H$ alespoň 
		$\frac{9}{4}\left(\frac{4}{3}\right)^{n-1}$ perfektních párování. Navíc 
		nahlédneme, že každé perfektní párování grafu $H$ odpovídá právě jednomu 
		v grafu $G$, věta tedy platí.
\begin{figure}[H]
\centering
\includegraphics{parovani-paralelni.pdf}
\caption{Nahrazení paralelní hrany}
\label{parovani:paralelni}
\end{figure}
		\item $w_1 \neq u \quad\&\quad w_1 = w_2$: Nechť $z$ je soused $w_1$ 
		různý od $v$ a $H$ graf získaný z $G \setminus \{v,w_1\}$ přidáním hrany 
		$e' = uz$, viz Obrázek \ref{parovani:sousedni}. Opět aplikujeme indukční 
		předpoklad, ale tentokrát získáme pouze polovinu potřebných párování.  
		Všimneme si ale, že na každé párování $H$ připadají dvě párování v $G$ 
		(podle volby paralelní hrany).
\begin{figure}[H]
\centering
\includegraphics{parovani-sousedni.pdf}
\caption{Nahrazení hrany sousední s paralelní}
\label{parovani:sousedni}
\end{figure}
		\item $u, w_1, w_2$ navzájem různé: Nejprve označme hrany $e_i$ jako 
		hrany vedoucí z $w_j$ do $y_i$. Vytvořme grafy $H_i$ pro $i=1..4$ ($H_1$ 
		na obrázku \ref{parovani:obecne}).
\begin{figure}[H]
\centering
\includegraphics{parovani-obecne.pdf}
\caption{Nahrazení hrany v obecném případě}
\label{parovani:obecne}
\end{figure}
		smazáním vrcholů $w_1,v$ a přidáním hrany $e_i'=(u,y_i)$ a zbylé $y_i$ 
		spojíme s $w_2$.  Zjevně $H_i$ mají každý z indukčního předpokladu 
		alespoň $3N/4$ perfektních párování.  Označíme $P_i$ množinu všech 
		párování grafu $G$, které neobsahují hranu $e$, ale obsahují hranu 
		$e_i'$.  Nahlédneme, že jsou množiny $P_i$ disjunktní.  Podíváme se na 
		párování $M$ v grafu $H_1$, které neobsahují hranu $e_1'$. Pokud takové 
		párování obsahuje hranu $y_2w_2$, odpovídá párování $P_2$ (kvůli vrcholu 
		$y_2$ a ekvivalentně pro ostatní $P_i$ a $H_i$. Stačí tedy sečíst 
		následující soustavu:
		\begin{align}
			|P_2|+|P_3|+|P_4| \geq 3N/4\\
			|P_1|+|P_3|+|P_4| \geq 3N/4\\
			|P_1|+|P_2|+|P_4| \geq 3N/4\\
			|P_1|+|P_2|+|P_3| \geq 3N/4\\
			3(|P_1|+|P_2|+|P_3| + |P_4|) \geq 3N
		\end{align}
		Což je ale počet perfektních párování $G$ neobsahujících hranu $e$, což 
		jsme chtěli dokázat.
	\end{enumerate}
\end{enumerate}
\qed

\section{Grafy na plochách}
\subsection{Homeomorfismus, plocha}
\paragraph{Definice} \textit{Homeomorfismus} je bijekce $f: A\to B$ taková, že 
$f$ i $f^{-1}$ jsou spojité. Existuje-li homeomorfismus z plochy $A$ do $B$, 
jsou homeomorfní.
\paragraph{Příklady} Homeomorfismy jsou například posunutí, otočení, zrcadlení, 
zmenšení, zvětšení, nebo složení dvou homeomorfismů.
\paragraph{Pozorování} Homeomorfismus zachovává nakreslitelnost grafu: obraz 
jednoduché spojité křivky je jednoduchá spojitá křivka, obraz nakreslení grafu 
je nakreslení grafu.
\paragraph{Konvence} Homeomorfní plochy budeme pro naše použití považovat za 
stejné.
\paragraph{Definice} \textit{Plocha} je kompaktní souvislá dvourozměrná varieta 
bez hranice, tj. každý její bod je homeomorfní otevřenému disku a z každého 
pokrytí otevřenými množinami lze vybrat konečné pokrytí.
\paragraph{Pozorování} Podle kompaktnosti lze plochu rozstříhat na konečné 
množství částí
\subsection{Buňkové nakreslení, rod plochy}
\paragraph{Definice} Nakreslení grafu $G$ na ploše je \textit{buňkové}, jestli 
každá stěna homeomorfní otevřenému disku.
\paragraph{Pozorování} Má-li graf $G$ buňkové nakreslení, pak je souvislý.
\paragraph{Věta} Jsou-li grafy $G_1$ a $G_2$ nakreslené na stejné ploše, pak 
$|E_{G_1}| - |V_{G_1}| - |F_{G_1}| = |E_{G_2}| - |V_{G_2}| - |F_{G_2}|$.
\paragraph{Důkaz} (TODO)
\paragraph{Lemma} Označme $t(G) = |E_G| - |V_G| - |F_G|$. Pak pro buňkově 
nakreslené grafy $H \subseteq G$ platí $t(G) = t(H)$.
\paragraph{Důkaz} Označme $G_f$ podgraf $G$ nakreslený v uzávěru nějaké stěny $f 
\in F_H$ (TODO: Obrázek). Takový graf je souvislý a rovinný, proto platí 
Eulerova formule: $|F_{G_f}| = |E_{G_f}| - |V_{G_f}| + 2$, pomocí které 
vyjádříme počet stěn $G$ (1 odečítáme za vnější stěnu):
\begin{align}
	|F_G| = \sum_{f\in F_H} (|F_{G_f}| - 1) = \sum_{f\in F_H} (|E_{G_f}| - 
	|V_{G_f}| + 1)
\end{align}
Dále si rozebereme jednotlivé části sumy. Uvědomíme si, že hrany z $G$ jsme 
započítali každou jednou, až na hrany použité uzávěrem, které jsme započítali 
každou dvakrát -- to jsou právě všechny hrany $H$. Proto:
\begin{align}
	\sum_{f\in F_H} |E_{G_f}| = |E_G| + |E_H|
\end{align}
Pro vrcholy je to podobně: započítali jsme každý vrchol z $G \setminus H$, 
vrcholy v $H$ ale vícekrát -- tolikrát, kolik mají stěn, což je jejich stupeň a 
tedy dvojnásobek hran v $H$.  Tedy:
\begin{align}
	\sum_{f\in F_H} |V_{G_f}| = |V_G| - |V_H| + \sum_{v\in V_G} \deg_H(v) = 
	|V_G| - |V_H| + 2|E_H|
\end{align}
To nyní sečteme a získáme:
\begin{align}
	|F_G| = |E_G| + |E_H| - (|V_G| - |V_H| + 2|E_H|) + |F_H| = |E_G| - |V_G| - 
	t(H)
\end{align}
Což po upravení dává žádanou rovnost $t(H) = t(G)$ a lemma je dokázáno.
\qed

\paragraph{Věta}
Jsou-li grafy $G_1$ a $G_2$ buňkově nakreslené na stejné ploše, pak $t(G_1) = 
t(G_2)$.
\paragraph{Důkaz}
Nejdříve si všimenem několika skutečností:
\begin{enumerate}
	\item Můžeme předpokládat, že grafy nemají smyčky (podrozdělením hrany 
	přidáme jednu hranu a jeden vrchol, počet stěn nezměníme)
	\item Můžeme předpokládat uzavřené buňkové nakreslení (přidáním vrcholu do 
	stěny délky $k$ a spojením se všemi hranami přidáme $k$ hran a $k-1$ stěn, 
	což se odečte)
	\item Můžeme předpokládat, že se hrany nekříží (pokud by se křížily, 
	podrozdělíme je tam a vytvoříme uzavřené nakreslení, čímž se z křížení stane 
	identita).
\end{enumerate}
Pak již triviálně podle předchozího lemmatu $t(G_1) = t(G_1 \cup G_2) = t(G_2)$ 
a věta je dokázána. \qed
\paragraph{Definice} \textit{Rod} plochy $\Sigma$ je celé číslo $g$ takové, že 
každý graf $G$ buňkově nakreslený na $\Sigma$ splňuje $|E_G| - |V_G| - |F_G| + 2 
= g$
\paragraph{Pozorování} Homeomorfismus zachovává nakreslení, tedy i rod.

Jak ale určíme rod toru? Všimeneme si, že mnohoúhelníková reprezentace dává 
buňkové nakreslení s jednou stěnou. Tedy například pro torus:
\begin{align}
	g = |E_G| - |V_G| - |F_G| + 2 = 2 -1 -1 +2 = 2
\end{align}

\paragraph{Definice} Plocha $\Sigma$ je \textit{orientovatelná}, jestliže každá 
uzavřená křivka má okolí homeomorfní otevřenému válci, jinak je $\Sigma$ 
\textit{neorientovatelná}.

\paragraph{Věta}
Mají-li dvě plochy stejný rod a orientovatelnost, pak jsou homeomorfní. Navíc 
rod každé orientovatelné plochy je sudý a nezáporný a každé neorientovatelné 
plochy je kladný.
\paragraph{Důkaz} (bez důkazu)

\paragraph{Lemma}
Přidáním ucha se zvýší rod plochy o 2 a nezmění orientovanost. Přidáním křižítka 
vznikne neorientovaná plocha s rodem o 1 vyšším.

\paragraph{Pozorování}
Každé nakreslení lze rozšířit na buňkové přidáním hran bez změny počtu stěn.  
Platí tedy:
\begin{align}
	|E_G| \leq |V_G| + |F_G| + g - 2
\end{align}

\paragraph{Věta} (Zobecněná Eulerova formule)
Jestliže $G$ je nakreslený na ploše rodu $g$, nemá smyčky ani násobné hrany a 
$|V_G| \geq 3$, pak $|E_G| \leq 3|V_G| + 3g - 6$.
\paragraph{Důkaz}
Můžeme předpokládat, že $G$ nemá izolované vrcholy. Pak má každá stěna délku 
alespoň $3$ a $2|E_G| \geq 3|F_G|$. Odvodíme z předešlého pozorování:
\begin{align}
	|E_G| \leq |V_G| + |F_G| + g - 2 \leq |V_G| + \frac{2}{3}|E_G| + g - 2
\end{align}
Což po upravení dává žádanou nerovnost. \qed

\paragraph{Věta} Jestliže je graf $G$ nakreslený na ploše rodu $g$ a nemá smyčky 
ani násobné hrany, pak je jeho průměrný stupeň nejvýše: \begin{align}6 + 
\frac{6g-12}{|V_G|}\end{align}
\paragraph{Důkaz}
Použijeme zobecněnou eulerovu formuli k sečtení stupňů:
\begin{align}
	\sum_{v\in V_G} \deg(v) = 2|E_G| \leq 6|V_G| + 6g - 12
\end{align}
Což po vydělení počtem vrcholů dává žádanou rovnost.
\paragraph{Důsledek}
Jestliže je graf $G$ nakreslený n a ploše rodu $g$ a nemá smyčky ani násobné 
hrany, pak
\begin{align}
	\delta(G) \leq 6 + \frac{6g-12}{|V_G|}
\end{align}

\section{Barevnost grafů}
\subsection{Barevnost na plochách}
\paragraph{Věta} (Headwoodova formule)
Nechť $G$ je nakreslený na ploše rodu $g > 0$, pak:
\begin{align}
	\chi(G) \leq \left\lfloor \frac{7 + \sqrt{1 + 24g}}{2}\right\rfloor
\end{align}
\paragraph{Důkaz} Pro $\chi(G) \leq 6$ nerovnost platí vždy. Dokážeme pro 
$\chi(G) \geq 7$: Nechť $G$ je nejmenší graf s $\chi(G) = c \leq 7$ nakreslený 
na ploše rodu $g > 0$.  Protože $c - 7 \geq 0$ a $|V_G| \geq c$, vynásobíme a 
získáme:
\begin{align}
	c(c-7) \leq (c-7)|V_G|
\end{align}
Navíc víme, že $\delta(G) \geq c-1$ (jinak odebereme vrchol s minimálním stupněm 
a dostaneme menší graf stejné barevnosti, což by byl spor s minimalitou).  
Aplikujeme tedy tento odhad dále a vyřešením kvadratické rovnice získáme:
\begin{align}
	(c-7)|V_G| = (c-1-6)|V_G| \leq (\delta(g)-6)|V_G| = 6g-12 \\
	c(c-7) \leq 6g-12 \\
	c^2 - 7c - 6g+12 \leq 0 \\
	c \leq \frac{7+\sqrt{1+24g}}{2}
\end{align}

\paragraph{Věta} (Ringel, Youngs) Headwoodova formule je těsné pro všechny 
plochy kromě Kleinovy láhve.
\paragraph{Důkaz} (bez důkazu)

\subsection{Vrcholová barevnost}
\paragraph{Lemma 1}
Je-li graf $G$ souvislý a ne všechny jeho vrcholy mají stejný stupeň, pak 
$\chi(G) \leq \Delta(G)$.
\paragraph{Důkaz}
Zvolme kostru $T$ grafu a zakořeňme ji ve vrcholu stupně nejvýše $\Delta(G)-1$.  
Kostru pak obarvíme od listů (v opačném pořadí DFS průchodu): všechny vrcholy 
mimo kořene pak mají nejvýše $\deg(u) -1$ předbarvených sousedů a tak je lze 
obarvit. Kořen jsme ale určili tak, aby měl nejvýše $\Delta(G)-1$ sousedů, takže 
ho lze také obarvit. \qed

\paragraph{Lemma 2}
Je-li $G$ 3-souvislý a není úplný, pak $\chi(G) \leq \Delta(G)$.
\paragraph{Důkaz}
Zvolme $u,v,w$ takové, že $uv, vw\in E_G$, ale $uw \notin E_G$. Obarvíme $u$ a 
$w$ stejnou barvou, zakořeníme kostru ve $v$ a opakujeme argument z předchozího 
lemmatu.

\paragraph{Věta} (Brooksova) Je-li $G$ souvislý a není ani úplný ani lichá 
kružnice, pak $\chi(G) \leq \Delta(G)$.
\paragraph{Důkaz} Bez újmy na obecnosti je graf $\Delta$-regulární a není 
3-souvislý (oba případy umíme řešit podle předchozích lemmat). Také $\Delta > 
3$, protože jinak by šlo o sudý cyklus nebo cestu a věta platí triviálně.

Dokážeme indukcí podle počtu vrcholů (například pro $n=4$ umíme obarvit pomocí 
Lemma 1).

Protože $G$ není 3-souvislý, existuje v něm řez velikosti nanejvýš $2$. Mějme 
řez $S$ nejmenší takový:
\begin{enumerate}
	\item $|S| = 1$, pak graf rozdělíme na dvě poloviny (vrchol z řezu necháme v 
	obou) a jednotlivé podgrafy obarvíme. Následně přepermujeme obarvení tak, 
	aby zdvojený vrchol mě v obou podgrafech stejné obarvení a graf slepíme 
	zpátky.
	\item $|S| = 2$, označme $\{u,v\} = S$. Rozdělíme graf opět na dvě části 
	$G_1 + uv$ a $G_2 + uv$ a indukčně obarvíme pomocí lemmat, protože nejsou 
	$\Delta$-regulární (pokud by byly, znamenalo by to, že jeden z vrcholů $u$, 
	$v$ má svého jediného souseda $v'$ v jednom z grafů $G_1$ a $G_2$ -- v tom 
	případě můžeme uvažovat řez $\{u,v'\}$).  Protože jsou vrcholy spojené 
	hranou, nemají stejnou barvu a můžeme tedy barvy přepermutovat, aby měly $u$ 
	a $v$ stejnou barvu.
\end{enumerate}
Na závěr si rozmyslíme, že pokud by se graf měl rozpadnout na více komponent, 
tak to nevadí -- pouze obarvíme každou komponentu zvlášť a následně všechny 
přepermutujeme.

\subsection{Hranová barevnost}
\paragraph{Poznámka} V této kapitole uvažujeme pouze grafy bez násobných hran.

\paragraph{Konvence} Hranovou barevnost grafu $G$ značíme $\chi'(G)$ a myslíme 
tím nejmenší počet barev, kterými umíme dobře obarvit hrany $G$ (tedy tak, aby 
žádný vrchol nebyl incidentní s dvěmi hranami stejné barvy).

\paragraph{Věta} (Vizingova) Nechť $G$ je graf bez smyček a násobných hran, pak 
$\chi'(G) \leq \Delta(G)+1$.
\paragraph{Důkaz} Dokážeme indukcí podle poču hran: nechť $e=xy$ hrana $G$.  
Označme $\varphi$ dobré obarvení hran $G-e$ a pro každ vrchol $v$ barvy, které 
jsou u něj použité $C(v)$ a nepoužité $N(v)$. Z předpokladů je zřejmé, že pro 
každý vrchol máme alespoň jednu nepoužitou barvu. Pokud $N(x) \cap N(Y) \neq 
\emptyset$, hranu můžeme obarvit a věta platí. Co když ale ne?

\begin{enumerate}
	\item Prvně se podíváme na sousedy vrcholu $y$, nechť to jsou $z_i$. Pokud 
	pro nějakého z nich $N(y) \cap N(z_i) \neq \emptyset$ a daná hrana je 
	obarvená barvou v $N(x)$, hranu $yz_i$ přebarvíme a získali jsme volnou 
	barvu pro $xy$.
	\item Pokud nalezneme vrchol $z_i$ takový, že má jednu volnou barvu stejnou 
	jako $y$, můžeme přebarvit hranu $yz_i$, čímž si uvolníme jinou barvu v $y$.  
	Takto postupujeme dokud neuvolníme barvu vhodnou pro obarvení $xy$.
	\item Pokud takový vrchol není, najdeme všechny vrcholy $z_i$, které mají 
	stejnou volou barvu jako $x$. Nechť $P$ je střídavá dvoubarevná cesta z hran 
	$N(x)\cap N(z_i)$ a $N(y)$. Taková cesta buď končí v $x$, $y$, nebo někde 
	jinde.  Pokud někde jinde, na cestě můžeme převrátit barvy a obarvit jako v 
	případe 2.  Jinak:
	\begin{enumerate}
		\item $P$ končí v $x$. V tom případě můžeme převrátit barvy na cestě a 
		obarvit hranu $xy$ nově uvolněnou barvou.
		\item $P$ končí v $y$. V tom případě cestu také přebarvíme a hranu $xy$ 
		obarvíme barvou volnou v $x$.
	\end{enumerate}
\end{enumerate}
\begin{figure}[H]
\centering
\includegraphics{barevnost-vizing.pdf}
\caption{Přebarvování cest (čárkovanou hranu chceme přebarvit, vrchol má barvu 
jedné z volných barev)}
\label{barevnost:vizing}
\end{figure}



\section{Zakázané minory a Kuratowského věta}
V této kapitole ukážeme obdobu Kuratowského věty s minory a ekvivalenci s 
klasickým zněním. Dříve než se k tomu dostaneme, zavedeme pojem minoru a 
dokážeme několik pomocných vět.

\paragraph{Definice} Graf $H$ je \textit{minorem} grafu $G$ právě tehdy, když ho 
lze získat z grafu $G$ posloupností operací odebírání vrcholů, hran a 
kontrakcemí. Pokud navíc byly všechny kontrakce topologické (tj. alespoň jeden 
vrchol každé kontrahované hrany měl stupeň nanejvýš 2), mluvíme o 
\textit{topologickém} minoru.

\paragraph{Pozorování} V posloupnosti operací tvořících minor můžeme 
předpokládat, že kontrakce byly poslední.

\paragraph{Tvrzení} Graf obsahuje $K_{3,3}$ jako podrozdělení právě tehdy, když 
obsahuje $K_{3,3}$ jako minor.
\paragraph{Důkaz}
Pokud $G$ obsahuje podrozdělení $K_{3,3}$, jistě ho obsahuje i jako minor.  
Ukážeme tedy implikaci opačnou, tedy pokud obsahuje minor, obsahuje i 
podrozdělení. Nechť $G'$ je graf vzniklý z $G$ po odebrání všech hran a vrcholů.  
Protože podrozdělení je opak topologické kontrakce, pro spor předpokládejme, že 
je potřeba udělat netopologickou kontrakci -- každou netopologickou kontrakcí 
ale vznikne vrchol stupně alespoň 4, abychom ale získali $K_{3,3}$ jako minor, 
museli bychom ještě jednu hranu odstranit -- to je ale ve sporu s tím, že jsme 
mazání hran prováděli před kontrakcemi. \qed

\begin{wrapfigure}{r}{0.2\textwidth}
\centering
\includegraphics{minor-k5.pdf}
\caption{Minor $K_5$ před poslední kontrakcí}
\label{minory:k5}
\end{wrapfigure}
\paragraph{Tvrzení} Graf obsahuje $K_5$ jako minor, pak obsahuje podrozdělení 
$K_5$ nebo $K_{3,3}$.
\paragraph{Důkaz}
Podívejme se opět na posloupnost operací vedoucí ke $K_5$ minoru. Buď není 
potřeba žádných netopologických kontrakcí a věta je dokázána, nebo jich potřeba 
je. Předpokládejme tedy, že netopologická kontrakce je jako poslední a podívejme 
se, jak vypadá -- bůno je to jako na obrázku \ref{minory:k5} (protože je to 
skoro úplný graf, ostatní možnosti jsou k této izomorfní).Je zjevné, že 
odstraněním čárkovaných hran získáme $K_{3,3}$ jako minor a tedy podle 
předchozího tvrzení i jako podrozdělení.


\paragraph{Věta} (Tutte) Nechť $G$ je 3-souvislý a není $K_4$, pak existuje 
hrana $e\in E_G$, že $G/e$ je 3-souvislý.
\paragraph{Důkaz}
Dokážeme sporem. Nechť neexistuje hrana $e$, kterou lze kontrahovat, aby zůstal 
graf 3-souvislý. Tedy pro každou hranu $xy\in G$ má $G/xy$ separátor velikosti 
nanejvýš $2$. Protože je $G$ 3-souvislý, musí tento separátor být velikosti 
právě dva a odpovídá mu nějaký separátor v $G$ a to takový obsahující hranu 
$xy$. Nechť tedy $\{v_{xy},z\}$ je separátor v $G/xy$ odpovídající separátoru $
T \{x,y,z\}$ v $G$. Protože žádná podmnožina $T$ neseparuje $G$, každý vrchol má 
hranu do každé komponenty souvislosti C grafu $G\setminus T$. Zvolme hranu $xy$,
vrchol $z$ a komponentu $C$ takové, aby byla $|C|$ nejmenší ze všech možných.  
Označme $zv$ nějakou hranu do C. Z předpokladů víme, že $G/zv$ není 3-souvislá, 
takže existuje $w$ takové, že $\{z,v,w\}$ separuje $G$ a opět každý vrchol z 
této množiny sousedí se všemi komponentami grafu $G\setminus\{z,v,w\}$. Navíc 
protože $xy$ spolu sousedí, existuje kompomenta $D$, která neobsahuje $xy$.  
Navíc z komponenty $D$ vede hrana do $v$, ale protože $v$ bylo v $C$ a $z$ není 
v $D$, musí být $D\subset D$, což je spor s volbou $C$.
\qed
\begin{figure}[H]
\centering
\includegraphics{minor-3souvislost.pdf}
\caption{Množiny $C$ a $D$ a jejich separátory s vyznačeným sousedstvím}
\label{minory:kontrakce}
\end{figure}

\paragraph{Věta} (Kuratowski, Wagner) Graf je rovinný právě tehdy, když 
neobsahuje $K_5$ ani $K_{3,3}$ jako (topologický) minor.
\paragraph{Důkaz}
Postupujeme indukcí dle velikosti grafu. Pro $n=4$ věta platí, protože $K_4$ je 
rovinný. Dále pro $n>4$:
\begin{enumerate}
	\item Graf je 1-souvislý, tedy obsahuje 1-řez: rozdělíme graf na dvě části v 
	tomto řezu a aplikujeme indukci.
	\item Graf je 2-souvislý, tedy obsahuje 2-řez: rozdělíme graf na dvě části, 
	vrcholy řezu v obou částech spojíme hranou a aplikujeme indukci.
	\item Graf je 3-souvislý. Pak použijeme Tutteho větu o 3-souvislých grafech 
	a aplikujeme indukci. Zde si ale musíme dát pozor, protože kontrakce obecně 
	nezachovává rovinnost. Ukážeme, že kdyby kontrakce rovinnost nezachovala, 
	graf obsahuje minor $K_5$ nebo $K_{3,3}$: Zjevně nás zajímají pouze 
	kontrakce, kde vrcholy mají 3 nebo 4 hrany (méně nemůže nastat, protože je 
	graf 3-souvislý a pro více hran stačí uvažovat ty "špatné"). Pro 
	zjednodušení si představme, že zbytek grafu se kontrahoval do kružnice 
	(resp. dejme tomu, že je rovinný a díváme se jenom na stěnu, která sousedí 
	se všemi vrcholy, do kterých chceme vést hrany). Není těžké si rozmyslet, že 
	záleží právě na pořadí hran na kružnici a mohou nastat následující dva 
	případy (tedy kromě toho, že je vše v pořádku), jako na obrázku 
	\ref{minory:kontrakce}, kde je zřejmě $K_5$, resp. $K_{3,3}$ jako minor a 
	tedy podle předchozích tvrzení také podrozdělení.
\begin{figure}[H]
\centering
\includegraphics{minor-kontrakce.pdf}
\caption{Minory $K_{3,3}$ a $K_{5}$ před kontrakcí}
\label{minory:kontrakce}
\end{figure}
	\item Graf je více-souvislý. Vezměme tedy řez velikosti alespoň $4$ Ten nám 
	rozdělí graf na alespoň $2$ komponenty souvislosti, my ale víme, že 
	vypuštěním libovolného vrcholu z řezu by graf zůstal souvislý -- tedy každý 
	z vrcholů má hranu do každé komponenty souvislosti. Zároveň obě komponenty 
	mají dohromady alespoň $4$ vrcholy, nebo mají vrcholy v řezu mezi sebou 
	hrany. V obou případech několika kontrakcemi získáme $K_5$ nebo $K_{3,3}$ 
	jako minor.
\end{enumerate}
\qed

\section{Perfektní grafy}
\paragraph{Definice}
Graf $G$ je \textit{perfektní}  právě tehdy, když každý jeho indukovaný podgraf 
$H$ má $\chi(H) = \omega(H)$.

\paragraph{Věta} (Silná věta o perfektních grafech) Graf $G$ je perfektní právě 
tehdy, když $G$ ani $\overline{G}$ neobsahují lichý cyklus délky alespoň 5 jako 
indukovaný podgraf.
\paragraph{Důkaz} (bez důkazu)

\paragraph{Věta} Graf $G$ je perfektní právě tehdy, když pro všechny indukované 
podgrafy $H$ grafu $G$ platí:
\begin{align}
	\label{perfgr:slaba-nerovnost} |V_H| \leq \alpha(H) \cdot \omega(H)
\end{align}
\paragraph{Důkaz}
Pokud je graf perfektní, nerovnost zjevně platí -- $\omega(H)$ je počet 
barevnostních tříd. Každá třída je ale nezávislá množina a proto v jedné může 
být nanejvýš $\alpha(H)$ prvků. Součin tedy vskutku dává horní odhad na počet 
vrcholů.

Stačí tedy dokázat, že platnost dané nerovnosti je postačující. Postupujme 
indukcí podle velikosti grafu. Předpokládejme tedy, že všechny indukované 
podgrafy $H$ grafu $G$ jsou perfektní, ale $G$ perfektní není. Rozlišme dva 
případy:
\begin{enumerate}
	\item $\exists$ nezávislá množina $A \subseteq V_G: \omega(G\setminus A) < 
	\omega(G)$. Odebráním nezávislé množiny ale snížíme barevnost nanejvýš o 1:
	\begin{align}
		\chi(G) \leq \chi(G\setminus A) + 1 = \omega(G \setminus A) + 1
	\end{align}
	Protože jsme ale vzali nezávislou množinu takovou, že zmenšila klikovost 
	alespoň o jedna, určitě také platí:
	\begin{align}
		\omega(G \setminus A) \leq \omega(G) -1
	\end{align}
	Čímž získáme spolu s triviálním odhadem $\chi(G) \geq \omega(G)$, že $G$ je 
	perfektní, což je spor:
	\begin{align}
		\omega(G) \leq \chi(G) \leq \omega(G \setminus A) + 1 \leq \omega(G) -1 
		+ 1
	\end{align}
	\item $\forall$ nezávislou množinu $A \subseteq V_G: \omega(G\setminus A) = 
	\omega(G)$, tedy $\forall A \exists K$ klika velikosti $\omega(G)$, že $A 
	\cap K = \emptyset$. Označme si:
	\begin{align}
		A_0 = \{v_1, v_2, \dots, v_{\alpha(G)}\} \quad \dots \quad & 
		\text{libovolnou nezávislou  množinu velikosti } \alpha(G)\\
		A_1, A_2, \dots, A_{\omega(G)} \quad \dots \quad & \text{třídy obarvení 
		} G-v_1 \\
		A_{\omega(G) + 1}, A_{\omega(G) + 2}, \dots, A_{2\omega(G)} \quad \dots 
		\quad & \text{třídy obarvení } G-v_2 \\
		\vdots\qquad\qquad\qquad&\qquad\vdots \\
		A_{\alpha(G)-1)\omega(G) + 1}, A_{(\alpha(G)-1)\omega(G)+2}, \dots, 
		A_{\alpha(G)\omega(G)} \quad \dots \quad & \text{třídy obarvení } G- 
		v_{\alpha(G)}
	\end{align}
	Dále podle předpokladů pro každou nezávislou množinu $A_i$ existuje klika 
	$K_i$ velikosti $\omega(G)$, která je s ní disjunktní. Nahlédneme, že každý 
	vrchol z $K_i$ patří do nějaké barevnostní třídy pro každý graf $G - v_j$, 
	tedy až na jednu třídu z grafu $G - v$ pro nějaké $v \in K_i$. Máme tedy 
	$\alpha(G)\omega(G)+1$ množin v $G$.

	Nechť $A$ je matice řádu $(\alpha(G)\omega(G) + 1)\times n$ a její i-tý 
	řádek nechť je chrakteristický vektor množiny $A_i$. Nechť $B$ je matice 
	řádu $n \times (\alpha(G)\omega(G) + 1)$ a její i-tý sloupec nechť je 
	charakteristický vektor $K_i$.

	Protože ale jsme si množiny $A_i$ a $K_j$ nadefinovali tak, že mají prázdný 
	průnik právě tehdy, když jsou $i=j$, tak matice $AB$ bude regulární. To 
	speciálně zanmená, že matice $A$ je regulární (protože pro dvě matice $A$, 
	$B$ platí $\rank(AB) \leq \min(\rank A, \rank B)$) a má rank 
	$\alpha(G)\omega(G) + 1$.  To ale speciálně znamená, že $|V_G| \geq 
	\alpha(G)\omega(G)+1$, což je ale spor s předpokladem 
	(\ref{perfgr:slaba-nerovnost}).
\end{enumerate}
\qed


\paragraph{Důsledek} (Slabá věta o perfektních grafech) Graf $G$ je perfektní, 
právě tehdy když je jeho doplněk perfektní.
\paragraph{Důkaz} Odvodíme z definice za vědomí, že $\omega(H) = 
\alpha(\overline{H})$:
\begin{align}
	|V_H| \leq \alpha(H)\omega(H)\\
	|V_H| \leq \omega(\overline H)\alpha(\overline H)
\end{align}
Což je ale to samé, jako bychom brali indukované podgrafy $\overline H$, čímž je 
věta dokázána.
\qed

Nyní máme celkem dobrou charakteristiku perfektních grafů. Jaké grafy to ale 
vlastně jsou?

\subsection{Třídy perfektních grafů}
Snadno můžeme nahlédnout, že \textit{bipartitní grafy} jsou perfektní (pokud 
mají alespoň jednu hranu, mají největší kliku velikosti 2 -- jinak 1 a to 
odpovídá jejich barevnost). Díky slabé větě také víme, že jejich doplněk je 
prefektní.

Také lze dokázat, že intervalové grafy jsou perfektní. My se ale podíváme na 
jednu třídu grafů a tou jsou cografy.

\paragraph{Definice} Graf $G$ je \textit{cograf} právě tehdy když:
\begin{enumerate}
	\item $G$ je izolovaný vrchol, nebo
	\item $G$ je disjunktní sjednocení cografů, nebo
	\item $G$ je úplné spojení cografů.
\end{enumerate}
\paragraph{Věta} Každý cograf $G$ je perfektní.
\paragraph{Důkaz}
Dokážeme indukcí podle počtu vrcholů: pro $G$ izolovaný vrchol platí triviálně.  
Mějme tedy $G$ vytvořený z (BÚNO) dvou grafů $G_1$ a $G_2$. Rozeberme zvlášť oba 
případy spojení:
\begin{enumerate}
	\item Disjunktní sjednocení -- pak výsledný graf není souvislý. Samozřejmě 
	platí rovnosti: 
	\begin{align}
	\omega(G) &= \max\{\omega(G_1),\omega(G_2)\} \\
	\chi(G) &= \max\{\chi(G_1),\chi(G_2)\}
	\end{align}
	Protože ale jsou ale $G_1$ i $G_2$ perfektní, platí rovnost pravých stran a 
	tedy i levých, tudíž je graf $G$ perfektní. Je snadné si rozmyslet, že 
	stejný vztah platí i pro libovolný podgraf.
	\item Úplné spojení -- obdobně jako v předchozím případě, nicméně místo 
	maxima bude součet.
\end{enumerate}
\qed

\section{Tutteho polynom}
V této kapitole nejdříve zavedeme pojem rankového polynomu a z něj odvodíme 
Tutteho polynom. Ten následně zobecníme univerzálním polynomem, to vše za pomocí 
jednoduchého předpisu pro rekurzivní výpočet.
\subsection{Rankový polynom}
\paragraph{Definice}
Pro graf $G$ značíme:
\begin{enumerate}
	\item $k(G)$ počet komponent grafu
	\item $r(G) = |V_G| - k(G)$ rank grafu
	\item $n(G) = |E_G| - |V_G| + k(G)$
\end{enumerate}
\paragraph{Konvence} V následující kapitole budeme značně využívat podgrafy 
vzniklé pouze odebíráním hran. Definujme tedy pro graf $G=(V,E)$ a $F\subseteq 
E$ hranový podgraf jako $\langle F \rangle = (V, F)$ a pro zjednodušení pišme 
počet komponent, rank a nulitu tohoto grafu jako $k\langle F\rangle$, $r\langle 
F\rangle$, $n\langle F\rangle$.
\paragraph{Definice} (Rankový polynom)
Pro nějaký graf $G=(V,E)$ definujeme \textit{rankový polynom} S(G, x, y) 
následovně:
\begin{align}
	S(G, x, y ) = \sum_{F \subset E} x^{r\langle E\rangle - r\langle F \rangle} 
	y^{n\langle F\rangle} = \sum_{F \subset E} x^{k\langle F\rangle - k\langle E 
	\rangle} y^{n\langle F\rangle}
\end{align}

Což není zrovna pohodlné pro výpočet. Podívejme se, jak to lze udělat 
rekurzivně:

\paragraph{Věta} Nechť $G=(V,E)$ je graf a $e\in E$ jeho hrana. Potom:
\begin{align}
	S(G, x, y) = \left\{ \begin{array}{ll}
	(x+1) S(G - e, x, y) & e \text{ je most} \\
	(y+1) S(G - e, x, y) & e \text{ je smyčka}\\
	S(G-e, x, y) + S(G/e, x, y) & \text{jinak}
	\end{array}
	\right.
\end{align}
\paragraph{Důkaz}
Důkaz provedeme pro případ, že $e$ je most -- ostatní případy jsou analogické.  
Vycházejme ze vzorce s komponentami. Podívejme se na to, jak se změní suma podle 
definice, pokud odebereme most -- rozdělíme na případy, kdy hrana patřila do 
$F$a kdy ne:
\begin{enumerate}
	\item pro $e\in F$: počet komponent v $E$ se zvětšil o 1, stejnětak se o 1 
	zvětšil počet komponent o $F$. Nulita zůstane nezměněná (zvýšil se počet 
	komponent, ale zato snížil počet hran). Suma se tedy nezmění.
	\item pro $e \notin F$: počet komponent v $E$ se zvětšil o 1, ale počet 
	komponent v $F$ se nezměnil. Exponent $x$ se tedy snížil o 1. Nulita se 
	nezmění, protože se hrana $e$ formule neúčastní.
\end{enumerate}
Následně složíme sumy a zjistíme, že věta platí. \qed

\subsection{Tutteho polynom}
\paragraph{Definice} (Tutteho polynom) Definujme \textit{Tutteho polynom}
$T(G, x,y)$ roven $S(G, x-1, y-1)$
\paragraph{Tvrzení} Nechť $G=(V,E)$ je graf a $e\in E$ jeho hrana. Potom:
\begin{align}
	T(G, x, y) = \left\{ \begin{array}{ll}
	x T(G - e, x, y) & e \text{ je most} \\
	y T(G - e, x, y) & e \text{ je smyčka}\\
	T(G-e, x, y) + T(G/e, x, y) & \text{jinak}
	\end{array}
	\right.
\end{align}
\paragraph{Důkaz} Tvrzení plyne triviálně z vlastností rankového polynomu. \qed

Celkem přirozené rozšíření by bylo přidat do rekurzivního vzorce vhodné 
konstanty, podívejme se tedy na jedno takové rozšíření:

\subsection{Univerzální polynom}
\paragraph{Věta} (Univerzální polynom)
Existuje právě jedna prostá funkce $U: G \to \Z[x,y,\alpha,\sigma,\tau]$ taková, 
že (kde $E_n$ je prázdný graf na $n$ vrcholech a psaním $U(G)$ myslíme $U(G, x, 
y, \alpha, \sigma, \tau)$):
\begin{align}
	\label{polynomy:univerzalni:prazdny-graf} &\forall n \geq 0 \quad U(E_n) = 
	U(E_n, x, y, \alpha, \sigma, \tau) = \alpha^n \\
	&U(G) = \left\{ \begin{array}{ll}
	\alpha x U(G - e) & e \text{ je most} \\
	\alpha y U(G - e) & e \text{ je smyčka}\\
	\sigma U(G-e) + \tau U(G/e) & \text{jinak}
	\end{array} \right.
\end{align}
a navíc:
\begin{align}
	\label{polynomy:univerzalni:tutte} U(G) = 
	\alpha^{k(G)}\sigma^{n(G)}\tau^{r(G)} T\left(G, \frac{\alpha x}{\tau}, 
	\frac{y}{\sigma}\right)
\end{align}
\paragraph{Důkaz} Jedinečnost je zřejmá z rekurzivní definice. Je však potřeba 
dokázat potřebné vlastnosti. Nejprve dokážeme 
(\ref{polynomy:univerzalni:prazdny-graf}). Pro prázdný graf zjevně platí: 
$k(E_n) = n$ a $r(E_n) = n(E_n) = 0$. Podle (\ref{polynomy:univerzalni:tutte}) a 
definice Tutteho polynomu pomocí rankového získáme:
\begin{align}
	U(E_n) = \alpha^n T\left(E_n, \frac{\alpha x}{\tau}, \frac{y}{\sigma}\right) 
	= \alpha^n
\end{align}
Samotný vztah (\ref{polynomy:univerzalni:tutte}) bychom získali pomocí porovnání 
rozhodovacích stromů rekurzivního výpočtu Tutteho a Univerzálního polynomu.  
Důkaz je ale značně technický a proto ho necháme čtenáři na rozmyšlení (čti: 
nechce se mi ho sepisovat, opravdu -- zájemci ho najdou v B. Bollobásovi, strana 
340). \qed

\paragraph{Pozorování}
Je vidět, že $T(G, x, y) = U(G, x, y, 1, 1, 1)$.

\subsection{Specializace grafových polynomů}
\subsubsection{Počet koster}
\paragraph{Definice}
Označme $t(G)$ jako počet koster grafu $G$. Pak celkem přirozeně můžeme 
definovat:
\begin{align}
	&t(E_n) = 1 \\
	&t(G) = \left\{ \begin{array}{ll}
	t(G - e) = t(G/e)& e \text{ je most (patří do všech koster)} \\
	t(G - e) = t(G/e)& e \text{ je smyčka (nepatří do žádné kostry)}\\
	t(G-e) + t(G/e) & \text{jinak (počet koster bez a s hranou $e$)}
	\end{array}
	\right.
\end{align}
\paragraph{Pozorování} Je vidět, že $t(G) = T(G, 1, 1)$


\subsubsection{Chromatická funkce}
\paragraph{Definice} Označme $c_G(k)$ počet obarvení $G$ pomocí $k$ barev. Pak 
definujme:
\begin{align}
	&c_{E_n}(k) = k^n \\
	&c(G) = \left\{ \begin{array}{ll}
	c_{G-e}(k) \cdot\frac{k-1}{k}& e \text{ je most} \\
	0 & e \text{ je smyčka}\\
	c_{G-e}(k) - c_{G/e}(k) & \text{jinak}
	\end{array}
	\right.
\end{align}
Jinými slovy pokud je $e$ most, započítáme všechna obarvení grafu bez $e$, ale 
$\frac{1}{k}$ z nich by musely přiřazovat oboum koncům mostu stejnou hranu -- ty 
nepočítáme.  Pokud je $e$ smyčka, tak zjevně nemá graf žádné dobré obarvení a 
ostatních případech spočítáme počet obarvení bez hrany a odečteme počet obarvení 
s kontrahovanou hranou (tj. oba konce hrany dostaly stejnou barvu).

\paragraph{Tvrzení} $c_G(k) = k^{k(G)} \cdot (-1)^{\chi(G)} \cdot T(G, 1-k, 0)$.
\paragraph{Důkaz} Snadno ověříme dosazením z rekurentní formule do vztahu 
(\ref{polynomy:univerzalni:tutte}). \qed

\subsubsection{Počet acyklických orientací}
\paragraph{Definice} Označme $a(G)$ počet acyklických orientací $G$. Pak 
rekurzivně definujme:
\begin{align}
	&a(E_n) = 1 \\
	&a(G) = \left\{ \begin{array}{ll}
	2a(G-c) & e \text{ je most (lze zorientovat libovolně)} \\
	0 & e \text{ je smyčka (nelze zorientovat)}\\
	a(G-e) + a(G/e) & \text{jinak (pokud existuje a neexistuje jiná orineotvaná 
	cesta směrem $e$)}
	\end{array}
	\right.
\end{align}
\paragraph{Pozorování} Opět je vidět, že $a(G) = T(G, 2, 0)$.

\bigskip

\noindent To je bohužel skoro vyčerpávající seznam užitečných věcí, které 
Tutteno polynom počítá.

\section{Hamiltonovské kružnice}
\paragraph{Definice} Hamiltonovská kružnice grafu je uzavřená cesta procházející 
každým vrcholem.

Nyní ukážeme několik kritérií existence hamiltonovské kružnice v grafu, 
konkrétně Direcovu a Oreho větu a obě dokážeme pomocí silnějšího tvrzení, 
Chvátalova uzávěru.

\paragraph{Věta} (Dirac) Nechť $G$ je graf, potom pokud $\delta(G) \geq 
|V_G|/2$, tak $G$ má hamiltonovskou kružnici.

\paragraph{Věta} (Ore) Nechť $G$ je graf a platí:
\begin{align}
	\forall u,v \in V_G: \quad uv\notin E_G \impl d(u) + d(v) \geq |V_G|
\end{align}
Pak má graf $G$ hamiltonovskou kružnici.

\paragraph{Důkaz} (Dirac) Dokážeme použitím Oreho věty ($n/2 + n/2 \geq n$).  
\qed

\paragraph{Věta} (Chvátalův uzávěr) Nechť $G$ je graf a platí>
\begin{align}
	\forall u,v \in V_G: \quad uv\notin E_G \impl d(u) + d(v) \geq |V_G|
\end{align}
Potom má $G$ hamiltonovskou kružici právě tehdy, když $G+uv$ má hamiltonovskou 
kružnici.


\begin{wrapfigure}{r}{0.3\textwidth}
\centering
\includegraphics{hamiltonovky-chvatal.pdf}
\caption{Nahrazení hrany $uv$}
\label{hamiltonovky:chvatal}
\end{wrapfigure}

\paragraph{Důkaz} (Ore) Protože má Oreho věta stejné předpoklady, jako Chvátalův 
uzávěr, stačí nahlédnout, že budeme aplikovat ,,uzávěr``, dokud nezískáme úplný 
graf, který určitě hamiltonovskou kružnici má. \qed

\paragraph{Důkaz} (Chvátal) Pokud $G$ má hamiltonovskou kružnici, tak ji má 
určitě i $G+uv$. Zbývá dokázat opačnou implikaci.  

Nechť tedy $G+uv$ má hamiltonovskou kružnici a $G$ ji nemá. Označme si takovou 
kružnici $C$. Ta určitě používá hranu $uv$, označme tedy $P=C-uv$, což je cesta 
z $u,v$ přes všehchny vrcholy grafu. Protože $P$ má $n-2$ vnitřních vrcholů a 
$u$ s $v$ mají dohromady alespoň $n$ vrcholů, musí mít z principu holubníku 
alespoň dva společné sousedy. Stačí si rozmyslet, že díky tomu existují dva 
sousedá $u'$ a $v'$ vrcholů $u$ a $v$ takoví, že $u'v'\in E$ a navíc 
$P=u...v'u'...v$ (opět z principu holubníku a zdravé úvahy). Potom ale můžeme 
vytvořit následující cyklus $C'=u...v'v...u'u$ a věta je dokázána. \qed

\section{Vytvořující funkce}
\paragraph{Definice} Pro posloupnost $a_n$ definujme vytvořující funkci 
nekonečnou řadu $\sum_i a_ix^i$, kde $a_i$ jsou koeficienty posloupnosti a řada 
konverguje pro nějaké $x$.

\paragraph{Definice} Vytvořující funkce v exponencielním tvaru je:
\begin{align}
	f(x) = \sum_{n=0}^\infty \frac{a_n}{n!}x^n
\end{align}

\section{Pólyova enumerace a Burnsideovo lemma}
\paragraph{Definice} Nechť $G$ je grupa působící na množinu $X$. Pak označme:
\begin{align}
	[x] &= \{y \in X: \exists \alpha \in G \quad \alpha x = y\} 
	\quad&\quad\text{orbita $x$} \\
	G(x,y) &= \{\alpha \in G: \alpha x = y \} & \text{množina prvků 
	zobrazujících $x$ na $y$}\\
	G(x) &= G(x,x)  \quad&\quad\text{stabilizátor $x$}\\
	F(\alpha) &= \{x \in X: \alpha x = x \}  \quad&\quad\text{množina 
	pevných bodů $\alpha$}
\end{align}

\paragraph{Věta} (Pólyova enumerace) Nechť $G$ je grupa působící na množinu $X$, 
$A$ je libovolná Abelovská grupa a $w: X \to A$ je váhová funkce. A označme:
\begin{align}
	&O_1, \dots, O_l & \text{orbity $G$} \\
	&w(O_i) = w_x, \quad x \in O_i & \text{váhy orbit}
\end{align}
Potom:
\begin{align}
	|G| \sum_{i=1}^l w(O_i) = \sum_{\alpha\in G} \sum_{x \in F(\alpha)} w(x)
\end{align}
\paragraph{Pozorování} (Bursideovo lemma) Zjednodušená verze pro $A=\Z$ a 
$w(x)=1$ pak vypadá:
\begin{align}
	N(G) = \frac{1}{|G|} \sum_{\alpha \in G} \sum_{x \in F(\alpha)} 1 = 
	\frac{1}{|F|} = \sum_{\alpha \in G} |F(\alpha)|
\end{align}
Což je známé Burnsideovo lemma.

\paragraph{Důkaz}
\begin{align}
	\sum_{\alpha\in G} \sum_{x\in F(\alpha)} &= \sum_{x \in X} \sum_{\alpha \in 
	G(x)} w(x) \\
	&=\sum_{i=1}^l \sum_{x \in O_i} \sum_{\alpha \in G(x)} w(x) \\
	&= \sum_{i=1}^l w(O_i) \sum_{x\in O_i} \sum_{\alpha \in G(x)} 1 \\
	&= \sum_{i=1}^l w(O_i) |O_i| = |G| \sum_{i=1^l} w(O_i)
\end{align}
Podívejme se nejdříve počítané rovnost. Začneme s pravou stranou výrazu. Nejprve 
si rozmyslíme, že pokud počítáme přes všechny permutace a jejich pevné body, je 
to to samé jako počítat přes všechny prvky a jejich stabilizátory, z čehož plyne 
první rovnost. Další rovnost je snadná -- můžeme počítat přes orbity a jejich 
prvky, protože každý prvek je v orbitě. Následně vytkneme váhu prvku před sumu 
jako váhu orbity (podle definice váhy orbity je stejná jako váha libovolného 
prvku). Následně si všimneme, že zbylé sumy prostě posčítají všechny prvky 
orbity a součet prvků orbit dá velikost celé grupy. \qed

\section{Extrémální grafy}
\subsection{Turánovy grafy}
\paragraph{Definice}
Označme číslo $\ex(n, H)$ jako maximální počet hran grafu na $n$ vrcholech, aby 
neobsahoval $H$ jako podgraf.
\paragraph{Definice}
Turánův graf $T(n, r)$ je maximální $r$-partitní graf s partitami co 
nejpodobnější velikosti (tj. po dvou se liší jejich velikost maximálně o 1).
\paragraph{Pozorování} Zjevně $T(n,r)$ neobsahuje $K_{r+1}$ jako podgraf.
\paragraph{Věta} (Turán) Nechť $r > 1$, pak:
\begin{align}\ex(n, K_{r+1}) \leq \frac{r-1}{2r} n^2\end{align}
Navíc to jsou právě Turánovy grafy.
\paragraph{Důkaz} Nejdříve ukážeme, že Turánův graf má správný počet hran.  
Rozmyslíme si, že maximálního počtu bude nabývat skutečně v případě, kdy se liší 
jednotlivé partity maximálně o 1 (protože přesunutím vrcholu z větší partity do 
menší získáme větší počet hran, pokud se ale liší pouze o 1, počet hran se 
nezmění). Ideální případ bude nabývat, když jsou si všechny partity rovny. Z 
každého vrcholu vede $n-n/r$ hran a každou hranu jsme započítali dvakrát. Tedy:
\begin{align}
	\frac{1}{2}
	n \left(n-\frac{n}{r}\right) = \frac{1}{2} n \frac{n(r-1)}{r}
\end{align}
To nám dává ale kýženou rovnost (pilný čtenář si doplní dolní celé části a 
odhadne i s nimi).

Nyní stačí ukázat, že Turánovy grafy jsou právě ty největší. Zafixujme si $r$. 
Pro spor předpokládejme, že existuje graf, který není turánovský a přitom je 
alespoň stejně tak velký.  Vezměme si největší takový a nazvěme ho $G$. Protože 
není $r$-partitní, není v něm relace ,,nebýt hranou`` ekvivalence, existuje tam 
tedy trojice vrcholů $x,y_1,y_2$ taková, že $y_1y_2 \in E_G$, ale $xy_1, xy_2 
\notin E_G$. Rozlišme dva případy:
\begin{enumerate}
	\item $\deg(y_1) > \deg(x)$, potom ale můžeme smazat vrchol $x$ a nahradit 
	ho kopií $y_1$, čímž získám větší neturánovský graf a je to spor s 
	maximalitou.
	\item $\deg(y_1), \deg(y_2) \leq \deg(x)$, potom ale můžeme smazat vrcholy 
	$y_1,y_2$ a nahradit je kopiemi $x$. Tím získáme určitě větší graf a pokud 
	není Turánovský, tak je to opět spor s maximalitou. Pokud náhodou 
	Turánovský, tak je to spor s volbou $G$ alespoň tak velkého jako příslušný 
	Turánovský graf.
\end{enumerate}
\qed


\paragraph{Věta} (Erdös, Stone) Pro každý graf $H$ s alespoň jednou hranou 
platí:
\begin{align}
	\lim_{n\to \infty} = \frac{\ex(n, H)}{\binom{n}{2}} = \frac{\chi(H) 
	-1}{\chi(H)-2}
\end{align}
\paragraph{Důkaz} (bez důkazu)

\subsection{Stupeň grafu a degenerovanost}
\paragraph{Definice} Nechť $G=(V,E)$, potom $\epsilon(G) = \frac{|E|}{|V|}$ je 
\textit{průměrný stupeň} grafu.
\paragraph{Věta}Každý graf s průměrným stupněm alespoň $2^{r-2}$ má $K_r$ minor.
\paragraph{Důkaz} Dokážeme indukcí podle $r$. Pro $r=2$ platí, protože každý 
graf průměrného $2^0$ má alespoň jednu hranu. Nechť tedy $r \geq 3$ a $G$ je 
libovolný graf průměrného stupně alespoň $2^{r-2}$. Podívejme se na jeho 
nejmenší minor $H$ takový, že $\epsilon(H) \geq 2^{r-3}$ a zvolme $x\in H$
libovolný.
Nyní si všimneme, že každý soused $y$ vrcholu $x$ má s $x$ alespoň $2^{r-3}$ 
společných sousedů -- jinak bychom kontrakcí vhodné $xy$ získali menší minor s 
dostatečným průměrným stupněm. Z toho víme, že podgraf indukovaný sousedy $x$ má 
minimální stupeň alespň $2^{r-3}$ a tedy obsahuje $K_{r-1}$ jako minor. Přidáním 
vrcholu $x$ k tomuto minoru již získáme kýžený $K_r$.

\paragraph{Věta} (Kostochka) Existuje konstanta $c\in\R$ taková, že $\forall 
r\in\N\forall G\quad \epsilon(G) \geq cr\sqrt{\log r}$ obsahuje $K_r$ jako 
minor. Tato mez je navíc ostrá až na konstantu $c$.
\paragraph{Důkaz} (bez důkazu)

\subsection{Protínající se množiny a body v rovině}

\paragraph{Věta} (Erdös, Ko, Rado) Nechť $2k \leq n$, potom každá množina 
k-prvkových podmnožin n-prvkové množiny může mít maximálně $\binom{n-1}{k-1}$ 
prvků, aby měly prvky po dvou neprázdný průnik.
\paragraph{Důkaz} (G. Katona, 1972) Nechť $M=\{0, ..., n-1\}$ je naše množina.  
Seřaďme ji do nějakého cyklického uspořádání, a to očíslujeme. Dále definujeme 
intervaly $B_s$ jako k-prvkové množiny po sobě jdoucích prvků začínajících 
prvkem $s$ a končících prvkem $s+k-1$ (a počítáme $\mod n$). Všimneme si, že ne 
všechny intervaly mohou být ve společné množině, protože jsou disjunktní.  
Označme si $S(\pi, s)$ jako dvojici označující permutaci prvků $M$ a $s$ nějaký 
prvek takový, že:
\begin{align}
	\pi(B_s) := \{ \pi(s), \pi(s+1), \dots, \pi(s+k-1)\}
\end{align}
Označme číslo $L$ jako počet takových dvojic. Pro fixní $\pi$ může obsahovat 
nanejvýš $k$ prvků z množiny, protože na fixním uspořádání množiny vzdálené víc 
jak $k$ jsou disjunktní. Tedy $L \leq k n!$. Naopak $s$ můžeme zvolit $n$ 
způsoby a pro každé takové existuje $k!(n-k)!$ možností, jak zvolit permutaci 
$\pi$, tedy každých $nk!(n-k)!$ párů sestrojí stejnou množinu. Takže $L = 
|\mathcal{F}| nk!(n-k)!$, kde $\mathcal{F}$ je kýžená množina. Zkombinujeme 
odhady a získáme:
\begin{align}
	|\mathcal F| \leq \frac{kn!}{nk!(n-k)!} = \frac{k}{n}\binom{n}{k} = 
	\binom{n-1}{k-1}
\end{align}
Čímž je věta dokázána. \qed

\paragraph{Definice} Řekneme, že množina $X$ je $m$-hrnek, pokud z ní lze udělat 
cestu dlouhou $m$ takovou, bude konvexní. Analogicky čepici je konkávní.
\paragraph{Pozorování} Rozmyslíme si, že pokud hrnek a čepice sdílejí poslední 
vrcholy, můžeme hrnek nebo čepici prodloužit (protože sdílenému vrcholu náleží 
dvě hrany, ale součet úhlů na obou polovinách nemůže být větší než $2\pi$.
\paragraph{Věta} (Erdös, Szekeres)
Označme $f(n,m)$ nejmenší přirozené číslo takové, že každá množina o $f(n,m)$ 
bodů v rovině v obecné poloze obsahuje buď $n$-čepici, nebo $m$-hrnek. Potom:
\begin{align}
	f(n,m) = \binom{n+m-4}{n-2} + 1
\end{align}
\paragraph{Důkaz} Dokážeme indukcí. Prvně si všimneme, že pro $m\leq 2$ nebo 
$n\leq 2$ je věta triviální, protože alespoň dvouprvková čepice nebo šálek 
existují vždy, pokud jsou v rovině alespoň dva body. Předpokládejme tedy, že 
$m,n\geq 3$. My budeme postupovat podle $n$ a fixního $m$, opačný postup je 
identický, jenom se zamění $m$ za $n$ a ověří, že to funguje. My se nebudeme 
opakovat.

Nechť tedy chceme dokázat větu pro nějaké $m$ a $n$ a z indukčního předpokladu 
víme, že pro $m-1$ a $n$ věta platí. Předpokládejme nyní, že je věta splněna 
tím, že existuje $m-1$ hrnek (pokud by platila čepicí, tak máme hotovo).

Najděme tedy největší hrnek a odeberme mu nejlevější vrchol. Takto postupujeme, 
dokud nám nezbyde pouze
\begin{align}
	\binom{n+(m-1)-4}{n-2} + 1
\end{align}
vrcholů. Vrcholy, které jsme odebrali nechť jsou v množině $R$. Podívejme se, 
kolik jich je (používáme vzorec pro součet kombinačních čísel a symetrii):
\begin{align}
	|R| = \binom{n+m-4}{n-2} + 1 - \binom{n+m-5}{n-2} - 1 = \binom{n+m-5}{n-3} = 
	\binom{n+m-5}{n-2}
\end{align}
Což je ale přesně tolik, kolik nám stačí na uplatnění indukce. Tedy víme, že 
mezi odebranými vrcholy existuje buď $m-1$-hrnek,nebo $n$ čepice. Podívejme se 
tedy na takovou čepici. Protože jsme oba její konce odebrali, tak určitě byly na 
konci alespoň jednoho $m-1$ hrnku. To máme ale hrnek a čepici sdílející poslední 
vrchol. Rozmyslíme si, že v tomto případě můžeme hrnek nebo čepici zvětšit a 
větu jsme dokázali. \qed


\section{Ramseyovské teorie}
\paragraph{Věta} (Ramseyova) $\forall n \exists N$ takové, že obarvením $K_N$ 
dvěma barvami dostaneme monochromatickou $K_n$ jako podgraf.
\paragraph{Důkaz} (bez důkazu, dělá se v DM)
\paragraph{Poznámka} Podobné věty mluví o obarvení více barvami a nebo jiných 
rozšířeních.

\paragraph{Věta} (Ramseyova nekonečná)
Pokud obarvíme $t$-tice přirozených čísel (z $\N^t$) $k$ barvami, pak existuje 
nekonečná množina $X$ taková, že všechny její prvky jsou stejné barvy.
\paragraph{Důkaz} Všimneme si, že když rozdělíme nekonečnou množinu na dvě 
poloviny (nebo více částí), alespoň jedna bude konečná. Pro obarvení 2mi-barvami 
to zjevně platí. Dále postupujeme indukcí: Rozdělme vrcholy podle první barvy a 
Označme množinu $C_1$ takovou největší -- odstraníme první barvu a aplikujeme 
indukci na problém obarvení $k-1$ barvami. \qed

\paragraph{Věta} (Ramseyova konečná) $\forall k \forall n \exists N$ takové, že 
obarvíme-li dvojice čísel $1..N$ k barvami, pak existuje množina $X\subset 
\{1,...,N\}$ taková, že všecny dvojice prvků z $X$ mají stejnou barvu, $|X| \geq 
n$ a navíc $|X| \geq \min\{X: x\in X\}$.
\paragraph{Důkaz} Sporem z nekonečné verze: Tvrdíme, že pro libovolně velké $N$ 
existuje obarvení $K_N$ bez velké monochromatické kliky. Podívejme se ale na 
dvojice čísel: pro každou dvojici čísel existuje nekonečně mnoho dvojic, které 
jsou stejně obarvené (kdyby ne, máme spor okamžitě). Z nekonečné věty víme, že 
existuje nekonečná monochromatická množina $Y$. Vezměme si tedy prvních $\max(n, 
\min Y)$ prvků a podívejme se na ně. Všechny mají stejnou barvu, je jich alespoň 
$n$ a navíc jich je alespoň tolik, kolik je velikost nejmenšího prvku v množině, 
což je spor.




\end{document}



























